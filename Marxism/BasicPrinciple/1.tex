%─────────────────%
% Header Settings %
%─────────────────%
\def\lecturer{李家丽}
\def\noter{THF}
\def\className{Marxism Basic Principle}
\def\term{III-B}
%─────────────────────%
% Undefined Variables %
%─────────────────────%
\ifx\noter\undefined
    \def\noter{THF}
\else
\fi
\ifx\lecturer\undefined
    \def\lecturer{None}
\else
\fi

%───────────────────%
% Document Settings %
%───────────────────%
\documentclass[12pt,a4paper]{article}

%─────────────────%
% Package Imports %
%─────────────────%
\usepackage[]{amsmath}
\usepackage[]{amssymb}
\usepackage[]{amsthm}
\usepackage[]{array}
\usepackage[]{bm}
\usepackage[]{booktabs}
\usepackage[UTF8]{ctex}
\usepackage[]{fancyhdr}
\usepackage[]{float}
\usepackage[]{geometry}
\usepackage[]{graphicx}
\usepackage[]{hyperref}
\usepackage[]{import}
\usepackage[]{inputenc}
\usepackage[]{mathrsfs}
\usepackage[]{multirow}
\usepackage[]{pdfpages}
\usepackage[]{pgfplots}
\usepackage[]{stmaryrd}
\usepackage[]{tabu}
\usepackage[]{tcolorbox}
\usepackage[]{textcomp}
\usepackage[]{thmtools}
\usepackage[]{tikz}
\usepackage[]{tkz-euclide}
\usepackage[]{url}
\usepackage[]{wrapfig}
\usepackage[dvipsnames]{xcolor}
\usepackage[]{xifthen}
\usepackage[]{yhmath}

%────────────────%
% Fancy Settings %
%────────────────%
\fancypagestyle{CustomStyle}{%
    \fancyhf{}
    \setlength{\headheight}{14.49998pt}
    \fancyhead[R]{\thepage}
    \fancyhead[L]{\lecturer: \className}
}

%──────────────────%
% pgfplot Settings %
%──────────────────%
\usepgfplotslibrary{external}
\pgfarrowsdeclarecombine{twolatex'}{twolatex'}{latex'}{latex'}{latex'}{latex'}
\pgfplotsset{compat=1.12}

%───────────────%
% Tikz Settings %
%───────────────%
\usetikzlibrary{arrows.meta}
\usetikzlibrary{decorations.markings}
\usetikzlibrary{decorations.pathmorphing}
\usetikzlibrary{positioning}
\usetikzlibrary{fadings}
\usetikzlibrary{intersections}
\usetikzlibrary{cd}
\tikzset{->/.style = {decoration={markings,mark=at position 1 with {\arrow[scale=2]{latex'}}},postaction={decorate}}}
\tikzset{<-/.style = {decoration={markings,mark=at position 0 with {\arrowreversed[scale=2]{latex'}}},postaction={decorate}}}
\tikzset{<->/.style = {decoration={markings,mark=at position 0 with {\arrowreversed[scale=2]{latex'}},mark=at position 1 with {\arrow[scale=2]{latex'}}},postaction={decorate}}}
\tikzset{->-/.style = {decoration={markings,mark=at position #1 with {\arrow[scale=2]{latex'}}},postaction={decorate}}}
\tikzset{-<-/.style = {decoration={markings,mark=at position #1 with {\arrowreversed[scale=2]{latex'}}},postaction={decorate}}}
\tikzset{->>/.style = {decoration={markings,mark=at position 1 with {\arrow[scale=2]{latex'}}}postaction={decorate}}}
\tikzset{<<-/.style = {decoration={markings,mark=at position 0 with {\arrowreversed[scale=2]{twolatex'}}},postaction={decorate}}}
\tikzset{<<->>/.style = {decoration={markings,mark=at position 0 with {\arrowreversed[scale=2]{twolatex'}},mark=at position 1 with {\arrow[scale=2]{twolatex'}}},postaction={decorate}}}
\tikzset{->>-/.style = {decoration={markings,mark=at position #1 with {\arrow[scale=2]{twolatex'}}},postaction={decorate}}}
\tikzset{-<<-/.style = {decoration={markings,mark=at position #1 with {\arrowreversed[scale=2]{twolatex'}}},postaction={decorate}}}
\tikzset{circ/.style = {fill, circle, inner sep = 0, minimum size = 3}}
\tikzset{scirc/.style = {fill, circle, inner sep = 0, minimum size = 1.5}}
\tikzset{mstate/.style={circle, draw, blue, text=black, minimum width=0.7cm}}
\tikzset{eqpic/.style={baseline={([yshift=-.5ex]current bounding box.center)}}}
\tikzset{commutative diagrams/.cd,cdmap/.style={/tikz/column 1/.append style={anchor=base east},/tikz/column 2/.append style={anchor=base west},row sep=tiny}}

%──────────────────────%
% Theorem Environments %
%──────────────────────%
\theoremstyle{definition}
\declaretheoremstyle[
    headfont=\bfseries\sffamily\color{ForestGreen!70!black}, bodyfont=\normalfont,
    mdframed={
        linewidth=2pt,
        rightline=false, topline=false, bottomline=false,
        linecolor=ForestGreen, backgroundcolor=ForestGreen!5,
    }
]{thmgreenbox}
\declaretheoremstyle[
    headfont=\bfseries\sffamily\color{NavyBlue!70!black}, bodyfont=\normalfont,
    mdframed={
        linewidth=2pt,
        rightline=false, topline=false, bottomline=false,
        linecolor=NavyBlue, backgroundcolor=NavyBlue!5,
    }
]{thmbluebox}
\declaretheoremstyle[
    headfont=\bfseries\sffamily\color{NavyBlue!70!black}, bodyfont=\normalfont,
    mdframed={
        linewidth=2pt,
        rightline=false, topline=false, bottomline=false,
        linecolor=NavyBlue
    }
]{thmblueline}
\declaretheoremstyle[
    headfont=\bfseries\sffamily\color{RawSienna!70!black}, bodyfont=\normalfont,
    mdframed={
        linewidth=2pt,
        rightline=false, topline=false, bottomline=false,
        linecolor=RawSienna, backgroundcolor=RawSienna!5,
    }
]{thmredbox}
\declaretheoremstyle[
    headfont=\bfseries\sffamily\color{RawSienna!70!black}, bodyfont=\normalfont,
    numbered=no,
    mdframed={
        linewidth=2pt,
        rightline=false, topline=false, bottomline=false,
        linecolor=RawSienna, backgroundcolor=RawSienna!1,
    },
    qed=\qedsymbol
]{thmproofbox}
\declaretheoremstyle[
    headfont=\bfseries\sffamily\color{NavyBlue!70!black}, bodyfont=\normalfont,
    numbered=no,
    mdframed={
        linewidth=2pt,
        rightline=false, topline=false, bottomline=false,
        linecolor=NavyBlue, backgroundcolor=NavyBlue!1,
    },
]{thmexplanationbox}
\declaretheorem[style=thmblueline,numbered=no,name=Notation]{notation}
\declaretheorem[style=thmgreenbox,numbered=no,name=Definition]{defi}
\declaretheorem[style=thmproofbox,numbered=no,name=Proof]{replacementproof}
\newtheorem*{aim}{Aim}
\newtheorem*{assumption}{Assumption}
\newtheorem*{axiom}{Axiom}
\newtheorem*{claim}{Claim}
\newtheorem*{conjecture}{Conjecture}
\newtheorem*{cor}{Corollary}
\newtheorem*{eg}{Example}
\newtheorem*{exercise}{Exercise}
\newtheorem*{ex}{Exercise}
\newtheorem*{fact}{Fact}
\newtheorem*{law}{Law}
\newtheorem*{lemma}{Lemma}
\newtheorem*{prop}{Proposition}
\newtheorem*{question}{Question}
\newtheorem*{remark}{Remark}
\newtheorem*{rrule}{Rule}
\newtheorem*{thm}{Theorem}
\newtheorem*{warning}{Warning}
% \newtheorem{ncor}[nthm]{Corollary}
% \newtheorem{nlemma}[nthm]{Lemma}
% \newtheorem{nprop}[nthm]{Proposition}
% \newtheorem{nthm}{Theorem}[section]
\renewenvironment{proof}[1][\proofname]{\vspace{-10pt}\begin{replacementproof}}{\end{replacementproof}}

%─────────────%
% Math Symbol %
%─────────────%

%────────────%
% Beginnings %
%────────────%
\title{\textbf{Part III-B: \className}}
\author{Lecture by \lecturer\\Note by \noter}
\pagestyle{CustomStyle}
%────────────────────────────────%
% Page Settings (set when print) %
%────────────────────────────────%
% \addtolength{\parskip}{-1mm}
% \addtolength{\parindent}{-2mm}
% \geometry{left=0.5cm,right=0.5cm,top=0.5cm,bottom=0.5cm}


%──────────%
% Document %
%──────────%
\begin{document}
\maketitle
\tableofcontents
\section*{课程介绍}%
\label{subsub:课程介绍}
24节课(1-18周,48学时)

3学分

\subsection*{要求}%
\label{sub:要求}
闭卷考试

成绩组成:平时(40\%)+期末(60\%)

平时成绩:点名(10\%,一次缺勤-2\%)+读书报告(10\%,学期内阅读一本经典文献)+课堂作业(20\%,共2-3次,随堂写作)

请假时未在课堂上完成的作业与假条一同提交

不提交假条的补交作业较平时评分低

教材:马克思主义基本原理

\subsection*{期末考试范围}%
\label{sub:期末考试范围}
导论,第一到三章,第四章1,2节

总分(100\%)=单选(20\%,20道)+多选(10\%,5道,选多与选少都不得分)+辨析(30\%,需写出判断和正误原因)+简答(20\%,题干上存在题目要点)+材料分析(20\%)
\subsection*{课堂要求}%
\label{sub:课堂要求}

\section{导论}%
\label{sec:导论}
马克思主义基本原理

马克思:千年第一思想家,千年伟人
\subsection{什么是马克思主义}%
\label{sub:什么是马克思主义}
\begin{notation}
    马克思主义是由马克思与恩格斯创立并为后继者所不断发展的科学理论体系,是关于自然、社会和人类思维发展一般规律的学说,是关于社会主义必然代替资本主义,最终实现共产主义的学说,是关于无产阶级解放、全人类解放和每个人资源而全面发展的学说,是指引人民创作美好生活的行动指南
\end{notation}
马克思主义理论是一个博大精深的理论体系:
\[
    \mbox{马克思主义}
    \begin{cases}
        \mbox{马克思主义哲学}\\ 
        \mbox{马克思主义政治经济学}\\ 
        \mbox{科学社会主义}
    \end{cases}
.\] 
还包括其他如历史学,政治学,法学,文化学,新闻学,军事学等,并随着实践和科学的发展而不断丰富自身的内容

党中央成立了中央编译局,系统地编译马克思主义经典著作,至今已有2版翻译,全集共30卷

\[
    \mbox{马克思主义基本原理}
    \begin{cases}
        \mbox{马克思主义基本立场}\\ 
        \mbox{马克思主义基本观点}\\ 
        \mbox{马克思主义基本方法}
    \end{cases}
.\] 
\begin{notation}
    马克思主义的基本立场是\textbf{人民立场},以人民为中心,一切为了人民,一切依靠人民
\end{notation}
\begin{notation}
    马克思主义的基本方法:
\end{notation}
\begin{notation}
    马克思主义的基本观点:
\end{notation}

\subsection{马克思主义的创立}%
\label{sub:马克思主义的创立}
马克思主义产生于19世纪40年代,创始人是马克思与恩格斯

马克思主义的产生与创立具有深刻的社会根源、阶级基础和思想渊源

\begin{notation}
    社会根源:18世纪60年代到19世纪,工业革命和科技进步极大地提高了劳动生产率,促进生产力的发展

    工业革命一方面带来了社会化大生产的迅猛发展,另一方面造成了深重的社会灾难

    财富增加伴随着贫困的扩散,生产的发展却引发经济危机

    阶级基础:无产阶级反对资产阶级的斗争,19世纪30-40年代爆发了“三大工人运动”:丝织工人起义(1831,法国里昂)、宪章运动(1836,英国)、纺织工人起义(1841,德国)

    工人阶级作为独立的政治力量已经登上了历史舞台,这种斗争必须有革命理论的知道和无产阶级政党的领导

    思想渊源:
    \[
        \mbox{直接理论来源}\\ 
        \begin{cases}
            \mbox{德国古典哲学(黑格尔):辩证法}\\ 
            \mbox{英国古典政治经济学(亚当斯密):劳动价值论}\\ 
            \mbox{空想社会主义}
        \end{cases}
    .\] 
\end{notation}
\begin{notation}
    古典政治经济学认为资本是永恒的自然关系,掩饰资本主义的基本矛盾,否定资本主义经济危机的可能性
\end{notation}
\begin{notation}
    空想社会主义主要的缺陷:其描述的世界运作状态尚未提出合理的实现方式

    1516年:《乌托邦》确定了空想社会主义

    柏拉图著有《理想国》

    1848年:科学社会主义建立
\end{notation}
\begin{eg}
    马克思主义创立的直接理论来源:(BCD)

    A. 中国古典哲学

    B. 德国古典哲学

    C. 英国古典政治经济学

    D. 空想社会主义
\end{eg}
\begin{notation}
    马克思:律师家庭出身,17岁写下《青年在选择职业时的考虑》,1841年就读波恩、柏林大学,在耶拿大学获得PhD;1842年5月,任《莱茵报》编辑工作;1843年4月,因工作问题离开祖国终身漂泊

    1844年2月,马克思和恩格斯统一了唯物主义思想;1844年8月,马克思与恩格斯相遇

    马克思和恩格斯合作了《神圣家族》;1845-1846,合作《德意志意志形态》;1848年,《共产党宣言》正式发表,标志着马克思主义的公开问世;后来马克思回到德国创建了《新莱茵报》

    1867年9月,发表《资本论》第一卷,系统阐述了剩余价值学说;1871年代表第一国际著《法兰西内战》;1876-1878年,恩格斯著《反杜林论》,全面阐述马克思主义理论体系
    
    马克思一生流离各地,途经德国、法国、比利时等地区,于1883年去世

    马克思去世后,恩格斯整理了《资本论》二、三卷
\end{notation}
\subsection{马克思主义的发展}%
\label{sub:马克思主义的发展}
\begin{notation}
    时代背景

    客观:战争加剧了资本主义国家的内部矛盾(第一次世界大战)

    主观:俄国成为帝国主义各种矛盾的焦点与集合点(无产阶级/封建地主阶级/资产阶级,自助游/农奴制,民族矛盾等),因此成为了帝国主义提醒等中最薄弱的环节
\end{notation}
列宁认为:资本主义发达国家已经发展到帝国主义阶段,经济政治发展的不平衡已经成为资本主义发展的绝对规律,并提出社会主义革命可能在一国或数国首先发生并取得胜利的论断

\begin{notation}
    十月革命的胜利确定了列宁的正确,列宁把马克思主义基本原理和俄国本地情况相结合,成立了列宁主义(布尔什维克主义),后来出版列宁全集
\end{notation}
李大钊、陈独秀等人受马列主义影响,以《新青年》和《每周评论》为阵地传播马克思主义

中共一大成立了中国共产党,并印制了共产党宣言

\begin{notation}
    马克思主义中国化的伟大成果:

    1. 毛泽东思想

    2. 邓小平理论

    3. “三个代表”重要思想

    4. 科学发展观

    5. 习近平新时代中国特色社会主义思想
\end{notation}
\begin{notation}
    现存的社会主义国家:中国、朝鲜、越南、古巴、老挝
\end{notation}

\subsection{马克思主义的基本特征}%
\label{sub:马克思主义的基本特征}
\[
    \begin{cases}
        \mbox{科学的理论}\\ 
        \mbox{人民的理论}\\ 
        \mbox{实践的理论}\\ 
        \mbox{不断发展的开发的理论}
    \end{cases}
.\] 
\subsubsection{科学的理论}%
\label{subsub:科学的理论}
马克思主义是对自然、社会、人类思维发展本质和规律的\textbf{正确反映}

1. 马克思主义在社会实践和科学发展的基础上产生,并在发展过程中不断总结经验,\textbf{吸取自然科学和社会科学发展的最新成就}

2. 马克思主义具有科学的世界观和方法论基础

3. 马克思主义理论是一个逻辑严密的有机整体

\subsubsection{人民的理论}%
\label{subsub:人民的理论}
“人民至上”是马克思主义的政治立场:人民群众是历史的创造者,是社会主义事业的依靠力量
\begin{notation}
    中国共产党人的初心和使命就是\textbf{为中国人民谋幸福},为中华民族谋复兴
\end{notation}
\subsubsection{实践的理论}%
\label{subsub:实践的理论}
1. 马克思主义的使命和作用决定了它是直接服务于无产阶级和人民群众改造世界的实践活动的科学理论

2. 马克思主义的内容反映了实践观点是马克思主义首要的和基本的观点,始终强调理论和实践统一
\subsubsection{发展的理论}%
\label{subsub:发展的理论}
马克思主义具有与时俱进的理论品质,是时代的产物

1. 马克思主义理论体系是开放的

2. 当今世界和我们所处的新时代在不断变化
\begin{notation}
    马克思主义的鲜明特征是\textbf{科学性与革命性的统一}
\end{notation}


\end{document}

