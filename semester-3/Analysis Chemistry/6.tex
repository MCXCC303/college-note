\lecture{6}{10.22}
\subsection{酸碱指示剂}%
\label{sub:酸碱指示剂}
\subsubsection{指示剂的特点}%
\label{subsub:示剂的特点-}
指示剂(Indicator, In)

1. 弱的有机酸/碱

2. 共轭酸碱对的颜色明显不同

3. 在不同pH下结构变化
\[
    \ce{HI <=> H+ +In-}
.\] 
\begin{notation}
    指示剂的作用原理:指示剂的分子结构变化
\end{notation}
\begin{eg}
    常用指示剂:

    1. 甲基橙(MO):$\text{pK}_a=3.4$ ,碱性呈黄色,酸性下质子化呈红色

    2. 酚酞(PP):$\text{pK}_a=9.1$,酸性下呈无色,碱性呈红色
\end{eg}
\begin{notation}
    不同指示剂变色点和变色范围不同
\end{notation}
对于解离平衡:
\[
    \ce{HI <=> H+ +In-} \hspace{0.5cm} K_{\text{HIn}}=\frac{[\ce{H+}][\ce{In-}]}{[\text{HIn}]}
.\] 
\[
    \implies \frac{K_{\text{HIn}}}{[\ce{H+}]}=\frac{[\ce{In-}]}{[\text{HIn}]}
.\] 

当$\text{pH}\ge \text{p}K_{\text{HIn}}+1$或$[\ce{In-}] / [\text{HIn}]\ge 10$ 时看到的是碱式色

当$\text{pH}\le  \text{p}K_{\text{HIn}}-1$或$[\ce{In-}] / [\text{HIn}]\le  \frac{1}{10}$ 时看到的是碱式色

\begin{notation}
    理论变色范围:pH=$\text{p}K_{\text{HIn}}\pm 1$

    理论变色点:$\text{pH}= \text{p}K_{\text{HIn}}$或$[\ce{In-}] / [\text{HIn}]=1$

    甲基橙的理论变色点:4.4 $\sim $ 2.4

    甲基橙的实际变色点:4.4 $\sim $ 3.1
\end{notation}
\begin{notation}
    指示剂的变色范围越窄,变色越敏锐
\end{notation}
\subsubsection{指示剂变色范围的影响}%
\label{subsub:指示剂变色范围的影响}
1. 温度:$T\to K_{\text{HIn}}\to $变色范围变化
\begin{eg}
    甲基橙( $18^\circ C$):$3.1\sim 4.4$ 

    甲基橙($100^\circ C$):$2.5\sim 3.7$
\end{eg}
2. 电解质:$c_\text{离子浓度}\to K_{\text{HIn}}\to$ 变色范围变化

3. 滴定次序:无色$\to $ 有色,浅色$ \to $深色

4. 指示剂用量:
\begin{eg}
    单色指示剂:

    设指示剂浓度为$C_{\text{HIn}}$,当[$\ce{In-}$]=$a$达到一定浓度时观察到颜色发生变化
    \[
        \frac{K_{\text{HIn}}}{[\ce{H+}]}=\frac{a}{C-a}
    .\] 

    当$C_{\text{HIn}}$ 变化时pH 也变化,导致变色点偏移,即浓度可影响变色范围
\end{eg}
双色指示剂:与$C_{\text{HIn}}$ 无关
\subsubsection{混合指示剂}%
\label{subsub:混合指示剂}
\begin{notation}
    混合指示剂:变色更敏锐、范围更窄
\end{notation}
1. 指示剂+惰性染料
\begin{eg}
    甲基橙+靛蓝:变色范围$4.4\sim 3.1$ ,变化颜色:绿色$\to $ 无色$\to $ 紫色
\end{eg}
2. 混合两种或两种以上的指示剂
\begin{eg}
    溴甲酚绿+甲基红:变色范围$4.9\sim 5.1\left( \pm 0.1 \right) $ ,变色:橙红 $\to $ 灰色$\to $ 绿色
\end{eg}
\subsection{酸碱滴定曲线}%
\label{sub:酸碱滴定曲线}
\begin{notation}
    $x$ 轴的两种:

    1. 滴定体积$V_T\text{ or }V_t$ 

    2. 滴定分数$\displaystyle{\frac{V_T}{V_{\text{Total}}}}$
\end{notation}
\subsubsection{强酸碱的滴定}%
\label{subsub:强酸碱的滴定}
滴定常数:\[
    K_t=\frac{1}{[\ce{H+}][\ce{OH-}]}=\frac{1}{K_w}=10^{14}
.\] 
\begin{align*}
    \text{滴定的四个阶段}
    \begin{cases}
        \text{滴定开始前}\\
        \text{化学计量点前}\\
        \text{化学计量点}\\
        \text{化学计量点后}
    \end{cases}
.\end{align*}
\begin{eg}
    使用NaOH($V_b$)滴定HCl($V_a,c_a=0.1\text{mol/L}$)
\end{eg}
1. 滴定开始前($V_b=0$):溶液组成:HCl

2. 滴定至化学计量点前(before sp.):溶液:HCl+NaCl
\[
    [\ce{H+}]=c_a \frac{V_a-V_b}{V_a+V_b}
.\] 
\begin{notation}
    化学剂量点前$0.1\%$:pH=4.3
\end{notation}
3. 化学计量点(sp.):溶液:NaCl,pH=7

4. 化学计量点后(after sp.):溶液:NaCl+NaOH
\begin{notation}
    化学剂量点后$0.1\%$:pH=9.7,滴定剂仅多加0.2mL
\end{notation}
\begin{notation}
    化学计量点前后滴定分数0.1 \%为滴定突越范围($\Delta \text{pH}$)
\end{notation}
\begin{notation}
    被滴定试剂$\ce{H+ / OH-}$浓度越大,可选指示剂越多(突越范围越大)

    浓度每增加10倍,突越范围($\Delta \text{pH}$)增大2个单位
\end{notation}
\begin{notation}
    指示剂选择原则:指示剂的变色范围部分或全部落在$\Delta \text{pH}$ 内
\end{notation}
\subsubsection{弱酸碱(一元)的滴定}%
\label{subsub:弱酸碱-一元-的滴定}
\begin{eg}
    NaOH0.1000mol/L滴定醋酸(HAc)0.1000mol/L
\end{eg}
滴定常数: \[
    K_t=\frac{1}{K_b}=\frac{K_a}{K_w}
.\]

1. 滴定开始前(起点较高):\[
    [\ce{H+}]=\sqrt{K_ac_a}\implies\text{pH}\approx 2.88 
.\] 

2. 化学计量点前(存在缓冲作用,pH增加速率减缓):
\begin{align*}
    [\ce{Ac-}]&= \frac{c_bV_b}{V_a+V_b} \\
    [\text{HAc}]&= \frac{c_aV_a-c_bV_b}{V_a+V_b}\\
    \text{pH}&=\text{p}K_a+\lg \frac{[\ce{Ac-}]}{[\text{HAc}]}\approx 7.76
.\end{align*}

3. 化学计量点(滴定突越范围减小)
\begin{align*}
    [\ce{OH-}]&=\sqrt{K_bc_b} =\sqrt{\frac{K_a}{K_w}c_b} \\
    \text{pOH}&\approx 5.28 \hspace{0.5cm} \text{pH}\approx 8.72
.\end{align*}

4. 化学计量点后与强酸碱滴定一样

指示剂选择:酚酞
\begin{notation}
    强酸滴定弱碱:\[
        \ce{H_3O+A- <=> HA+H_2O}
    .\] 

    指示剂选择:甲基橙、甲基红
\end{notation}

