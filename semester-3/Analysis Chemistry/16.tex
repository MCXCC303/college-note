\lecture{16}{12.31}
\begin{notation}
    沉淀形成的过程:
    \begin{enumerate}
        \item 成核作用
        \item 聚集
    \end{enumerate}
    第一步:构晶离子$\xrightarrow[\text{均相、异相}]{\text{成核作用}}$ 晶核,均相成核为自发过程;第二步:溶液中的离子聚集到晶核上,形成沉淀
\end{notation}
\begin{eg}
    醋酸钠的结晶可以使用异相成核(投入一些晶核)
\end{eg}
\begin{notation}
    Eiichi Nakamura等人\cite{doi:10.1021/jacs.0c12100} 2021年在JACS发表文章,将超高分辨率的电子显微镜与快速灵敏的成像传感器结合,实现对分子的成像精度<0.1mm;通过将约100个氯化钠分子装入碳纳米管网兜中而观察。
\end{notation}
\begin{notation}
    沉淀的聚集速度主要由溶液的过饱和度决定:\[
        V=\frac{K\left( Q-S \right)}{S}
    .\]
    其中$Q$ 为加入沉淀剂是溶质的瞬间总浓度,$K$ 是一个性质常数
\end{notation}
\subsubsection*{沉淀纯度的影响因素}%
\label{subsub:沉淀纯度的影响因素}
\begin{enumerate}
    \item 共沉淀(表面吸附$\to $ 洗涤,形成混晶$\to $ 除杂、缓慢加入沉淀剂、陈化,包埋或吸留$\to $ 陈化、重结晶)
    \item 后沉淀
\end{enumerate}
\begin{notation}
陈化的作用:利用不同沉淀之间晶格的能量不同除去混晶
\end{notation}
\begin{notation}
    选择沉淀条件:

    形成晶型沉淀(降低聚集速度、生产大颗粒纯净晶体):
    \begin{itemize}
        \item 稀溶液
        \item 热溶液
        \item 搅拌
        \item 缓慢加入沉淀剂
        \item 加热陈化
        \item 陈化
    \end{itemize}
    形成无定形沉淀:
    \begin{itemize}
        \item 浓溶液
        \item 热溶液
        \item 搅拌
        \item 快速加入沉淀剂
        \item 无需陈化
    \end{itemize}
\end{notation}
\begin{notation}
沉淀的过滤和干燥:
\begin{itemize}
    \item 无灰滤纸
    \item 溶解度小的使用蒸馏水洗涤
    \item 溶解度大的使用沉淀的饱和溶液洗涤
    \item 易胶溶的沉淀使用易挥发的电解质稀溶液洗涤
    \item 沉淀溶解度随温度变化不大的用热溶液洗涤
    \item 110至120 $^\circ\text{C}$加热40至60分钟烘干
    \item 800 $^\circ\text{C}$ 以上灼烧
\end{itemize}
干燥要求:恒重,两次重量差异不超过0.3mg
\end{notation}
\begin{notation}
    重量因数或换算因数的计算:
    \[
        F=\frac{a\times MR_{a}}{b\times MR_{b}}
    .\]
    质量分数:\[
        \omega\left( \% \right) = \frac{m\times F}{m'}\times 100\%
    .\]
    其中$m$ 为称量形式的质量,$m'$ 为试样的质量
\end{notation}
{\centering{\section*{结课}%
\label{sec:结课}
}}
考试时间1月10号,无计算题(大题),在小题中有计算,可以使用计算器
