\subsection{如何提高分析结果的准确度}%
\label{sub:如何提高分析结果的准确度}
1. 选择合适的分析方法:
\begin{notation}
    对于化学分析法:适用于常量组分(>1\%)的分析

    对于仪器分析法:适用于微量组分(0.01 $\sim $ 1\%)或痕量组分(<0.01\%)的分析
\end{notation}

2. 减少相对误差:增大取样量

3. 减少偶然误差:增加平行测试次数($\ge 3$,活体测试$\ge 6$)

4. \textbf{消除}系统误差:
\begin{notation}
    常用方法:

    1. 与经典方法比较:(用其他方法测试该方法)

    2. 对照试验(control test):(用其他样本对比)

    3. 回收试验/标准加入法:测量原样本$x_1$ ,加入某标准量$x_2$ ,测量加入后的样本$x_3$ ,计算回收率:\[
        \text{Recovery}\%=\frac{x_3-x_1}{x_2}\times 100\%\approx 95\%\sim 105\%
    .\] 

    4. 空白试验:不加试样测试,得到并扣去空白值,用于检验由试剂、容器等引入杂质导致的系统误差

    5. 仪器校正
\end{notation}

\subsection{有效数字}%
\label{sub:有效数字}
\begin{defi}
    有效数字:分析工作中实际上能测量到的数字
\end{defi}
原则上:

1. 在记录测量数据时,只允许保留一位可疑数字(欠准数字)

2. 误差是末尾数的$\pm 1$个单位
\begin{eg}
    在分析天平上称出$m=21.5370\text{g}$ ,则真值为$21.5370\pm 0.0001\text{g}$
\end{eg}
\begin{notation}
    有效数字的末尾0不可省略:反映相对误差
\end{notation}

\begin{question}
    如何判断有效数字的位数?
\end{question}
1. 在数据中1 $\sim $ 9均为有效数字(0待定)

2. 算式中的倍数、分数及某些常数($\pi,\text{e}$ 等)可看为无数位有效数字

3. 变化单位时有效数字的位数必须保持不变,如$0.0015\text{g}=1.5\text{mg}$ 

4. pH和$\text{pK}_{\text{a}}$ 等对数值,有效数字仅取决于小数部分,例:$\text{pH}=4.23$,有效数字2位

\begin{notation}
    0的位置和有效数字:

    1. 小数前的0起定位作用,后面的0为有效数字

    2. 整数后的0不一定是有效数字

    \begin{eg}
        $36000 \Rightarrow 3.60\times 10^4$:3位有效数字
    \end{eg}    
\end{notation}
\subsubsection{修约规则}%
\label{subsub:修约规则}
1. 四舍六入五留双,五后有数需进位

2. 修约标准偏差:只进不舍,降低精密度,提高可信度

\begin{table}[htpb]
    \centering
    \caption{修约为两位}
    \label{tab:修约为两位}
    \begin{tabular}{ccc}
    \toprule
    原值 & 修约值 & 原因\\
    \midrule
    3.249 & 3.2 & 四舍\\
    8.361 & 8.4 & 六入\\
    6.550 & 6.6 & 五留双\\
    6.250 & 6.2 & 五留双\\
    6.252 & 6.3 & 五后有数\\
    \bottomrule
    \end{tabular}
\end{table}

2.1. 可多保留一位有效数字进行计算

2.2. 与标准限度值比较不应修约
\subsubsection{运算规则}%
\label{subsub:运算规则}
1. 加减法:结果的小数位数以小数点后位数最少的为标准
\begin{eg}
\[
    0.0121+25.64+1.057=26.7091\approx 26.71
.\]
\end{eg}

2. 乘除法:取相对误差最大的为标准
\begin{eg}
    \[
        \frac{0.0325\times 5.10\times 60.1}{139.8}\approx 0.0712
    .\]     
\end{eg}

\begin{notation}
    对于高含量组分$\left( w>10\% \right) $,分析结果一般保留4位有效数字

    对于中等含量组分$\left( 1\%<w<10\% \right) $,保留3位

    对于微量组分$\left( w<1\% \right) $,保留2位
\end{notation}

\subsection{有限量分析数据的统计处理}%
\label{sub:有限量分析数据的统计处理}
\subsubsection{偶然误差的正态分布}%
\label{subsub:偶然误差的正态分布}
正态分布的概率密度函数:\[
    y=f\left( x \right) =\frac{1}{\sigma \sqrt{2\pi} }\text{e}^{-\left( x-\mu \right) ^{2}/2\sigma^{2}}
.\] 
\begin{notation}
    $x$ :测量值

    $\mu$ :无限次测量的总体平均值

    $\sigma$ :总体偏差
\end{notation}
特点:

1. 当$x=\mu$ 时,$y$ 最大:大部分测量值集中在算术平均值附近

2. 函数图像以$x=\mu$ 的直线对称:正负误差出现的概率相同

3. $x\to -\infty$或$x\to +\infty$ 时:无限趋近$x$ 轴

4. $\mu$ 越大,数据越分散,函数图像矮小、坡度较缓

5. 
\[
    \int_{-\infty}^{+\infty} f\left( x \right) \text{d}x=1
.\] 

\begin{notation}
    若令\[
        u=\frac{x-\mu}{\sigma}
    .\] 

    则:\[
        y=f\left( x \right) =\frac{1}{\sigma\sqrt{2\pi}} \text{e}^{-\frac{u^{2}}{2}}
    .\] 
    
    称为标准正态分布曲线
\end{notation}

\begin{notation}
    $3\sigma$ 标准:测量结果需要有$99.7\%$ 以上的数据在真值范围内,即:
    \[
        \int_{\mu-3\sigma}^{\mu+3\sigma}f\left( x \right) \text{d}x\approx 99.7\%
    .\] 
\end{notation}

\subsubsection{T分布}%
\label{subsub:T分布}
\begin{defi}
    T分布为有限量数据n平均值的概率误差分布
\end{defi}
\[
    t=\frac{\left| \bar{x}-\mu \right| }{S}\sqrt{n} 
.\] 

$x$ :样本平均值

$S$ :样品标准偏差

$t$ 值随自由度$f\left( f=n-1 \right) $ 而变

当$f\to +\infty$ 时为正态分布
\begin{notation}
    T分布和正态分布的异同:

    相同:形状相似,积分面积表示概率

    不同:T不同时概率不同
\end{notation}
\begin{notation}
    T分布相关概念:

    1. 自由度$f$ 

    2. 置信区间:以测定区间为中心,包括总体平均值在内的可信范围\[
        \bar{x}\pm \frac{tS}{\sqrt{n} }
    .\]

    3. 置信水平(置信度$P$):样本平均值落在置信区间的概率\[
        P = \mu\pm \frac{tS}{\sqrt{n} }
    .\]

    4. 显著性水平$\alpha=1-P$
\end{notation}
$t$值的表达:一定P下,$t\to t_{\alpha,f}$ 
\subsubsection{平均值的精密度和置信区间}%
\label{subsub:平均值的精密度和置信区间}
\begin{eg}
    有一个样品,$m$个人各测量$n$次,计算出每个人测得的平均值,平均值的分布符合正态分布
\end{eg}
平均值的标准偏差:
\[
    S_{\bar{x}}=\frac{S_{x}}{\sqrt{n} }
.\] 

即:增加平行测定次数$n$ 可以减小平均值的标准偏差$S_{\bar{x}}$ 
\begin{notation}
    增加测定次数,平均值的标准偏差呈反比变化,一般3到4次已经可以达到目标,继续增加效果不显著
\end{notation}
