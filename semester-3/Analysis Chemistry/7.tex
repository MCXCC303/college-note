\lecture{7}{10.28}
\begin{notation}
    $K_a<10^{-9}$的弱酸无法准确滴定

    判断弱酸/弱碱能否被准确滴定:
    \[
        c_aK_a\ge 10^{-8}\hspace{0.5cm} c_bK_b\ge 10^{-8}
    .\] 
\end{notation}
\begin{eg}
    酸的浓度为0.1000mol/L,则其 $K_a\ge 10^{-7}$ 才能被准确滴定
\end{eg}
\subsubsection{多元酸/碱的滴定}%
\label{subsub:多元酸/碱的滴定}
\begin{notation}
    首先解决:

    1. 能准确滴定至第几级解离产物

    2. 是否能准确滴定、能形成几个pH突越

    3. 选择什么指示剂
\end{notation}
\subsubsection*{准确滴定}%
\label{subsub:准确滴定}
\[
    c_aK_a\ge 10^{-8} \hspace{0.5cm} c_bK_b\ge 10^{-8}
.\] 
\subsubsection*{分布滴定}%
\label{subsub:分布滴定}
\[
    \frac{K_{a_1}}{K_{a_2}} \ge 10^{4}\hspace{0.5cm} \frac{K_{b_1}}{K_{b_2}} \ge 10^{4}
.\] 
判断第二级解离的$\ce{H+}$ 是否影响第一步
\begin{eg}
    用0.1000mol/L NaOH滴定0.1000mol/L磷酸:

    磷酸:$K_{a_1}=10^{-2.16}$ ,$K_{a_2}=10^{-7.12}$,$K_{a_3}=10^{-12.32}$ 

    判断是否能准确滴定:

    第一步:$c_aK_{a_1}=10^{-1.16}>10^{-8}$ 且$K_{a_1}/K_{a_2}=10^{4.96}>10^{4}$

    第二步同理:可以准确分步滴定

    第三步:$c_aK_{a_3}<10^{-8}$ ,不能准确滴定
\end{eg}
\subsubsection*{指示剂选择}%
\label{subsub:指示剂选择}
只看化学计量点的pH
\begin{eg}
    以磷酸为例:第一化学计量点pH=4.68:甲基橙、甲基红、溴甲酚绿+甲基橙

    第二化学计量点pH=9.76:酚酞、百里酚酞、酚酞+百里酚酞
\end{eg}
\subsection{滴定应用}%
\label{sub:滴定应用}
\subsubsection{酸碱标准溶液的配置}%
\label{subsub:酸碱标准溶液的配置}
\begin{defi}
    基准物质:用于直接配置或标定标准溶液的物质
\end{defi}
\begin{notation}
    基准物质常用纯金属或纯化合物
\end{notation}
对基准物质的要求:

1. 组成与化学式完全相符

2. 纯度足够高(主成分含量>99.9\%)

2.1. 杂质不能影响反应

3. 性质稳定

4. 有较大的摩尔质量

5. 按滴定反应式定量反应
\begin{defi}
    标准溶液:已知\textbf{准确}浓度的试剂溶液
\end{defi}
标准溶液浓度:物质的量浓度$c$
\begin{notation}
    滴定度(\textit{titer}):每\textbf{毫升}标准溶液相当于被测物质的\textbf{质量},用$\bm{T}_{T / B}$ 表示
    \[
        \bm{T}_{T/B}=\frac{m_B}{V_T} 
    .\] 
\end{notation}
配置标准溶液:
\begin{notation}
    直接法:称量$\Rightarrow $ 溶解$\Rightarrow $ 定容$\Rightarrow $ 标签
\end{notation}
\begin{notation}
    标定法(非标准物质的标准溶液配置):

    配置为近似于所需浓度的溶液后,使用标定后的标准溶液标定该溶液
\end{notation}
\begin{eg}
    配置0.1 mol/L HCl 标准溶液:

    1. 浓盐酸稀释为近似0.1 mol/L

2. 用基准物质硼砂$\ce{Na_2B_4O_7}\cdot \ce{10H_2O}$ 标定
\end{eg}
\subsubsection{常用酸碱标准溶液的配置与标定}%
\label{subsub:常用酸碱标准溶液的配置与标定}
\begin{notation}
    酸标准溶液:最常用0.1 mol/L

    最常用HCl,配置方法:浓盐酸间接法

    标定使用的基准物质:无水碳酸钠(易吸湿),硼砂(易风化)

    常用指示剂:甲基橙、甲基红
\end{notation}
\begin{notation}
    碱标准溶液:最常用NaOH,配置方法:浓碱间接法(NaOH易吸水和$\ce{CO_2}$,KOH较贵)

    标定使用基准物质:邻苯二甲酸氢钾(纯净、易保存、摩尔质量大)、草酸
\end{notation}
\subsubsection{酸碱滴定分析中的计算}%
\label{subsub:酸碱滴定分析中的计算}
计量关系:
\[
    t \text{T}+b \text{B}=c \text{C}+ d \text{D}
.\] 
 
\begin{notation}
    标定法配置:
    \[
        c_T=\frac{t}{b} \times \frac{m_b}{M_bV_T} 
    .\] 
\end{notation}
\begin{notation}
    物质的量浓度和滴定度之间的关系:
    \[
        \frac{n_b}{n_t} =\frac{c_T\cdot V_T}{T_{T/B}\cdot V_T/M_B} =\frac{c_b\times 10^{-3}\times M_B}{T_{T /B}} 
    .\] 
\end{notation}
\begin{notation}
    被测组分百分含量:
    \[
        \omega_B\%=\frac{m_B}{m} =\frac{n_BM_B}{m} =\frac{b}{t} \times \frac{c_tV_tM_B}{m} \times 100\%
    .\] 
\end{notation}
\begin{eg}
    $T_{\ce{K_2Cr_2O_7 /Fe}}=0.005022\text{g/mL}$,测定0.5000g Fe,用去标准溶液25.10 mL,计算 $T_{\ce{K_2Cr_2O_7 /Fe_3O_4}}$ 和试样中Fe和$\ce{Fe_3O_4}$ 的质量分数
\end{eg}
\subsubsection{滴定方式}%
\label{subsub:滴定方式}
\[
    \begin{cases}
        \text{直接滴定}\\
        \text{间接滴定}\\
        \text{返滴定}\\
        \text{置换滴定}
    \end{cases}
.\] 
\begin{notation}
    直接滴定要求(重点):

    1. 反应必须反应完全、定量进行

    2. 反应必须较快

    3. 反应必须有确定的化学计量关系

    4. 必须有适当简便的方法确定终点
\end{notation}
\begin{eg}
    用NaOH滴定乙酰水杨酸

    缺点:乙酰基可能被碱水解

    改进:使用中性乙醇溶解,使用已知滴定度计算
\end{eg}
\begin{notation}
    返滴定:适用于反应较慢、难溶、无合适的指示剂

    1. 准确加入定量且过量的标准溶液A

    2. 加入待测物质

    3. 等待彻底反应完全

    4. 使用另一种标准溶液B滴定剩余的标准溶液A
\end{notation}
\begin{eg}
    HCl标定固体ZnO(难溶)、HCl标定$\ce{CaCO_3}$,$\ce{AgNO_3}$ 标定$\ce{Cl-}$
\end{eg}
\begin{notation}
    置换滴定:适用于无明确定量关系、有副反应

    1. 用适当试剂与待测物质反应,定量置换出另一种物质

    2. 用标准物质滴定置换出的物质
\end{notation}
\begin{notation}
    间接滴定:适用于不能与滴定剂直接反应
\end{notation}
\begin{eg}
    $\ce{KMnO_4}$ 滴定$\ce{Ca^2+}$ :先使用草酸沉淀,使用硫酸溶解,用高锰酸钾测定脱落的草酸根浓度
\end{eg}
\begin{eg}
    硼酸(酸性极弱,不能直接滴定):使用甘油结合生成甘油硼酸($K_a=4.26$)后可以滴定
\end{eg}






