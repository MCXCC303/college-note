\lecture{4}{10.08}
\begin{notation}
    常用估计法:点估计

    求$x$ 和$\bar{x}$ 进行比较
\end{notation}
平均值的置信区间:

1. 单次
 \[
    \mu=x\pm u\sigma
.\] 

其中$u=(x-\mu)/\sigma$(见标准正态分布曲线)

2. 多次:使用$\sigma_{\bar{x}}$代替$\sigma$
\[
    \mu=\bar{x} \pm u\sigma_{\bar{x}}=\bar{x}\pm u\cdot \frac{\sigma}{\sqrt{n} }
.\] 

3. 少量:$t$ 分布
\[
    \mu=\bar{x}\pm tS_{\bar{x}}=\bar{x}\pm\frac{tS}{\sqrt{n} }
.\] 

可以用$X_{U}$ 表示上限,$X_{L}$ 表示下限

\begin{notation}
    置信水平越高,置信区间越宽

    通常置信水平取$P=0.95$ 或$\alpha=0.05$ 

    置信区间反映估计的精密度

    置信水平说明估计的把握程度
\end{notation}
\begin{notation}
    1. $\mu$ 为一个定值,无随机性

    2. 单侧检验大于或小于总体均值,双侧检验同时大于和小于总体均值
\end{notation}
\subsubsection{可疑数据的取舍和显著性检验}%
\label{subsub:可疑数据的取舍和显著性检验}
主要使用$G$ 检验(可疑数据取舍)$\to $ $F$检验 (精密度检验)$\to $ $t$ 检验(准确度检验)
\begin{notation}
    $F$ 检验和$t$ 检验合称显著性检验
\end{notation}

\begin{notation}
    $G$ 检验

    在一组平行测量数据中有过高或过低的数据,称为可疑数据/异常值/逸出值

    1. 确定可疑值$x_{q}$ ,求出包括可疑值在内的平均值$\bar{x}$

    2. 求出可疑值与平均值之差的绝对值$\left| x_{q}-\bar{x} \right| $ 

    3. 计算包括可疑值在内的标准偏差$S$ 
    \[
        G=\frac{\left| x_{q}-\bar{x} \right| }{S}
    .\] 

    4. 根据置信度$P$ 得到$\alpha$ ,查表得$G$ 的临界值$G_{\alpha,n}$,若$G>G_{\alpha,n}$ ,则数据应当舍弃
\end{notation}

\begin{notation}
    $F$ 检验

    判断两组数据间存在偶然误差是否有显著不同
    \[
        F=\frac{S_1^2}{S_2^2},\left( S_1>S_2 \right) 
    .\]

    $f_1,f_2$ 为$S_1,S_2$ 的自由度

    $P$ 一定时,查表得$F_{\alpha,f_1,f_2}$,比较$F<F_{\alpha,f_1,f_2}$ 则无显著性差异
\end{notation}

\begin{notation}
    $t$ 检验

    判断某一分析方法或操作过程中是否存在较大的系统误差

    1. 使用$\bar{x}$ 和$\mu$ 的比较:已知真值的$t$ 检验
    \[
        t=\frac{\left| \bar{x}-\mu \right| }{s}\sqrt{n} 
    .\] 
    
    2. 两组样本平均值比较:未知真值的$t$ 检验

    两组数据:
    \[
        \begin{cases}
            n_1,s_1,\bar{x_1}\\
            n_2,s_2,\bar{x_2}
        \end{cases}
    .\] 
    当$s_1\approx s_2$ ,令合并标准偏差:
    \[
        s_{R}=\sqrt{\frac{\text{偏差平方和}}{\text{总自由度}}} =\sqrt{\frac{s_1^2\left( n_1-1 \right) +s_2^2\left( n_2-1 \right) }{n_1-1+n_2-1}}  
    .\] 
    \[
        \implies t=\frac{\left| \bar{x_1}-\bar{x_2} \right| }{s_{R}}\sqrt{\frac{n_1\cdot n_2}{n_1+n_2}} 
    .\] 

    查表得$t_{\alpha,f}$ (若无真值使用总自由度$f=f_1+f_2$),比较$t,t_{\alpha,f}$
\end{notation}

\begin{notation}
    更常用的数据组比较准确度方法:ANOVA分析

    通过比较$p$ 值判断,用*号个数表示差异的大小
\end{notation}

\subsubsection{相关和回归}%
\label{subsub:相关和回归}
\[
    r = \frac{\displaystyle{\sum_{i=1}^{n}} \left( x_{i}-\bar{x} \right) \left( y_{i}-\bar{y} \right) }{\sqrt{\displaystyle{\sum_{i=1}^{n}} \left( x_{i}-\bar{x} \right) ^2\times \displaystyle{\sum_{i=1}^{n}} \left( y_{i}-\bar{y} \right)^2 } }
.\] 

作业:P25,8、9题

\section{酸碱滴定法}%
\label{sec:酸碱滴定法}
\subsection{概述}%
\label{sub:概述}
滴定分析法(titrimetry)是一种定量分析方法

\begin{notation}
    将已知准确浓度的试剂溶液低价到待测物质的溶液中,通过滴加试剂的浓度和体积,定量计算待测物质的含量

    滴定分析法又称容量分析法
\end{notation}
特点:

1. 准确度高、适于常量组分分析(组分含量$>1\%$ ,取样量$>0.1\text{g}$)

2. 易于操作

3. 快速

\begin{notation}
    基本术语:

    滴定剂:浓度准确已知的试样溶液

    滴定:将滴定剂通过滴定管逐滴滴加到被测物质溶液中的过程

    化学计量点:滴定剂与待测溶液按化学计量关系完全反应的点,用sp表示

    指示剂:发生颜色改变指示终点的物质

    滴定终点:指示剂发生颜色改变的点,用ep表示

    滴定误差:滴定终点和化学计量点的差距,用$TE$ 表示
\end{notation}
\subsection{基本原理}%
\label{sub:基本原理}
\subsubsection{酸碱定义}%
\label{subsub:酸碱定义}
1. 电离理论
\[
    \begin{cases}
        \text{酸:能电离出}\text{H}^{+} \text{的物质}\\
        \text{碱:能电离出}\text{OH} ^{-}\text{的物质}
    \end{cases}
.\] 

2. 质子理论
\[
    \begin{cases}
    \text{Br}\varnothing\text{nster酸:能给出质子的物质}\\
    \text{Br}\varnothing\text{nster碱:能接受质子的物质}
    \end{cases}
.\] 

\begin{notation}
    共轭酸碱对:酸$\leftrightarrow$ 碱(质子)

    两性物质:能给出和接受质子的物质(例:水)
\end{notation}
\subsubsection{酸碱的强度}%
\label{subsub:酸碱的强度}
1. 酸HA的解离常数: \[
    K_{a}=\frac{[\ce{H_3O+}][\ce{A-}]}{[\text{HA}]}
.\] 

2. 碱BOH的解离常数:
\[
    K_{b}=\frac{[\ce{OH-}][\ce{B+}]}{[\text{BOH}]}
.\] 
