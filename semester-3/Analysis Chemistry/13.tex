\lecture{13}{12.10}
\subsubsection*{碘量法的标准溶液}%
\label{subsub:碘量法的标准溶液}
\begin{description}
    \item[碘单质溶液] 避免接触有机物
    \item [碘离子溶液] 
    \item [硫代硫酸钠溶液]
\end{description}
\begin{notation}
标定$\ce{I_2}$ 标准溶液:\textbf{过量KI,加入少量盐酸,置于棕色试剂瓶}

加入盐酸:除去少量的$\ce{IO_3-}$ 并中和稳定剂$\ce{Na_2CO_3}$
\end{notation}
\begin{notation}
    标定$\ce{As_2O_3}$ :在碱性条件下溶解为偏亚砷酸盐,用HCl酸化后加入$\ce{NaHCO3}$ 至弱碱性
\end{notation}
\begin{notation}
    标定$\ce{Na_2S_2O_3}$ 

    五水和硫代硫酸钠的特点:杂质多、易风化、潮解、易分解

    易分解的原因:水中的$\ce{CO_2, O_2}$ 会使其分解,微生物也会分解硫代硫酸钠

    解决方法:配置前煮沸(杀菌、除溶解氧),加入$\ce{Na_2CO_3}$ 使$\text{pH}=9\sim 10$ ,\textbf{放置7至10天},待浓度稳定后标定

    配定好的标准溶液可以用于滴定:$\ce{K_2Cr_2O_7, KIO_3, KBrO_3, K_3[Fe(CN)_6]}$ 等,一般使用置换滴定法
\end{notation}
\begin{eg}
    0.4903 g $\ce{K_2Cr_2O_7}$ 标准物质,溶解定容至100 mL,取25 mL加入硫酸和KI,反应完全后用$\ce{Na_2S_2O_3}$ 滴定生成的$I_2$ ,到达终点时消耗24.95 mL,求该$\ce{Na_2S_2O_3}$ 标准溶液的浓度($MR_{\ce{K_2Cr_2O_7}}=294.18$)
\end{eg}
\begin{notation}
    反应方程式:还原部分\[
        \ce{6I^- + Cr_2O_7^{2-} + 14H^+ = 3I_2 + 2Cr^{3+} + 7H_2O}
    .\]
    滴定部分\[
        \ce{I_2 + 2S_2O_3^{2-} = 2I^- + S_4O_6^{2-}}
    .\]
\end{notation}
\begin{eg}
    设计试验滴定测定葡萄糖酸亚铁原料药中高铁盐的含量
\end{eg}
\begin{sol}
    精密称取约5.0 g葡萄糖酸亚铁原料药,置入250 ml碘瓶中

    加入100 ml水和10 ml盐酸,加热使葡萄糖酸亚铁溶解,迅速冷却

    加入3 g碘化钾,密塞碘瓶,摇匀,在暗处放置5分钟

    加入75 ml水,立即用0.1 mol/L硫代硫酸钠滴定液滴定,至近终点时加入2 ml淀粉指示液,继续滴定至蓝色消失

    将滴定结果用空白试验校正

    计算高铁盐含量
\end{sol}
\begin{notation}
    有关反应:溶解:\[
        \ce{Fe(C_6H_{11}O_7)_2 + 2HCl = FeCl_2 + 2C_6H_{11}O_7H_2} 
    .\]
    还原高铁离子:\[
        \ce{2Fe^{3+} + 2I^- = 2Fe^{2+} + I_2}
    .\]
    滴定$I_2$ :\[
        \ce{I_2 + 2S_2O_3^{2-} = 2I^- + S_4O_6^{2-}}
    .\]
\end{notation}
\begin{notation}
    亚硝基化法滴定:\[
        \ce{ArNHR +NaNO_2 \ce{<=>[][]}ArN( NO )R +NaCl +H_2O}
    .\]
\end{notation}
\begin{notation}
    溴酸钾法/溴量法

    $\circ$ 溴酸钾法:\[
        \ce{BrO_3- +6H+ +6e- \ce{<=>[][]} Br- +3H_2O}\quad \varphi^{\ominus'} _{\ce{BrO_3-/Br-}} =1.44V
    .\]
    加入的指示剂为甲基红/甲基橙,为\textbf{不可逆指示剂},可以测定还原性物质,但是2020版中国药典未收载溴酸钾法

    $\circ$ 溴量法:\[
        \ce{Br_2 +2e- \ce{<=>[][]} 2Br-}\quad \varphi^{\ominus'} _{\ce{Br_2 /Br-}}=1.065V
    .\]
    滴定使用碘的置换滴定法,常用于苯酚和苯胺类化合物的滴定
\end{notation}
\begin{notation}
重铬酸钾法:\[
    \ce{6I^- + Cr_2O_7^{2-} + 14H^+ = 3I_2 + 2Cr^{3+} + 7H_2O}\quad \varphi^\ominus=1.33V
    .\]
使用HCl介质,用于测定还原性物质(盐酸小檗碱、二价铁离子)
\end{notation}
\begin{notation}
铈量法:\[
    \ce{Ce^4+ +e- \ce{<=>[][]} Ce^3+}\quad \varphi^{\ominus'} _{\ce{Ce^4+ /Ce^3+}} =1.45V
.\]
铈量法中硫酸铈溶液配置简单、选择性高、反应简单,常用于\textbf{制剂中}亚铁含量的测定
\end{notation}
\begin{notation}
    高碘酸钾法:\[
        \ce{IO_6^5- +H+ +2e- \ce{<=>[][]} IO_3- +3H_2O}\quad \varphi^{\ominus'} _{\ce{H_5IO_6 /IO_3-}} = 1.60V
    .\]
    测定对象:$\alpha$-羟基醇/氨基醇/羰基醇等醇类有机物
\end{notation}
作业:习题1,4,7(第九版)
