\lecture{9}{11.12}
\textit{Review:}

反应\[
    \ce{M+Y}\ce{<=>[][]}\ce{MY}
.\]
有大量副反应存在,如$\ce{M +L}\ce{<=>[][]}\ce{ML},\ce{H+ +Y}\ce{<=>[][]}\ce{HY},\ce{MY +H+}\ce{<=>[][]}\ce{MHY+}$

\begin{notation}
    条件稳定常数:\[
        K_{\text{MY}}'=\frac{[\text{MY}']}{[\text{M}'][\text{Y}']}
    .\]
    对比:稳定常数:\[
        K_{\text{MY}}=\frac{[\text{MY}]}{[\text{M}][\text{Y}]}
    .\]
    由$[\text{M}]=\alpha_\text{M}\cdot [\text{M}']$ 等:可得条件稳定常数和稳定常数的关系:$\lg K_\text{MY}'=\lg K_\text{MY}-\lg\alpha_\text{M}-\lg\alpha_\text{Y}+\lg\alpha_\text{MY}$;条件稳定常数在所有条件确定后是一个常数
\end{notation}
在没有副反应时稳定常数为1:$\frac{\alpha_\text{MY}}{\alpha_\text{M}\alpha_\text{Y}}=1,K_\text{MY}'=K_\text{MY}$,但只要有副反应条件稳定常数就小于1:\[
    K_\text{MY}'=K_\text{MY}\cdot \frac{\alpha_\text{MY}}{\alpha_\text{M}\alpha_\text{Y}}
.\]
\begin{notation}
    条件稳定常数$K_{\text{MY}}'$ 越大,所形成的配合物越稳定
\end{notation}
\begin{eg}
    求$\text{pH}=2,\text{pH}=5$ 时ZnY的条件稳定常数
\end{eg}
解:对比公式$\lg K_\text{ZnY}'=\lg K_\text{ZnY}-\lg\alpha_\text{Zn}-\lg\alpha_\text{Y}+\lg\alpha_\text{ZnY}$可得:不同pH下的$\alpha_\text{Y(H)},\alpha_\text{Zn(OH)}$和$K_{\text{ZnY}}$ 可查表,即原式变为$\lg K_\text{ZnY}'=\lg K_{\text{ZnY}}-\lg\alpha_\text{Y(H)}$,即可以计算出$K_\text{ZnY}'$
\subsection{配位滴定曲线}%
\label{sub:配位滴定曲线}
\begin{notation}
    滴定曲线的横坐标恒为滴定剂的体积$V_\text{Y}$
\end{notation}
建立对于M和Y的物料平衡:\[
    \begin{cases}
        [\text{M}']+[\text{MY}']=\frac{V_\text{M}}{V_\text{M}+V_\text{Y}}\cdot c_\text{M}\\
        [\text{Y}']+[\text{MY}']=\frac{V_\text{Y}}{V_\text{M}+V_\text{Y}}\cdot c_\text{Y}
    \end{cases}
.\]
结合条件稳定常数$K_\text{MY}'=\frac{[\text{MY}']}{[\text{M}'][\text{Y}']}$ ,解该三元一次方程组即可算出不同滴定剂下的结合情况:\[
    K_\text{MY}'[\text{M}']^2 +\left( \frac{V_\text{Y}c_\text{Y}-V_\text{M}c_\text{M}}{V_\text{M}+V_\text{Y}}\cdot K_\text{MY}+1 \right)\cdot [\text{M}']-\frac{V_\text{M}}{V_\text{M}+V_\text{Y}}\cdot c_\text{M}=0
.\]
\begin{itemize}
    \item 影响滴定突越上限:$\text{pM}'=\lg K_\text{MY}'-3$
    \item 影响滴定突越下限:$\text{pM}'=\text{p}C_\text{M}^{\text{sp}}+3$
\end{itemize}
即:
\begin{itemize}
    \item 金属离子浓度越大,滴定突越前侧的滴定突越范围越大
    \item $K_\text{MY}'$ 越大,滴定突越后侧的突越范围越大
\end{itemize}
能影响$K_{\text{MY}}'$ 的因素:\textbf{酸效应、配位效应}等

近似计量点时的情况:$\text{pM}'_\text{sp}=\frac{1}{2}\left( \lg K_\text{MY}'+\text{p}C_\text{M}^{\text{sp}} \right)$,即两个影响因素的平均值
\begin{eg}
    pH=10的氨性缓冲溶液$[\ce{NH_3}]=0.2$ 中用0.02mol/L的EDTA滴定0.02mol/L的$\ce{Cu2+}$,计算$\text{p}\ce{Cu}'_{\text{sp}}$
\end{eg}
\subsection{金属指示剂}%
\label{sub:金属指示剂}
\begin{notation}
    通过指示金属离子的含量确定终点
\end{notation}
$\circ$ 铬黑T(EBT):本身为蓝色,Mg和EBT可以配对而变为红色
\begin{notation}
    指示原理:

    加入滴定剂前:金属离子和指示剂结合:$\ce{M +In}\ce{<=>[][]}\ce{MIn}$ 

    滴定开始时:金属离子和滴定剂反应:$\ce{M +Y}\ce{<=>[][]}\ce{MY}$ 

    滴定结束时:指示剂配合物消耗完后显示指示剂本身的颜色:$\ce{MIn +Y}\ce{<=>[][]}\ce{MY +In}$
\end{notation}
\begin{notation}
    指示剂本身为弱酸,因此需要控制溶液pH;EDTA与有色金属离子结合生成颜色更深的配合物
\end{notation}
\subsubsection*{指示剂应具备的条件}%
\label{subsub:指示剂应具备的条件}
\begin{eg}
    以EBT为例:
    \[
        \underset{\text{紫红}}{\ce{H_2In-}}\ce{<=>[][]} \underset{\text{蓝色}}{\ce{HIn^2-}}\ce{<=>[][]} \underset{\text{橙色}}{\ce{In^3-}}
    .\]
\end{eg}
$\circ$ 显色反应灵敏、快速、具有良好的变色可逆性

$\circ$ MIn稳定性要适当:

比滴定生成的配合物更不稳定,但需要在溶液中稳定存在:$K_\text{MIn}'>10^{-4}$

在滴定未到达终点时不被滴定剂置换,但到达终点时可以被置换:$\frac{K_\text{MY}'}{K_\text{MIn}'}>10^2 $
\begin{notation}
    指示剂的封闭现象:指示剂和金属离子结合太稳定,滴定剂无法将金属离子置换出来,无法观察到颜色的变化;在不更换指示剂的前提下,首先确定是由待测离子还是干扰离子引起的封闭
\end{notation}
\begin{eg}
    干扰离子$\ce{Fe^3+}$和 $\ce{Al^3+}$ 可以封闭EBT,解决方法是:加入三乙醇胺(掩蔽剂)结合这两个离子(更稳定的配合物),使这两个离子无法与EBT结合
\end{eg}
\begin{eg}
    如果要测$\ce{Al^3+}$ :使用返滴定

    先加入定量过量的EDTA与Al离子反应,反应完全后加EBT:此时EBT不会和Al反应$^\star $ ,此时再用Zn离子滴定EBT
\end{eg}
常用掩蔽剂表格:P82
\begin{notation}
    指示剂的僵化现象:有些指示剂和金属离子结合后的配合物不溶于水,此时EDTA和该配合物反应速率非常慢,使终点延长;可以通过加热或加入其他溶剂提高溶解度解决
\end{notation}
\begin{eg}
    PAN(1-(2-吡啶偶氮)-2-萘酚)与金属离子形成的配合物难溶,通过加热或加入EtOH将其溶解
\end{eg}
常用金属指示剂:EBT、二甲酚橙(XO),酸性铬蓝K,磺基水杨酸(Ssal)、PAN等
\subsection{配位滴定误差}%
\label{sub:配位滴定误差}
\[
    TE\%=\frac{[\text{Y}']_\text{sp}-[\text{M}']_\text{sp}}{C_\text{M}^{\text{sp}}}\times 100\%=\frac{10^{\Delta \text{pM}'}-10^{-\Delta \text{pM}'}}{\sqrt{K_\text{MY}'C_\text{M}^{\text{sp}}}}
.\]
\subsection{配位滴定条件选择}%
\label{sub:配位滴定条件选择}
$\circ$ 酸度(单一离子测定的最高酸度和最低酸度):使$\lg K_\text{MY}'\ge 8$

当$\lg K_\text{MY}'=\lg K_\text{MY}-\lg\alpha_\text{Y(H)}=8$ 时的pH为单一金属滴定的最高酸度

通过金属离子和氢氧根形成的沉淀的$K_\text{sp}$ 可以求得氢氧根浓度,为防止形成沉淀:$[\ce{OH-}]\le \sqrt[a]{K_\text{sp}/c_\text{M}}$

$\circ$ 最佳酸度:$\text{pM}'_\text{指示剂变色}=\text{pM}'_\text{sp}$
\begin{notation}
    可以通过缓冲溶液来保持滴定过程中的酸度基本不变
\end{notation}
