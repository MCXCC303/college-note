\lecture{10}{11.19}
\textit{Review:}
$\circ$ 准确测定单一金属离子:确定最高酸度等

\subsubsection*{多种金属离子共同滴定}%
\label{subsub:多种金属离子共同滴定}
共存离子M,N:能选择性滴定出M离子的条件:$\frac{c_\text{M}K_\text{MY}}{c_\text{N}K_\text{NY} }\ge 10^5$

当满足条件时可以使用控制酸度的方法分布滴定,若不满足条件使用\textbf{掩蔽法}
\begin{notation}
    掩蔽法:当$c_\text{N}$或$K_\text{NY}$ 较大:用掩蔽剂降低$\lg c_\text{N}K_\text{NY}$ (降低了$c_\text{N}$)
\end{notation}
1. 常见掩蔽剂:三乙醇胺(TEA):可以与$\ce{Fe^3+}$ 和$\ce{Al^3+}$ 形成更稳定的配合物,而不影响$\ce{Ca^2+}$ 和$\ce{Mg^2+}$

2. 通过沉淀也可以掩蔽:测定$\ce{Ca^2+}$ 和$\ce{Mg^2+}$ 中钙离子的浓度:两种离子$\lg K$ 相似,无法通过TEA掩蔽,但$K_\text{sp}$ 相差较大,可以将镁离子沉淀后滴定钙离子

3. 通过氧化还原反应掩蔽:使杂质离子转变价态
\begin{eg}
    使用EDTA测定铋离子和铁(II)离子的浓度时,加入还原剂使铁(III)离子转为铁(II)离子,转换后$\lg K$ 降低
\end{eg}
\subsection{配位滴定法的应用}%
\label{sub:配位滴定法的应用}
常用标准溶液:
\begin{description}
    \item [EDTA(0.05 mol/L)] 使用间接法配置,装在硬质玻璃瓶或塑料瓶中,使用ZnO滴定,使用EBT/XO指示
    \item [$\ce{ZnSO_4}$ ] 仅了解
\end{description}
\subsubsection*{配位滴定法滴定方式}%
\label{subsub:配位滴定法滴定方式}
\begin{itemize}
    \item 直接滴定
    \item 间接滴定
    \item 置换滴定
    \item 返滴定
\end{itemize}
\begin{notation}
直接滴定:条件和酸碱滴定法相似
\end{notation}
\begin{notation}
    返滴定:适用于金属离子和EDTA反应慢和指示剂的封闭现象
\end{notation}
\begin{notation}
间接滴定:不和EDTA反应的物质(磷酸根离子:生成磷酸铋沉淀后测定铋离子)
\end{notation}
\begin{notation}
置换滴定:形成的配合物不稳定,可以置换出金属离子和配体离子,使置换出的金属离子和EDTA稳定;也可以置换出EDTA,再用更强的金属离子滴定EDTA
\end{notation}
\begin{eg}
设计方案滴定药用氢氧化铝中氢氧化铝中含量
\end{eg}
在药典中的记载:使用盐酸溶解,加入\textbf{氨水}中和至恰好析出沉淀,在滴加盐酸至刚好溶解;加入醋酸-醋酸铵缓冲溶液(pH=6),使用乙二胺四醋酸二钠滴定,煮沸完全反应,最后使用锌离子标准溶液滴定
\section{氧化还原滴定法}%
\label{sec:氧化还原滴定法}
\begin{itemize}
    \item 概述
    \item 氧化还原滴定法的基本原理
    \item 常用的氧化还原滴定法
\end{itemize}
重点内容:
\begin{itemize}
    \item 条件电位
    \item 条件平衡常数
    \item 指示剂的选择
    \item 碘量法
    \item 高锰酸钾法
    \item $\ldots $
\end{itemize}
\begin{notation}
氧化还原滴定法和配位滴定法是药学中应用最多的滴定方法
\end{notation}
最常用的滴定方法:碘量法(使用硫代硫酸钠滴定)

目标滴定物质本身具有氧化还原性时可以使用氧化还原滴定法
\begin{defi}
氧化还原滴定法:以氧化还原反应为基础的滴定分析方法
\end{defi}
氧化还原滴定的本质为电子的转移:还原剂$\to $ 氧化剂

特点:
\begin{itemize}
    \item 机理复杂,常常分布进行
    \item 反应慢
    \item 常伴有副反应(无明确的计量关系)
    \item 可以滴定无机物和有机物
\end{itemize}
分类:
\begin{enumerate}
    \item 碘量法
    \item 高锰酸钾法
    \item 重铬酸钾法
    \item 亚硝酸钠法
\end{enumerate}
\subsection{配位滴定法基本原理}%
\label{sub:配位滴定法基本原理}
条件电位:
\begin{defi}
Ox/Red称为氧化还原电对,简称电对,其中Ox为氧化态,Red为还原态
\end{defi}
\begin{notation}
    电对可以分为可逆电对和不可逆电对:
    \begin{enumerate}
        \item 可逆电对:电势使用能斯特方程计算
        \item 不可逆:无平衡,不能使用能斯特方程计算
    \end{enumerate}
\end{notation}
\begin{eg}
     \[
         \ce{Ce^4+ +Fe^2+}\ce{<=>[][]}\ce{Ce^3+ +Fe^3+}
     .\]
     可以表示为:
      \[
         \text{Ox}_1+\text{Red}_2 \ce{<=>[][]} \text{Red}_1+\text{Ox}_2
     .\]
\end{eg}
电对的电极电位(potential):
\begin{align*}
    \begin{cases}
        \varphi_\text{Ox/Red}&= \varphi_\text{Ox/Red}^\theta+\frac{2.303RT}{nF}\lg \frac{\alpha_\text{Ox}}{\alpha_\text{Red}} \\
        \varphi_\text{Ox/Red}&= \varphi_\text{Ox/Red}^\theta+\frac{0.059}{n}\lg \frac{\alpha_\text{Ox}}{\alpha_\text{Red}}
    \end{cases}
.\end{align*}
\begin{notation}
    高电位的氧化态和低电位的还原态反应,电位差越大反应越完全
\end{notation}
条件电位:\[
    \varphi_\text{Ox/Red}^{\ominus'}=\varphi_\text{Ox/Red}^\theta+\frac{2.303RT}{nF}\lg \frac{\gamma_\text{Ox}\alpha_{Red}}{\gamma_\text{Red}\alpha_\text{Ox}}
.\]
其中:\[
    \alpha_\text{Ox}=\gamma_\text{Ox}[\text{Ox}]=\frac{\gamma_\text{Ox}c_\text{Ox}}{\alpha_\text{Ox}}
.\]
$\alpha_\text{Red}$ 同理,条件电位查表可得且条件电位不能通过计算获得
\begin{notation}
条件电位的影响因素:
\begin{enumerate}
    \item 盐效应
    \item 生成沉淀
    \item 生成配合物
    \item 酸效应
\end{enumerate}
当所有条件一定的时候,条件电位是一个常数
\end{notation}
$\circ$ 盐效应:

离子强度的改变会改变活度系数$\gamma$ ,由$\varphi'=\varphi^{\ominus}+\frac{0.059}{n}\lg \frac{\gamma_\text{Ox}}{\gamma_\text{Red}}$ 改变$\varphi'$

盐效应可以忽略

$\circ$ 生成沉淀、配合物
