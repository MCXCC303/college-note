\section{概论}%
\label{sec:概论}
20世纪20-30年代:分析化学出现四大反应平衡理论的建立

20世纪40-50年代:光电色谱仪器设备出现
\begin{notation}
    Bloch F and Purcell E M 建立核磁共振测定

    Martin A J P and Synge R L M 建立气相色谱

    Heyrovsky J 建立极谱分析法
\end{notation}
20世纪70年代以来:计算机参与自动化
\begin{notation}
    分析化学分析方法:3S+2A

    3S: Sensitivity, Selectivity, Speediness
    
    2A: Accuracy, Automatics
\end{notation}
分析化学主要发展趋势:
\[
    \begin{cases}
        \mbox{在线分析}\\ 
        \mbox{原位分析}\\ 
        \mbox{实时分析}\\ 
        \mbox{活体分析}\\ 
        \ldots
    \end{cases}
.\] 
\begin{notation}
    分析过程的步骤:

    1. 分析方法选择

    2. 取样(Sampling,具代表性的样本)

    3. 制备试样(适合与选定的分析方法,消除可能的干扰)

    4. 分析测定(优化条件,仪器校正,方法验证)

    5. 结果处理和表达(统计学分析,测量结果的可靠性分析,书面报告)
\end{notation}
\begin{notation}
    制备试样首先需要进行样品前处理

    方法验证:线性性,灵敏性等
\end{notation}
\section{误差和分析数据处理}%
\label{sec:误差和分析数据处理}
\begin{notation}
    “挑数据”:做标准曲线

    标品浓度(0,1,2,3,4,5,6,7)

    吸光度(0,0.1,0.2,0.3,0.4,0.001,0.6,0.7)

    对于可疑数据,需要通过其他方法进行确认(至少3次试验)
\end{notation}
\subsection{准确度和精密度}%
\label{sub:准确度和精密度}
\subsubsection{准确度和误差}%
\label{subsub:准确度和误差}
\begin{defi}
    准确度(Accuracy):测量值和真值的接近程度,准确度的高低用误差大小衡量
\end{defi}
\begin{defi}
    误差(error):测量结果和真值的差值
\end{defi}
误差具有客观性和普遍性

实验结果都有误差,测量值只能尽可能接近真实值
\begin{defi}
    约定真值:由国际计量大会定义的单位及我国的法定计量单位
    \begin{eg}
        国际单位制基本单位“米”、“克”等
    \end{eg}
    \begin{notation}
        约定真值是有一个量的真值的近似值,误差可以忽略不计
    \end{notation}
\end{defi}

\begin{defi}
    标准值:采用可靠的分析方法、在不同实验室、由不同的分析人员、对同一个试样反复多次测定后将大量数据用数理统计求得的测量值
\end{defi}
误差的表示方法:\[
    \delta=x-\mu
.\] 
\begin{defi}
    绝对误差:$\delta$

    测量值: $x$

    真值: $\mu$ 

    相对误差:$RE\%$
\end{defi}
绝对误差(Absolute Error)可正可负,单位为测量值的单位

绝对误差的绝对值越小,准确度越高

相对误差(Relative Error, RE):\[
    RE\% = \frac{\delta}{\mu}\times 100\%
.\] 

或:\[
    RE\% = \frac{\delta}{x}\times 100\%
.\]

相对误差无单位,可正可负
\begin{eg}
    有真实值为0.0020g 和0.5000g 的两个样品,称量结果分别为0.0021g 和0.5001g,计算相对误差和绝对误差

$\delta_1=0.0021-0.0020=0.0001$,$\delta_2=0.5001-0.5000=0.0001=\delta_1$

\[
    RE_1\%=\frac{0.0001}{0.0020}\times 100\%=5\%
\]
\[
    RE_2\%=\frac{0.0001}{0.5000}\times 100\%=0.02\%
.\]  
\end{eg}
\begin{notation}
    RE要求:测高含量组分,RE可小;测低含量组分,RE可大

    高含量组分对应化学分析法;低含量组分对应仪器分析法
\end{notation}
\subsubsection{精密度和偏差}%
\label{subsub:精密度和偏差}
\begin{defi}
    精密度:在规定的测定条件下,多次平行测定结果相互吻合的程度,精密度高低用偏差衡量

    偏差:单个测量值和测量平均值的差距
\end{defi}
\begin{defi}
    绝对偏差:d

    相对偏差:d\%

    平均偏差:$\bar{d}$

    标准偏差:SD

    相对标准偏差:RSD
\end{defi}

绝对偏差:\[
    d=x_i-\bar{x}
.\]

相对偏差:\[
    d\%=\frac{d}{\bar{x}}
.\]

平均偏差:\[
    \bar{d}=\frac{\sum |x_{i}-\bar{x}|}{n}
.\]

相对平均偏差:\[
    \frac{\bar{d}}{\bar{x}}\times 100\%=\frac{\sum |x_{i}-\bar{x}|}{n\cdot \bar{x}}\times 100\%
.\]

标准偏差:\[
    S_x=\sqrt{\frac{\displaystyle{\sum_{i=1}^{n} \left( x_{i}-\bar{x} \right) ^2}}{n-1}} \left(n\le 20  \right) 
.\] 

$n-1 $称为自由度

相对标准偏差(RSD,变异系数):\[
    \text{RSD}=\frac{S_x}{\bar{x}}\times 100\%
.\] 

RSD越小,数据越集中,精密度越高

\begin{notation}
    方法的精密度考察:

    1. 重复性(repeatability):同一实验室,较短时间间隔,同一分析人员对同一试样测定所得结果的接近程度

    2. 中间精密度(intermediate precision):改变某些试验条件对同一试样测定结果的接近程度

    3. 重现性(reproducibility):不同实验室,不同人员对同一试样测试结果的接近程度
\end{notation}

\begin{notation}
    精密度和准确度的关系:

    1. 精密度高是准确度高的前提

    2. 精密度高,准确度不一定高

    3. 只有精密度和准确度都高的数据才可取
\end{notation}
\begin{notation}
    准确度体现结果的正确性,精密度体现结果的重现性
\end{notation}

\subsubsection{误差}%
\label{subsub:误差}
\begin{notation}
    误差分类:系统误差、偶然误差(随机误差)
\end{notation}
\begin{defi}
    系统误差(可定误差):分析中由某些确定原因造成的误差

    特点:

    1. 重现性

    2. 单向性

    3. 可以校正消除

    4. 影响准确度
\end{defi}
\begin{notation}
    系统误差分类:
    \[
        \begin{cases}
            \mbox{方法误差:方法不完善}\\ 
            \mbox{仪器误差:仪器本身缺陷}\\ 
            \mbox{试剂误差:试剂有杂质}\\ 
            \mbox{操作误差、主观误差}
        \end{cases}
    .\] 
\end{notation}
\begin{defi}
    偶然误差(随机误差):由偶然因素影响

    特点:

    1. 无单向性(方向大小不确定)

    2. 符合统计学规律(大误差出现的概率小,随机误差的正态分布)

    3. 可以通过增加平行测定次数减小

    4. 影响精密度
\end{defi}
\begin{defi}
    过失:由分析工作者粗心大意造成,可以避免
\end{defi}
