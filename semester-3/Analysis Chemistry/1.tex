\lecture{1}{}
\section{概论}%
\label{sec:概论}
20世纪20-30年代:分析化学出现四大反应平衡理论的建立

20世纪40-50年代:光电色谱仪器设备出现
\begin{notation}
    Bloch F and Purcell E M 建立核磁共振测定

    Martin A J P and Synge R L M 建立气相色谱

    Heyrovsky J 建立极谱分析法
\end{notation}
20世纪70年代以来:计算机参与自动化
\begin{notation}
    分析化学分析方法:3S+2A

    3S: Sensitivity, Selectivity, Speediness
    
    2A: Accuracy, Automatics
\end{notation}
分析化学主要发展趋势:
\[
    \begin{cases}
        \text{在线分析}\\ 
        \text{原位分析}\\ 
        \text{实时分析}\\ 
        \text{活体分析}\\ 
        \ldots
    \end{cases}
.\] 
\begin{notation}
    分析过程的步骤:

    1. 分析方法选择

    2. 取样(Sampling,具代表性的样本)

    3. 制备试样(适合与选定的分析方法,消除可能的干扰)

    4. 分析测定(优化条件,仪器校正,方法验证)

    5. 结果处理和表达(统计学分析,测量结果的可靠性分析,书面报告)
\end{notation}
\begin{notation}
    制备试样首先需要进行样品前处理

    方法验证:线性性,灵敏性等
\end{notation}
