\lecture{8}{11.05}
\begin{notation}
    药物本身含有金属离子:通过对金属离子定量分析控制药品质量
\end{notation}
\begin{eg}
    复方葡萄糖酸钙口服溶液:
    \begin{description}
        \item[葡萄糖酸钙] 50g
        \item[乳酸钙] 50g
        \item[辅料] 适量
        \item [水] 适量
    \end{description}
\end{eg}
\begin{eg}
    重质碳酸镁:治疗胃溃疡,含杂质氧化钙

    药典规定该药品氧化钙杂质不得超过0.60\%
\end{eg}
\section{配位滴定法}%
\label{sec:配位滴定法}
\begin{itemize}
    \item 概述
    \item 基本原理
    \item 条件选择
    \item 应用
\end{itemize}
\subsection{概述}%
\label{sub:概述:配位滴定法}
\begin{defi}
    配位滴定法又称络合滴定法,以\textbf{配位反应} 为基础的滴定方法
\end{defi} 
配合物的生成:中心元素(阳离子、原子,提供空轨道)+配体(阴离子、分子,提供电子对)
\begin{notation}
    \textit{Review:直接滴定的四大要求}
    \begin{itemize}
        \item 明确的剂量比
        \item 反应完全
        \item 反应迅速
        \item 明确的指示终点的方式
    \end{itemize}
    $\Rightarrow $配位反应对反应的要求:
    \begin{itemize}
        \item 配位比恒定
        \item 生成的配合物足够稳定(可逆反应)
        \item 反应迅速
        \item 合适的方法判断终点
    \end{itemize}
\end{notation}
配位剂种类:

$\circ$ 无机配位剂:逐级配位,速率较慢
\begin{eg}
    $\ce{SCN-,CN-,CO}$ 等
\end{eg}

$\circ$ 有机配位剂:氨羧类配位剂
\begin{eg}
    EDTA:乙二胺四乙酸,可用于数十种金属离子的滴定

    EDTA可以提供大量孤对电子:-NHR,-COOR

    EDTA为六齿配体,且由于其中的氨基可以结合质子,因此认为EDTA为六元酸
\end{eg}
\begin{notation}
    EDTA难溶于水(0.2g/L)但EDTA可溶于碱,一般使用EDTA的二钠盐(111g/L)

    EDTA在水溶液中存在七种型体:$\chemfig{H_iY^{i-4}} ,i\in [0,6]$,其中$\chemfig{Y^{-4}} $ 为最佳型体
\end{notation}
EDTA的特点:
\begin{itemize}
    \item 广泛
    \item 稳定
    \item 配位比简单(1:1)
    \item 配位反应速度快
    \item 反应完全,水溶性好
    \item 与无色金属离子生成无色配合物,与有色金属离子生成颜色更深的配合物
\end{itemize}
其他氨羧类配位剂:
\begin{enumerate}
    \item EGTA
    \item NTA(氨三乙酸)
    \item EDTMP(乙二胺四甲叉膦酸)
    \item EDTP(乙二胺四丙酸)
\end{enumerate}
\subsection{配位滴定法的基本原理}%
\label{sub:配位滴定法的基本原理}
\subsubsection*{配位平衡}%
\label{subsub:配位平衡}
\begin{notation}
    配合物的稳定常数:

    反应\[
        \chemfig{M+Y}\ce{<=>[][]} \chemfig{MY}  
    .\] 
    对应的稳定常数:
    \[
        K_\text{MY} =\frac{[\chemfig{MY} ]}{[\text{M}][\text{Y}] } 
    .\] 
\end{notation}
\begin{notation}
    逐级配位常数:
    \begin{align*}
        \chemfig{M+L=ML} &\Rightarrow K_1=\frac{[\text{ML}]}{[ \text{M}][\text{L}]} \\
        \chemfig{ML+L=ML_2} &\Rightarrow K_2=\frac{[\chemfig{ML_2} ]}{[\text{ML}][\text{L}]} 
    .\end{align*}
    累计稳定常数:
     \[
         \beta_n=\prod_{i=1}^{n} K_{i} =\frac{[\chemfig{ML_{n}} ]}{[\text{M}][\text{L}]^{n}} 
    .\] 
\end{notation}
\begin{notation}
    配位反应的副反应系数:
    \begin{enumerate}
        \item 辅助配位效应:与其他单齿配体配位

        \item 羟基配位效应:与羟基配位

        \item 酸效应:有氢离子存在时配体的型体发生变化

        \item 共存离子效应:配体与其他中心离子配位

        \item 混合配位效应:中心离子同时和目标配体和其他配体配位(利于主反应进行)
    \end{enumerate}
    副反应系数$\alpha$ :
    \[
        \alpha=\frac{[\text{X'}]}{[\text{X}]} 
    .\] 
    [\text{X}]代表总平衡浓度,[\text{X'}]代表参与反应的平衡浓度
\end{notation}
\begin{notation}
    配位剂的副反应系数:
    \[
        \begin{cases}
            c_\text{Y} \begin{cases}
                [\text{Y'}]\begin{cases}
                    [\text{Y}]\\
                    [\ce{HY+H_2Y+\ldots +H_{n}Y+NY} ]
                \end{cases}\\
                [\text{MY}]
            \end{cases}
        \end{cases}
    .\] 
    副反应系数
    \[
        \alpha_\text{Y} =\frac{[\text{Y'}]}{[\text{Y}]} 
    .\] 
    副反应产物HY,NY等和游离态Y为[\text{Y'}],只有Y为[\text{Y}],所有包含Y的部分为$c_\text{Y} $

    可再分为$\alpha_\text{Y(H)} $ 和$\alpha_\text{Y(N)} $
\end{notation}
酸效应系数$\alpha_\text{Y(H)} $ :
\begin{align*}
    \alpha_\text{Y(H)} &= \frac{[\text{Y'}]}{[\text{Y}]}  \\
                       &= 1+\frac{[\text{H}^+]}{K_{a_6}} +\ldots .+\frac{[\text{H}^+]^6}{K_{a_6}K_{a_5}K_{a_4}\ldots K_{a_1} }
.\end{align*}
当pH>12时可以忽略酸效应

共存离子效应系数:
\begin{align*}
    \alpha_\text{Y(N)} &= \frac{[\text{Y'}]}{[\text{Y}]} =\frac{[\text{Y}]+[\text{NY}]}{[\text{Y}]}  \\
                       &= 1+K_\text{NY} [\text{N}]
.\end{align*}

\subsubsection*{金属离子的副反应系数}%
\label{subsub:金属离子的副反应系数}
\begin{defi}
    $\alpha_\text{M} $ :未与EDTA配位的金属离子以各种形式存在的浓度与总浓度比
\end{defi}
\begin{align*}
    \alpha_\text{M(L)} &= 1+\frac{[\text{ML}]}{[\text{M}]} +\ldots +\frac{[\text{ML}]_{n}}{[\text{M}]}  \\
                       &= 1+\sum_{i=1}^{n} \beta_{i}[\text{L}]^{i}
.\end{align*}
同理可得$\alpha_\text{M(OH)} =1+\sum_{i=1}^{n} \beta_{i}[\text{OH}]^{i}$

$\lg\alpha$ 均可以查表得知

若溶液中配位剂有$P$种:
\begin{align*}
    \alpha_\text{M} =\frac{[\text{M}^+]}{[\text{M}]} =\sum_{i=1}^{n} \alpha_{\text{M(L)}_i} +\left( 1-P \right) 
.\end{align*}
\subsubsection*{配合物的副反应系数}%
\label{subsub:配合物的副反应系数}
配合物的副反应可以推动主反应的进行
\[
    \alpha_\text{MY} =\frac{[\text{MY'}]}{[\text{MY}]} \approx 1
.\] 
即一般情况下生成的副产物稳定性非常低,该反应几乎不发生
\begin{notation}
    在酸性/碱性较强的条件下不能忽略:
    \begin{align*}
        \alpha_\text{MY(H)} &=1+K_\text{MHY} \times [\text{H}^+]\\
        \alpha_\text{MY(OH)} &=1+K_\text{M(OH)Y} \times [\text{OH}^-]
    .\end{align*}
\end{notation}
重点题目:例5-3
