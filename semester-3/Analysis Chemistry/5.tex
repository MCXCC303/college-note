\lecture{5}{10.15}
\subsubsection{分布系数}%
\label{subsub:分布系数}
\begin{defi}
    分布系数:$\delta$

    用$\delta_0,\delta_1,\delta_2\ldots$ 来表示电荷数为$0,1,2\ldots$ 的组分的分布系数
\end{defi}
一元弱酸的浓度(分析浓度)为$c$ ,解离平衡后:
\begin{align*}
    \delta_{\text{HA}}&=\frac{[\text{HA}]}{c}=\frac{[\text{HA}]}{[\text{HA}]+[\ce{A-}]}\\
                      &= \frac{1}{1+K_a/[\ce{H+}]} =\frac{[\ce{H+}]}{[\ce{H+}]+K_a} \\
    \delta_{\ce{A-}}&= \frac{K_a}{\ce{[H+]}+K_a}
.\end{align*}

易得$\delta_{\text{HA}}+\delta_{\ce{A-}}=1$
\begin{eg}
    计算pH=5是0.10mol/L HAc溶液中各组分(HAc,$\ce{Ac-},\ce{H+}$)的分布系数$\delta$和平衡浓度$c$
\end{eg}
解:$\ce{[H+]}=10^{-5}\text{mol/L}$,$K_a=1.7 \times 10^{-5}$

计算$\delta$ :\[
    \delta_0=\delta_{\text{HAc}}=\frac{[\ce{H+}]}{[\ce{H+}]+K_a}=0.37
.\] 
\[
    \delta_1=\delta_{\ce{Ac-}}=1-\delta_{\text{HAc}}=1-0.37=0.63
.\] 

\[
    \ldots
.\] 
\begin{notation}
    一元强酸$\ce{H+}$ 的浓度(精确式)
    \[
        [\ce{H+}]=\frac{c_a+\sqrt{c_a^2+4K_w} }{2}
    .\] 

    忽略水的解离($c_a\ge 20[\ce{OH-}]$),可以近似为下式(最简式):
    \[
        \begin{cases}
            [\ce{H+}]=[\ce{A-}]=c_a\\
            \ce{pH}=-\lg [\ce{H+}]=-\lg c_a
        \end{cases}
    .\] 
\end{notation}
\begin{notation}
    一元弱酸的pH计算:

    精确式:\[
        [\ce{H+}]^2 = K_a\left( c_a-[\ce{H+}] \right) 
    .\] 
    \[
        [\ce{H+}]=\frac{-K_a+\sqrt{K_a^2+4c_a\cdot K_a} }{2}
    .\]

    近似式($c_aK_a<20K_w, c_a/K_w\ge 500$)
    \[
        [\ce{H+}]=\sqrt{c_aK_a+K_w} 
    .\]

    最简式($c_aK_a\ge 20K_w, c_a/K_w\ge 500$)
    \[
        [\ce{H+}]=\sqrt{K_a\cdot c_a} 
    .\] 
\end{notation}
\begin{notation}
    两性物质pH计算

    精确式:
    \[
        [\ce{H+}]=\sqrt{\frac{K_{a_1}\left( K_{a_2}[\ce{HB}]+K_w \right) }{K_{a_1}+[\ce{HB}]}} 
    .\] 
    
    近似式:
    \[
        [\ce{H+}]=\sqrt{\frac{K_{a_1}\left( K_{a_2}c+K_w \right) }{K_{a_1}+c}} 
    .\] 

    最简式:
    \[
        [\ce{H+}]=\sqrt{K_{a_1}K_{a_2}} 
    .\] 
    \[
        \ce{pH}=\frac{1}{2}\left( \text{p}K_{a_1}+\text{p}K_{a_2} \right) 
    .\] 
\end{notation}
\subsubsection{缓冲溶液的pH}%
\label{subsub:缓冲溶液的pH}
\begin{notation}
    缓冲溶液的种类:

    1. 共轭酸碱对

    2. 两性物质,如$\ce{H_2PO_4}\sim \ce{HPO_4^{2+}}$(PBS,磷酸缓冲盐)

    3. 高浓度的强酸/强碱
\end{notation}
\begin{notation}
    缓冲溶液有缓冲能力/缓冲容量$\beta$,与缓冲溶液的总浓度与组分有关

    1. 总浓度$\propto \beta$
    
    2. 缓冲组分的浓度比 $\to 1\propto \beta$ 

    3. 
\end{notation}
缓冲溶液的Henderson缓冲公式:
 \[
     [\ce{H+}]=K_a \frac{c_a}{c_b}
.\] 
\[
    \ce{pH}=\text{p}K_a+\lg \frac{c_b}{c_a}
.\] 
\begin{notation}
    生物上常用的缓冲溶液:

    1. 三(羟甲基)氨基甲胺+HCl(tris-HCL)
    
    2. HEPES
\end{notation}

