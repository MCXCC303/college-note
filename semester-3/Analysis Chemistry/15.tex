\lecture{15}{12.24}
\textit{Review:}

\textbf{沉淀反应}:定量分析

\textbf{溶解平衡:}与溶解度$S$ ,活度积$K_\text{ap}$ ,溶度积$K_\text{sp}$ ,条件溶度积$K_\text{sp}'$ 有关;影响因素有\textbf{同离子效应、酸效应、配位效应、盐效应},其中只有同离子效应降低溶解度

\textbf{银量法:}滴定曲线相交并对称,$c_x$ 和$K_\text{sp}$ 越大,滴定突越范围$\Delta \text{p}X$ 越大

\subsection{银量法指示终点的方法}%
\label{sub:银量法指示终点的方法}
\begin{itemize}
    \item 铬酸钾指示剂法(莫尔法)
    \item 铁铵钒指示剂法
    \item 吸附指示剂法    
\end{itemize}
\subsubsection*{铬酸钾指示剂法}%
\label{subsub:铬酸钾指示剂法}
\begin{notation}
铬酸钾指示剂法:铬酸钾为黄色,使用时只能用银离子做滴定剂滴定$Cl-$ 和$Br-$ ,主反应:\[
    \begin{cases}
        \text{终点前:}\ce{Ag+ +Cl- = AgCl\downarrow} &K_\text{sp}=1.8\times 10^{-10}\\ 
        \text{终点时:}\ce{2Ag+ +CrO_4^2- = Ag_2CrO_4\downarrow} &K_\text{sp}=1.2\times 10^{-12}
    \end{cases}
.\]
易得氯离子浓度与铬酸根浓度相等时,通过\[
    S=\sqrt[m+n]{\frac{K_\text{sp}}{m^{m}\cdot n^{n}}}
.\]
计算可得氯化银先沉淀
\end{notation}
\begin{notation}
    铬酸钾指示剂需要适量:过量时终点提前($S_{\ce{Ag_2CrO_4}}$ 所需的银离子减少),同时铬酸根本身的黄色也会影响;计算得100 mL被滴定液一般使用1-2 mL指示剂,必要时做空白校正

    酸度规定:中性或弱碱性(pH=6.5-10.5),酸性条件下形成铬酸,碱性条件下形成氧化银沉淀

    不能加入如铵盐等可以与阴离子形成配合物的离子

    滴定时需\textbf{剧烈震摇}

    干扰离子有: \begin{itemize}
        \item 沉淀离子:$\ce{PO_4^3-, CO_3^2-, S^2-, Ba^2+, Pb^2+,}\ldots $
        \item 有色离子:$\ce{Fe^{3+}, Cu^{2+},} \ldots $
        \item 易水解的离子:$\ce{Sn^{2+}, Al^{3+}, }\ldots $
    \end{itemize}
    应用范围:\begin{itemize}
        \item 滴定$\ce{Cl-,Br-,CN-}$ 
        \item 不能滴定$\ce{I-, SCN-}$ ($\ce{AgI, AgSCN}$ 对$\ce{I-, SCN-}$ 有强吸附性)
        \item 用$\ce{Cl-}$ 滴定$\ce{Ag+}$ :返滴定(用于$\ce{I-, SCN-}$ 等的滴定)
    \end{itemize}
\end{notation}
\begin{eg}
铬酸钾指示剂的应用:复方氯化钠滴眼液中氯化钠的滴定
\end{eg}
\subsubsection*{铁铵钒指示剂法}%
\label{subsub:铁铵钒指示剂法}
\begin{notation}
    使用$\ce{SCN-}$ 直接滴定$\ce{Ag+}$ \[
        \begin{cases}
            \ce{Ag+ +SCN- =AgSCN\downarrow} \text{ 白色}\\
            \ce{Fe^3+ +SCN- =Fe(SCN)^{2+}} \text{ 红色}\\
        \end{cases}
    .\]
    使用$\ce{NH_4Fe(SO_4)_2\cdot}\ce{12H_2O}$ 指示
\end{notation}

滴定条件:\begin{itemize}
    \item 在$0.1\sim 1\text{mol/L }\ce{HNO_3}$介质中进行
    \item 终点时$c_{\ce{Fe^3+}}\le 0.015\text{mol/L}$
    \item 充分震摇,降低沉淀附着银离子
    \item 预先去除干扰物质:强氧化剂、N的氧化物、铜盐、汞盐等
    \item 滴定$\ce{I-}$ 时,加入过量$\ce{AgNO_3}$ 再加入指示剂
    \item 滴定$\ce{Cl-}$ 时,注意沉淀转化
\end{itemize}
\begin{notation}
    $\ce{Cl-}$ 的沉淀转化:到终点时用力震摇,会发现红色消失,发生了沉淀转化;解决方法:\begin{itemize}
        \item 过滤
        \item 加入硝基苯
        \item 加入过量$\ce{Fe^3+}$
    \end{itemize}
\end{notation}
\begin{eg}
    林旦的测定:有机氯
\end{eg}
\subsubsection*{吸附指示剂法}%
\label{subsub:吸附指示剂法}
指示剂被吸附前后的颜色不一样
\begin{notation}
    指示剂:荧光黄\[
        HFIn \ce{<=>[][]} H+ +FIn-\text{(黄绿色)}
    .\]
\end{notation}
\begin{eg}
    $Ag+$ 滴定$Cl-$ 时,不过量时溶液中含$Cl-$ ,与荧光黄离子产生静电排斥,不吸附,呈黄绿色;过量时$Ag+$ 与荧光黄离子可以吸附,呈粉红色
\end{eg}
滴定条件:
\begin{itemize}
    \item 加入糊精保护胶体
    \item 溶液pH有利于吸附
    \item AgX吸附能力要略大于对指示剂的吸附能力
    \item 避光
\end{itemize}
应用范围:卤素离子,硫氰根离子,硫酸根离子,银离子,\textbf{不包含氰根离子}
\begin{notation}
    一般指示剂离子和滴定剂的电荷相反,与被测离子相同
\end{notation}
\begin{eg}
    单硝酸异山梨酯氯化钠注射液的含量测定,氯化琥珀胆碱注射液的含量测定
\end{eg}
\subsection{重量分析法}%
\label{sub:重量分析法}
\begin{defi}
    用适当方法将试样中的待测组分与其他组分分离,然后称重测定组分含量
\end{defi}
按分离方法分类:\begin{itemize}
    \item 沉淀重量法
    \item 挥发重量法
    \item 萃取法
\end{itemize}
\begin{notation}
    沉淀重量法的特点:准确度高,不需要特殊仪器和设备;繁琐费时

    操作过程:$\text{试样}\to \text{溶解}\to \text{沉淀}\to \text{过滤洗涤}\to \text{烘干灼烧}\to \text{称重}$
\end{notation}
\subsubsection*{沉淀重量法}%
\label{subsub:沉淀重量法}
\begin{eg}
测定$SO_4^2-$ 的含量:使用$BaCl_2$ 沉淀得到沉淀形式的$BaSO_4$,过滤时使用定量滤纸(无灰滤纸:灼烧后产生的灰量$<0.2\text{mg}$ 灰分/张),得到纯净沉淀,灼烧后得到称量形式的$BaSO_4$ 后称量
\end{eg}
\begin{notation}
    沉淀形式和称量形式可以不一样
    \begin{eg}
        沉淀形式:$Al(OH)3$ ;称量形式:$Al_2O_3$
    \end{eg}
    对沉淀形式的要求:沉淀要完全且溶解度必须要小,沉淀需要纯净,容易过滤和洗涤,易于转化为称量形式

    对称量形式的要求:组成固定,称重形式稳定(不吸收水、$CO_2$ 等),摩尔质量尽可能大

    对沉淀剂的要求:选择性好,易挥发、易灼烧除去
\end{notation}
\begin{notation}
沉淀形态:
\begin{table}[htpb]
    \centering
    \caption{沉淀形态}
    \label{tab:沉淀形态}
    \begin{tabular}{|c|c|c|c|}
    \hline
    类别 & 颗粒直径 & 特性 & 示例 \\
    \hline
晶型沉淀 & 0.1-1 $\mu$m & 颗粒大、排列规则、紧密,易于过滤 &  \\
\hline
无定形沉淀 &  &  &  \\
\hline
凝乳状沉淀 &  &  &  \\
\hline
    \end{tabular}
\end{table}
\end{notation}
