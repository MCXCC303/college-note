\lecture{11}{11.26}
\begin{notation}
    氧化态发生副反应时条件电位降低
\end{notation}
$\circ$ 酸效应:分两种情况:\begin{enumerate}
    \item $\ce{H+}$ 或$\ce{OH-}$ 参与反应,直接影响
    \item 氧化态或还原态为多元酸碱,酸度变化改变型体的浓度
\end{enumerate}
\begin{eg}
    25度时,酸性($c_{\ce{H+}}\approx 5$ mol/L)和碱性($\text{pH}\approx 8.0$)下,$\ce{H_3AsO_4}/\ce{HAsO_2}$ 的条件电位,判断和$\ce{I_2}/\ce{I-}$ 的反应方向
\end{eg}
解:半反应:
\begin{align*}
    \ce{H_3AsO_4 +2H+ + 2e-}&\ce{<=>[][]}\ce{HAsO_2 +2H_2O}\quad \varphi^{\ominus}=0.56V\\
    \ce{I_2 +2e-}&\ce{<=>[][]}\ce{2I-}\quad \varphi ^{\ominus'}\approx \varphi^{\ominus}=0.54
.\end{align*}
计算条件电位:
\[
    \varphi_\text{As}=\varphi^{\ominus}_\text{As}+\frac{0.059}{2}\lg \frac{c_{\ce{H_3AsO_4}}c_{\ce{H+}}}{c_{\ce{HAsO_2}}}
.\]
$\ldots $
\begin{notation}
    在酸性介质中铁的条件电位都会有所降低:
    \begin{table}[htpb]
        \centering
        \caption{不同介质中的条件电位}
        \label{tab:不同介质中的条件电位}
        \begin{tabular}{cccccc}
        \toprule
        介质 & 高氯酸 & 稀盐酸 & 硫酸 & 稀磷酸 & 浓磷酸\\
        \midrule
        条件电位 & 0.767 & 0.7 & 0.68 & 0.44 & 0.32\\
        \bottomrule
        \end{tabular}
    \end{table}
\end{notation}
\subsection{氧还反应进行的程度和速度}%
\label{sub:氧还反应进行的程度和速度}
使用平衡常数的大小衡量:
\begin{itemize}
    \item 绝对平衡常数
    \item 相对平衡常数
\end{itemize}
\begin{eg}
    写出两个半反应:
    \begin{align*}
        \ce{\ce{Ox_1} +n_1e- =\ce{Red_1}}\quad &\varphi_1=\varphi_1^{\ominus'}+\frac{0.059}{n_1}\lg \frac{c_{\ce{Ox_1}}}{c_{\ce{Red_1}}}\\
        \ce{\ce{Ox_2} +n_2e- =\ce{Red_2}}\quad &\varphi_2=\varphi_2^{\ominus'}+\frac{0.059}{n_2}\lg \frac{c_{\ce{Ox_2}}}{c_{\ce{Red_2}}}
    .\end{align*}
    当平衡时:$\varphi_1=\varphi_2$ 即:
    \[
        \varphi_1^{\ominus'}+\frac{0.059}{n_1}\lg \frac{c_{\ce{Ox_1}}}{c_{\ce{Red_1}}}=\varphi_2^{\ominus'}+\frac{0.059}{n_2}\lg \frac{c_{\ce{Ox_2}}}{c_{\ce{Red_2}}}
    .\]
    两边同乘$n=n_1\cdot n_2$ 整理后得:\[
        \lg K'=\lg \frac{c^{b}_{\ce{Ox_2}}\cdot c^{a}_{\ce{Red_1}}}{c^{a}_{\ce{Ox_1}}\cdot c^{b}_{\ce{Red_2}}}=\frac{n\left( \varphi_1^{\ominus'}-\varphi_2^{\ominus'} \right)}{0.059}=\frac{n\Delta\varphi^{\ominus'}}{0.059}
    .\]
    即电位差越大,反应越完全
\end{eg}
\begin{notation}
    一般根据滴定要求,反应的完全程度需要在$99.9\%$ 以上,即:$\frac{c_{\ce{Red_1}}}{c_{\ce{Ox_1}}}\ge 10^3 $ ,计算发现条件与电子转移数有关:
    \begin{itemize}
        \item $n=1$ 时,$\Delta\varphi'=0.35V$
        \item $n=2$ 时,$\Delta\varphi'=0.18V$
        \item $n_1:n_2=2$ 时,$\Delta\varphi^{\ominus'}\ge 0.27V$
    \end{itemize}
    即只要满足$\Delta\varphi^{\ominus'}\ge 0.4V$ 即可完全反应
\end{notation}
\subsubsection*{氧化还原反应的速度}%
\label{subsub:氧化还原反应的速度}
\begin{notation}
    部分反应理论上可行,但反应速度极慢
\end{notation}
\begin{eg}
    水溶液中的溶解氧相较于Sn:$\Delta\varphi^{\ominus '}\approx 1.08\gg 0.4$,理论上可以完全反应,但该反应太慢了,因此锡制品可以用于锅碗瓢盆等
\end{eg}
\begin{notation}
热力学只考虑反应程度,反映了反应的可能性;动力学考虑反应速度,反映了反应的现实性
\end{notation}
一般来说,反应物浓度越大,反应速度越快
\begin{eg}
$\ce{Cr_2O_7^2- +6I- +14H+} \ce{<=>[][]} \ce{2Cr^3+ +3I_2 +H_2O}$ ,加入过量$\ce{KI}$ ,置于暗处10min即可反应完全
\end{eg}
\begin{notation}
一般情况下:温度每升高$10^\circ\text{C}$ ,反应速度加快2到3倍
\end{notation}
催化剂:分为正催化和负催化
\begin{notation}
    自动催化反应:反应中产生了能催化反应的物质
\end{notation}
\subsection{氧化还原滴定曲线}%
\label{sub:氧化还原滴定曲线}
滴定反应:
\[
    n_2\ce{Ox_1}+n_1\ce{Red_2} \ce{<=>[][]}n_2\ce{Red_1} + n_1\ce{Ox_2}
.\]
\begin{notation}
滴定曲线的横坐标:滴定剂的体积或滴定百分数

酸碱滴定的纵坐标:pH;络合滴定的纵坐标:$\text{p}C_x $

氧还滴定的纵坐标:任意一个电对的电极电位(常用量大的/被滴定物质进行计算;随着滴定规律变化,动态平衡)
\end{notation}
滴定时计算电极电势:

滴定前:溶液中只有被滴定物质,且不知道已经发生了什么副反应,因此\textbf{氧还滴定图像没有起点}

滴定中:使用大量物质(被滴定物质)计算:

计量点处:完全反应,但仍有残留,把两个电对的电势相加:$2\varphi_\text{sp}=\varphi_\text{A}^{\ominus'} +\varphi_\text{B}^{\ominus'} $

过量时使用滴定物质计算
\begin{figure}[ht]
    \centering
    \incfig[0.6]{氧化还原反应滴定曲线}
    \caption{氧化还原反应滴定曲线}
    \label{fig:氧化还原反应滴定曲线}
\end{figure}
\begin{itemize}
    \item 对称电对:氧化态和还原态的系数相同
    \item 不对称电对:系数不相同
\end{itemize}
对称电对的电位值计算:\[
    \varphi_\text{sp}=\frac{n_1\varphi_1 ^{\ominus'} +n_2\varphi_2 ^{\ominus'} }{n_1+n_2}
.\]
\subsection{滴定突越范围的影响}%
\label{sub:滴定突越范围的影响}
$\Delta\varphi ^{\ominus'} $ 越大,滴定突越范围越大
\begin{figure}[ht]
    \centering
    \incfig{突越范围与电位差}
    \caption{突越范围与电位差}
    \label{fig:突越范围与电位差}
\end{figure}

