\lecture{12}{12.03}
\textit{条件电位}

$\circ$ 影响条件电位的因素:氧化态发生副反应(降低),还原态发生副反应(升高)

$\circ$ 氧还反应的进行程度:$\lg K'=\frac{n\Delta\varphi ^{\ominus'}  }{0.059}$,当$\Delta\varphi ^{\ominus'} >0.4V$ 时符合滴定要求

$\circ$ 影响氧还反应进行速度的因素:浓度、温度、催化剂

$\circ$ 氧还反应的滴定曲线:电极电势为纵坐标,化学计量点电位$\varphi_\text{sp}=\frac{n_1\varphi_1 ^{\ominus'} +n_2\varphi_2 ^{\ominus'} }{n_1+n_2}$,影响滴定突越范围的因素有$\Delta\varphi ^{\ominus'} $ ,与滴定剂的浓度无关

\subsection{氧还滴定的指示剂}%
\label{sub:氧还滴定的指示剂}
\begin{notation}
    配位滴定指示剂:金属指示剂(铬黑T)
\end{notation}
氧还滴定共五类指示剂:
\begin{description}
    \item[自身指示剂] 滴定剂或被测物本身有颜色,滴定产物无色或浅色,使用本身颜色的变化指示
    \item [特殊指示剂] 物质本身不具有氧化还原性,但可以和氧化还原电对形成有色配合物
    \item [氧化还原指示剂] 自身为弱氧化剂或还原剂:$\underset{\text{颜色1}}{\text{In}_\text{Ox}}+n\mathrm{e}^{-}\ce{<=>[][]}\underset{\text{颜色2}}{\text{In}_\text{Red}}$
    \item [外指示剂] 本身具有氧化还原性,但能与标准溶液或与被测溶液反应,不能加入溶液中,在化学计量点附近取出一些被滴定溶液在外界与指示剂反应
    \item [不可逆指示剂] 与微过量标准溶液作用,会发生不可逆的颜色变化
    
\end{description}
\begin{eg}
第一类:高锰酸钾$\underset{\text{粉红色}}{\ce{MnO_4-}} \to \underset{\text{无色}}{\ce{Mn_2+}}$ 

碘量法:$\underset{\text{浅黄色}}{\ce{I_2}}\to \underset{\text{无色}}{\ce{I-}}$

优点:无需选择指示剂
\end{eg}
\begin{eg}
    第二类:淀粉+单质碘$\to \ce{I_3-}$ :显蓝色,可逆,碘量法常用
\end{eg}
\begin{notation}
    第三类:电对电位使用能斯特方程计算,计算可得理论变色点:
    \[
        \frac{c_{\ln (\text{O})}}{c_{\ln (\text{R})}}=1
    .\]理论变色范围:\[
        \varphi=\varphi_{\ln (\text{O})/\ln (\text{R})}^{\ominus'} \pm \frac{0.059}{n}
    .\]
    常见指示剂变色电位和颜色:P94
\end{notation}
\begin{eg}
    第四类:重氮化滴定法,使用碘化钾+淀粉糊指示
\end{eg}
\begin{notation}
    第五类:只能在滴定终点时滴入指示
    \begin{eg}
        在溴酸钾中滴入甲基红或甲基橙,单质溴会破坏指示剂
    \end{eg}
\end{notation}
\subsection{氧还滴定的终点误差}%
\label{sub:氧还滴定的终点误差}
误差的产生:EP(End Point)和SP(化学计量点)不一致
\begin{notation}
    林邦误差公式:
    \[
        TE\left( \% \right)=\frac{[\ce{Ox_1}]_{\text{sp}}-[\ce{Red_2}]_{\text{sp}}}{c_{\text{sp}}}\times 100\%
    .\]
\end{notation}
\subsection{氧还滴定预处理}%
\label{sub:氧还滴定预处理}
\begin{notation}
    将待测物质转化为可配滴定价态的过程
\end{notation}
\begin{eg}
重铬酸钾滴定二价铁:将所有Fe先转换为Fe(II)
\end{eg}
要求:
\begin{itemize}
    \item 迅速、定量、完全转换为目标价态
    \item 选择性好,不生成干扰物质
    \item 过量处理物质易于除去
\end{itemize}
\subsection{常用的氧还滴定法}%
\label{sub:常用的氧还滴定法}
\subsubsection*{碘量法}%
\label{subsub:碘量法}
使用$\ce{I_2}$ 的弱氧化性和$\ce{I-}$ 的中等还原性

半反应:\[
    \ce{I_2 +2e-} \ce{<=>[][]} \ce{I-} \quad \varphi_{\ce{I_2 /I-}}^{\ominus'} =0.535V
.\]
原理:\[
    \ce{I_2 +I-}\ce{<=>[][]}\ce{I_3-}
.\]
直接碘量法/碘滴定法:使用碘单质作为滴定剂,利用氧化性,要求$\varphi^{\ominus'} <\varphi_{\ce{I_2 /I-}}^{\ominus'} $ ,可以测定:$\ce{S^2-, Sn^2+, AsO_2^-, S_2O_3^2-, VC}$等;酸度要求:弱酸碱性(pH<9)或中性,强酸下发生氧化反应,强碱性下发生歧化反应
\begin{eg}
VC的滴定:碱性条件下VC极易氧化,因此使用HAc提供的弱酸性;使用淀粉指示

$\ce{Na_2SO_3}$ 的测定:基本反应:\[
    \begin{cases}
        \ce{I_2 +2e-}\ce{<=>[][]}\ce{2I-}\\
        \ce{SO_3^2- +H_2O - 2e-}\ce{<=>[][]}\ce{SO_4^2- +2H+}
    \end{cases}
.\]
\end{eg}
间接碘量法/滴定碘法:
\begin{itemize}
    \item 测定物质为氧化性物质,$\varphi^{\ominus'} >\varphi_{\ce{I_2 /I-}}^{\ominus'} $ ,产生定量的$\ce{I_2}$ ,又称置换碘量法
    \item 测定物质为还原性物质时,加入定量且过量的$\ce{I_2}$ ,测定剩余的$\ce{I_2}$,属于返滴定法,称为剩余碘量法
\end{itemize}
\begin{notation}
药典规定,使用$\ce{Na_2S_2O_3}$ 滴定间接滴定法中产生的$\ce{I_2}$,反应为:\[
    \ce{I_2 +2S_2O_3^2-} \ce{<=>[][]}\ce{2I- +S_4O_6^2-}
.\]
可以测定大部分氧化还原物质
\end{notation}
\begin{notation}
    间接碘量法无法在强酸性中使用,碱性条件下无论是强碱性还是弱碱性都不可以:硫代硫酸钠会在碱性条件下和碘发生副反应
\end{notation}
\begin{eg}
    漂白粉中有效成分次氯酸钙的测定(置换碘量法)

    剩余碘量法测定葡萄糖的含量
\end{eg}
\begin{notation}
碘量法误差的两大来源:碘挥发、碘离子被空气氧化

解决方法:
\begin{enumerate}
    \item 防止碘挥发:
        \begin{itemize}
            \item 加入过量的KI,使$\ce{I_2}$ 生成$\ce{I_3-}$ 
            \item 在室温中滴定
            \item 使用碘量瓶,水封瓶口,快滴慢摇
        \end{itemize}
    \item 防止氧化:
        \begin{itemize}
            \item 降低酸度
            \item 防止阳关直射
            \item 使用碘量瓶,降低溶解氧的含量
        \end{itemize}
\end{enumerate}
\end{notation}
\begin{notation}
    碘量法的指示剂:
    \begin{description}
        \item[自身指示剂] 在$\ce{CCl_4}$ 中为紫红色
        \item [淀粉] $\ce{I_3 }+\text{直链可溶淀粉}\ce{<=>[][]}\text{深蓝色吸附物}$
    \end{description}
    注意事项:
    \begin{itemize}
        \item \begin{description}
            \item[直接碘量法] 滴定前加入,终点为蓝色,30秒不变色即可
            \item [间接碘量法] 只能在临近终点时加入,通过碘的自身指示大概判断,终点时蓝色褪去,5秒不变蓝即可
        \end{description}
        \item 不推荐加热
        \item 控制酸度:pH<2时成为糊精
        \item 淀粉质量控制:使用直链淀粉,加入防腐剂
        
    \end{itemize}
\end{notation}
