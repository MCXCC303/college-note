\lecture{14}{12.17}
\section{沉淀滴定法和重量分析法}%
\label{sec:沉淀滴定法和重量分析法}
\subsection{沉淀平衡}%
\label{sub:沉淀平衡}
\[
    \text{\ce{MA}}(s) \ce{<=>[][]}\text{\ce{MA}}(a.q.) \ce{<=>[][]} \ce{M+ +A-}
.\]
$\circ$ 第一步:固体转移至液相
\begin{notation}
    固有溶解度$S^{0}$ :从固相转移到液相的溶解度,或一分子状态存在的活度\[
        S^{0}=\frac{\alpha_{\text{\ce{MA}}\left( a.q. \right)}}{\alpha_{\text{\ce{MA}}\left( s \right)}}
    .\]
    一定温度下$\alpha_{\text{\ce{MA}}\left( s \right)}=1$ ,即$S^{0}=\alpha_{\text{\ce{MA}}\left( a.q. \right)}$
\end{notation}
\begin{notation}
    温度一定,沉淀的固有溶解度是一个常数($10^{-6}\sim 10^{-9}$ mol/L)
\end{notation}
$\circ$ 第二步:离子平衡
\begin{notation}
    $K_\text{ap}$ :活度积\[
        K_\text{ap}=a_{\ce{M+}}\cdot a_{\ce{A-}}
    .\]
    当温度一定时$K_\text{ap}$ 为常数\[
        K_\text{ap}=[\ce{M+}]\cdot \gamma_{\ce{M+}}\cdot [\ce{A-}]\cdot \gamma_{\ce{A-}}
    .\]
    转换得:\[
        \frac{K_\text{ap}}{\gamma_{\ce{M+}}\gamma_{\ce{A-}}}=K_\text{sp}
    .\]
    $K_\text{sp}$ 即为溶度积,一般情况下,由于难溶性盐的溶解度极小,即$\gamma\approx 1$ ,因此$K_\text{ap}\approx K_\text{sp}$
\end{notation}
\begin{notation}
    沉淀的溶解度$S$ :水相分子态+水相离子态$=S^{0}+c_{\ce{M+}}$ ,由于$S^{0}$ 很小,可以忽略不计

    对于$\ce{MA}$ 型沉淀:\[
        S=[\ce{M+}]=[\ce{A-}]=\sqrt{K_\text{sp}}=\sqrt{\frac{K_\text{ap}}{\gamma_{\ce{M+}}\gamma_{\ce{A-}}}} 
    .\]
    对于任意型沉淀$M_mA_n$ :由于$K_\text{sp}=[M^{n+}]^{m}\cdot [A^{m+}]^{n}=m^{m}\cdot n^{n}\cdot S^{m+n}$,有以下结论:\[
        S=\frac{[\ce{M+}]}{m}=\frac{[\ce{A-}]}{n}\quad S=\sqrt[m+n]{\frac{K_\text{sp}}{m^{m}\cdot n^{n}}}
    .\]
\end{notation}
\subsection{条件溶度积}%
\label{sub:条件溶度积}
当有副反应存在:$K_\text{sp}'>K_\text{sp}$

此时对于$\ce{MA}$ 型沉淀:$S=\sqrt{K_\text{sp}'} =\sqrt{K_\text{sp}\cdot \alpha_{\ce{M}}\cdot \alpha_A}$
\subsection{影响沉淀溶解平衡的因素}%
\label{sub:影响沉淀溶解平衡的因素}
\begin{itemize}
    \item 同离子效应
    \item 酸效应
    \item 配位效应
    \item 盐效应
    \item \ldots 
\end{itemize}
\subsubsection*{同离子效应}%
\label{subsub:同离子效应}
\begin{defi}
    沉淀达到平衡时向溶液中加入\textbf{构晶离子},可以降低溶解度
\end{defi}
\begin{eg}
    用$\ce{BaSO_4}$ 重量法测定$\ce{SO_4^{2-}}$ 时,用$\ce{BaCl_2}$ 沉淀,计算等量和过量0.01mol/L $\ce{Ba^{2+}}$ 时在200mL溶液中$\ce{BaSO_4}$ 沉淀的溶解损失
\end{eg}
\subsubsection*{酸效应}%
\label{subsub:酸效应}
\begin{defi}
溶液pH对溶解度的影响:$S=\sqrt{K_\text{sp}'}=\sqrt{K_\text{sp}\cdot \alpha_\text{H}}$
\end{defi}
\begin{eg}
    草酸存在时$CaC_2O_4$ 沉淀平衡将向沉淀溶解移动:\[
        \ce{CaC_2O_4 \ce{<=>[][]}Ca^{2+} +C_2O_4^{2-}}
    .\]
\end{eg}
\begin{notation}
    酸度对弱酸型或多元酸型沉淀溶解度影响较大
\end{notation}
\subsubsection*{配位效应}%
\label{subsub:配位效应}
\begin{defi}
    配位剂与构晶离子形成配合物,使平衡朝溶解方向移动
\end{defi}
\begin{eg}
    $Ag+$ 和 $Cl-$ 配位:
    \begin{align*}
        \ce{AgCl &\ce{<=>[][]}Ag+ +Cl-}\\
        \ce{AgCl + Cl- &\ce{<=>[][]} AgCl_2^{-}}\\
                       &\ldots 
    .\end{align*}
    在该过程中:一开始由沉淀反应的同离子效应主导,过量的$Cl-$ 后以配位效应主导
\end{eg}
\begin{notation}
    配位效应与$S$ 和$K_\text{sp}$ 有关
\end{notation}
\subsubsection*{盐效应}%
\label{subsub:盐效应}
\begin{defi}
当溶液中有大量其他离子(尤其是高价态电解质)存在:活度系数$\gamma$ 降低,由:\[
    K_\text{sp}=\frac{K_\text{ap}}{\gamma_{\ce{M^{n+}}}\cdot \gamma_{\ce{A^{m-}}}}
.\]
使得$K_\text{sp}$ 增大,即溶解度$S$ 增大
\end{defi}
\subsubsection*{其他因素}%
\label{subsub:其他因素}
\begin{description}
    \item [温度$T$] 显然当温度增大时溶解度增大(溶解损失增大)
    \item [溶剂极性] 溶剂极性越小,溶解度越小(如在乙醇中析出沉淀)
    \item [颗粒大小] 同种沉淀颗粒越大溶解度越小(与溶解时间相关较大)
    \item [胶体] 形成胶体时为“胶溶”状态(可以穿过滤纸),因此溶解度增大(可以加入电解质破坏)
    \item [水解] 溶解度增大:$\ce{Fe^{3+} + 3H_2O \ce{<=>[][]} Fe(OH)_3 +3H+}$
\end{description}
\subsection{沉淀滴定法}%
\label{sub:沉淀滴定法}
\begin{defi}
    以沉淀反应为基础的滴定分析方法
\end{defi}
适用条件:
\begin{enumerate}
    \item 恒定组成
    \item 溶解度小
    \item 快速沉淀
    \item 适当的指示终点方法
    \item 沉淀的\textbf{吸附现象}不会引起显著的误差
\end{enumerate}
\begin{notation}
    吸附现象:沉淀作为细小颗粒可能吸附其他颗粒从而改变沉淀质量
\end{notation}
同时满足以上条件的反应非常少,符合的案例:
\begin{notation}
    催眠药:异戊巴比妥,小剂量产生镇静作用,大剂量催眠、麻醉、昏厥等作用

    根据中国药典2020版,测定异戊巴比妥使用银量法(使用$\ce{AgNO_3}$ 沉淀)
\end{notation}
\begin{eg}
    银量法:\[
        \ce{Ag+ +X- \ce{<=>[][]} AgX \downarrow}
    .\]
    其中$\ce{X=Cl, Br, I, CN, SCN}\ldots $,与碘量法类似,银量法也有直接法和间接法(直接滴定、返滴定)
\end{eg}
\begin{notation}
    滴定曲线:加入的沉淀剂的体积$V$为横坐标,溶液中金属离子浓度或阴离子浓度的负对数pX绘制
\end{notation}
\subsubsection*{银量法基本原理}%
\label{subsub:银量法基本原理}
\begin{eg}
    使用0.1M $\ce{AgNO_3}$ 滴定 20.00mL 的0.1M NaCl:

    滴定开始前:$[\ce{Cl-}]=0.1$ 即$\text{pCl}=1$ 

    滴定至化学计量点前:\[
        [\ce{Cl-}]=\frac{\left( 20.00-V \right)\times 10^{-3}\times 0.1}{\left( 20.00+V \right)\times 10^{-3}}
    .\] 
    化学计量点时:$\text{pAg}+\text{pCl}=\frac{1}{2}\text{p}K_\text{sp}$

    化学计量点后: \[
        [\ce{Ag+}]=\frac{V\times 10^{-3}\times 0.1}{\left( 20.00+V \right)\times 10^{-3}}
    .\]
    绘制出的曲线:两种离子在化学计量点相交,完全对称
\begin{figure}[ht]
    \centering
    \incfig{沉淀反应滴定曲线}
    \caption{沉淀反应滴定曲线}
    \label{fig:沉淀反应滴定曲线}
\end{figure}
\end{eg}

