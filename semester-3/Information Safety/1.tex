\section*{课程要求}%
\label{sec:课程要求}
教师邮箱:yinyunfei@cqu.edu.cn

课程:信息技术与社会(信社)

课程目标:32学时,课上考核

课程内容:信息安全、网络技术、信息处理、电子商务、网络游戏与人类社会的关系,大数据、云计算、深度学习

成绩组成:平时表现(50\%)+课程报告(50\%,带PPT讲解,心得报告,技术调研,小论文)

平时成绩:出勤(10\%)+课堂分享(20\%)+回答问题、讨论、作业(20\%)

采用翻转课堂:教师40\%,主体转换分享10\%,主体转换报告50\%,期末考核(与主体转换报告结合)

\section{信息安全与社会}%
\label{sec:信息安全与社会}
\begin{question}
    何为信息安全?
\end{question}
\begin{notation}
    网络攻击:利用漏洞进行对硬件、软件和系统数据的攻击
\end{notation}
攻击可分为主动式攻击(篡改、伪造、拒绝服务)和被动攻击(流量分析、窃听)

攻击层次:简单拒绝服务,非读写权限,管理员权限

攻击方法:特洛伊木马,www欺骗,口令入侵,电子邮件,端口扫描等

攻击位置:本地,远程,伪远程
\begin{question}
    如何防范网络攻击?
\end{question}
1. 谨慎不明邮件、链接、软件、游戏

2. 复杂的密码设置

3. 系统补丁和防火墙

4. 代理服务器

\begin{notation}
    信息安全常用技术:

    1. 密码学(编码学和解码学)

    2. 认证技术(Authentication)

    3. PKI(公钥基础设施,Public Key Infrastructure):通过核对证书确认公钥所属

    4. 信息隐藏:对不需要隐秘信息的人来说该信息的载体正常且无法访问

    5. 访问控制

    6. 防火墙:位于内网和外网之间的网安系统

    7. 入侵检测:主动(特征)+被动(异常)
\end{notation}
\begin{notation}
    非对称加密(SHA,RSA等)中:公钥用于加密信息,私钥用于解密信息
\end{notation}
