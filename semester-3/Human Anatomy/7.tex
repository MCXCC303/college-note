\lecture{7}{10.24}
\begin{notation}
    十二指肠球:十二指肠溃疡的常发处
\end{notation}
\subsubsection{空肠、回肠}%
\label{subsub:肠-回肠-}
\begin{table}[htpb]
    \centering
    \caption{空肠与回肠}
    \label{tab:空肠与回肠}
    \begin{tabular}{ccc}
    \toprule
    区别 & 空肠 & 回肠\\
    \midrule
    位置 & 左上腹 & 右下腹\\
    颜色 & 粉红 & 粉灰\\
    淋巴滤泡 & 孤立 & 集合\\
    血管弓(动脉之间的交叉)& 较少 & 较多\\ 
    \bottomrule
    \end{tabular}
\end{table}
\subsection{大肠}%
\label{sub:大肠}
约1.5米长,位于空回肠的周围,可吸收水、维生素和无机盐

分为:盲肠、阑尾、结肠、直肠、肛管
\begin{notation}
    区分大肠和小肠:大肠含有脂肪垂
\end{notation}
\subsubsection{盲肠和阑尾}%
\label{subsub:肠和阑尾-}
盲肠位于:右髂窝,左接回肠,下端的盲囊可储存食物残渣

阑尾:盲肠的内下方蚯蚓状突起,尖端游离,根部附着在盲肠壁
\begin{notation}
    阑尾炎:食物残渣进入阑尾,导致发炎

    阑尾炎的特点:反绞痛

    阑尾位于麦氏点(肚脐与右髂连线的三分之一点)
\end{notation}
\subsubsection{结肠}%
\label{subsub:肠-}
\[
    \begin{cases}
        \text{升结肠}\\
        \text{横结肠}\\
        \text{降结肠}\\
        \text{乙状结肠}
    \end{cases}
.\] 
\subsection{直肠}%
\label{sub:直肠}
有两个弯曲:直肠骶区、直肠会阴区
\subsection{肛管}%
\label{sub:肛管}
长3-4厘米
\begin{notation}
    痔疮:肛门处静脉曲张

    齿状线以上称为内痔
\end{notation}
\subsection{肝}%
\label{sub:肝}
功能:分泌胆汁(胆管系统)

肝的四大系统:管道系统、动脉系统、门静脉系统、腔静脉系统
\begin{notation}
    前三套系统伴行
\end{notation}



