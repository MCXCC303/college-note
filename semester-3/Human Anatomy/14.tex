\lecture{14}{12.12}
\section{感觉器}%
\label{sec:感觉器}
\subsection{视器}%
\label{sub:视器}
\begin{notation}
    视器:眼球和眼副器
\end{notation}
\[
    \text{眼器}
    \begin{cases}
        \text{外膜:纤维膜}\begin{cases}
            \text{角膜}\\
            \text{巩膜}
        \end{cases}\\
        \text{中膜:血管膜}\begin{cases}
            \text{虹膜}\\
            \text{睫状体}\\
            \text{脉络膜}
        \end{cases}\\
        \text{内膜:视网膜}\begin{cases}
            \text{虹膜部分}\\
            \text{睫状体部分}\\
            \text{视部}
        \end{cases}
    \end{cases}
.\]
\begin{notation}
    虹膜和睫状体部分不成像
\end{notation}
\subsubsection*{虹膜}%
\label{subsub:虹膜}
\begin{notation}
眼球壁易发症:角膜溃疡、角膜白斑、巩膜色斑
\end{notation}
\begin{notation}
    虹膜的作用:由瞳孔开大肌和瞳孔括约肌控制进光量,临床上使用虹膜的对光反射性质判断生命体征
\end{notation}
\subsubsection*{睫状体}%
\label{subsub:睫状体}
睫状肌为环绕晶状体的一圈肌肉,内部的韧带和晶状体相连;睫状肌舒张时韧带拉紧,弧度减小,看的距离更远
\subsection{眼球内容物}%
\label{sub:眼球内容物}
\[
    \begin{cases}
        \text{房水}\\
        \text{晶状体}\\
        \text{玻璃体}
    \end{cases}
.\]
\begin{notation}
    房水循环:\[
        \text{睫状体上皮}\to \text{眼后房}\to \text{瞳孔}\to \text{虹膜角膜角}\to \text{巩膜静脉窦}\to \text{眼静脉}\to \text{睫状体上皮}
    .\]
\end{notation}
\begin{notation}
    房水过多导致眼压过大:青光眼,早期症状不明显,通过肾上腺素类药物降低房水分泌,瞳孔缩小,促进房水吸收
\end{notation}
\subsubsection*{晶状体}%
\label{subsub:晶状体}
支撑、调光
\begin{notation}
    晶状体病变:白内障、飞蚊症
\end{notation}
\ldots 
\section{前庭蜗器}%
\label{sec:前庭蜗器}
\[
    \text{前庭蜗器}\begin{cases}
        \text{外耳}\begin{cases}
            \text{耳廓}\\
            \text{耳屏}\\
            \text{耳垂}
        \end{cases}\\
        \text{中耳}\begin{cases}
            \text{鼓膜}\\
            \text{耳道}
            \text{颈动脉、静脉壁}
            \text{迷路壁}\\
        \end{cases}\\
        \text{内耳}\begin{cases}
            \text{听小骨}\\
            \text{咽鼓管}\\
            \text{骨迷路}\begin{cases}
                \text{耳蜗}\\
                \text{前庭}\\
                \text{骨半规管}
            \end{cases}
        \end{cases}
    \end{cases}
.\]
\begin{notation}
    鼓膜病变:鼓膜穿孔
\end{notation}
