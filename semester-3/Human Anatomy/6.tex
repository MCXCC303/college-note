\lecture{6}{10.17}
\begin{notation}
    作业讲解:

    1. 关节的主要结构:关节囊,关节腔,关节盘

    2. 胸大肌的作用:肩关节内收、旋外

    3. 前斜角肌的形态结构:膈神经后方
\end{notation}
\section{内脏}%
\label{sec:内脏}
\subsection{总论}%
\label{sub:内脏总论}
\begin{notation}
    内脏的一般结构:

    位于胸、腹、盆腔内,分为中空性和实质性器官

    中空性器官一般有3-4层组织
\end{notation}
胸部标志线:

1. 前正中线

2. 腋中线

3. 肩胛线

4. 后正中线

5. 锁骨中线

腹部分区

1. 四区分法:左右上下腹

2. 9区分法:
\[
    \text{9区分法}\begin{cases}
        \text{腹上区}\\
        \text{左季肋区}\\
    \end{cases}
.\] 
\section{消化系统}%
\label{sec:消化系统}
\begin{defi}
    消化管+消化腺

    有摄食、消化、吸收、排泄、内分泌功能
\end{defi}
\begin{align*}
    \text{消化管}\begin{cases}
        \text{口腔}\\
        \text{咽}\\
        \text{食管}\\
        \text{胃}\\
        \text{小肠}\begin{cases}
            \text{十二指肠}\\
            \text{空肠}\\
            \text{回肠}
        \end{cases}\\
        \text{大肠}\begin{cases}
            \text{盲肠}\\
            \text{阑尾}\\
            \text{结肠}\\
            \text{直肠}\\
            \text{肛管}
        \end{cases}
    \end{cases}\\
    \text{消化腺}\begin{cases}
        \text{大消化腺:唾液腺、肝、胰}\\
        \text{小消化腺:消化管黏膜内的小腺体}
    \end{cases}
.\end{align*}
\begin{notation}
    十二指肠及以上称为上消化道
\end{notation}
\subsection{口腔}%
\label{sub:口腔}
\begin{notation}
    牙齿前、唇后为前庭口腔

    口腔有两个开口:外界、咽腔
\end{notation}
\begin{notation}
    人中:鼻下方唇上方

    腮腺乳头:位于上颌第二磨牙牙冠相对的颊黏膜上
\end{notation}
\subsubsection{腭}%
\label{subsub:腭}
分硬腭软腭

硬腭:骨为主

软腭:肌腱、黏膜、肌为主
\subsubsection{牙}%
\label{subsub:牙}
根据形状和功能分为:切牙、尖牙、磨牙

小时候:乳牙共20个,上下各10个

换牙后:恒牙共32个,上下各16个
\begin{notation}
    记录牙的位置:

    1. 乳牙:罗马数字

    2. 恒牙:阿拉伯数字

    以中间为起始点(中切牙/门牙)
\end{notation}
\begin{eg}
    第三前磨牙:智齿
\end{eg}
\subsubsection*{牙的形态}%
\label{subsub:牙的形态}
1. 牙冠:外侧

2. 牙颈:内侧

3. 牙根:牙龈包围
\subsubsection*{牙组织}%
\label{subsub:牙组织}
釉质:第一层

牙质:第二层

牙骨质:牙根、牙颈表面

牙髓:牙腔内
\subsubsection{舌}%
\label{subsub:舌}
无骨,灵活

分为三部分:舌尖、舌体、舌根

功能:尝味道,发音,咀嚼
\begin{notation}
    界沟:倒V型
\end{notation}
\begin{notation}
    舌乳头:

    1. 丝状乳头(无味蕾)

    2. 菌状乳头

    3. 叶状乳头

    4. 轮廓乳头
\end{notation}
\begin{notation}
    舌下面有舌系带、舌下阜等
\end{notation}
\subsubsection*{舌的肌肉}%
\label{subsub:舌的肌肉}
舌内肌、舌外肌
\subsubsection{唾液腺}%
\label{subsub:唾液腺}
\begin{table}[htpb]
    \centering
    \caption{唾液腺}
    \label{tab:唾液腺}
    \begin{tabular}{ccc}
    \toprule
    名字 & 位置 & 导管开口\\
    \midrule
    腮腺 & 颧弓下 & 颊黏膜的腮腺乳头\\
    \bottomrule
    \end{tabular}
\end{table}
\subsection{咽}%
\label{sub:咽}
位置:颅底到第六颈椎下缘的脊柱前方,呼吸消化共用

分为鼻咽部(呼吸)、口咽部(口腔)、喉咽部(食道)
\subsubsection{鼻咽部}%
\label{subsub:鼻咽部}
含咽鼓管开口,与中耳相连

咽鼓管开口可以通过咽鼓管圆枕找到
\subsubsection{口咽部}%
\label{subsub:口咽部}
含会厌、腭扁桃体、咽淋巴环
\subsubsection{喉咽部}%
\label{subsub:喉咽部}
有梨状隐窝,容易卡食物
\subsection{食管}%
\label{sub:食管}
食管的三个狭窄(容易卡异物,易癌):

第一狭窄:起始处(第六颈椎,C6,15cm)

第二狭窄:左支气管(T4,T5,25cm)

第三狭窄:食管通过膈食管裂孔处(第十胸椎,T10,40cm)
\subsection{胃}%
\label{sub:胃}
收纳食物、分泌胃液、初步消化
\begin{notation}
    从上到下:

    贲门、胃体、幽门

    胃的疾病好发区:贲门到幽门的短边

    诊断胃溃疡:钡餐-X光/胃镜
\end{notation}
\subsection{小肠}%
\label{sub:小肠}
长5-7米
\subsubsection{十二指肠}%
\label{subsub:十二指肠}
呈C型,长25cm,由韧带牵拉
\begin{notation}
    十二指肠大乳头:消化液通过该处进入
\end{notation}

