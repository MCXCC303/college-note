\lecture{13}{12.05}
\textit{Review:}

$\circ$ 食管静脉丛、直肠静脉丛
\begin{notation}
通过口腔或直肠给药绕过首过消除效应
\end{notation}
\subsection{上肢动静脉}%
\label{sub:上肢动静脉}

\section{淋巴系统}%
\label{sec:淋巴系统}
\subsection{概述}%
\label{sub:概述}
\begin{defi}
    进入淋巴管道的组织液称为淋巴
\end{defi}
\begin{notation}
    淋巴器官:\begin{itemize}
        \item 淋巴结
        \item 脾脏
        \item 胸腺
    \end{itemize}
\end{notation}
\subsubsection*{脾脏}%
\label{subsub:脾脏}
位置:左季肋区,车祸易造成脾破裂和肝破裂
\ldots 
\section{感受器}%
\label{sec:感受器}
\begin{defi}
    感觉器:由感受器和附属器构成
\end{defi}
感受器分为:内感受器、外感受器、本体感受器
