\lecture{16}{12.29}
{\centering{\section*{复习}%
\label{sec:复习}
}}
\section{复习大纲}%
\label{sec:复习大纲}
\begin{itemize}
    \item 总论
    \item 运动系统
        \begin{itemize}
            \item 骨学
            \item 关节学
            \item 肌学
        \end{itemize}
    \item 内脏学
        \begin{itemize}
            \item 总论
            \item 消化系统
            \item 腹膜
            \item 呼吸系统
            \item 泌尿系统
            \item 男性生殖系统
            \item 女性生殖系统
        \end{itemize}
    \item 心血管系统
    \item 脉管系统
        \begin{itemize}
            \item 动脉
            \item 静脉
            \item 淋巴系统
        \end{itemize}
    \item 内分泌系统
    \item 感觉器系统
        \begin{itemize}
            \item 视器
            \item 耳
        \end{itemize}
    \item 神经系统
\end{itemize}
\subsection{总论}%
\label{sub:复习:总论}
\begin{notation}
    方位术语:
    \begin{itemize}
        \item 上下(距头/足)
        \item 前后(距身体腹侧面/背侧面)
        \item 内外(距正中矢状面)
        \item 内外侧(距内腔)
        \item 深浅(距皮肤)
        \item 远近(距躯干)
    \end{itemize}
\end{notation}
\begin{notation}
    轴与面:
    \begin{itemize}
        \item 矢状面$\to $ 矢状轴(左右两半)
        \item 冠状面$\to $ 冠状轴(前后两半)
        \item 水平面$\to $ 垂直轴(从上至下)
    \end{itemize}
\end{notation}
\begin{notation}
    九大系统:\textbf{运动、消化、呼吸、泌尿、生殖、脉管、感觉、神经、内分泌}
\end{notation}
\begin{notation}
    解剖学姿势:\textbf{人体直立、两眼平视、上肢下垂、掌心向前、下肢并拢、足尖向前}
\end{notation}
\subsection{运动系统}%
\label{sub:复习:运动系统}
\subsubsection{骨学}%
\label{subsub:复习:骨学}
\begin{notation}
    成人共\textbf{206}块骨,各部分的数量:
    \begin{itemize}
        \item 颅骨:29
            \begin{itemize}
                \item 脑颅骨:8
                    \begin{itemize}
                        \item 额骨:1
                        \item 枕骨:1
                        \item 蝶骨:1
                        \item 筛骨:1
                        \item 颞骨:$1\times 2$
                        \item 顶骨:$1\times 2$
                    \end{itemize}
                \item 面颅骨:15
                    \begin{itemize}
                        \item 上颌骨:$1\times 2$
                        \item 鼻骨:$1\times 2$
                        \item 颧骨:$1\times 2$
                        \item 泪骨:$1\times 2$
                        \item 下鼻甲:$1\times 2$
                        \item 腭骨:$1\times 2$
                        \item 犁骨:1
                        \item 下颌骨:1
                        \item 舌骨:1
                    \end{itemize}
                \item 听小骨:6
            \end{itemize}
        \item 躯干骨:51
            \begin{itemize}
                \item 椎骨:26
                    \begin{itemize}
                        \item 颈椎:7
                        \item 胸椎:12
                        \item 腰椎:5
                        \item 骶椎:1(骶骨,共5块)
                        \item 尾椎:1(尾骨,共2~3块)
                    \end{itemize}
                \item 肋骨:$12\times 2$
                    \begin{itemize}
                        \item 真肋:$7\times 2$
                        \item 假肋:$3\times 2$
                        \item 浮肋:$2\times 2$
                    \end{itemize}
                \item 胸骨:1
            \end{itemize}
        \item 四肢骨:126
            \begin{itemize}
                \item 上肢骨:64
                    \begin{itemize}
                        \item 上肢带骨:$2\times 2$
                            \begin{itemize}
                                \item 锁骨:$1\times 2$
                                \item 肩胛骨:$1\times 2$
                            \end{itemize}
                        \item 自由上肢骨:
                    \end{itemize}
                \item 下肢骨:62
            \end{itemize}
    \end{itemize}
\end{notation}
\begin{notation}
    手骨:\textbf{舟月三角豆、大小头状钩}
\end{notation}
\begin{notation}
    前囟位置:\textbf{头骨前方,额骨和顶骨之间的菱形缝隙}

    前囟闭合时间:\textbf{新生儿12至18个月}
\end{notation}
\subsubsection{肌学}%
\label{subsub:复习:肌学}
\begin{notation}
    斜方肌:\textbf{脖颈后侧}

    \textbf{竖脊肌}:从颅骨底部延伸至尾骨

    腹肌:\textbf{共4层,外斜、内斜、直、横}
\end{notation}
\begin{notation}
    腹股沟管:\textbf{两口四壁(内外口、前后上下壁)}
\end{notation}
\begin{notation}
    腹股沟管内口为\textbf{腹横筋膜},外口为\textbf{腹股沟浅环}

    前壁:\textbf{腹外斜肌腱膜、腹内斜肌},后壁:\textbf{腹横筋膜、腹膜},上壁:\textbf{腹内斜肌、腹横肌弓状下缘},下壁:\textbf{腹股沟韧带}
\end{notation}
\begin{notation}
    面肌包括表情肌和咀嚼肌

    或:头肌包括面肌(表情肌)和咀嚼肌
\end{notation}
\begin{notation}
    斜角肌间隙通过的结构:锁骨下动脉、臂丛
\end{notation}
\begin{notation}
    肱三头肌的起止点:[肱骨,尺骨]

    肱二头肌的起止点:[桡骨,肱骨]

    缝匠肌的起止点:[髂前上棘,胫骨上前缘]

    股二头肌的起止点:[坐骨结节,腓骨头]
\end{notation}
\begin{notation}
    下肢大肌肉的功能:
    \begin{description}
        \item[臀大肌] 后伸髋
        \item [股四头肌] 伸膝、屈髋
        \item [股二头肌] 屈膝
        \item [小腿三头肌] 提脚跟
    \end{description}
\end{notation}
\subsection{内脏学}%
\label{sub:复习:内脏学}
\begin{notation}
    食管的狭窄:\textbf{食管起始处15、左支气管交叉处25、食管裂孔处40}
\end{notation}
\subsubsection{呼吸系统}%
\label{ssub:复习:呼吸系统}
\begin{notation}
    肺的结构:\textbf{一尖、一底、三面、三缘}
\end{notation}
\subsubsection{泌尿系统}%
\label{subsub:复习:泌尿系统}
\begin{notation}
肾的高度:\textbf{左肾:11胸椎椎体下缘,右肾:12胸椎椎体上缘}
\end{notation}
\subsubsection{男性生殖系统}%
\label{ssub:复习:男性生殖系统}
\begin{notation}
男性尿道三个狭窄:\textbf{尿道内口、膜部、尿道外口(最狭窄)}

两个弯曲:\textbf{耻骨下弯、耻骨前弯}
\end{notation}
