\lecture{9}{11.07}
\textit{Review:}

$\circ$ 鼻旁窦有哪些:
\begin{itemize}
    \item 上颌窦
    \item 颌窦
    \item 筛窦
    \item 蝶窦
\end{itemize}
$\circ$ 异物易掉入右支气管

$\circ$ 胸膜包括壁胸膜和脏胸膜
\section{泌尿系统}%
\label{sec:泌尿系统}
组成:
\begin{itemize}
    \item 肾:产尿
    \item 输尿管
    \item 膀胱:储尿
    \item 尿道:排尿
\end{itemize}
功能:生成、排除尿液
\subsection{肾}%
\label{sub:肾}
形态描述:
\begin{enumerate}
    \item 上下两端
    \item 前后两面
    \item 内外侧两缘
\end{enumerate}
\begin{notation}
    肾内侧:
    \begin{itemize}
        \item 肾门:神经、血管、输尿管进出的部分
        \item 肾蒂:肾门处的一团组织
        \item 肾窦:肾门窝进去的一圈
    \end{itemize}
\end{notation}
位置上:左肾比右肾高一些
\begin{notation}
    肾门的体表投影称为肾区
\end{notation}
\subsubsection*{肾的被膜}%
\label{subsub:肾的被膜}
共三层膜:
\begin{itemize}
    \item 肾纤维囊
    \item 肾脂肪囊
    \item 肾筋膜
\end{itemize}
\subsubsection*{肾的构造}%
\label{subsub:肾的构造}
分为肾实质和肾窦内尿液引流管道,肾实质包括肾皮质、肾柱(伸入髓质)和肾髓质(淡红,肾椎体的尖端称为肾乳头),皮质内含有肾小球(过滤器)
\[
    \text{肾单位}\begin{cases}
        \text{肾小体}\begin{cases}
            \text{血管球}\\
            \text{肾小囊}
        \end{cases}\\
        \text{肾小管:重吸收原尿}
    \end{cases}
.\]
\subsubsection*{肾的病变与畸形}%
\label{subsub:肾的病变与畸形}
\begin{enumerate}
    \item 马蹄肾:长在一起的两个肾
    \item 多囊肾
    \item 交叉异位肾
    \item $\ldots$
\end{enumerate}
\subsection{输尿管}%
\label{sub:输尿管}
一般20~30cm,分为数段:
\begin{enumerate}
    \item 腹段
    \item 盆段
    \item 壁内段(0.5cm)
\end{enumerate}
\subsubsection*{输尿管的狭窄部}%
\label{subsub:输尿管的狭窄部}
\begin{enumerate}
    \item 起始处
    \item 小骨盆入口处(髂动脉和输尿管有一个交叉)
    \item 壁内段
\end{enumerate}
一旦肾结石在输尿管处发生堵塞将产生疼痛(类似胆结石)
\subsection{膀胱}%
\label{sub:膀胱}
\subsubsection*{内部结构}%
\label{subsub:内部结构}
$\circ$ 膀胱三角:位于左右输尿口和尿道内口之间,此处无皱襞,病变易发区

$\circ$ 输尿管间壁:两个输尿管之间的皱襞,膀胱镜下看到为一圈苍白带
\subsubsection*{膀胱位置}%
\label{subsub:膀胱位置}
位于盆腔前部、耻骨联合后方
\begin{itemize}
    \item 男性:毗邻前列腺、精囊、直肠
    \item 女性:毗邻子宫、阴道、直肠
\end{itemize}
\begin{notation}
    膀胱穿刺:在膀胱过分充盈时一般使用导尿,当导尿管无法插入或尿路感染时时需要进行膀胱穿刺,从耻骨联合上部1-2cm穿入

    由于超声无法穿过空气,因此进行b超前需憋尿,使肠道与膀胱贴合
\end{notation}
\subsection{尿道}%
\label{sub:尿道}
\begin{itemize}
    \item 男性尿道:长而弯,传输尿液和精液,长约18cm
    \item 女性尿道:短粗直,长约3-5cm,女性易发生尿路感染
\end{itemize}
\begin{notation}
    尿路结石:到达膀胱就可以自然排出,若结石卡到狭窄处则需要进行手术碎石
\end{notation}
\subsection{尿液的排泄途径}%
\label{sub:尿液的排泄途径}
\begin{enumerate}
    \item 肾小球滤过(原尿,含水、葡萄糖、尿素、无机盐)
    \item 肾小管重吸收(尿液,不含葡萄糖)
    \item 肾乳头
    \item 肾小盏
    \item 肾大盏
    \item 肾盂
    \item 输尿管$\to$ 膀胱$\to$ 尿道
\end{enumerate}
\section{男性生殖系统}%
\label{sec:男性生殖系统}
\begin{itemize}
    \item 内生殖器
        \begin{itemize}
            \item 生殖腺
            \item 生殖管道
            \item 附属腺
        \end{itemize}
    \item 外生殖器:阴囊、阴茎
\end{itemize}
\subsection{男性生殖腺}%
\label{sub:男性生殖腺}
男性生殖腺为睾丸,产生精子并分泌雄激素;睾丸位于阴囊内,有一定硬度(白膜和纤维隔),每个隔间中有精细小管和睾丸间质,产生精子
\begin{notation}
    男性睾丸为较脆弱部位
\end{notation}
\begin{notation}
    睾丸并非从始至终在阴囊内,一开始(三个月前)在腹腔内,逐渐转到阴囊内;如果睾丸一直未能掉入阴囊,称为隐睾症
\end{notation}
\subsection{生殖管道}%
\label{sub:生殖管道}
\begin{itemize}
    \item 附睾
    \item 输精管
    \item 射精管
    \item 尿道
\end{itemize}
\begin{notation}
    附睾:位于睾丸上后缘,可以储存、活化精子一周左右
\end{notation}
输精管:起于附睾尾部,长40cm
\begin{notation}
    精索:主要包含输精管、睾丸动脉、蔓状静脉丛、淋巴管、神经(类似于肾蒂等),输精管结扎手术常用精索结扎(剪断、拉开一段距离)
\end{notation}
\subsection{附属腺}%
\label{sub:附属腺}
从上到下:
\begin{itemize}
    \item 精囊
    \item 前列腺
    \item 尿道球腺(位于尿道生殖隔内)
\end{itemize}
\subsubsection*{精囊}%
\label{subsub:精囊}
可以分泌体液(前列腺液等)组成精液
\subsubsection*{前列腺}%
\label{subsub:前列腺}
呈板栗型/倒三角,常见病为前列腺炎,从上到下:底、体、尖;尿道和射精管穿过前列腺,共分为五叶:前、中、后、两侧
\subsection{男性外生殖器}%
\label{sub:男性外生殖器}
\subsubsection*{阴茎}%
\label{subsub:阴茎}
分为:根、体、头,根部紧贴前列腺;阴茎共三根海绵体:2条阴茎海绵体、1条尿道海绵体,海绵体充血后可勃起
\begin{notation}
    包皮内的包皮腔易藏污纳垢,是诱发宫颈癌的一大诱因
\end{notation}
海绵体外部包有白膜,较坚韧
\subsection{男性尿道}%
\label{sub:男性尿道}
特点:长而弯,分为前列腺部、膜部和海绵体部(大部分);有三个狭窄:内口、外口、膜部

尿道共两个弯曲:耻骨前弯、耻骨下弯,耻骨下弯难以插入导尿管
