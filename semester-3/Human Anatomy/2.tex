\lecture{2}{}
\section{运动系统}%
\label{sec:运动系统}
组成:骨(杠杆)+骨连结(枢纽)+骨骼肌(或骨骼+骨骼肌)

骨骼(skeleton)=骨+骨连结

\begin{notation}
    功能:

    1. 运动

    2. 支撑、保护(大脑、胸腔、盆腔)
\end{notation}
\begin{notation}
    体表标志:能在体表看到或摸到的一些骨性突起和肌性隆起
\end{notation}

\subsection{骨学}%
\label{sub:骨学}
\subsubsection{总论}%
\label{subsub:骨学总论}
人有206块骨头(6块听小骨归入感觉器)

按部位分类:
\[
    \begin{cases}
    \text{中轴骨}
    \begin{cases}
        \text{颅骨:29块}\\ 
        \text{躯干骨:51块}\\ 
    \end{cases}\\
    \text{四肢骨:126块}
    \end{cases}
.\] 

按形状分类:

1. 长骨:一体两端,呈管状,分布于四肢,两端膨大称骺,表面光滑称关节面,内有空腔称骨髓腔,容纳骨髓

2. 短骨:立方体,往往结在一起呈拱形,如腕骨

3. 扁骨:板状,分内板和外板,外板附有骨膜

4. 不规则骨:如椎骨、上颌骨

\begin{notation}
    骨的表面形态:

    1. 骨面突起、棘、隆起、粗隆、结节、嵴、线

    2. 骨面凹陷、窝、凹、小凹、沟、压迹

    3. 骨的空腔、窦、房、管、道、口、孔

    4. 骨端膨大、头、小头、颈、髁

    5. 平滑骨面、缘、切迹
\end{notation}

\subsubsection*{骨的构造}%
\label{subsub:骨的构造}
\[
    \text{骨}\\ 
    \begin{cases}
        \text{骨质}
        \begin{cases}
            \text{骨密质}\\ 
            \text{骨松质}
        \end{cases}\\
        \text{骨膜}\\ 
        \text{骨髓}
        \begin{cases}
            \text{红骨髓}\\ 
            \text{黄骨髓}
        \end{cases}\\
        \text{血管、淋巴、神经}
    \end{cases}
.\] 

\begin{notation}    
    骨密质:外部
    
    骨松质:内部
\end{notation}
\begin{notation}
    骨外膜:外层致密,内层疏松,有血管和神经分布

    骨内膜:菲薄结缔组织

    严重骨折时骨膜大量腐坏导致难以愈合
\end{notation}
\begin{notation}
    红骨髓:有造血功能

    黄骨髓:无造血功能,严重失血时转化为红骨髓

    临床上通过骨髓穿刺检查骨髓像
\end{notation}

\begin{notation}
    骨血管:滋养动脉、骺动脉、干骺端动脉、骨膜动脉

    骨淋巴管:主要位于骨膜

    骨神经:伴滋养动脉进入骨内
\end{notation}

\begin{notation}
    白血病的治疗方案:
    
    骨髓移植:

    1. 杀灭患者所有的血细胞

    2. 采集配型成功的供者骨髓的造血细胞/干细胞

    3. 去除干细胞中的恶性细胞、免疫细胞

    4. 将干细胞输给病人

    免疫细胞治疗(DC疗法,已淘汰;CAR-T疗法,主流):
    
    1. 培养出专一功能T细胞

    2. 输回人体,T细胞即可杀死对应癌细胞
\end{notation}
\subsubsection*{骨的化学成分和物理性质}%
\label{subsub:骨的化学成分和物理性质}
1. 有机质:弹性、韧性

2. 无机质:刚性、硬度

\subsubsection*{发生与发育}%
\label{subsub:发生与发育}

\subsubsection*{骨的重塑}%
\label{subsub:骨的重塑}
1. 血肿:激化

2. 成骨细胞附着

3. 骨细胞形成
\subsubsection{躯干骨}%
\label{subsub:躯干骨}
\[
    \text{躯干骨:51块}
    \begin{cases}
        \text{椎骨:26块}\\ 
        \text{胸骨:1块}\\ 
        \text{肋骨:12对}\\ 
    \end{cases}
.\] 

1. 椎骨:
\[
    \begin{cases}
        \text{颈椎:7块}\\ 
        \text{胸椎:12块,有肋骨连接}\\ 
        \text{腰椎:5块}\\ 
        \text{骶椎:5块}\implies \text{骶骨:1块}\\
        \text{尾椎:3-4块}\implies \text{尾骨:1块}
    \end{cases}
.\] 

\begin{notation}
    椎骨的形态:
    \[
        \text{椎骨}
        \begin{cases}
            \text{椎体}\\ 
            \text{椎弓}
            \begin{cases}
                \text{椎弓根:不同椎节之间组成椎间孔}
                \begin{cases}
                    \text{椎上切迹}\\ 
                    \text{椎下切迹}
                \end{cases}\\
                \text{椎弓板:七个突起}
                \begin{cases}
                    \text{棘突:1个}\\ 
                    \text{横突:1对}\\ 
                    \text{上下关节突:各1对}
                \end{cases}
            \end{cases}
        \end{cases}
    .\] 
    椎间孔中有神经穿过
\end{notation}

\begin{notation}
    颈椎:

    1. 椎体较小
    
    2. 有横突孔,椎动脉从中穿过

    3. 第2~6颈椎棘突短而分叉

    4. 第7颈椎棘突长
    \[
        \begin{cases}
            \text{第一颈椎:寰椎,无椎体}\\ 
            \text{第二颈椎:枢椎,椎体有齿突,齿突与寰椎的前突形成寰枢关节}\\ 
            \text{第七颈椎:隆椎,棘突长}\\ 
        \end{cases}
    .\] 
\end{notation}

\begin{notation}
    胸椎:
\end{notation}

\begin{notation}
    腰椎:

    椎体大,棘突宽短,水平向后伸

    棘突间隙宽,有利于腰椎穿刺
\end{notation}

\begin{notation}
    尾骨:
\end{notation}

\begin{notation}
    胸骨:

    柄、体、剑突

    柄与体连接处向前突称为胸骨角,连接第2肋
\end{notation}

\begin{notation}
    肋:

    包含肋骨和肋软骨,软骨终身不骨化

    共12对:
    \[
        \begin{cases}
            \text{1-7:真肋}\\ 
            \text{8-10:假肋}\\ 
            \text{11-12:浮肋}
        \end{cases}
    .\] 

    肋骨后端:肋头、肋颈、肋结节

    肋下端:肋沟、肋体、肋角
\end{notation}

\subsection{颅骨}%
\label{sub:颅骨}
分为:脑颅骨、面颅骨

共23块:脑颅骨8块,面颅骨15块

\subsubsection{脑颅骨}%
\label{subsub:脑颅骨}
\[
    \text{共8块}
    \begin{cases}
        \text{成对}
        \begin{cases}
            \text{颞骨}\\ 
            \text{顶骨}
        \end{cases}\\
        \text{不成对}
        \begin{cases}
            \text{额骨}\\ 
            \text{筛骨(嗅神经传到大脑)}\\ 
            \text{蝶骨}\\ 
            \text{枕骨}
        \end{cases}
    \end{cases}
.\] 
\begin{notation}
    额骨:
\end{notation}

\begin{notation}
    筛骨:含筛板、垂直板、筛骨迷路,较脆弱
\end{notation}

\begin{notation}
    蝶骨:体、大翼、小翼、翼突
\end{notation}

\begin{notation}
    颞骨:
\end{notation}

\begin{notation}
    枕骨:枕骨大孔(脊髓、脑干穿出)、枕髁、枕外隆突

    顶骨:外隆内凹
\end{notation}

\subsubsection{面颅骨}%
\label{subsub:面颅骨}
\[
    \text{共15块}
    \begin{cases}
        \text{成对}
        \begin{cases}
            \text{上颌骨}\\ 
            \text{鼻骨}\\ 
            \text{颧骨}\\ 
            \text{泪骨}\\ 
            \text{下鼻甲}\\ 
            \text{腭骨}
        \end{cases}\\
        \text{不成对}
        \begin{cases}
            \text{犁骨}\\ 
            \text{下颌骨}\\ 
            \text{舌骨}
        \end{cases}
    \end{cases}
.\] 

内部分为颅前窝、中窝、后窝

侧面:颞骨、额骨、顶骨、蝶骨交汇至翼点(太阳穴)
