\lecture{3}{}
\begin{notation}
    颅前面观:

    1. 眶

    2. 骨性鼻腔

    3. 骨性口腔
\end{notation}
\[
    \text{眶}
    \begin{cases}
        \text{底}\\ 
        \text{尖}\\ 
        \text{上/下壁}\\ 
        \text{内/外壁}
    \end{cases}
.\] 

\begin{notation}
    骨性鼻腔:为梯形管状腔隙

    分为上、中、下鼻甲,上、中鼻甲在蝶骨上,下鼻甲独立
\end{notation}
\begin{notation}
    鼻窦:含额窦、筛窦、蝶窦、上颌窦

    上颌窦最大,在上颌骨体内,窦口高于窦底,不易引流
\end{notation}
\begin{notation}
    新生儿颅骨中央有一块软骨
\end{notation}
颅骨的骨性标志:

1. 枕外隆突

2. 乳突:有肌肉与锁骨相连

3. 颧弓:颧骨和颞骨汇合

4. 下颌角:下巴

5. 眉弓/眶上缘/眶下缘
\subsubsection{附肢骨}%
\label{subsub:附肢骨}
附肢骨/四肢骨:126块

\[
    \begin{cases}
        \text{上肢骨:64}
        \begin{cases}
            \text{上肢带骨:锁骨、肩胛骨}\\ 
            \text{自由上肢骨:肱骨、桡骨、尺骨、手骨}
        \end{cases}\\
        \text{下肢骨:62}\begin{cases}
            \text{下肢带骨:髋骨}\\ 
            \text{自由下肢骨:股骨、髌骨、胫骨、腓骨、足骨}
        \end{cases}
    \end{cases}
.\] 
\subsubsection*{上肢带骨}%
\label{subsub:上肢带骨}
\begin{notation}
    锁骨:呈~型,易骨折
\end{notation}
\begin{notation}
    肩胛骨/琵琶骨:分三缘、三角、两面

    三角:外侧角(与关节连接)、上角、下脚

    含有喙突、关节盂、肩胛冈
\end{notation}
\subsubsection*{自由上肢骨}%
\label{subsub:自由上肢骨}
\begin{notation}
    肱骨:典型长骨,上端外科颈处脆弱、易骨折

    体端有桡神经沟(麻筋)
\end{notation}
\begin{notation}
    尺骨、桡骨:尺骨较大,尺骨位于内侧

    有尺骨、桡骨茎突组成关节
\end{notation}
\begin{notation}
    手骨:
    \begin{table}[htpb]
        \centering
        \caption{手骨}
        \label{tab:手骨}
        \begin{tabular}{ccc}
        \toprule
        腕骨 & 掌骨 & 指骨\\
        \midrule
        短骨,共8块,含腕骨沟 & 长骨,5块 & 长骨,14块\\
        \bottomrule
        \end{tabular}
    \end{table}
\end{notation}
\subsubsection*{下肢带骨}%
\label{subsub:下肢带骨}
\begin{notation}
    髋骨+股骨:骨性连接,十分牢固

    髋骨=耻骨+坐骨+髂骨,含有髂嵴、髂结节、髂前上下棘、弓状线等
\end{notation}
\subsubsection*{自由下肢骨}%
\label{subsub:自由下肢骨}
\begin{notation}
    股骨:关节大,股骨颈处较小、易骨折,大量激素治疗会造成后遗症股骨头坏死

    上端:股骨头、股骨颈、大转子

    下端:内/外侧髁、髁间窝
\end{notation}
\begin{notation}
    髌骨:最大的籽骨(游离于关节之间)

    另一块籽骨:舌骨
\end{notation}
\begin{notation}
    胫骨、腓骨:腓骨位于外侧
\end{notation}
\begin{notation}
    足骨:
    \begin{table}[htpb]
        \centering
        \caption{足骨}
        \label{tab:足骨}
        \begin{tabular}{ccc}
        \toprule
        跗骨 & 跖骨 & 趾骨\\
        \midrule
        短骨,7块 & 长骨,5块 & 长骨,14块\\
        \bottomrule
        \end{tabular}
    \end{table}
\end{notation}
\section{关节学}%
\label{sec:关节学}
\subsection{总论}%
\label{sub:关节学总论}
\begin{notation}
    骨连结:分为直接连结(纤维、软骨、骨性连接)和间接连结(关节)

    骨性连结强度最大

    纤维连结:部分颅骨

    软骨连结:肋骨

    骨性连接:股骨
\end{notation}
间接连结:称为关节或滑膜关节

\begin{notation}
    关节的基本构造:
    \[
        \begin{cases}
            \text{关节面:含关节头和关节窝}\\ 
            \text{关节囊:纤维层(外层)+滑膜层(内层)}\\ 
            \text{关节腔:内含滑液,负压}
        \end{cases}
    .\] 

    特点:

    1. 仅借周围纤维结缔组织连结

    2. 相对骨面间存在含滑液的腔隙

    3. 具有较大的活动性
\end{notation}
\begin{notation}
    关节的辅助结构:
    \[
        \begin{cases}
            \text{韧带}\\ 
            \text{关节盘和关节唇}\\ 
            \text{滑膜壁和滑膜囊}\\ 
        \end{cases}
    .\] 
\end{notation}
\begin{notation}
    关节的运动:屈伸收展,旋转,环转
    \begin{table}[htpb]
        \centering
        \caption{关节的运动}
        \label{tab:关节的运动}
        \begin{tabular}{cccc}
        \toprule
        屈伸 & 收展 & 旋转(旋内/外) & 环转\\
        \midrule
        沿冠状轴运动 & 沿矢状轴运动 & 环绕旋转轴 & 二轴或三轴关节\\
        \bottomrule
        \end{tabular}
    \end{table}
\end{notation}
\subsubsection*{关节分类}%
\label{subsub:关节分类}
\begin{notation}
    单轴关节:滑车关节、车轴关节

    双轴关节:椭圆关节、鞍状关节

    多轴关节:球窝关节、平面关节
\end{notation}
\subsection{中轴骨的连结}%
\label{sub:中轴骨的连结}
\subsubsection{颅骨的连结}%
\label{subsub:颅骨的连结}
\begin{notation}
    颞下颌关节:

    1. 由下颌头、下颌窝、关节结节构成
    
    2. 特点:囊内有关节盘将关节腔分为上下两部分

    3. 常见病:下巴脱臼
\end{notation}
\subsubsection{躯干骨的连结}%
\label{subsub:躯干骨的连结}
1. 椎骨连结形成脊柱

2. 肋椎连结和胸肋连结形成胸腔
\begin{notation}
    脊柱:侧面有四个生理弯曲:颈曲、腰曲、胸曲、骶曲,其中腰曲和骶曲先天形成

    可以保护脊髓,有较大幅度的运动
\end{notation}
\begin{notation}
    椎体间连结:一盘两韧带(椎间盘、前后纵韧带)

    椎弓间连结:三韧带一对关节(黄韧带、棘间韧带、棘上韧带、关节突关节)

    易发症:椎间盘突出症
\end{notation}
\begin{notation}
    胸廓连结:肋椎关节和胸肋关节
\end{notation}
\subsubsection{附肢骨的连结}%
\label{subsub:附肢骨的连结}
\begin{notation}
    上肢骨连结:
    \[
        \begin{cases}
            \text{上肢带连结}\begin{cases}
                \text{胸锁关节}\\
                \text{肩关节} 
            \end{cases}\\ 
            \text{自由上肢连结}
        \end{cases}
    .\] 
\end{notation}






