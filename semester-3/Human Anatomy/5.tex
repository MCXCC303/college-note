\begin{notation}
    腱鞘:含滑膜,摩擦过度易引起腱鞘炎、腱鞘囊肿
\end{notation}

\subsection{头颈肌}%
\label{sub:头颈肌}
\[
    \begin{cases}
        \text{头肌}\begin{cases}
            \text{面肌}\\
            \text{咀嚼肌}
        \end{cases}\\
        \text{颈肌}
    \end{cases}
.\] 
\begin{notation}
    面肌:又称表情肌

    分为环形肌和辐射肌,有闭合或开打孔裂的作用
\end{notation}
\begin{notation}
    咀嚼肌:分布于颞下颌关节周围,参与咀嚼运动
    \[
        \begin{cases}
            \text{闭口}
            \begin{cases}
                \text{咬肌}\\
                \text{颞肌}\\
                \text{翼内肌}
            \end{cases}\\
            \text{张口:翼外肌}
        \end{cases}
    .\] 
\end{notation}
\begin{notation}
    颈肌:
    \[
        \begin{cases}
            \text{颈浅肌}\begin{cases}
                \text{颈阔肌}\\
                \text{胸锁乳突肌}\\
                \text{舌骨上下肌群}
            \end{cases}\\
            \text{颈深肌:斜角肌}
        \end{cases}
    .\] 

    胸锁乳突肌:

    1. 起自锁骨柄前面和锁骨的胸骨端,止于乳突

    2. 两侧同时收缩可使头后仰,一侧收缩向同侧倾斜
\end{notation}
\subsection{躯干肌}%
\label{sub:躯干肌}
\[
    \begin{cases}
        \text{背肌}\\
        \text{胸肌}\\
        \text{膈肌}\\
        \text{腹肌}\\
        \text{会阴肌}
    \end{cases}
.\] 
\subsubsection{背肌}%
\label{subsub:背肌}
\[
    \begin{cases}
        \text{浅层}\begin{cases}
            \text{斜方肌}\\
            \text{背阔肌(最大扁肌)}
        \end{cases}\\
        \text{深层:竖脊肌}
    \end{cases}
.\] 
\begin{notation}
    临床上:

    1. 斜方肌瘫痪:塌肩

    2. 背阔肌常用于植皮(肌瓣)
\end{notation}
斜方肌:位于项、背部浅层,作用为上升、下降、内牵肩胛骨

背阔肌:位于腰、背部浅层,作用为运动臂(引体向上)

竖脊肌:位于项背腰骶部深层、棘突两侧,作用为后伸脊柱、仰头、维持直立
\subsubsection{胸肌}%
\label{subsub:胸肌}
\[
    \begin{cases}
        \text{胸上肢肌}\begin{cases}
            \text{胸大肌}\\
            \text{胸小肌}\\
            \text{前锯肌(翼状肩)}
        \end{cases}\\
        \text{胸固有肌}\begin{cases}
            \text{肋间外肌}\\
            \text{肋间内肌}
        \end{cases}
    \end{cases}
.\] 
\begin{notation}
    肋间肌:

    1. 肋间外肌:提肋,助吸气

    2. 肋间内肌:降肋,助呼气
\end{notation}
\subsubsection{膈肌}%
\label{subsub:膈肌}
结构:

穹隆状,肌性部、中心腱;分隔胸、腹腔

裂孔:

1. 主动脉裂孔(T12,主动脉、胸导管)

2. 食管裂孔(T10,食管、迷走神经)

3. 腔静脉裂孔(T8,下腔静脉)

作用:重要的呼吸肌

收缩时膈顶下降,助吸气

舒张时膈顶上升,助呼气
\subsubsection{腹肌}%
\label{subsub:腹肌}
\[
    \begin{cases}
        \text{前外侧群}\begin{cases}
            \text{腹直肌}\\
            \text{腹横肌}\\
            \text{腹外斜肌}\\
            \text{腹内斜肌}
        \end{cases}\\
        \text{后群:腰方肌}
    \end{cases}
.\] 
\begin{notation}
    腹肌形成的结构

    1. 腹直肌鞘

    2. 白线

    3. 腹股沟管

    4. 腹股沟三角(海氏三角)
\end{notation}
\begin{notation}
    海氏三角:位于腹前壁下部的三角区
    \[
        \begin{cases}
            \text{腹直肌外侧缘}\\
            \text{腹股沟韧带}\\
            \text{腹壁下动脉}
        \end{cases}
    .\] 
\end{notation}
\subsection{上肢肌}%
\label{sub:上肢肌}
\subsubsection{上肢带肌}%
\label{subsub:上肢带肌}
\[
    \begin{cases}
        \text{三角肌}\\
        \text{冈上、下肌}\\
        \text{大、小圆肌}\\
        \ldots
    \end{cases}
.\] 
\begin{notation}
    三角肌:

    起点为一条线,止于肱骨,运动肩关节

    三角肌注射安全区:

    在三角肌区画一个“井”字,第2,5格为注射安全区:肌质厚、无大血管和神经
\end{notation}
\subsubsection{臂肌}%
\label{subsub:臂肌}
\[
    \begin{cases}
        \text{前群:肱二头肌、喙肱肌,肱肌(屈肌)}\\
        \text{后群:肱三头肌(伸肌)}
    \end{cases}
.\] 
\begin{notation}
    手肌
    \[
        \begin{cases}
            \text{外侧群}\\
            \text{内侧群}\\
            \text{中间群}
        \end{cases}
    .\]     
\end{notation}
\subsection{下肢肌}%
\label{sub:下肢肌}
\begin{notation}
    髋肌:

    前群:腰大肌、髂肌

    后群:臀大肌等
\end{notation}
\begin{notation}
    大腿肌:

    前群肌:股四、缝匠肌

    内侧肌:内收肌

    后群肌:半腱肌、半膜肌、股二
\end{notation}
\begin{notation}
    小腿肌:基本同大腿肌
\end{notation}
\begin{notation}
    足肌:略
\end{notation}



