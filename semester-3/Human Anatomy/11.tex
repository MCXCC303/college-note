\lecture{11}{11.21}
\section{心血管系统}%
\label{sec:心血管系统}
\subsection{心血管系统概述}%
\label{sub:心血管系统概述}
组成:
\begin{itemize}
    \item 心脏(动力泵)
    \item 肺循环
    \item 体循环
\end{itemize}
循环途径:
\begin{description}
    \item[动脉血] 鲜红色,富氧,流经体动脉和肺静脉
    \item [静脉血] 暗红色,乏氧,流经体静脉和肺动脉
\end{description}
肺循环:起始于右心室$\to $ 肺动脉$\to $ 肺毛细血管$\to $ 其他静脉$\to $ 肺静脉$\to $ 左心房

左心房血液进入左心室

体循环:起始于左心室$\to $ 主动脉(上下腔静脉)$\to $ 全身动脉$\to $ 毛细血管$\to $ 体静脉$\to $ 右心房

右心房经三尖瓣膜又进入右心室,进入下一次肺循环
\begin{notation}
肺动脉高压:引发肺部纤维化
\end{notation}
\subsubsection*{血管吻合}%
\label{subsub:血管吻合}
最明显的:掌动脉弓(另一个动脉可以代偿供血)

包括:\begin{description}
    \item [侧支] 动脉旁边出现几条小血管,可以代替大血管
    \item [终动脉] 无侧支的动脉,截断即坏死
    \item [功能性终动脉] 较少的侧支(脾脏、肾脏),不足以代偿
    \item [动静脉吻合] 微动脉和微静脉之间有直通血管,肌肉较发达,平时关闭,远端组织不需要大量血液时打开使血液流回
\end{description}
\subsection{心脏}%
\label{sub:心脏}
位置:胸腔(纵隔)内,两肺之间
\begin{notation}
心包裸区:在胸骨测沿1-2cm,左侧4-6肋软骨的位置,心脏没有胸膜覆盖,心内注射使用该区域
\end{notation}
\subsubsection*{心脏的外形}%
\label{subsub:心脏的外形}
\textbf{一尖、一底、两面、三缘、四沟}
\begin{description}
    \item[一尖] 心尖:左前下方
    \item [一底] 心底:左心房和小部分的右心房,右后上方
    \item [两面] 前后面
    \item [三缘] 
    \item [四沟] 冠状沟、前室间沟、后室间沟、房间沟(只有后面能看到)
\end{description}
\begin{notation}
心尖切迹:室间沟交叉处有一个小凹陷
\end{notation}
\subsubsection*{心腔}%
\label{subsub:心腔}
\begin{itemize}
    \item 左右心房
    \item 左右心室
\end{itemize}
右心房:\textbf{四口、一窝、一三角}
\begin{description}
    \item[四口] 三入口(上腔静脉口、下腔静脉口、冠状窦口:心脏自身血液单独回流),一出口(右房室口,经三尖瓣)
    \item [一窝] 卵圆窝,婴幼儿时期为卵圆孔
    \item [一三角] Koch三角
\end{description}
\begin{notation}
    Koch三角:房室结定位
\end{notation}
右心室:\textbf{一嵴、一圆锥、两道、两口、瓣膜}
\begin{description}
    \item[一圆锥] 肺动脉圆锥,在肺动脉口和肺动脉瓣膜下方
    \item [两道] 
    \item [两口] 右房室口和三尖瓣、肺动脉口和肺动脉瓣膜
\end{description}
\begin{notation}
瓣膜:单向阀,由纤维环构建结构
\end{notation}
左心室:\textbf{两道、两口、瓣膜}
\begin{description}
    \item[两道] 主动脉前庭、主动脉窦
    \item [两口] 主动脉口、左房室口
    \item [瓣膜] 二尖瓣、主动脉瓣膜
\end{description}
左心房:
\subsubsection*{心脏构造}%
\label{subsub:心脏构造}
内到外分三层:
\begin{enumerate}
    \item 心内膜
    \item 心肌层
    \item 心外膜
\end{enumerate}
\begin{notation}
心纤维支架:瓣膜的底座,共4个纤维环(二尖瓣、三尖瓣环:大环,肺动脉瓣、主动脉瓣环:小环,笼状)
\end{notation}
\begin{notation}
心瓣膜:单向阀
\begin{description}
    \item[二尖瓣] 心房$\to $ 心室
    \item [三尖瓣] 心室抽血时打开
    \item [动脉瓣] 射血时打开
\end{description}
\end{notation}
\begin{notation}
瓣膜病:

1. 腱索断裂

2. 风湿性心脏病引发心内膜炎导致的硬化,心房室室关闭不全导致射血效率降低,使心脏代偿、巨大化直至心衰

治疗:换瓣手术(人工瓣膜/生物瓣膜+支架,内窥镜)
\end{notation}
\subsubsection*{心传导系统}%
\label{subsub:心传导系统}
\begin{description}
    \item[窦房结] 正常起搏点
    \item [结间束] 兴奋经此传自左右心房和房室结
    \item [房室结] 延迟电冲动向心室传导
    \item [房室束] 刺激心肌收缩
\end{description}
\subsubsection*{心血管}%
\label{subsub:心血管}
\begin{description}
    \item [左冠状动脉] 向下迅速分为前室间支和旋支
    \item [右冠状动脉] 在远端分为右冠脉主干、右室前支和室间隔前支
\end{description}
\begin{notation}
冠脉易发生的病变:

1. 动脉粥样硬化

2. 冠脉痉挛:硝酸甘油治疗

3. 粥样硬化堵塞:心脏搭桥手术或冠状动脉支架(70\%堵塞)

4. 急性心梗:完全阻塞
\end{notation}
心静脉:
\begin{itemize}
    \item 心小静脉:右冠脉的伴行支
    \item 心中静脉:后室间支伴行支
    \item 心大静脉
\end{itemize}
\subsubsection*{心包}%
\label{subsub:心包}
\begin{description}
    \item[纤维心包] 结实、弹性小
    \item [浆膜心包] 脏层和壁层
    \item [心包腔]
\end{description}
\subsubsection*{心的体表投影}%
\label{subsub:心的体表投影}
\begin{notation}
听诊时:听心尖处声音最大
\end{notation}
大部分在左边
\subsubsection*{人工心脏}%
\label{subsub:人工心脏}
在心尖处开孔,将动脉血直接泵至主动脉
