\lecture{1}{}
\section*{课程需求}%
\label{sec:课程需求}
课堂签到:雨课堂

成绩组成:平时(40\%,出勤、课堂表现等)+期末考试(60\%)

教师手机:15902399317

使用教材:系统解剖学第九版

参考教材:人体解剖彩色图谱第三版

学习方法:预习复习、多看、多想、多抹、多讨论、注意探讨解剖学名词的命名和记忆规律
\section{绪论}%
\label{sec:绪论}
解剖学是基础医学、临床医学等学科的基石
\begin{notation}
    Appendectomy:阑尾切除术
\end{notation}
\begin{notation}
    最早研究人体的古希腊医生:盖伦
\end{notation}
\begin{notation}
    著名的解剖学奠基者:

    达芬奇
    
    维萨里(Vesalius,现代解剖学)
    
    哈维(William Harvey,计算心脏泵出的血液量提出血液循环)
    
    列文虎克(观察到毛细血管)
    
    卡米洛高尔基(硝酸银染色法,神经元学说)
\end{notation}
\subsection{解剖学基础的定义}%
\label{sub:解剖学基础的定义}
是研究人体正常形态结构的科学,包括解剖学、组织学、胚胎学

解剖学分科:巨视解剖学(系解,局解等)、微视解剖学(组织学histology、胚胎学embryology、细胞学cytology)、其他

\subsection*{系统解剖学}%
\label{sub:系统解剖学}
人体分为9大系统:运动、消化、呼吸、泌尿、生殖\ldots
\subsection*{局部解剖学}%
\label{sub:局部解剖学}
研究局部的细微结构
\subsection*{断面解剖学}%
\label{sub:断面解剖学}
Sectional Anatomy:以CT/X-ray/MR/NMR等断面扫描为工具研究人体结构
\subsection*{显微解剖学}%
\label{sub:显微解剖学}
\begin{notation}
    世界首例断肢再植:1963,陈中伟
\end{notation}
缝合微小血管、神经

\subsection{人体的组成}%
\label{sub:人体的组成}
细胞$\to $组织$\to $器官$\to $ 系统$\to $ 人体

九大系统:
\[
    \begin{cases}
        \text{运动}\\
        \text{循环}\\
        \text{呼吸}\\
        \text{消化}\\
        \text{泌尿}\\
        \text{生殖}\\
        \text{神经}\\
        \text{内分泌}\\
        \text{感觉系统}\\
    \end{cases}
.\] 
\subsubsection*{运动系统}%
\label{subsub:运动系统}
由骨骼、骨连结、骨骼肌组成
\subsubsection*{消化系统}%
\label{subsub:消化系统}
消化道由消化腺、上消化道和下消化道组成
\subsubsection*{呼吸系统}%
\label{subsub:呼吸系统}
由呼吸道和肺组成
\subsubsection*{泌尿系统}%
\label{subsub:泌尿系统}
由肾、输尿管、膀胱和尿道组成
\subsubsection*{生殖系统}%
\label{subsub:生殖系统}
分为男性和女性生殖系统
\subsubsection*{脉管系统}%
\label{subsub:脉管系统}
由心血管系统和淋巴系统组成
\begin{notation}
    淋巴管末端为盲端,传输组织液
\end{notation}
\subsubsection*{感觉系统}%
\label{subsub:感觉系统}
眼睛、耳朵等
\subsubsection*{神经系统}%
\label{subsub:神经系统}
分为中枢神经系统和周围神经系统

中枢神经:大脑、脊髓

周围神经:脑神经、脊髓神经
\begin{notation}
    脑科学计划:对人类大脑意识工作原理的研究
\end{notation}
\subsubsection*{内分泌系统}%
\label{subsub:内分泌系统}
含有内分泌腺等
\subsection{人体分部}%
\label{sub:人体分部}
\subsection{解剖学姿势}%
\label{sub:解剖学姿势}
标准姿势:人体直立,两眼平视,上肢下垂,掌心向前,下肢并拢,足尖向前
\subsubsection{方位术语}%
\label{subsub:方位术语}
1. 上和下(颅侧、尾侧)

2. 前和后(腹侧、背侧)

3. 内侧和外侧(尺侧/胫侧、桡侧/腓侧)

4. 内和外:腔道里为内,腔道外侧为外

5. 浅和深:皮肤

6. 近侧和远侧:距离躯干近处为近端
\begin{notation}
    三个轴:

    矢状轴:前后轴,y

    冠状轴:左右轴,x

    垂直轴:上下轴,z

    三个面:

    矢状面(正中矢状面):矢状轴和垂直轴

    冠状面:冠状轴和垂直轴

    水平面(横断面):矢状轴和冠状轴
\end{notation}
\begin{eg}
    眼睑:上眼睑、下眼睑

    尺桡骨:近端,中近端,中远端,远端

    心脏:心内,心外
\end{eg}
\subsection{总结}%
\label{sub:总结}
九大系统,解剖学姿势
