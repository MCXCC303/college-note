\section*{课程需求}%
\label{sec:课程需求}
课堂签到:雨课堂

成绩组成:平时(40\%,出勤、课堂表现等)+期末考试(60\%)

教师手机:15902399317

使用教材:系统解剖学第九版

参考教材:人体解剖彩色图谱第三版

学习方法:预习复习、多看、多想、多抹、多讨论、注意探讨解剖学名词的命名和记忆规律
\section{绪论}%
\label{sec:绪论}
解剖学是基础医学、临床医学等学科的基石
\begin{notation}
    Appendectomy:阑尾切除术
\end{notation}
\begin{notation}
    最早研究人体的古希腊医生:盖伦
\end{notation}
\begin{notation}
    著名的解剖学奠基者:

    达芬奇
    
    维萨里(Vesalius,现代解剖学)
    
    哈维(William Harvey,计算心脏泵出的血液量提出血液循环)
    
    列文虎克(观察到毛细血管)
    
    卡米洛高尔基(硝酸银染色法,神经元学说)
\end{notation}
\subsection{解剖学基础的定义}%
\label{sub:解剖学基础的定义}
是研究人体正常形态结构的科学,包括解剖学、组织学、胚胎学

解剖学分科:巨视解剖学(系解,局解等)、微视解剖学(组织学histology、胚胎学embryology、细胞学cytology)、其他

\subsection*{系统解剖学}%
\label{sub:系统解剖学}
人体分为9大系统:运动、消化、呼吸、泌尿、生殖\ldots
\subsection*{局部解剖学}%
\label{sub:局部解剖学}
研究局部的细微结构
\subsection*{断面解剖学}%
\label{sub:断面解剖学}
Sectional Anatomy:以CT/X-ray/MR/NMR等断面扫描为工具研究人体结构
\subsection*{显微解剖学}%
\label{sub:显微解剖学}
\begin{notation}
    世界首例断肢再植:1963,陈中伟
\end{notation}
缝合微小血管、神经

\subsection{人体的组成}%
\label{sub:人体的组成}
细胞$\to $组织$\to $器官$\to $ 系统$\to $ 人体

九大系统:
\[
    \begin{cases}
        \mbox{运动}\\
        \mbox{循环}\\
        \mbox{呼吸}\\
        \mbox{消化}\\
        \mbox{泌尿}\\
        \mbox{生殖}\\
        \mbox{神经}\\
        \mbox{内分泌}\\
        \mbox{感觉系统}\\
    \end{cases}
.\] 
\subsubsection*{运动系统}%
\label{subsub:运动系统}
由骨骼、骨连结、骨骼肌组成
\subsubsection*{消化系统}%
\label{subsub:消化系统}
消化道由消化腺、上消化道和下消化道组成
\subsubsection*{呼吸系统}%
\label{subsub:呼吸系统}
由呼吸道和肺组成
\subsubsection*{泌尿系统}%
\label{subsub:泌尿系统}
由肾、输尿管、膀胱和尿道组成
\subsubsection*{生殖系统}%
\label{subsub:生殖系统}
分为男性和女性生殖系统
\subsubsection*{脉管系统}%
\label{subsub:脉管系统}
由心血管系统和淋巴系统组成
\begin{notation}
    淋巴管末端为盲端,传输组织液
\end{notation}
\subsubsection*{感觉系统}%
\label{subsub:感觉系统}
眼睛、耳朵等
\subsubsection*{神经系统}%
\label{subsub:神经系统}
分为中枢神经系统和周围神经系统

中枢神经:大脑、脊髓

周围神经:脑神经、脊髓神经
\begin{notation}
    脑科学计划:对人类大脑意识工作原理的研究
\end{notation}
\subsubsection*{内分泌系统}%
\label{subsub:内分泌系统}
含有内分泌腺等
\subsection{人体分部}%
\label{sub:人体分部}
\subsection{解剖学姿势}%
\label{sub:解剖学姿势}
标准姿势:人体直立,两眼平视,上肢下垂,掌心向前,下肢并拢,足尖向前
\subsubsection{方位术语}%
\label{subsub:方位术语}
1. 上和下(颅侧、尾侧)

2. 前和后(腹侧、背侧)

3. 内侧和外侧(尺侧/胫侧、桡侧/腓侧)

4. 内和外:腔道里为内,腔道外侧为外

5. 浅和深:皮肤

6. 近侧和远侧:距离躯干近处为近端
\begin{notation}
    三个轴:

    矢状轴:前后轴,y

    冠状轴:左右轴,x

    垂直轴:上下轴,z

    三个面:

    矢状面(正中矢状面):矢状轴和垂直轴

    冠状面:冠状轴和垂直轴

    水平面(横断面):矢状轴和冠状轴
\end{notation}
\begin{eg}
    眼睑:上眼睑、下眼睑

    尺桡骨:近端,中近端,中远端,远端

    心脏:心内,心外
\end{eg}
\subsection{总结}%
\label{sub:总结}
九大系统,解剖学姿势

\section{运动系统}%
\label{sec:运动系统}
组成:骨(杠杆)+骨连结(枢纽)+骨骼肌(或骨骼+骨骼肌)

骨骼(skeleton)=骨+骨连结

\begin{notation}
    功能:

    1. 运动

    2. 支撑、保护(大脑、胸腔、盆腔)
\end{notation}
\begin{notation}
    体表标志:能在体表看到或摸到的一些骨性突起和肌性隆起
\end{notation}

\subsection{骨学}%
\label{sub:骨学}
\subsubsection{总论}%
\label{subsub:总论}
人有206块骨头(6块听小骨归入感觉器)

按部位分类:
\[
    \begin{cases}
    \mbox{中轴骨}
    \begin{cases}
        \mbox{颅骨:29块}\\ 
        \mbox{躯干骨:51块}\\ 
    \end{cases}\\
    \mbox{四肢骨:126块}
    \end{cases}
.\] 

按形状分类:

1. 长骨:一体两端,呈管状,分布于四肢,两端膨大称骺,表面光滑称关节面,内有空腔称骨髓腔,容纳骨髓

2. 短骨:立方体,往往结在一起呈拱形,如腕骨

3. 扁骨:板状,分内板和外板,外板附有骨膜

4. 不规则骨:如椎骨、上颌骨

\begin{notation}
    骨的表面形态:

    1. 骨面突起、棘、隆起、粗隆、结节、嵴、线

    2. 骨面凹陷、窝、凹、小凹、沟、压迹

    3. 骨的空腔、窦、房、管、道、口、孔

    4. 骨端膨大、头、小头、颈、髁

    5. 平滑骨面、缘、切迹
\end{notation}

\subsubsection*{骨的构造}%
\label{subsub:骨的构造}
\[
    \mbox{骨}\\ 
    \begin{cases}
        \mbox{骨质}
        \begin{cases}
            \mbox{骨密质}\\ 
            \mbox{骨松质}
        \end{cases}\\
        \mbox{骨膜}\\ 
        \mbox{骨髓}
        \begin{cases}
            \mbox{红骨髓}\\ 
            \mbox{黄骨髓}
        \end{cases}\\
        \mbox{血管、淋巴、神经}
    \end{cases}
.\] 

\begin{notation}    
    骨密质:外部
    
    骨松质:内部
\end{notation}
\begin{notation}
    骨外膜:外层致密,内层疏松,有血管和神经分布

    骨内膜:菲薄结缔组织

    严重骨折时骨膜大量腐坏导致难以愈合
\end{notation}
\begin{notation}
    红骨髓:有造血功能

    黄骨髓:无造血功能,严重失血时转化为红骨髓

    临床上通过骨髓穿刺检查骨髓像
\end{notation}

\begin{notation}
    骨血管:滋养动脉、骺动脉、干骺端动脉、骨膜动脉

    骨淋巴管:主要位于骨膜

    骨神经:伴滋养动脉进入骨内
\end{notation}

\begin{notation}
    白血病的治疗方案:
    
    骨髓移植:

    1. 杀灭患者所有的血细胞

    2. 采集配型成功的供者骨髓的造血细胞/干细胞

    3. 去除干细胞中的恶性细胞、免疫细胞

    4. 将干细胞输给病人

    免疫细胞治疗(DC疗法,已淘汰;CAR-T疗法,主流):
    
    1. 培养出专一功能T细胞

    2. 输回人体,T细胞即可杀死对应癌细胞
\end{notation}
\subsubsection*{骨的化学成分和物理性质}%
\label{subsub:骨的化学成分和物理性质}
1. 有机质:弹性、韧性

2. 无机质:刚性、硬度

\subsubsection*{发生与发育}%
\label{subsub:发生与发育}

\subsubsection*{骨的重塑}%
\label{subsub:骨的重塑}
1. 血肿:激化

2. 成骨细胞附着

3. 骨细胞形成
\subsubsection{躯干骨}%
\label{subsub:躯干骨}
\[
    \mbox{躯干骨:51块}
    \begin{cases}
        \mbox{椎骨:26块}\\ 
        \mbox{胸骨:1块}\\ 
        \mbox{肋骨:12对}\\ 
    \end{cases}
.\] 

1. 椎骨:
\[
    \begin{cases}
        \mbox{颈椎:7块}\\ 
        \mbox{胸椎:12块,有肋骨连接}\\ 
        \mbox{腰椎:5块}\\ 
        \mbox{骶椎:5块}\implies \mbox{骶骨:1块}\\
        \mbox{尾椎:3-4块}\implies \mbox{尾骨:1块}
    \end{cases}
.\] 

\begin{notation}
    椎骨的形态:
    \[
        \mbox{椎骨}
        \begin{cases}
            \mbox{椎体}\\ 
            \mbox{椎弓}
            \begin{cases}
                \mbox{椎弓根:不同椎节之间组成椎间孔}
                \begin{cases}
                    \mbox{椎上切迹}\\ 
                    \mbox{椎下切迹}
                \end{cases}\\
                \mbox{椎弓板:七个突起}
                \begin{cases}
                    \mbox{棘突:1个}\\ 
                    \mbox{横突:1对}\\ 
                    \mbox{上下关节突:各1对}
                \end{cases}
            \end{cases}
        \end{cases}
    .\] 
    椎间孔中有神经穿过
\end{notation}

\begin{notation}
    颈椎:

    1. 椎体较小
    
    2. 有横突孔,椎动脉从中穿过

    3. 第2~6颈椎棘突短而分叉

    4. 第7颈椎棘突长
    \[
        \begin{cases}
            \mbox{第一颈椎:寰椎,无椎体}\\ 
            \mbox{第二颈椎:枢椎,椎体有齿突,齿突与寰椎的前突形成寰枢关节}\\ 
            \mbox{第七颈椎:隆椎,棘突长}\\ 
        \end{cases}
    .\] 
\end{notation}

\begin{notation}
    胸椎:
\end{notation}

\begin{notation}
    腰椎:

    椎体大,棘突宽短,水平向后伸

    棘突间隙宽,有利于腰椎穿刺
\end{notation}

\begin{notation}
    尾骨:
\end{notation}

\begin{notation}
    胸骨:

    柄、体、剑突

    柄与体连接处向前突称为胸骨角,连接第2肋
\end{notation}

\begin{notation}
    肋:

    包含肋骨和肋软骨,软骨终身不骨化

    共12对:
    \[
        \begin{cases}
            \mbox{1-7:真肋}\\ 
            \mbox{8-10:假肋}\\ 
            \mbox{11-12:浮肋}
        \end{cases}
    .\] 

    肋骨后端:肋头、肋颈、肋结节

    肋下端:肋沟、肋体、肋角
\end{notation}

\subsection{颅骨}%
\label{sub:颅骨}
分为:脑颅骨、面颅骨

共23块:脑颅骨8块,面颅骨15块

\subsubsection{脑颅骨}%
\label{subsub:脑颅骨}
\[
    \mbox{共8块}
    \begin{cases}
        \mbox{成对}
        \begin{cases}
            \mbox{颞骨}\\ 
            \mbox{顶骨}
        \end{cases}\\
        \mbox{不成对}
        \begin{cases}
            \mbox{额骨}\\ 
            \mbox{筛骨(嗅神经传到大脑)}\\ 
            \mbox{蝶骨}\\ 
            \mbox{枕骨}
        \end{cases}
    \end{cases}
.\] 
\begin{notation}
    额骨:
\end{notation}

\begin{notation}
    筛骨:含筛板、垂直板、筛骨迷路,较脆弱
\end{notation}

\begin{notation}
    蝶骨:体、大翼、小翼、翼突
\end{notation}

\begin{notation}
    颞骨:
\end{notation}

\begin{notation}
    枕骨:枕骨大孔(脊髓、脑干穿出)、枕髁、枕外隆突

    顶骨:外隆内凹
\end{notation}

\subsubsection{面颅骨}%
\label{subsub:面颅骨}
\[
    \mbox{共15块}
    \begin{cases}
        \mbox{成对}
        \begin{cases}
            \mbox{上颌骨}\\ 
            \mbox{鼻骨}\\ 
            \mbox{颧骨}\\ 
            \mbox{泪骨}\\ 
            \mbox{下鼻甲}\\ 
            \mbox{腭骨}
        \end{cases}\\
        \mbox{不成对}
        \begin{cases}
            \mbox{犁骨}\\ 
            \mbox{下颌骨}\\ 
            \mbox{舌骨}
        \end{cases}
    \end{cases}
.\] 

内部分为颅前窝、中窝、后窝

侧面:颞骨、额骨、顶骨、蝶骨交汇至翼点(太阳穴)
