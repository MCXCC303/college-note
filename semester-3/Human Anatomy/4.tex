\begin{notation}
	肩关节:
	\[
		\text{肩关节}
		\begin{cases}
			\text{构成}\begin{cases}
				\text{肱骨头}\\
				\text{肩胛骨关节盂}
			\end{cases}\\
			\text{特点}\begin{cases}
				\text{头大、关节盂浅小}\\
				\text{关节囊薄而松弛}\\
				\text{下方的关节囊的韧带少而弱}
			\end{cases}
		\end{cases}
	.\] 

	肩关节的运动特点:幅度最大、形式最多、最灵活,“全能关节”

	运动形式包含屈伸、收展、内外旋、环转
\end{notation}
\begin{notation}
	肘关节:由肱尺关节、肱桡关节、桡尺近测关节组成
\end{notation}
\begin{notation}
	手关节:

	1. 桡腕关节/腕关节:由桡骨下端、尺骨下端关节盘、舟月、三角骨组成,可以进行屈伸、收展、环转运动
\end{notation}
\begin{notation}
	常见的骨连结病变:
	
	肩关节脱位(希波克拉底法复位)

	肘关节脱位:后上脱位,三个关节呈尖朝上的等腰三角形提示脱位

	桡骨头脱位:桡骨小头脱出环状韧带
\end{notation}
\subsubsection{下肢骨连结}%
\label{subsub:下肢骨连结}	
\begin{notation}
	骨盆:由髋骨、骶骨、尾骨组成,含骶髂关节、耻骨联合、韧带(骶结节韧带、骶棘韧带)组成

	男女骨盆差异:男性上大下小、女性呈较均匀的圆柱形
\end{notation}
\begin{notation}
	髋关节:
	\[
		\text{髋关节}
		\begin{cases}
			\text{构成}\begin{cases}
				\text{髋臼}\\
				\text{股骨头}
			\end{cases}\\
			\text{特点}\begin{cases}
				\text{大、臼深}\\
				\text{关节囊厚}
			\end{cases}
		\end{cases}
	.\] 
\end{notation}
\begin{notation}
	膝关节:
	\[
		\text{膝关节}\begin{cases}
			\text{构成}\begin{cases}
				\text{股骨内外侧髁}\\
				\text{胫骨内外侧髁}\\
				\text{髌骨}
			\end{cases}\\
			\text{特点}\begin{cases}
				\text{髌韧带、腓侧、胫侧副韧带、膝交叉韧带}\\
				\text{半月板:内C外O}\\
				\text{髌上囊、翼状襞}
			\end{cases}
		\end{cases}
	.\] 
\end{notation}
\begin{notation}
	足关节:拒小腿关节(踝关节)

	由胫腓骨下端、距骨组成

	结构特点:

	1. 关节面前宽后窄

	2. 三角韧带

	3. 外侧韧带:较薄弱,活动度小
\end{notation}
\section{肌学}%
\label{sec:肌学}
\subsection{肌总论}%
\label{sub:肌总论}
肌分为:骨骼肌、心肌、平滑肌

骨骼肌约有640块,占人体体重的40\%

骨骼肌分为:头颈肌、躯干肌、四肢肌
\subsection{肌的形态和构造}%
\label{sub:肌的形态和构造}
\begin{notation}
	骨骼肌包含肌腹和肌腱
\end{notation}
按形态分类:长肌、短肌、扁肌和轮匝肌

长肌$\to $ 四肢

短肌$\to $ 胸、腹壁

阔肌$\to $ 躯干部的深层

轮匝肌$\to $ 孔、裂周围
\subsection{肌的起止、配置和作用}%
\label{sub:肌的起止、配置和作用}
起点:近正中矢状面

止点:相对起点
\begin{notation}
	肌的起止点是相对的,可以相互转换
\end{notation}
拮抗肌:位于运动轴的相对侧,作用相反

协同肌:位于关节运动轴同侧且作用相同的两块或多块肌
\subsection{肌的命名}%
\label{sub:肌的命名}
1. 按位置:肋间内肌、肋间外肌

2. 按形态:斜方肌、三角肌

3. 按位置和形态:肱二头肌

4. 按位置和大小:胸大肌、臀大肌

5. 按起止点:胸锁乳突肌、肩胛舌骨肌

6. 按作用:旋后肌、拇收肌

7. 按位置和肌束走行方向:腹外斜肌、腹横肌
\subsection{肌的辅助装置}%
\label{sub:肌的辅助装置}
\[
	\begin{cases}
		\text{筋膜}\begin{cases}
			\text{浅筋膜:皮下筋膜,脂肪}\\
			\text{深筋膜:肌肉间筋膜,弹性差}
		\end{cases}\\
		\text{滑膜囊}\\
		\text{腱鞘}\begin{cases}
		    \text{纤维层:腱纤维鞘}\\
            \text{滑膜层:腱滑膜鞘}
		\end{cases}
	\end{cases}
.\] 
