\lecture{8}{10.31}
\subsection{胰}%
\label{sub:胰}
位置:胃的后方,紧贴腹后壁,被十二指肠包裹

外分泌部可分泌胰液,经胰管排入十二指肠,腐蚀性强,含大量酶

内分泌部即胰岛,分泌胰高血糖素、胰岛素
\begin{notation}
    胆(管)结石超声:从十二指肠$\to $ 十二指肠乳头$\to $ 胆管$\to $ 胆囊管(狭窄)伸入内窥镜
\end{notation}
\section{呼吸系统}%
\label{sec:呼吸系统}
组成:呼吸道和肺
\[
    \text{呼吸系统}\begin{cases}
        \text{呼吸道}\begin{cases}
            \text{上呼吸道}\begin{cases}
                \text{鼻}\\
                \text{咽}\\
                \text{喉}
            \end{cases}\\
            \text{下呼吸道}\begin{cases}
                \text{气管}\\
                \text{支气管}
            \end{cases}
        \end{cases}\\
        \text{肺}\begin{cases}
            \text{肺实质}\\
            \text{肺间质}
        \end{cases}
    \end{cases}
.\] 

功能:进行气体交换、嗅觉(通过筛板传递到大脑)、发声(喉腔:声带)
\subsection{鼻}%
\label{sub:鼻}
组成:外鼻、鼻腔、鼻旁窦

功能:嗅觉、通气
\subsubsection*{外鼻}%
\label{subsub:外鼻}
含鼻尖、鼻翼、鼻背、鼻根

鼻翼、臂外侧、鼻中隔为软骨
\subsubsection*{鼻腔}%
\label{subsub:鼻腔}
由骨和软骨组成,被鼻中隔分为两半
\begin{notation}
    鼻中隔偏曲可能造成某一边鼻子堵
\end{notation}
鼻腔分为鼻前庭和固有鼻腔(与口腔类似),鼻阀分隔

固有鼻腔含大量鼻腔黏膜,上部分为嗅区,中间为呼吸区,鼻阀处易出血
\subsubsection*{鼻旁窦}%
\label{subsub:鼻旁窦}
\[
    \begin{cases}
        \text{上颌窦}\\
        \text{颌窦}\\
        \text{筛窦}\\
        \text{蝶窦}
    \end{cases}
.\] 
\begin{notation}
    鼻窦的各个开口:

    筛窦的后组开口于上鼻道,蝶窦开口于蝶筛隐窝,其他开口于中鼻道
\end{notation}
\subsection{咽}%
\label{sub:鼻咽}
\[
    \begin{cases}
        \text{口咽}\\
        \text{鼻咽}\\
        \text{舌咽}
    \end{cases}
.\] 
\subsection{喉}%
\label{sub:喉}
\[
    \begin{cases}
        \text{喉软骨}\\
        \text{喉的连接}\\
        \text{喉肌}\\
        \text{喉腔}
    \end{cases}
.\] 
功能:通气、发声
\subsubsection*{喉软骨}%
\label{subsub:喉软骨}
\[
    \begin{cases}
        \text{甲状软骨:喉结处,通过韧带与舌骨连接}\\
        \text{环状软骨}\begin{cases}
            \text{软骨弓}\\
            \text{软骨板}
        \end{cases}\\
        \text{会厌软骨:最上方,盖住喉部使食物进入食道}\\
        \text{杓状软骨:成对,旋内/外}\begin{cases}
            \text{声带突}\\
            \text{肌突}
        \end{cases}
    \end{cases}
.\] 
\subsubsection*{喉连接}%
\label{subsub:喉连接}
\[
    \begin{cases}
        \text{甲状舌骨膜:连接甲状软骨和舌骨}\\
        \text{环甲关节}\\
        \text{环杓关节}\\
        \text{弹性圆锥:连接杓状软骨、杓状软骨和甲状软骨}
    \end{cases}
.\] 
\begin{notation}
    弹性圆锥:上缘游离部分为声韧带/声带,中间较厚为环甲正中韧带

    喉腔最狭窄的部分,喉窒息(喉水肿)时可以在甲状软骨和舌状软骨之间的空隙插入粗针头临时呼吸
\end{notation}
\subsubsection*{*喉肌}%
\label{subsub:-喉肌}
附着在喉软骨上的肌,共三组:
\subsubsection*{喉腔}%
\label{subsub:喉腔}
\begin{notation}
    前庭裂处最狭窄
\end{notation}
\subsubsection*{气管、支气管}%
\label{subsub:气管-支气管}
最大的叫气管,第一层为主支气管、两根(左侧较细长,偏移较多)
\begin{table}[htpb]
    \centering
    \caption{左右支气管区别}
    \label{tab:左右支气管区别}
    \begin{tabular}{ccc}
    \toprule
    特点 & 左 & 右\\
    \midrule
    管径 & 细 & 粗\\
    长度 & 长 & 短\\
    走向 & 平 & 陡直\\
    \multicolumn{3}{c}{异物易调入右主支气管} \\
    \bottomrule
    \end{tabular}
\end{table}
\begin{notation}    
    气管的软骨呈C型

    由外到内:外膜、软骨、平滑肌(只有一侧)、粘膜下层、黏膜层
\end{notation}
\begin{notation}
    哮喘:呼吸道水肿
\end{notation}
\subsection{肺}%
\label{sub:肺}
形态:

外层圆润,膈面较平整
\begin{notation}
    一尖(肺尖)、一底(肺底)、三面(肋面、膈面、纵膈面)、三缘(较尖锐的部位:前缘、后缘、下缘)
\end{notation}
\begin{notation}
    肺的分叶:

    靠心脏的左肺(仅斜裂)少一叶,且缺一块:心切迹

    右肺(水平裂、斜裂)分三叶
\end{notation}
\subsubsection*{肺门与肺根}%
\label{subsub:肺门与肺根}
肺根:血管、淋巴、支气管
\begin{notation}
    胎儿肺与成人肺的区别:胎儿肺无空气,比重较大

    成人肺有空气,可以浮于水面
\end{notation}
\subsubsection*{支气管树}%
\label{subsub:支气管树}
共分23-25级

肺段支气管:3级支气管
\subsubsection*{肺段}%
\label{subsub:肺段}
按照支气管的级数分隔,左右肺各为10段(类似肝脏,便于手术切除)
\subsubsection*{胸膜}%
\label{subsub:胸膜}
光滑的浆膜,一层覆盖于肺表面(脏胸膜),另一层覆盖于纵隔表面(壁胸膜)

可以减少摩擦
\begin{notation}
    气胸:外伤或内发损伤胸膜和胸壁,负压使胸内空间迅速减小

    自发性:脏层破裂;外伤性:壁层破裂
\end{notation}
\subsubsection*{胸膜腔}%
\label{subsub:胸膜腔}
特点:腔内为负压,有少量浆液
\begin{notation}
    肋膈隐窝:肺最低的部位,很尖锐,临床易造成胸膜积液

    胸膜积液治疗:胸膜腔穿刺
\end{notation}
\subsubsection*{纵隔}%
\label{subsub:纵隔}
食管、气管、主动脉经过的区域
