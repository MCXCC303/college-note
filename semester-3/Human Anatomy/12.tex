\lecture{12}{11.28}
\textit{Review:}

$\circ$ 心脏的结构:左右心房、心室,右心房有4个开口(上下腔静脉,冠状窦、右房室口),心室一个出口(肺动脉,主动脉),左心室有四个入口(四个肺静脉)

$\circ$ 冲动传导:心房先收缩,收缩有一个间隔使血液传到心室,心室再收缩;起搏点为窦房结,传到房室结延迟一段时间,等血液充盈后传到心室
\subsection{动脉、静脉}%
\label{sub:动脉、静脉}
部分分类:
\begin{itemize}
    \item 体位置
    \begin{itemize}
        \item 壁支
        \item 脏支
    \end{itemize}   
    \item 深浅
    \begin{itemize}
        \item 躯侧
        \item 深部
    \end{itemize}
\end{itemize}
分布特点:
\begin{enumerate}
    \item 分布形式与器官形态有关
    \item 以最短距离到达组织和器官
\end{enumerate}
结构特点:
\begin{description}
    \item[动脉] 壁厚、弹性大
    \item [静脉] 壁薄、弹性小、易塌陷、呈不规则椭圆状
\end{description}
\subsubsection*{静脉}%
\label{subsub:静脉}
特点:\begin{itemize}
    \item 有瓣(静脉瓣,防止倒流)
    \item 腔体大(容纳70\%人体循环血液,静脉=容量血管)
    \item 流速慢、压力低
    \item 向心回流
    \item 壁薄
    \item 有肌肉收缩挤压血液回流
\end{itemize}
分布特点:
\begin{itemize}
    \item 体循环分深浅静脉两套,表浅静脉较粗、可触可见;深静脉和动脉同行
    \item 迷宫式吻合(网、丛、弓等)
    \item 板障静脉:与骨密质连接
    \item 硬脑膜窦:没有平滑肌,可以取到脑脊液
\end{itemize}
\begin{notation}
静脉曲张:静脉瓣膜关闭不全
\end{notation}
\subsubsection*{肺循环的动静脉}%
\label{subsub:肺循环的动静脉}
标记颜色和体动静脉相反,肺动脉流动脉血
\begin{notation}
    动脉韧带:动脉导管关闭时形成的结缔组织,先天性心脏病多在此形成(六个月以上不关闭等)
\end{notation}
