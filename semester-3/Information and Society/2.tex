\begin{notation}
    信息安全的两大方面:攻击和防守
\end{notation}
\begin{notation}
    主动入侵检测:检测主体活动是否符合入侵活动的特征,效率较高

    被动入侵检测:假设入侵者活动异常于正常主体的活动,效率较低
\end{notation}
\begin{defi}
    计算机病毒:编写者在程序中插入的破坏计算机功能或者数据的代码,能影响计算机使用、自我复制的一段计算机指令或程序代码

    计算机病毒具有传播性,隐蔽性,感染性,潜伏性,可激发性,表现性/破坏性

    生命周期:

    开发$\implies$传播$\implies$ 潜伏$\implies$ 发作$\implies$ 发现$\implies$ 消化$\implies$ 消亡
\end{defi}
其他常用技术:

安全扫描

系统安全

安全风险评估

信息安全管理
\subsection{信息安全与社会的关系}%
\label{sub:信息安全与社会的关系}
\begin{defi}
    社会即“关系”:在特定环境下形成的个体间的存在关系的总和
\end{defi}
使用信息技术属于社会,部分使用行为导致了安全问题
\begin{eg}
    云计算、云存储:非个人控制信息安全

    人肉搜索:威胁日常生活安全

    数据集中:风险集中

    系统复杂:难以解决系统安全
\end{eg}

