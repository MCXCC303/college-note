\lecture{7}{11.05}
\textit{Review:}

$\circ$ NFC是一种近距离无线通讯技术,特点为:点对点,安全

$\circ$ IPS:室内定位系统,相对GPS:全球定位系统
\section{人工智能与数据挖掘}%
\label{sec:人工智能与数据挖掘}
\begin{defi}
    人工智能:构造智能机器或智能系统,是使用机器模拟、延伸、扩展人的智能的技术
\end{defi}
人工智能在棋类游戏和专家系统方面得到了广泛应用
\begin{itemize}
    \item 1956年夏季Minsky和McCarthy提出人工智能学科
    \item 1956年Samuel研究出跳棋程序
    \item 1958年机器证明出现
    \item Selfride提出模式识别程序
    \item 1965年Robert编写积木构造程序
    \item 1968年DENDRAL专家系统出现
    \item 1972年专用于人工智能语言PROLOG出现
    \item 1972年MYCIN专家系统出现
    \item 1977年首次提出“知识工程”的概念
    \item 1981年日本宣布开发第五代计算机
    \item 1997年5月IBM“深蓝”击败国际象棋大师
    \item 2016年3月15日AlphaGo V18击败李世石九段,后来的AlphaGo Master击败八冠王柯洁九段
\end{itemize}
人工智能应用领域:
\[
    \begin{cases}
        \text{早期}\begin{cases}
            \text{专家系统}\\
            \text{机器学习}\\
            \text{模式识别}\\
            \text{自然语言理解}\\
            \text{自动定理证明}\\
            \text{自动程序设计}\\
            \text{机器博弈}\\
            \ldots
        \end{cases}\\
        \text{近期}\begin{cases}
            \text{深度学习}\\
            \text{强化学习}\\
            \text{数据挖掘}\\
            \ldots
        \end{cases}
    \end{cases}
.\] 
\subsection{数据挖掘}%
\label{sub:数据挖掘}
\begin{notation}
    早期人工智能发展存在问题:交互问题、扩展问题
    \begin{description}
        \item[交互问题] 只能按原先设计的状态进行
        \item [扩展问题] 只适用于建造狭窄领域的专家系统
    \end{description}
\end{notation}
\begin{notation}
    数据挖掘的过程:
    \begin{enumerate}
        \item 数据抽取:统计学
        \item 知识形成
    \end{enumerate}
\end{notation}
\section*{分享课专题}%
\label{sec:分享课专题}

