\lecture{12}{10.25}
\subsubsection{药品质量标准制定}%
\label{subsub:药品质量标准制定}
\[
    \begin{cases}
        \text{制订基础}\begin{cases}
            \text{文献查询和整理}\\
            \text{有关研究资料}
        \end{cases}\\
        \text{制定原则}\begin{cases}
            \text{安全有效}\\
            \text{技术先进}\\
            \text{经济合理}\\
            \text{不断完善}
        \end{cases}\\
        \text{主要内容}\begin{cases}
            \text{专属性、灵敏性、耐用性}\\
            \text{便于推广}\\
            \text{尽可能采用药典收载的方法}
        \end{cases}\\
    \end{cases}
.\] 
\begin{notation}
    检查:

    杂质限量检查:专属性、耐用性、检测限

    杂质定量检查:准确度、精密度、专属性、定量限、线性、范围、耐用性

    确定杂质检查及限度:针对性、合理性
\end{notation}
\begin{notation}
    含量测定:

    常见测定方法:容量分析、重量分析、光谱、色谱(GC、TLC、HPLC)等
\end{notation}
\begin{notation}
    含量限度测定:

    1. 根据不同剂型

    2. 根据生产水平

    3. 根据主药含量多少
\end{notation}
\begin{eg}
    1. VB1:原料药$\ge 99.0\%$,片剂$90.0\sim 110.0\%$ ,注射液$93.0\sim 107.0\%$ 

    2.1. 积雪草:不易提纯,原料药以积雪草总甘计量$\ge 60.0\%$ 

    2.2. 盐酸罂粟碱:提纯工艺好,含量标准$\ge 99.0\%$,注射液$95.0\sim 105.0\%$

    3.1. 阿司匹林片:分布均匀,要求高:$95.0\sim 105.0\%$ 

    3.2. 炔雌醇:分布不均匀 (5 $\mu$g/L),要求$80.0\sim 120.0\%$
\end{eg}
\subsubsection{药品稳定性研究}%
\label{subsub:药品稳定性研究}
\[
    \text{稳定性试验}
    \begin{cases}
        \text{目的}\begin{cases}
            \text{考察贮藏情况}\\
            \text{建立有效期}
        \end{cases}\\
        \text{方法}\begin{cases}
            \text{影响因素试验:1批原料药}\\
            \text{加速试验:3批供试品}\\
            \text{长期试验}
        \end{cases}
    \end{cases}
.\] 
\begin{notation}
    影响因素试验:高温、高湿、光照

    加速试验:市售包装下于$40^{\circ}C\pm 2^{\circ}C$,相对湿度$75\%\pm 5\%$放置6个月

    长期试验:市售包装下于$25^{\circ}C\pm 2^{\circ}C$,相对湿度$60\%\pm 10\%$放置12个月
\end{notation}
\begin{notation}
    药品质量标准的长期性

     $\circ$ 质量标准伴随产品终身

     $\circ$ 不断发展和提高
\end{notation}
\section{药剂学}%
\label{sec:药剂学}
\[
    \text{药剂学}
    \begin{cases}
        \text{基本概念}\\
        \text{\textbf{剂型分类}}\\
        \text{药物递送}\\
        \text{发展与任务}\\
        \text{分支学科}\\
        \text{制备工艺}
    \end{cases}
.\] 
\subsection{基本概念}%
\label{sub:基本概念}
\begin{defi}
    药剂学(Pharmaceutics):研究\textit{药物制剂}的基本理论、处方设计、制备工艺和合理用药的综合性技术科学
\end{defi}
\subsection{重要性}%
\label{sub:重要性}
\[
    \text{不同制剂}
    \begin{cases}
        \text{改变药物的作用性质}\\
        \text{改变药物的作用速度}\\
        \text{降低或消除药物的不良反应}\\
        \text{靶向作用}\\
        \text{影响疗效}\\
        \text{提高药物稳定性}\\
        \text{改善患者的用药依从性}
    \end{cases}
.\] 
\begin{notation}
    按形态分类

    1. 液体

    2. 固体

    3. 半固体
\end{notation}
\begin{notation}
    按分散系分类

    1. 真溶液型

    2. 胶体溶液型

    3. 乳剂型

    4. 混悬型

    5. 气体分散型

    6. 固体分散型
\end{notation}
\begin{notation}
    按给药途径分类

    1. 经胃肠道给药

    2. 非经胃肠道给药
\end{notation}
\subsection{递送系统}%
\label{sub:递送系统}
\begin{defi}
    DDS:将必要的药物在必要的时间递送至必要位置的技术
\end{defi}
分类:
\[
    \begin{cases}
        \text{缓控释}\\
        \text{经皮药物}\\
        \text{靶向药物}\\
        \text{智能型药物}\\
        \text{生物大分子}
    \end{cases}
.\] 
\subsection{药剂学发展与任务}%
\label{sub:药剂学发展与任务}
发展:

传统剂型$\Rightarrow $ 缓控释剂型、肠溶剂型$\Rightarrow $ 靶向给药(Paul Ehrlich)$\Rightarrow $ 自调式给药
\begin{notation}
    被动靶向:被吞噬

    主动靶向:修饰为导弹

    物理化学靶向:利用靶-特殊物化性质区
\end{notation}
基本任务:制备安全、有效、稳定、使用方便的药物制剂
\[
    \text{药剂学的任务}
    \begin{cases}
        \text{基本理论}\\
        \text{基本药物剂型}\\
        \text{新技术和新剂型}\\
        \text{新型药用辅料}\\
        \text{中药新剂型}\\
        \text{生物技术药物制剂}\\
        \text{制剂机械和设备}
    \end{cases}
.\] 
\subsection{药剂学分支学科}%
\label{sub:药剂学分支学科}
 \[
    \begin{cases}
        \text{生物药剂学}\\
        \text{工业药剂学:大量制备}\\
        \text{药用高分子材料学:辅料}\\
        \text{药物动力学:含量-时间过程}\begin{cases}
            \text{时辰药物动力学}\\
            \text{手性药物动力学}\\
            \text{群体药物动力学}
        \end{cases}\\
        \text{临床药学:制剂临床评价、监控}
    \end{cases}
.\] 
\subsection{制备工艺}%
\label{sub:制备工艺}
1. 片剂

2. 注射剂:无菌

3.1. 缓释片

3.2. 结肠靶向

3.3. 胃漂浮片剂

4. 经皮给药:离子导入、超声波

5. 靶向给药体系
