\lecture{19}{12.11}
\subsection{药剂学}%
\label{sub:复习:药剂学}
\begin{notation}
    药剂学的研究任务:\textbf{制备安全、有效、稳定、使用方便的药物制剂}
\end{notation}
\begin{notation}
    药物剂型分类:
    \begin{description}
        \item[按形态分类] 液体、固体、半固体
        \item [按分散系分类] 真溶液、胶体溶液、乳剂、混悬、气体分散、固体分散
        \item [按给药途径分类] 经胃肠道、不经胃肠道
    \end{description}
\end{notation}
\begin{notation}
    药物递送系统的分类:\textbf{缓控释、经皮药物、靶向药物、智能型药物、生物大分子}
\end{notation}
\subsection{生物信息学}%
\label{sub:复习:生物信息学}
\begin{notation}
    现代生物技术类型:\textbf{基因组学、生物信息学等}
\end{notation} 
\begin{notation}
    基因工程在医药科学中的应用:\textbf{大量生产难以获取的蛋白和多肽、提供足够的生理活性物质、挖掘内源性生理活性物质、改造或消除内源性生理活性物质的不足、获得新型化合物并扩大药物筛选来源}
\end{notation}
\begin{notation}
    合成生物学的主要研究内容:\textbf{全合成DNA、人工创建全新的生物系统、利用现有的天然生物模块构建新的调控网络}

    合成生物学在天然产物药物开发中的应用:
\end{notation}
