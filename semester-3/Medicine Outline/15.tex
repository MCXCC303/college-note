\lecture{15}{11.30}
\section{复习}%
\label{sec:复习}
\subsection*{大纲}%
\label{sub:大纲}
\begin{itemize}
    \item 药学
    \begin{itemize}
        \item 药学的任务
        \item 药学教育的四大专业基础课
    \end{itemize}
    \item 药物化学
    \begin{itemize}
        \item 药物化学在创新药物研究与开发中的作用。
        \item 新化学实体
        \item 新药研究的特点
        \item 创新药物研究常见的靶标
        \item 构效关系
        \item 先导化合物及其主要发现途径
    \end{itemize}
    \item 中药学
    \begin{itemize}
        \item 中药炮制的目的
        \item 中药饮片
        \item 生药的鉴定
        \item 生药学的研究内容和任务
        \item 中药的四气五味
        \item 天然药物化学研究的主要内容
        \item 天然化合物的结构分类
    \end{itemize}
    \item 药物作用
    \begin{itemize}
        \item 药物的双重性
        \item 治疗作用及其分类
        \item 常见药物作用不良反应类型
        \item 受体的特点及药物分类
        \item 肝肠循环
        \item 首过效应
        \item 药物毒性评价“三致实验”
    \end{itemize}
    \item 药物分析学
    \begin{itemize}
        \item 药物分析学的任务
        \item 中国药典及其内容
        \item 药品稳定性试验方法
        \item 药品质量标准制定的原则
    \end{itemize}
    \item 药剂学
    \begin{itemize}
        \item 药剂学研究任务
        \item 药物剂型分类
        \item 药物递送系统的分类
    \end{itemize}
    \item 生物信息学
    \begin{itemize}
        \item 现代生物技术类型
        \item 基因工程在医药科学中的应用
        \item 合成生物学的主要研究内容及其在天然产物药物开发中的应用
    \end{itemize}
    \item 药事管理学
    \begin{itemize}
        \item 管理的基本方法
        \item 管理的基本原理
        \item 药品的特殊性
        \item 药物质量管理规范
        \item 人工智能与药物研发
    \end{itemize}
\end{itemize}
\subsection{药学}%
\label{sub:复习:药学}
\begin{notation}
    药学的任务:\textbf{研究新药;阐明药物的作用机理;研究新的制剂;制定药品的质量标准、控制药品质量;开拓医药市场、规范药品管理}
\end{notation}
\begin{notation}
    药学教育的四大专业基础课:\textbf{药理、药分、药化、药剂}
\end{notation}
\subsection{药物化学}%
\label{sub:复习:药物化学}
\begin{notation}
    药化在创新药物研究与开发中的作用:\textbf{研究和开发新药}
\end{notation}
