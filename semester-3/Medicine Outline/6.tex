药物研发基本流程:

药物发现阶段$\to $ 临床前研究$ \to $临床研究$\to $ 注册、营销

\begin{defi}
    靶点:药物的作用位置
\end{defi}
\[
    \text{靶分子}\begin{cases}
        \text{受体:52\%}\\
        \text{酶:22\%}\\
        \text{离子通道:6\%}\\
        \text{核酸:3\%}
    \end{cases}
.\] 
\begin{defi}
    受体(Receptor)$\leftrightarrow$ 配体(Ligand)

    受体:能与细胞外专一信号分子(配体)结合引起细胞反应的蛋白质

    配体:在受体介导的内吞中,与细胞质膜受体蛋白结合,最后被吞入细胞的物质
\end{defi}
\begin{defi}
    酶(enzyme):催化特定化学反应的蛋白质、RNA或其复合体

    酶抑制剂(enzyme inhibitor):特异性作用于酶的某些基团,降低酶的活性或使酶完全丧失活性的物质
\end{defi}
\begin{notation}
    酶能通过降低化学反应的活化能,但不改变平衡点

    酶的作用条件温和、专一性强、催化效率高
\end{notation}
\begin{defi}
    抗原(antigen):一类能刺激机体免疫系统产生特异性免疫应答并能与相应免疫应答产物在体内外发生特异性结合的物质

    抗体(antibody):机体由于抗原的刺激而产生的具有保护作用的蛋白质
\end{defi}
\begin{defi}
    苗头化合物(hit)对特定靶标或作用环节具有初步活性的化合物
\end{defi}
\begin{notation}
    苗头化合物的标准:

    1. 与靶标的结合强度不低于10 $\mu$mol/L

    2. 有一定的水溶性,溶解度不低于10 $\mu$g/mL

    3. 可穿越细胞膜

    4. 细胞水平上显示生物活性

    5. 无细胞毒性作用

    6. 有化学稳定性

    7. 可以制备获得

    8. 具有知识产权的保护
\end{notation}
\begin{defi}
    先导化合物(lead compound):具有一定的生物化学并可以进行修饰的化合物
\end{defi}
\begin{defi}
    先导化合物优化(optimization):使化合物具有一系列药物特征的过程
\end{defi}
\begin{defi}
    药物候选物(drug candidate):具有了药物的特征,拟进行系统的临床前试验并进入临床研究的活性化合物
\end{defi}
\begin{notation}
    药物候选物的特征:

    1. 在动物上可以治疗某些人类疾病的适应症

    2. 药物结合方式明确

    3. 有合适的用药方式

    4. 存在最大生物利用度

    5. 分子可控、毒性耐受
\end{notation}
\begin{defi}
    构效关系:药物的化学结构与药理作用之间的特定关系,是药物化学最基本的核心理论
\end{defi}
\subsubsection{研究内容}%
\label{subsub:研究内容}
1. 基于生物学科研究揭示的潜在药物作用靶点,参考内源性配体或已知活性物质的结构特征,设计新的活性化合物分子

2. 化学药物的制备原理、合成路线及其稳定性

3. 研究化学药物和生物体相互作用的方式,吸收、分布和代谢的规律和代谢产物

4. 研究化学药物的构效关系、构代关系、构毒关系

5. 寻找发现新药
\begin{eg}
    立普妥(Lipitor, atorvastatin calcium)
\end{eg}
\begin{notation}
    血脂(Blood-lipid):血浆或血清中含的脂质,以及与载脂蛋白所形成的各种可溶性脂蛋白

    脂质包含胆固醇、胆固醇酯、甘油三酯和磷脂
\end{notation}
\begin{notation}
    脂蛋白(Lipoproteins):含有乳糜颗粒、极低密度脂蛋白VLDL、低密度脂蛋白LDL、高密度脂蛋白HDL

    高浓度的胆固醇、低密度脂蛋白、载脂蛋白会促进动脉粥状硬化,高密度脂蛋白、HDL-胆固醇或载脂蛋白A低于相应浓度也会导致相关疾病
\end{notation}
\begin{notation}
    高血脂的标准:血浆总胆固醇>5.7mmol/L、甘油三酯>1.7mmol/L

    高脂血症是心脑血管病的主要病理基础

    如何降低胆固醇含量:抑制3-羟基-3-甲基-戊二酰辅酶A还原酶的活性,进而抑制胆固醇的合成

    1973年,Dr. Akira Endo发现美伐他汀(Mevastatin)可以降低胆固醇(第一个他汀类药物)
\end{notation}


