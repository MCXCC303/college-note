\lecture{13}{10.30}
\section{生物技术、生物工程和应用}%
\label{sec:生物技术、生物工程和应用}
\[
    \text{主要内容}
    \begin{cases}
        \text{生物技术发展与现状}\\
        \text{现代生物工程简介}
    \end{cases}
.\] 
\subsection{生物技术发展与现状}%
\label{sub:生物技术发展与现状}
\[
    \text{影响国计民生的四大学科}
    \begin{cases}
        \text{生物技术}\begin{cases}
            \text{自然更新}\\
            \text{组织培养}\\
            \text{现代生物技术}\begin{cases}
                \text{基因工程}\\
                \text{酶工程}\\
                \ldots 
            \end{cases}\\
            \ldots 
        \end{cases}\\
        \text{微电子}\\
        \text{新材料}\\
        \text{新能源}
    \end{cases}
.\] 
\begin{defi}
    生物技术:以生命科学为基础,利用生物体的特性和功能,设计具有预期性状的新物种或新品系,与工程相结合,加工生产商品的综合性技术
\end{defi}
\subsubsection*{生物技术发展历史}%
\label{subsub:生物技术发展历史}
\begin{notation}
    19世纪自然科学三大发现:进化论、细胞学说、能量守恒定律
\end{notation}
孟德尔:遗传学基本规律

1. 基因分离定律

2. 基因自由组合定律

3. 基因连锁和交换定律
\begin{notation}
    摩尔根:遗传因子/基因的提出,染色体遗传理论
\end{notation}
艾弗里:肺炎双球菌转化实验
\begin{notation}
    艾弗里:DNA为遗传物质,建立现代分子遗传学
\end{notation}
沃森、克里克:DNA双螺旋结构,生物学进入分子阶段
\begin{notation}
    中心法则: DNA $\ce{->[\text{转录}]}$ RNA $\ce{->[\text{翻译}]} $ 蛋白质
\end{notation}
1966年破译核苷酸与氨基酸的对应关系:64种密码子
\subsubsection*{生物技术现状}%
\label{subsub:生物技术现状}
\begin{notation}
    基因组学:结构基因组学+功能基因组学/后基因组

    涉及基因作图、测序和基因组功能分析
\end{notation}
\begin{notation}
    生物信息学:研究生物信息的采集、处理、存储、传播、分析和解释等各方面的学科

    由生命科学和计算机科学结合

    研究内容:序列对比、结构比对和预测、药物设计等
\end{notation}
代表:\textit{Nuclear Acid Research}:核酸研究
\subsubsection*{生命科学和生物制药前沿}%
\label{subsub:生命科学和生物制药前沿}
1. 合成生物学
\begin{notation}
    2000年完成人类基因组计划
\end{notation}
主要研究内容:

$\circ$ 利用现有天然生物模块构建新的调控网络

$\circ$ 全合成DNA

$\circ$ 人工创建全新生物系统
\begin{notation}
    \textbf{天然产物的挑战}:

    1. 产量极低

    2. 有限来源

    3. 分子结构复杂

    4. 难以发现新的分子骨架

    5. 大规模合成较困难
\end{notation}
合成生物学可以从根本上改变这些挑战

2. 生物催化(绿色制造)

最开始利用酶/活细胞进行简单合成,后来用酶定向合成
\subsection{现代生物工程}%
\label{sub:现代生物工程}
\subsubsection*{基因工程}%
\label{subsub:基因工程}
\begin{defi}
    基因工程:将所要重组的对象的目的基因插入载体、拼接转入新的宿主细胞,构建为\textbf{工程菌},是目的基因在工程菌内进行复制和表达
\end{defi}
上游阶段:基因分离、工程菌构建(实验室内)

下游阶段:大规模培养生产
\begin{notation}
    现代生物技术的核心是\textbf{基因工程}
\end{notation}
\begin{eg}
    人生长激素(治疗侏儒症):

    基因工程1~2L细菌培养液提取量=50具新鲜尸体脑下垂体提取
\end{eg}
基因工程的诞生:

1. Sanger:胰岛素的氨基酸序列,噬菌体的一级结构,因此两次获得诺贝尔化学奖

2. Gilbert:DNA测序

3. 保罗$\cdot $ 伯格:重组DNA技术之父

4. 穆利斯:PCR技术,获诺贝尔化学奖
\begin{notation}
    主要步骤:

    获得目的基因 $\to $ 体外重组DNA $\to $ 转移到受体$\to $ 筛选重组DNA分子,受体细胞克隆$\to $ 提取扩增目的基因$\to $ 克隆到表达载体,导入宿主细胞

    主要工具与技术:

    1. 工具酶:核酸内切限制酶、DNA连接酶

    2. 基因克隆载体

    3. 获取DNA
\end{notation}
\begin{notation}
    \textbf{在医药科学中的应用}:

    1. 大量生产以前难以获取的蛋白和多肽

    2. 提供足够的生理活性物质

    3. 挖掘内源性生理活性物质

    4. 改造或消除内源性生理活性物质的不足

    5. 获得新型化合物,扩大药物筛选来源
\end{notation}
\subsubsection*{酶工程}%
\label{subsub:酶工程}
\begin{defi}
    酶学和工程学结合,从应用出发研究应用酶的特异性催化
\end{defi}
酶的来源:动物、微生物和化学合成

大部分由微生物而来:种类多、生长繁殖快、产量高、适应性强、可控

对菌种的要求:产量高、非致病、胞外酶、廉价、稳定等

目前常用的产酶微生物:大肠杆菌、枯草杆菌、曲霉、青霉菌等
\subsubsection*{细胞工程}%
\label{subsub:细胞工程}
\begin{defi}
    以细胞为对象,改变细胞的某些形状,培育新的品种或获得珍贵生物产品
\end{defi}
可以操作原核细胞和真核细胞,包括体外培养、组织培养、细胞融合、细胞器移植、胚胎移植和基因转移等
\begin{notation}
    基本操作:

    1. 无菌操作

    2. 细胞培养

    3. 细胞融合

    4. 淋巴细胞杂交瘤和单克隆抗体技术

    5. 干细胞
\end{notation}
\subsubsection*{发酵工程}%
\label{subsub:发酵工程}
又称微生物工程

利用微生物制造工业原料与产品的技术,是一个十分复杂的自催化过程

分为好氧发酵和厌氧发酵
\begin{notation}
    发酵工程的特点:自发调节,多个反应一次完成,条件温和,耗能少,设备简单,易生产高分子化合物,设备需灭菌
\end{notation}
\section{药事管理学}%
\label{sec:药事管理学}
\[
    \text{药学主要就业岗位}
    \begin{cases}
        \text{药品研发}\\
        \text{药品注册与管理}\\
        \text{生产与质量管理}\\
        \text{医药销售与市场的推广}\\
        \text{临床监察与药剂师}
    \end{cases}
.\] 

