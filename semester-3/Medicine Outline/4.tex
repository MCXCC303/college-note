\lecture{4}{}
同名:桂皮(肉桂/天竺桂的皮)

同物:三七
\subsubsection*{调查考证生药资源}%
\label{subsub:调查考证生药资源}
未在古籍中记载而发现的新生药:长春新碱、紫杉醇、喜树碱

\subsubsection*{评价生药的品质,制定标准}%
\label{subsub:评价生药的品质-制定标准}

\subsubsection*{生产规范化}%
\label{subsub:生产规范化}
执行GAP标准、实现生产规范化
\begin{defi}
    中药标准化:Traditional Chinese Medicines Standardization

    包括中药材标准化(基础)、饮片标准化、中成药标准化
\end{defi}

\subsection{鉴定生药}%
\label{sub:鉴定生药}
\subsubsection*{基原鉴定}%
\label{subsub:基原鉴定}
用分类学的方法把各种动植物来源鉴定清楚,确定学名和药用部位
\begin{notation}
    必须要用完整的标本
\end{notation}
\subsubsection*{性状鉴定}%
\label{subsub:性状鉴定}
利用看、摸、闻、尝等描述中药的性状
\begin{eg}
    党参:皮松肉紧,似“狮子盘头”

    松贝:如“怀中抱月”

    海马:马头蛇尾瓦楞身

    黄芪:菊花心

    苍术:朱砂点

    何首乌:云锦纹
\end{eg}
\subsubsection*{显微鉴定}%
\label{subsub:显微鉴定}
使用显微技术对细胞、组织进行观察
\begin{notation}
    组织鉴定:观察切片或磨片,适合于完整的药材或粉末特征相似的药材

    粉末鉴定:观察粉末制片或解离片,观察分子及内涵物的特征,适合于破碎、粉末状的药材或中成药
\end{notation}
\subsubsection*{理化鉴定}%
\label{subsub:理化鉴定}
通过物理和化学的方法对生药所含成分的定性定量分析
\begin{notation}
    部分鉴定方法:

    紫外分光光度法

    红外分光光度法

    高效液相色谱

    气象色谱

    荧光分析

    质谱
\end{notation}
\subsubsection*{DNA分子鉴定}%
\label{subsub:DNA分子鉴定}
比较DNA的差异鉴别物种
\begin{notation}
    利用了:
    \[
        \begin{cases}
            \text{遗传稳定性}\\ 
            \text{遗传多样性}\\ 
            \text{化学稳定性}
        \end{cases}
    .\] 
\end{notation}
鉴定技术包含:
\[
    \begin{cases}
        \text{杂交基础鉴定}\\ 
        \text{聚合酶链式PCR分子标记}\\ 
        \text{重复序列分子标记}\\ 
        \text{mDNA分子标记}\\ 
        \text{DNA序列分析}
    \end{cases}
.\] 
\subsection{天然药物化学}%
\label{sub:天然药物化学}
\begin{defi}
    天然药物化学:运用现代科学理论与方法研究天然药物化学成分的学科
\end{defi}
研究内容:天然药物中的化学成分(活性成分/药效成分)的结构特点、理化性质、提取分离方法、结构鉴定、合成途径
\begin{eg}
    人参:

    提取:人参皂苷

    结构鉴定:人参皂苷类

    结构鉴定$\to$ 理化性质$\to$ 提取分离方法
    
    结构特点$\to$ 生物合成路径
\end{eg}
\begin{table}[htpb]
    \centering
    \caption{天然药物化学成分的主要分类}
    \label{tab:天然药物化学成分的主要分类}
    \begin{tabular}{ccccc}
    \toprule
    化学结构 & 酸碱性 & 溶解性 & 有无活性 & 生物合成途径\\
    \midrule
    $\begin{cases}
        \text{苯丙素类}\\ 
        \text{醌类}\\ 
        \text{酮类}\\ 
        \text{萜类}\\ 
    \end{cases}$ & $\begin{cases}
        \text{酸性}\\ 
        \text{碱性}\\ 
        \text{中性}\\ 
        \text{两性}
    \end{cases}$ & $\begin{cases}
        \text{非极性}\\ 
        \text{中等极性}\\ 
        \text{极性}
    \end{cases}$ & $\begin{cases}
        \text{活性成分}\\ 
        \text{无效成分}
    \end{cases}$ & $\begin{cases}
        \text{一次代谢}\\ 
        \text{二次代谢}
    \end{cases}$ \\
    \bottomrule
    \end{tabular}
\end{table}
\begin{notation}
    苯丙素类:一个苯环与三个直链连在一起形成单元$\text{C}_6-\text{C}_3$的化合物,再分为苯丙酸类、香豆素类、本脂素类
\end{notation}
\begin{notation}
    醌类:通常由酚氧化而来,分为:苯醌、萘醌、菲醌、蒽醌等
    \begin{eg}
        苯醌:辅酶类

        萘醌:胡桃醌

        菲醌:丹参醌IIA

        蒽醌:大黄素
    \end{eg}
\end{notation}
\begin{notation}
    黄酮类:两个具有酚羟基的苯环通过中央三碳原子相互连接的化合物$\text{C}_6-\text{C}_3-\text{C}_6$

    萜类:甲戊二甲酸的衍生物且为$\left( \text{C}_5\text{H}_x \right)_{n}$

    甾体:含环戊烷并多氢菲母核的化合物

    生物碱:含氮原子,多有复杂的环状结构、呈碱性,可与酸成盐,具有显著而特殊的生物活性
\end{notation}
