\subsubsection*{中药的采收}%
\label{subsub:中药的采收}
1. 植物药材的采收:
\[
    \begin{cases}
        \text{全草类:生长、枝叶茂盛的花前期、刚开花时}\\ 
        \text{叶类:开花蕾时}\\ 
        \text{花类:开花时}\\ 
        \text{果实、种子类:果实成熟后、将成熟时}\\ 
        \text{根茎类:阴历二、八月}\\ 
        \text{树皮、根皮类:清明至夏至期间剥皮}\\ 
    \end{cases}
.\] 

2. 动物药物的采收:种类和部位不同,采集事件不同

\subsubsection*{中草药贮存}%
\label{subsub:中草药贮存}
\begin{notation}
    不科学保存可能出现的现象:霉烂、虫蛀、变色、泛油等
\end{notation}
\subsubsection*{中药的药性}%
\label{subsub:中药的药性}
\begin{defi}
    药性:药物的性味和功能,包含对机体的反应和对疾病的疗效,包括四气、五味、升降沉浮、归经
\end{defi}
\begin{table}[htpb]
    \centering
    \caption{四气(四性)}
    \label{tab:四气(四性)}
    \begin{tabular}{cccc}
    \toprule
    寒性 & 凉性 & 温性 & 热性\\
    \midrule
    \multicolumn{2}{c}{石膏、黄连、栀子} & \multicolumn{2}{c}{附子、干姜}\\
    \midrule
    \multicolumn{2}{c}{治疗热性病} & \multicolumn{2}{c}{治疗寒性病}\\
    \multicolumn{2}{c}{清热泻火、阴性} & \multicolumn{2}{c}{温里散寒、阳性}\\
    \bottomrule
    \end{tabular}
\end{table}
\begin{table}[htpb]
    \centering
    \caption{五味(药物作用的标志)}
    \label{tab:五味-药物作用的标志-}
    \begin{tabular}{ccccc}
    \toprule
    辛 & 甘 & 酸 & 苦 & 咸\\
    \midrule
    发散、行气、行血 & 补益、和中 & 收敛、固涩 & 燥湿、泻降 & 软坚、泻下\\
    \bottomrule
    \end{tabular}
\end{table}
\begin{table}[htpb]
    \centering
    \caption{升浮沉降}
    \label{tab:升浮沉降}
    \begin{tabular}{cccc}
    \toprule
    升(黄芪)& 浮(麻黄) & 沉(大黄)& 降(石决明)\\
    \midrule
    \multicolumn{2}{c}{发汗、升阳、散寒、催吐} & \multicolumn{2}{c}{降逆、清热、泻火、收敛}\\
    \midrule
    \multicolumn{2}{c}{阳性} & \multicolumn{2}{c}{阴性}\\
    \bottomrule
    \end{tabular}
\end{table}
\[
    \text{辩证}
    \begin{cases}
        \text{气味、质地轻重}
        \begin{cases}
            \text{味属辛、甘;气属温热,阳性:升浮}\\ 
            \text{味属苦、酸、咸;气属寒凉,阴性:沉降}
        \end{cases}\\ 
        \text{受配伍的影响}\\ 
        \text{受炮制的影响}
    \end{cases}
.\]
\begin{defi}
    归经:药物对于机体某部分的选择作用

    以脏腑、经络为基础
\end{defi}
\begin{notation}
    寒性药物:清肺热、心热、胃热(清热)
\end{notation}

\subsubsection*{炮制中药}%
\label{subsub:炮制中药}
炮制中药以中医基础理论位置的,依据医疗、制剂和调剂的不同要求对原药材进行各种加工处理的总称

经过加工炮制后的中药称为\textbf{中药饮片}

\begin{notation}
    炮制中药的作用(易考):

    1. 降低、消除药物的毒性或副作用

    2. 转变药物的功能

    3. 增强疗效

    4. 改变、增强药物作用的部位和倾向

    5. 适应调剂制剂的需要

    6. 保证药物纯度,利于保存

    7. 便于服用
\end{notation}
\begin{eg}
    马钱子中包含士的宁(疗效好)和马钱子碱(有毒),使用马钱子需要通过沙烫或油炙,由于士的宁分解温度高达286摄氏度,马钱子碱仅在178摄氏度分解,因此可以去除毒性
\end{eg}
\begin{eg}
    麻黄:挥发油可治感冒,生物碱可治疗气管炎

    通过使用蜂蜜炒制(蜜炙),可以使挥发油降低$\frac{1}{2}$ ,生物碱含量不降,以此增强治疗气管炎的疗效
\end{eg}
\begin{eg}
    元胡:有效成分为生物碱,通过使用醋炒制,形成有机酸盐,提高提取率
\end{eg}
\begin{eg}
    黄柏树皮可入药,使用酒炙后治疗部位转变
\end{eg}
\begin{eg}
    石决明:为壳状,经煅烧、捣碎后质地疏松,药物易于溶出
\end{eg}
\begin{eg}
    动物类药物有腥味,通过酒炙可除去
\end{eg}
\[
    \text{中药配伍}
    \begin{cases}
        \text{单行:单味药}\\ 
        \text{相须:性能功效相类似的药物配合}\\ 
        \text{相使:一药为主另一药为辅}\\
        \text{相畏:药物的毒性或副作用可以被另一种药消除}\\ 
        \text{相杀:一种药能减轻另一种药的毒性}\\ 
        \text{相恶:在另一种药物下药效降低、消失}\\ 
        \text{相合:一种药使另一个药物毒性增强}\\
        \text{十八反:剧烈毒性}\\ 
        \text{十九畏:同上}
    \end{cases}
.\] 
\subsection{生药学}%
\label{sub:生药学}
\begin{notation}
    古代“生药”与“熟药”相对,均属中药
\end{notation}
生药最早由德国提出(pharmakognosie),由日本翻译,1905年由中国学者带回中国

生药分类:\[
    \begin{cases}
        \text{植物类}\begin{cases}
            \text{全草}\\ 
            \text{部分}\\ 
            \text{渗出物}
        \end{cases}\\ 
        \text{动物类(同上)}\\ 
        \text{矿物类}
    \end{cases}
.\] 

\begin{notation}
    生药发展分三个时期:

    1. 有生药记载(公元前4000~4500年)到19世纪中叶:研究医疗效用,凭感官和实践经验

    2. 近代商品生药学时期:生药成为国际贸易的特殊商品,研究内容为商品生药的来源(生物、地理)和鉴定商品生药的真伪、优劣,研究手段主要为形态学、化学(显微观察、定性定量方法),标志为\textbf{吗啡的分离}

    3. 20世纪60年代至今(现代生药学时期):研究内容和研究手段增加,形成了海洋生药学等新的学科
\end{notation}
\begin{table}[htpb]
    \centering
    \caption{近代生药提取物}
    \label{tab:近代生药提取物}
    \begin{tabular}{ccc}
    \toprule
    年代 & 提取物 & 生药 \\
    \midrule
    1806 & morphine & 阿片 \\
    1816 & emetine & 吐根\\
    1818 & strychnine & 番木鳖\\
    1820 & quinine,caffeine & 金鸡纳树皮,咖啡豆\\
    1828 & nicotnie & 烟草\\
    1829 & atropine & 颠茄根\\
    1860 & cocaine & 古柯\\
    1864 & eserine & 毒扁豆\\
    1887 & ephedrine & 麻黄\\
    \bottomrule
    \end{tabular}
\end{table}
生药学的发展方向:
\[
    \begin{cases}
        \text{道地药材的研究、建立生产基地、制定生产标准}\\ 
        \text{生药制剂科学研究}\\ 
        \text{生药智能化}\\ 
        \text{开发利用生药资源}\\ 
        \text{细胞培养、细胞工程、基因工程利用}
    \end{cases}
.\] 
\subsection{生药学研究内容和任务}%
\label{sub:生药学研究内容和任务}
\subsubsection*{准确识别、鉴定}%
\label{subsub:准确识别-鉴定}
大量同名异物、同物异名,掺假造假
