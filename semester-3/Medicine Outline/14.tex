\lecture{14}{11.01}
\begin{itemize}
    \item 定义
    \item 重要性
    \item 原理方法
    \item 我国药事管理机构
    \item 我国药事管理主要内容
\end{itemize}
\subsection{药事管理定义}%
\label{sub:药事管理定义}
\begin{defi}
    管理:

    设计和维持一种环境,使集体工作的人能有效完成预定目标的工作

    为实现预期的目标,以人为中心进行的协调活动
\end{defi}
\begin{defi}
    药事:一切与药相关的事务

    药事管理:对药学事业的综合管理,为有效率地实现药事组织的目标
\end{defi}
\begin{notation}
    狭义的药事管理:行政的管理,如drug administration或pharmaceutical affair administration

    广义的药事管理:一切的药物事业经营:pharmacy adminisration
\end{notation}
管理学是药事管理学的\textbf{理论基础} ,药事管理学是药学的\textbf{服务对象} 
\begin{notation}
    研究和任务:

    探索药事活动的管理规律,以促进医药行业的健康发展,保障用药的安全性、有效性和经济性
\end{notation}
\subsection{药事管理的重要性}%
\label{sub:药事管理的重要性}
\begin{notation}
    管理的重要性:

    科学技术$\xrightarrow[]{\text{相辅相成}} $ 管理学$\to $ 高效实现组织的目标
\end{notation}
药品的特殊性:
\begin{itemize}
    \item 专属性强
        \begin{itemize}
            \item 对症治疗
        \end{itemize}
    \item 时效性强
        \begin{itemize}
            \item 药等病而非病等药
        \end{itemize}
    \item 选择代理性
        \begin{itemize}
            \item 医生代理 
        \end{itemize}
    \item 作用两重性
        \begin{itemize}
            \item 防治病和不良反应
        \end{itemize}
    \item 兼备作用于非健康状态和作用最内在直接的双重性
\end{itemize}
\subsection{药事管理的原理和方法}%
\label{sub:药事管理的原理和方法}
\begin{notation}
    管理的基本原理:
    \begin{itemize}
        \item 系统原理
        \item 人本原理
        \item 责任原理
        \item 效益原理
    \end{itemize}
\end{notation}
\begin{notation}
    管理的基本方法:
    \begin{itemize}
    \item 法律方法
    \item 行政方法
    \item 经济方法
    \item 技术方法
    \item 宣传教育方法
    \end{itemize}
\end{notation}
\subsection{我国药事管理机构}%
\label{sub:我国药事管理机构}
\begin{notation}    
    国家药监局:
    \begin{itemize}
        \item 药品、医疗器械安全监督
        \item 药品、医疗器械标准
        \item 注册管理
        \item 质量管理
        \item 上市后风险管理
        \item 药师资格准入
        \item $\ldots $
    \end{itemize}
\end{notation}
\begin{notation}
    省级药品监管局:
    \begin{itemize}
        \item 辖区内药品管理法律法规和规章
        \item 核发许可证
        \item 新药受理和初审
        \item 辖区内医药企业监督管理
        \item 辖区内药事组织人员培训
    \end{itemize}
\end{notation}
\begin{notation}
    药品监管技术机构:
    \begin{itemize}
        \item 中国食品药品监督研究院
        \item 省级药品检验所
        \item 国家药典委员会
        \item 国家药品监督管理局药品审评中心
        \item $\ldots $
    \end{itemize}
\end{notation}
\subsection{我国药事管理主要内容}%
\label{sub:我国药事管理主要内容}
\begin{notation}
    药品研发管理
    \begin{itemize}
        \item 临床前研究管理:执行GLP
        \item 临床研究管理:执行GCP
    \end{itemize}
\end{notation}
\begin{notation}
    药品生产管理
    \begin{itemize}
        \item 药品生产活动的条件管理
        \item 药品生产活动的审批
        \item 药品生产质量管理要求
    \end{itemize}
\end{notation}
\begin{notation}
    药品经营管理
    \begin{itemize}
        \item 药品经营活动的条件管理
        \item 药品经营活动审批
        \item 经营质量管理
    \end{itemize}
\end{notation}
\begin{notation}
    药品使用管理
    \begin{itemize}
        \item 人员要求
        \item 审批程序
        \item 药品管理及配制制剂 
    \end{itemize}
\end{notation}
其他管理:特殊药品(毒品、放射性、预防类疫苗、兴奋剂)
{\centering{\section*{结课}%
\label{sec:结课}
}}
期末考试为闭卷考试
