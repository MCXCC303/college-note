\subsubsection*{主要研究内容}%
\label{subsub:主要研究内容}
1. 物理化学性质

2. 提取分离方法(研发新药)

3. 结构鉴定(IR、NMR、MR等,可以预测性质)

4. 生物合成途径(仿生合成)

\subsubsection*{主要任务}%
\label{subsub:主要任务}
1. 研究开发创新药物:
\[
    \begin{cases}
        \text{直接从天然药物中发现新药}\begin{cases}
            \text{麻黄碱}\\
            \text{利血平}\\
            \text{长春碱}
        \end{cases}\\
        \text{以活性成分为先导开发新药}\begin{cases}
            \text{青蒿素$\to $ 蒿甲醚(油溶性增强)}\\
            \text{长春碱$\to $ 长春酰胺}\\
            \text{吗啡}\begin{cases}
                \text{$\to $ 非那佐辛(效果强)}\\
                \text{$\to $ 喷那佐辛(副作用小)}
            \end{cases}\\
            \text{可卡因$\to $ 普鲁卡因}
        \end{cases}
    \end{cases}
.\] 
\begin{notation}
    创新药物研究的一般过程:

    原生生物活性成分研究:药效学筛选$\to $ 粗分确定有效成分$\to $ 分离追踪活性成分$\to $ 确定活性单体结构$\to $ 药理、毒理、临床评价$\to $ 结构修饰

    前体活性成分的研究
\end{notation}

2. 推进中药现代化、使中药早日进入国际医药主流市场
\[
    \begin{cases}
        \text{研究开发服用方便、安全有效、质量可靠的中药}\\
        \text{降低毒性、提高疗效、科学用药}\\
        \text{扩大范围}
    \end{cases}
.\] 
\begin{notation}
    我国是最早进行天然药物化学研究的国家:1575年李挺著,《医学入门》,五倍子$\to $ 没食子酸
\end{notation}

3. 天然产物的结构修饰
\begin{eg}
    根皮苷$\to $ 达格列净$\to $ 列净类药物
\end{eg}
作业:中药的现代化-屠呦呦与青蒿素:屠呦呦发现青蒿素及青蒿素药品的研制成功对整理发掘中药资源的启示

\section{药物化学}%
\label{sec:药物化学}
\subsection{基本定义、研究内容和任务}%
\label{sub:基本定义、研究内容和任务}
\subsubsection{基本定义}%
\label{subsub:基本定义}
\begin{defi}
    药物化学:关于药物发现、发展和确证,并在分子水平研究药物作用方式的一门学科

    药物化学基于化学,涵盖生物学、医学和药学等学科
\end{defi}
药物化学涉及的阶段:

1. 新药研究的发现阶段(Discovery)

2. 新药研究的开发阶段(Development)
\begin{defi}
    化学药物:已知确切结构的单一化合物

    新药:新的化学实体(New Chemical Entities, NCE),指从前没有用于人体治疗并注定可用作处方药的产品

    新分子实体(New Molecular Entities, NME):指具有某种生物活性的化学结构,由于活性不强、选择性底、吸收性差、毒性较大等特点不能直接药用,但可用对其进行改造优化修饰生产新的药物
\end{defi}

