\lecture{10}{10.18}
\begin{notation}
    药物效应和作用机制:

    药物为什么起作用、如何起作用
\end{notation}
药物以机理分类:
\begin{align*}
    \begin{cases}
        \text{非特异性药物:抗酸药物治疗胃酸过多}\\
        \text{补充机体所缺的物质:Vetamin-B12}\\
        \text{影响神经递质和激素:他汀类、生长激素}\\
        \text{作用于特定靶点}\begin{cases}
            \text{受体}\\
            \text{影响酶活性}\\
            \text{离子通道}\\
            \text{影响核酸代谢}\\
            \text{免疫系统}\\
            \text{基因}
        \end{cases}
    \end{cases}
.\end{align*}
\subsubsection{受体}%
\label{subsub:体-}
\begin{notation}
    受体学说:

    药效学的基本理论之一,从分子层面解释药理、作用机制、药效关系的基本理论
\end{notation}
\begin{defi}
    受体:能与药物结合相互作用发动细胞反应的大分子或大分子复合物
\end{defi}
\[
    \text{药物}+\text{内源性配体}+\text{受体}\ce{->[\text{信息放大系统}]}\text{生理、药理学反应}
.\] 

1878:Langley发现受体与药物作用的现象

1908:Ehrlic提出receptor(受体)概念,提出抗体的概念

1905:Langley命名受体(receptive substance),受体具有特异性和高亲和力

1933:Alfred J Clark用数学提出占领理论
\begin{notation}
    受体的特征:

    1. 灵敏性:低浓度配体产生大效应

    2. 特异性:引起受体兴奋反应的配体化学结构相似(合成相似效应的配体)

    3. 饱和性:受体数目一定

    4. 可逆性:配体+受体$\ce{<=>[\text{结合}][\text{解离}]}$ 结合体

    5. 多样性:同一受体可分布到不同细胞,产生不同效应
\end{notation}
\begin{notation}
    作用于受体的药物:

    1. 激动剂(agonist):产生最大效应

    2. 拮抗剂(antagonist):与受体结合但不产生效应,降低抗体反应活性,可再分为竞争性与非竞争性
\end{notation}
\begin{align*}
    \text{受体分类}\begin{cases}
        \text{酪氨酸激酶中介}\\
        \text{离子通道型}\\
        \text{细胞内}\\
        \text{G-蛋白偶联受体(2012年诺贝尔奖)}\\
        \ldots
    \end{cases}
.\end{align*}
\begin{notation}
    G-蛋白偶联受体:最大的膜受体家族
\end{notation}
\subsubsection{药物分布}%
\label{subsub:药物分布}
\begin{notation}
    药理效应与剂量在一定范围内成一定比例:剂量-效应关系

    1. 量反应:过量

    2. 质反应:过敏
\end{notation}
\begin{notation}
    半数量/浓度:

    $\text{EC}_{50}$:半数有效浓度

    $\text{ED}_{50}$ 半数有效剂量

    $\text{LC}_{50}$ :半数致死浓度

    $\text{LD}_{50}$ 半数致死剂量

    有效浓度越小越好,致死浓度越大越好
\end{notation}
\begin{notation}
    1. 治疗指数TI(therapeutic index):
    \[
        \text{TI}=\frac{\text{LD}_{50}}{\text{ED}_{50}}
    .\] 

    2. 安全指数$\displaystyle{\frac{\text{LD}_{5}}{\text{ED}_{95}}} $

    3. 强化安全指数$\displaystyle{\frac{\text{LD}_1}{\text{ED}_{99}}}$
\end{notation}
\begin{notation}
    药物的体内过程:

    吸收$\to $ 分布$\to $ 代谢$\to $ 排泄(ADME)

    转运方式:

    1. 被动转运:类似渗透,由浓度差驱动,不消耗ATP;共三种类型:简单扩散、易化扩散、滤过

    2. 主动转运
\end{notation}
\begin{defi}
    吸收:药物从给药位置进入血液循环

    静脉内给药不经过吸收过程:直接循环
\end{defi}
按速度排序给药方式:

静脉注射$\to $ 吸入$\to $ 舌下$\to $ 直肠$\to $ 肌注$\to $ 皮下$\to $ 口服$\to $ 皮肤
\begin{notation}
    首过效应

    口服药物在肠腔内未被破坏,进入肝脏不被吸收,经肠道转化排出体内

    口腔吸收、直肠吸收无首过效应
\end{notation}
药物在体内分布大多数不均匀
\begin{notation}
    再分布:首先到血流丰富的器官,再朝容量大的组织转移
\end{notation}
\begin{notation}
    屏障效应

    1. 血脑屏障:阿托品季铵转化为甲基阿托品后不能通过血脑屏障

    2. 胎盘屏障
\end{notation}
血浆蛋白结合力、局部器官血流量也会影响分布
\subsubsection{代谢}%
\label{subsub:代谢}
狭义的药物代谢:药物在体内化学结构改变

1. 第一时相反应:氧化还原水解

2. 第二时相反应:结合葡萄藤醛酸、乙酰基、氨基酸等

意义:促进药物排泄,改变药理活性
\subsubsection{排泄}%
\label{subsub:排泄}
可以从肾、胆汁、肺、腺体排出
\begin{notation}
    肝肠循环:肝脏$\to $ 胆汁$\to $ 肠腔$\ce{->[\text{重吸收(有时不利)}]}$
\end{notation}
\subsubsection{代谢动力学}%
\label{subsub:代谢动力学}
用量时曲线表示:治疗和毒性的作用,剂量和给药时间
\begin{notation}
    静脉注射$\to $ 肌肉注射$\to $ 口服:血药浓度逐渐降低
\end{notation}
\begin{defi}
    半衰期$t_{\frac{1}{2}}$ :药物在体内消耗一半的时间
\end{defi}
\begin{defi}
    生物利用度:非血管给药时实际吸收与总药量的比值:$F=\displaystyle{\frac{A}{D}}\times 100\%$
\end{defi}
绝对利用度$=\displaystyle{\frac{\text{AUC}_{\text{口服}}}{\text{AUC}_{\text{静脉注射}}}}\times 100\%$

相对利用度$=\displaystyle{\frac{\text{AUC}_{\text{待测}}}{\text{AUC}_{\text{已知}}}}\times 100\%$
\begin{notation}
    连续恒速给药:等剂量等时间间隔多次重复给药,约4-5个半衰期后达到血液药物浓度平衡
\end{notation}
\subsubsection{毒理学}%
\label{subsub:毒理学}
\begin{defi}
    毒理学:研究药物代谢、入侵途径、中毒机理、病理过程
\end{defi}
一般药理学/安全药理学:观察新药对机体的影响

毒理学方法:半数致死量测量试验

长期毒性:在动物上观察蓄积与慢性毒性

特殊毒理学方法:
\begin{align*}
    \begin{cases}
        \text{致突变}\\
        \text{生殖毒性}\\
        \text{致癌}\\
        \text{药物依赖性}\\
        \text{毒物代谢动力学}
    \end{cases}
.\end{align*}

药理学研究方法:生物化学、行为心理学等

\begin{eg}
    乙酰水杨酸:

    小肠吸收,关节分布,血液转运,尿液排出

    临床应用:略

    不良反应:凝血等
\end{eg}

