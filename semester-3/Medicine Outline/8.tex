\lecture{8}{10.11}
\begin{notation}
    设计阶段:

    发生了“反应停”事件,药物安全性要求提高
\end{notation}
\begin{notation}
    药物设计学进程

    1. 定量构效关系

    2. 三维定量构效关系

    3. 计算机辅助

    4. 基因组学

    5. 蛋白组学
\end{notation}
\subsubsection{中国发展现状}%
\label{subsub:中国发展现状}
规模大、产值低、工艺水平领先、创新不足
\begin{notation}
    新药研究已经得到重视,形成了基于天然活性成分结构为基础的新药设计和发现的特色
\end{notation}
\subsection{药物化学在新药研究与开发中的作用}%
\label{sub:药物化学在新药研究与开发中的作用}
\begin{notation}
    药物化学的根本任务:研究和开发新药
\end{notation}
\[
    \text{药物设计学}\begin{cases}
        \text{数学}\\
        \text{统计学}\\
        \text{计算机科学}\\
        \text{计算化学}\\
        \text{药理学}\\
        \text{药物化学}\\
        \text{分子生物学}\\
        \text{生物信息学}\\
        \ldots
    \end{cases}
.\] 
新药研究与开发的现状:三高一长
\begin{notation}
    新药研究的挑战

    1. 药物分子要求多且高

    2. 药物分子具有成药性、类药性(ADME:吸收、分布、代谢、排布)

    3. 药物的成药性与药物分子结构密切相关
\end{notation}
\begin{notation}
    药物化学发展新方向:

    1. 新药研究新模式

    2. 计算机辅助设计

    3. 手性药物
\end{notation}
\begin{notation}
    新药研究开发过程

    1. 研究方针

    2. 发现有效化合物、先导化合物优化

    3. 药理、毒理临床研究

    4. 新药临床研究
\end{notation}
\begin{eg}
    先导化合物的发现:

    1. 从天然资源中筛选:青蒿素$\to $ 蒿甲醚

    2. 以生物化学或药理学为基础:卡托普利(副作用:味觉丧失、抑制骨生长、皮疹)$\to $ 依那普利拉$\to $ 依那普利(降血压)

    3. 从药物副作用中发现:某种治疗伤寒的药物(副作用降血糖)$\to $ 氯磺丁脲/甲苯磺丁脲

    4. 通过药物代谢研究发现:美伐他汀$\to $ 美伐他汀代谢产物

    5. 药物合成的中间体

    6. 利用组合化学和高通量筛选
\end{eg}
临床四期:

临床一期:评价药理学安全问题

临床二期:人体试验,有效性安全性初步评价,推荐给药剂量

临床三期:大规模、多中心试验,进一步确定疗效和不良反应

临床四期:新药上市后监测,收集罕见不良反应

\section{药理学}%
\label{sec:药理学}
\subsection{性质与任务}%
\label{sub:性质与任务}
\begin{defi}
    药理学(Pharmacology):研究药物与机体相互作用的规律与原理的医学基础学科
\end{defi}
\begin{defi}
    药效学(药物效应动力学,Pharmacodynamics),研究药物对机体的作用及作用机制的学科
\end{defi}

