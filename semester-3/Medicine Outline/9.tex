\lecture{9}{10.16}
$\ldots$ 
\subsection{药理学的研究内容}%
\label{sub:药理学的研究内容}
\begin{notation}
    药物作用具有双重性:
    \[
        \text{药物作用}
        \begin{cases}
            \text{治疗作用}\begin{cases}
                \text{对因治疗}\\
                \text{对症治疗}\\
                \text{补充治疗}
            \end{cases}\\
            \text{不良反应}\begin{cases}
                \text{副反应}\\
                \text{毒性反应}\\
                \text{后遗效应}\\
                \text{继发反应}\\
                \text{停药反应}\\
                \text{变态反应}\\
                \text{特异质反应}\\
                \text{耐受性和依赖性}
            \end{cases}
        \end{cases}
    .\] 
\end{notation}
\begin{notation}
    治疗作用:目标药物的目标作用
\end{notation}
\begin{notation}
    不良反应:
    \begin{table}[htpb]
        \centering
        \caption{不良反应}
        \label{tab:不良反应}
        \begin{tabular}{|c|c|c|c|c|}
        \hline
        A型 & 与剂量有关 & 可预测 & 发生率高、死亡率低 & 药物动力学 \\
        \hline
        B型 & 与剂量无关 & 难预测 & 发生率低、死亡率高 & 遗传、杂质等\\
        \hline
        \end{tabular}
    \end{table}
\end{notation}
\begin{eg}
    典型B型不良反应:青霉素过敏
\end{eg}
\subsubsection*{副作用}%
\label{subsub:副作用}
\begin{defi}
    副作用:药物在治疗量时,出现的与治疗目的无关的不适反应
\end{defi}
\begin{eg}
    阿托品:可抑制腺体分泌、松弛平滑肌、血管和心脏收缩

    在治疗其他疾病时以上效果为副作用
\end{eg}
\subsubsection*{毒性反应}%
\label{subsub:毒性反应}
\begin{defi}
    毒性反应:剂量过大或药物在体内蓄积过多时发生的危害性反应
\end{defi}
\[
    \begin{cases}
        \text{剂量过大}\implies \text{急性毒性:损害循环呼吸神经系统}\\
        \text{用药时间过长}\begin{cases}
            \text{慢性毒性:损害肝肾骨髓内分泌}\\
            \text{特殊毒性或潜在毒性}\begin{cases}
                \text{致突变}\\
                \text{致畸形}\\
                \text{致癌}
            \end{cases}
        \end{cases}
    \end{cases}
.\] 
\subsubsection*{变态反应(过敏反应)}%
\label{subsub:变态反应-过敏反应-}
\begin{defi}
    机体受药物刺激时发生的异常免疫反应,极少量过敏原即可引起剧烈反应
\end{defi}
机理:\[
    \underset{\text{半抗原}}{D}+\underset{\text{过敏原}}{P}\ce{->} \underset{\text{全抗原}}{DP}
.\] 
\begin{notation}
    常见变态反应:

    1. 青霉素:过敏性休克

    2. 氯霉素:再生障碍性贫血
\end{notation}
\subsubsection*{继发性反应}%
\label{subsub:继发性反应}
\begin{defi}
    又称治疗矛盾
\end{defi}
\begin{eg}
    久用广谱抗生素导致二重感染

    糖尿病性便秘使用乳果糖、麻仁丸等
\end{eg}
\subsubsection*{后遗效应}%
\label{subsub:后遗效应}
\begin{defi}
    停药后血药浓度降至有效浓度下后,有残留的生物效应
\end{defi}
\begin{eg}
    苯巴比妥催眠导致次日早晨头晕、困倦
\end{eg}
\subsubsection*{特异质效应}%
\label{subsub:特异质效应}
某些药物使少数并出现特异性的不良反应,不可预知,多与遗传有关
\begin{eg}
    某些人对琥珀胆碱呈现特异性反应:先天性血浆胆碱酶缺失
\end{eg}
\begin{eg}
    乙醇脱氢酶/乙醛脱氢酶的缺失
\end{eg}





