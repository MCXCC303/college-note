\lecture{11}{10.23}
\subsection{药物分析学的研究内容}%
\label{sub:药物分析学的研究内容}
\begin{notation}
    药品质量标准:

    国家对药物质量规格和检验方法制定的规定标准
\end{notation}
\[
    \begin{cases}
        \text{非法定标准}\\
        \text{法定标准}\begin{cases}
            \text{中国药典}\\
            \text{局颁标准}
        \end{cases}\\
        \text{临时标准}\\
        \text{正式标准}\\
        \text{内部标准}\\
        \text{公开标准}
    \end{cases}
.\] 
\begin{notation}
    中华人民共和国药典:简称\textbf{中国药典}(ChP),由国家药典委员会编撰,由NMPA(CFDA)颁布实施的\textbf{记载药品标准的法典},是国家监督管理药品质量的技术标准
\end{notation}
\begin{notation}
    国家药品监督管理局药品标准:简称居颁标准:主要收载\textbf{新药标准}、新版药典未收载但尚未淘汰的药品标准和原地方标准经规范后适用于全国范围的药品标准
\end{notation}
省(自治区、直辖市)药品标准:与2001年取消
\begin{notation}
    临床研究用药品质量标准:非公开的药品标准,仅在临床试验期间(1,2,3期)有效

    暂行和试行药品质量标准:4期使用暂行药品质量标准,经稳定后成为试行药品质量标准,正式生产后两年稳定后成为居颁标准
\end{notation}
\begin{notation}
    企业标准:非公开标准,非法定标准,用于控制内部质量
\end{notation}
\subsubsection{中国药典与主要国外药典}%
\label{subsub:中国药典与主要国外药典}
中国药典分为凡例、正文和通则三部分,现行2020版(第11版,共11版,5年更新一次),具有法律效应
\begin{notation}
    凡例:对正文和通则的规定
\end{notation}
\begin{notation}
    正文:检测药品质量是否达到用药要求并衡量质量是否稳定统一的技术规定

    正文包括:品名、来源、处方、制法、性状等
\end{notation}
\begin{notation}
    通则:包括制剂通则、通用检测方法、指导原则,按分类编码

    包括:一般杂质检查方法、一般鉴别试验等
\end{notation}
其他药典:
\begin{table}[htpb]
    \centering
    \caption{外国药典}
    \label{tab:外国药典}
    \begin{tabular}{cc}
        \toprule
    美国 & 美国国家处方集、美国药典\\
    英国 & 英国药典\\
    日本 & 日本药局方\\
    欧洲 & 欧洲药典\\
    国际 & 国际药典\\
    \bottomrule
    \end{tabular}
\end{table}
\subsection{药品质量管理与监督}%
\label{sub:药品质量管理与监督}
国际上的:《国际人用药品注册技术协调会》

中国的:GLP等

\begin{notation}
    药品质量监督管理行政机构:

    $\circ$ NMPA/CFDA:负责全国

    $\circ$ 省级食品药品监督管理局:负责省级以下

    $\circ$ 市、县级食品药品监督管理局
\end{notation}
\begin{notation}
    药品质量监督管理技术机构:国家药典委员会等
\end{notation}
\subsubsection{药品检验工作的基本内容}%
\label{subsub:药品检验工作的基本内容}
1. 药品的质量检验
\begin{notation}
    基本程序:取样、性状鉴别、检查、含量测定、检验报告

    取样要求:均匀合理,样品总件数为$n$,$n\le 3$ 每件都取,$n\in [4,300]$ 时抽样量为$\sqrt{n}+1 $

    性状:测定物理常数

    鉴别:化学方法测定离子反应、IR、UV、官能团等

    检查:纯度要求、一般/特殊杂质
    
    含量测定:化学分析、仪器分析、生物鉴别

    检验报告:完整清晰的原始记录,不得随意涂改,若更改:画两条细线并在右上角写出正确内容并签名
\end{notation}
\begin{notation}
    制剂检验包含性状到含量测定4步

    检查:无需检查原料药相关的检查项目

    常规性检查:崩解时限(片剂、胶囊),装量(注射液),溶出度(难溶性片剂),释放度(缓控释、肠溶制剂),含量均匀度(小剂量片剂)

    制剂的主要成分含量限度较原料药宽:片剂含量限度$[95.0\%,105.0\%]$ ,原料药$[99.0\%,100\%]$,小剂量$[90.0\%,110.0\%]$ 

    测定含量:UV,HPLC等
\end{notation}
\begin{notation}
    中药制剂:多一个指纹图谱

    鉴别:贵重药、毒剧药、君药、臣药,使用显微鉴别法(含原生药粉末的中药制剂)、化学鉴别法(荧光法、显色法、沉淀/升华/结晶)、色谱法(TLC、GC、HPLC等)

    检查:水分、灰分、酸不溶性灰分、砷盐、重金属、农药残留、有毒组分

    含量测定:HPLC(高灵敏度)

    指纹图谱:HPLC图谱内有相对稳定的色谱信号、GC、TLC、MS等
\end{notation}
\begin{notation}
    生化药物:

    分子量测定(一个范围)、生物活性、安全性检查(热源、过敏等)

    鉴别:酶法、电泳法(糖凝胶)、生物法

    检查:胰蛋白酶

    安全性检查:残留异性蛋白质(过敏)、致病微生物、降压物质等

    含量测定/效价测定:百分含量表示(小分子药物)、生物效价和酶活力单元(酶类、蛋白质类)

    酶活力:酶催化一定化学反应的能力
\end{notation}
\subsubsection{药房制剂快速化学检验}%
\label{subsub:药房制剂快速化学检验}
速度快


    


