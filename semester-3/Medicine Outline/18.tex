\lecture{18}{12.07}
\begin{notation}
    药物分析学的任务:\textbf{药品质量管理和监督、药品的质量检验、药品质量标准的制定、药品稳定性研究}
\end{notation}
\begin{notation}
中国药典:(重点)\textbf{中华人民共和国药典,由国家药典委员会编撰,由NMPA颁布实施的记载药品标准的法典,是国家监督管理药品质量的技术标准}

中国药典的内容:\textbf{凡例、正文、通则}
\begin{description}
    \item[凡例] 对正文和通则的规定
    \item [正文] 检测药品质量是否达到用药要求并衡量质量是否稳定统一的技术规定
    \item [通则] 制剂通则、通用检测方法(一般杂质检查方法、一般鉴别试验等)、指导原则,按分类编码
\end{description}
\end{notation}
\begin{notation}
    药品稳定性检验方法:\textbf{影响因素试验、加速试验、长期试验}
    \begin{description}
        \item[影响因素试验] 在高温、高湿度、光照下测试
        \item [加速试验] 使用市售包装在$40^\circ\text{C}\pm 2^\circ\text{C}$,相对湿度$75\%\pm 5\%$ 下放置6个月
        \item [长期试验] 使用市售包装在$25^\circ\text{C}\pm 2^\circ\text{C}$,相对湿度$60\%\pm 10\%$ 下放置12个月
    \end{description}
\end{notation}
\begin{notation}
    药品质量标准制定的原则:\textbf{安全有效、技术先进、经济合理、不断完善}
\end{notation}

