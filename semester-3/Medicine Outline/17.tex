\lecture{17}{12.03}
\begin{notation}
    天然药物化学研究的主要内容:\textbf{天然药物中的化学成分的结构特点、理化性质、提取分离方法、结构鉴定、合成途径}
\end{notation}
\begin{notation}
    天然化合物的结构分类:\textbf{苯丙素类、醌类、黄酮类、萜类、甾体、生物碱}
\end{notation}
\subsection{药物作用}%
\label{sub:复习:药物作用}
\begin{notation}
    药物的双重性:\textbf{治疗作用和不良反应}
\end{notation}
\begin{notation}
    治疗作用:\textbf{目标药物的目标作用}

    分类:\textbf{对因治疗、对症治疗、补充治疗}
\end{notation}
\begin{notation}
常见药物作用的不良反应类型:\textbf{副反应、毒性反应、后遗效应、继发反应、停药反应、变态反应、特异质反应、耐受性和依赖性}
\end{notation}
\begin{notation}
    受体的特点:\textbf{灵敏性、特异性、饱和性、可逆性、多样性}

    作用于受体的药物分类:\textbf{激动剂、拮抗剂}
\end{notation}
\begin{notation}
    肝肠循环:肝脏 $\to $ 胆汁$\to $ 肠腔$\to $ 重吸收$\to $ 肝脏
\end{notation}
\begin{notation}
首过效应:\textbf{口服药物在肠腔内不被破坏、进入肝脏不被吸收、经肠道排出体内},对于口腔吸收和直肠吸收无首过效应
\end{notation}
\begin{notation}
药物毒性评价“三致实验”:\textbf{致癌、致畸、致突变}
\end{notation}
\subsection{药物分析学}%
\label{sub:复习:药物分析学}
