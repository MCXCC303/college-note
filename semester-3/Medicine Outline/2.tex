\section{中药、生药与天然药物化学}%
\label{sec:中药、生药与天然药物化学}
\subsection{中药的起源和发展}%
\label{sub:中药的起源和发展}
    我国历史上的本草著作:《神农本草经》、《唐本草》、《本草纲目》等
\begin{notation}
    《神农本草经》:3卷,东汉25-200年,载药365种

    上药:120种为君,养命以应天

    中药:120种为臣,养性以应人

    下药:120种佐使,治病以应地
\end{notation}
\begin{notation}
    《本草纲目》:1892种药物,11000附方

    分为16部纲,62目,参考书籍800余本

    正名为纲,附释名为目,故命名为《本草纲目》
\end{notation}
\begin{eg}
    胎盘可入药
\end{eg}
中药现代化:屠呦呦与青蒿素(由黄花蒿乙醚萃取而来)
\begin{notation}
    青蒿素可以杀疟虫的关键:过氧键
\end{notation}
\[
    \mbox{中药分类}
    \begin{cases}
        \mbox{现代医学理论体系}
        \begin{cases}
            \mbox{化学药}\\ 
            \mbox{生物制品}\\ 
            \mbox{天然药物}
        \end{cases}\\
        \mbox{传统医学理论体系}
        \begin{cases}
            \mbox{中草药}
            \begin{cases}
                \mbox{中药}\\ 
                \mbox{草药}
            \end{cases}\\
            \mbox{藏药}\\ 
            \mbox{蒙药}\\ 
            \mbox{维药}\\
            \ldots
        \end{cases}
    \end{cases}
.\] 
\begin{defi}
    中药:依据中医学理论和中医临床药物,应用于医疗保健的药物
\end{defi}
\begin{defi}
    生药:来源于植物、动物和矿物的新鲜品或经过简单加工,直接用于医疗保健或作为医药用原料药的天然药物
\end{defi}
\begin{table}[htpb]
    \centering
    \caption{生药分类}
    \label{tab:生药分类}
    \begin{tabular}{ccccc}
    \toprule
    植物生药 & 植物中制取 & 动物生药 & 动植物中制取 & 矿物生药\\
    \midrule
    人参、洋地黄 & 淀粉、挥发油 & 水蛭 & 油脂、蜡 & 朱砂\\
    \bottomrule
    \end{tabular}
\end{table}
\begin{notation}
    一些医用敷料、滤材以及具有杀虫作用的草也属于生药
\end{notation}
\begin{defi}
    天然药物:在现代医学理论体系指导下,来自于天然植物、动物和矿物中的、非化学单体的药物(本质上为生药)
\end{defi}

民族药用资源:

藏药:2805 种

蒙药:1340 种

壮药:2076 种

维药:1917 种

$\ldots\ldots$
\begin{notation}
    保护中药:以保护求发展

    就地保护、迁地保护结合

    天然更生、人工培育结合
\end{notation}
\begin{notation}
    中药资源创新:

    1. 转基因植物生产活性物质(转基因烟草、甘草等)
    
    2. 细胞组织培养(人参、三七、丹参等)
\end{notation}
\subsection{中药学}%
\label{sub:中药学}
\begin{defi}
    中药学是研究中药的基本理论和各味中药的来源、性味、功效和应用方法的学科
\end{defi}
\begin{notation}
    四大怀药:怀菊花、怀帝黄、怀山药、怀牛膝
\end{notation}
