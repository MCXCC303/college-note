\section*{课程简介}%
\label{sec:课程简介}
教师邮箱:jdeng@cqu.edu.cn

教师微信:ytyr88

教师电话:18223244276

成绩组成(100\%)=出勤,课堂小测试(<20\%)+课堂表现(10\%)+课后作业(10\%)+期末考试(>60\%)

缺勤三次取消成绩

教材需求:药学概论第五版,共8节

共16节课,32学时,1-8周

考试为闭卷考试,有选择、名词解释、材料分析等题型
\section{绪论}%
\label{sec:绪论}
\subsection{药学概念}%
\label{sub:药学概念}
\begin{question}
    什么是药物?什么是药学?
\end{question}
东汉《说文解字》有“药,治病草也”,将药分作草、木、虫、石、谷五种

药的定义广泛,作动词为治疗,作名词为花名、火药等

\begin{notation}
    今天所说的药物:用于防病、治病、诊断疾病的物质
\end{notation}
药物通常具有明确的适应症、禁忌症、用法和用量

药物的基本属性为安全性和有效性(在一定剂量内)

在我国药品专指人用药品

\begin{notation}
    药品分为处方药(R-receptor, Rx)和非处方药(over the counter, OTC),红标OTC/甲类OTC药物只能在医院与药房购买
\end{notation}
\begin{notation}
    药物具有双重性:治疗作用和不良反应
\end{notation}
坏血病:牙龈出血,牙齿松动脱落,手指关节肿痛等

林德发现维生素C可以治疗坏血病
\begin{eg}
    维生素C适量摄入可以提升人体免疫力、治疗坏血病、缓解关节疼痛、预防癌症,但过量摄入将导致一系列不良反应:腹泻,胃酸增加,溶血等
\end{eg}
\begin{eg}
    肾上腺皮质分泌的可的松可以治疗炎症、免疫抑制等,但过度使用会导致身体对类固醇的依赖加强、溃疡、免疫功能下降、骨质疏松等不良反应
\end{eg}
\begin{eg}
    吗啡由德国化学家Serturner于1805年首次从鸦片中分离,具有镇痛作用,但大量长期使用成瘾,吗啡双乙酰化后成为海洛因,成瘾性更强
\end{eg}
\begin{question}
    药物与保健食品的区别:
\end{question}
1. 保健食品是具有特定保健功能的\textbf{食品},不限定剂量,包装管理为国食健字G(J),无药用价值,无适应症

2. 药品包装管理为国药准字H(或Z,S,J,B,F)

\begin{notation}
    药学是以现代化学和医学为主要指导,研究、开发、生产、销售、使用、管理用于预防、治疗、诊断疾病的药物的科学
\end{notation}
药学的主要学科:
\[
    \begin{cases}
        \text{药物化学}\\ 
        \text{药理学}\\ 
        \text{药物分析学}\\ 
        \text{药剂学}\\ 
        \text{生药学}\\ 
        \text{微生物与生化药学}
    \end{cases}
.\] 
\begin{question}
药学与化学、医学的关系:
\end{question}
\begin{notation}
    研究药学要以化学为基础:人体本质上是化学物质的组合

    研究药物要以临床医学为指导:先有病后有药

    药学是医学和化学的桥梁
\end{notation}
\begin{eg}
    帕金森病的发病机理是缺少多巴胺(快乐因子),5-羟色胺用于抑制情绪:爱情催化剂
\end{eg}
\begin{eg}
    阿尔茨海默病由临床发现脑内胆碱是神经系统退化,因此研制乙酰胆碱酯酶抑制剂延缓
\end{eg}
\subsection{药学的起源与发展}%
\label{sub:药学的起源与发展}
\subsubsection{药学起源}%
\label{subsub:药学起源}
现代药学起源可追溯至远古时代
\begin{eg}
    公元前6世纪通过酒曲治疗胃病,利用酵母菌促进消化,发展为如今的酵母片
\end{eg}
\begin{eg}
    现代黑猩猩学会利用特定植物来治疗肠道疾病
\end{eg}
\begin{notation}
    最早记载人类医学活动:巴比伦时代(公元前2600年)

    埃及的《Papyrus Ebers》记载于公元前1500年前,记录了800个处方,700种药物

    第一家私人药店:阿拉伯人于公元8世纪开创了\textbf{医药的分家}

    第一个国家药店:北宋与公元1076年开办的熟药所

    第一部官方组织编篡的药典:公元659年唐政府颁布《新修本草》或《唐本草》

    第一个从植物中提取的活性成分:吗啡(1805 $\to $1809,由德国药剂师Sertuner从鸦片中提取)
\end{notation}
著名药学典籍:
\begin{notation}
    神农本草经:东汉出版,由多方补充而成,共三卷

    收录365种药物(252种植物药,67种动物药,46种矿物药)
\end{notation}
\begin{notation}
    本草纲目:由李时珍在明代历时30年完成,成书于1578年,全书共52卷,约190万字

    共收录1892种药物,11000余处方,插图1160幅
\end{notation}
李时珍对生物学、化学、矿物学、地质学也有贡献,是一个杰出的科学家,药学家
\begin{notation}
    现代医学之父:希波克拉底

    古希腊医师,提出了“四体液学说”:人体由四属性的体液组成

    提出了《希波克拉底誓言》:医学与药学学生入学
\end{notation}
\begin{notation}
    盖伦:古罗马医师

    主要贡献:提倡使用生药制剂(盖伦制剂,多为膏剂),强调按季节、地区和气候用药,在欧洲盛名
\end{notation}
\begin{notation}
    阿维森纳:著有《医典》,是医学史上最著名的系统的医药学百科学书

    与盖伦和希波克拉底共称为西方医学三巨匠
\end{notation}
\begin{notation}
    药剂学之父:席勒/舍勒

    制备$\text{O}_2,\text{Cl}_2$,发现众多金属元素,从自然界提取多种有机酸,开创了近代以天然药物为原料的药剂学基础
\end{notation}
\begin{notation}
    药理学之父:施来台德勒

    微生物学奠基者:巴斯德,发明了巴氏消毒法和微生物纯培养法,首次实现了手性化合物的分离

    细菌学奠基者:科赫,主要研究结核杆菌,提出的方法用于验证细菌与病害的关系

    巴斯德和科赫奠定了微生物学最基本的原理和方法,为微生物学发展指明了方向
\end{notation}
\subsubsection{现代药学发展}%
\label{subsub:现代药学发展}
现代药学发展分为以下时期:

1. 古代至19世纪末:利用天然药物

2. 19世纪末:药物合成(1910年,德国科学家Paul Ehrlich合成606用于杀灭梅毒杆菌)
\begin{notation}
    Paul Ehrlich(欧立希):化学疗法的先驱,合成梅毒特驱药606并改进为914
\end{notation}
\begin{notation}
    百浪多息:一种磺胺染料,对链球菌和金黄色葡萄球菌感染有特效,是第一个对任何全身细菌性感染有效的化学治疗剂,由克拉尔于1932年合成,由多马克发现疗效
\end{notation}
\begin{notation}
    法国夫妇特雷福埃尔发现百浪多息并不能在体外抗菌,其真正的抗菌物质是在人体内转换后的对氨基苯磺酰胺(磺胺),二人研究了相似的结构发现具有类似效果(SD,ST,SMZ,SDM),提出了构象关系理论
\end{notation}
\begin{notation}
    西德公司生产的一对手性分子(R/S)-Thalidomide所组成的药物“反应停”,(R)-Thalidomide有镇定疗效,(S)-Thalidomide有生理毒性,产生了大量畸形胎儿
\end{notation}
3. 20世纪40-60年代:合成药物大量上市

4. 20世纪70年代至今:生物药学时期(医学、化学、生物学、计算化学等相互结合,多学科交叉渗透)
\begin{notation}
    我国药学发展仅次于美国,是世界原料药生产的第二大国

    我国97\%以上的药物是外国研制,仅在国内仿制生产

    1993年转为以创新为主,仿制为辅
\end{notation}
\subsubsection*{药物化学现状}%
\label{subsub:药物化学现状}
1. 随机合成、逐个合成$\implies$计算机辅助设计、定向合成

2. 多步骤液相合成$\implies$一步固相合成

\subsubsection*{药物制剂现状}%
\label{subsub:药物制剂现状}
1. 一般制剂$\implies$缓释、控释、速释

2. 工艺为主$\implies$与生物相结合

\begin{notation}
    现在我国已生产3000余种制剂,中成药制剂9600余种
\end{notation}
\subsubsection*{药理学现状}%
\label{subsub:药理学现状}
1. 新药筛选$\implies$高质量的机器人筛选、酶、细胞、受体筛选

2. 作用机理:整体、器官、细胞$\implies$分子、量子水平

\begin{notation}
    陈克恢院士进行了对麻黄碱的研究,成为中国药理学的奠基人

    我国在心血管药理、神经药理、生化药理等一部分已达到国际先进水平
\end{notation}
\subsubsection*{药物分析现状}%
\label{subsub:药物分析现状}
1. 化学比色$\implies$HPLC,GC(气相色谱),MS(质谱)

2. 对体内药物分析的灵敏度不断提高
\begin{notation}
    理化测试、分析仪器和计算机技术的发展大大促进了药物分析的发展
\end{notation}
\subsubsection*{生物技术与生物制药现状}%
\label{subsub:生物技术与生物制药现状}
1. 广泛应用生物技术、转基因生产药物

2. 酶不断分离纯化

3. 基因治疗
\subsubsection*{抗生素现状}%
\label{subsub:抗生素现状}
1. 单纯的开发抗菌药物$\implies$以微生物为主要来源的研究

2. 产生了酶抑制剂、免疫调节剂、受体阻断剂等
\begin{notation}
    1929年亚历山大弗莱明发现了第一个抗生素:青霉素(盘尼西林)

    中国抗生素历史:

    1949前(完全依赖进口)$\to $抗日战争间(汤飞凡开始研究)$\to $ 1950(陈毅批准建立青霉素试验所)$\to $ 1950.9(得到青霉素钾结晶)$\to $ 1953.5.1(正式生产)$\to $ 氯霉素(沈家祥)$\to $ 至今(广开菌源,应用新的筛选体系及基因工程技术)
\end{notation}

\subsubsection*{中药与天然药物现状}%
\label{subsub:中药与天然药物现状}
1. 形态学、显微水平$\implies$化学、基因水平

2. 陆地药物$\implies $ 海洋药物
\subsection{药学的任务}%
\label{sub:药学的任务}
\subsubsection*{研究新药}%
\label{subsub:研究新药}
原有的非主要致命性疾病成为主要致命疾病(癌症)

\subsubsection*{阐明药物的作用机理}%
\label{subsub:阐明药物的作用机理}
\begin{eg}
    消炎痛(吲哚美辛):副作用胃溃疡

    COX(环氧化酶)分为COX1,COX2

    COX1(结构型):副作用胃溃疡

    COX2(诱导型):副作用心血管疾病
\end{eg}
\subsubsection*{研究新的制剂}%
\label{subsub:研究新的制剂}
\subsubsection*{制定药品的质量标准、控制药品质量}%
\label{subsub:制定药品的质量标准、控制药品质量}
\subsubsection*{开拓医药市场,规范药品管理}%
\label{subsub:开拓医药市场,规范药品管理}
药物具有一般性(可以购买)和特殊性(使用不当将造成不可挽救的结果)

\begin{notation}
    研究过程的各种规范:

    1. 药用植物栽培:GAP

    2. 临床前研究:GLP

    3. 临床研究:GCP

    4. 生产:GMP(Good Manufacture Practice,药品生产质量管理规范)

    5. 销售:GSP
\end{notation}
\subsection{药学地位}%
\label{sub:药学地位}
1. “医药不分家”

2. “药食同源”

3. 药学是\textbf{独立的一级学科}
\begin{notation}
    研究一种新药需要10-15年,10-20亿美元,途径临床前、临床1、2、3期,临床末期(4期)、FDA审核

    药学“三高一长”:高技术、高投入、高风险、长周期
\end{notation}
\subsubsection*{药学和其他学科的关联}%
\label{subsub:药学和其他学科的关联}
\begin{table}[htpb]
    \centering
    \caption{药学课程}
    \label{tab:药学课程}
    \begin{tabular}{ccc}
    \toprule
    专业基本课程 & 专业课程 & 拓展课程\\
    \midrule
    化学 & 药工 & 药概\\
    高数 & 药理 & 药剂\\
    \ldots&\ldots&\ldots\\
    \bottomrule
    \end{tabular}
\end{table}

\subsection{总结}%
\label{sub:总结}
\[
    \begin{cases}
        \text{药学的定义}\\ 
        \text{研究任务}\\ 
        \text{学科关系}\\ 
        \text{药学地位}
    \end{cases}
.\] 
课后作业:从以上四个方面谈谈对药学的认识,500字以上,第3周周二前提交至djyxgl@163.com
