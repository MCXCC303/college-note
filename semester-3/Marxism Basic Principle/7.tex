\lecture{7}{10.08}
\subsubsection*{意识与人工智能}%
\label{subsub:意识与人工智能}
\begin{defi}
    人工智能:对人脑组织结构与思维运动机制的模仿,是人类智能的物化,是人的本质力量的对象化、现实化
\end{defi}
人类意识是知情意的统一体,人工智能只是对人类的理性智能的模拟和扩展,不具备情感、信念、一直等人类意识形式
\subsection{世界的物质统一性}%
\label{sub:世界的物质统一性}
世界的统一性来源于世间万物有没有共同本质或本原的问题

世界的统一性在于其物质统一性,世界统一于物质
\subsubsection*{自然界是物质的}%
\label{subsub:自然界是物质的}
自然界先于人类而存在,并不依赖人类意识而存在

自然界按是否有人类改造分为自在自然和人造自然/第二自然
\subsubsection*{人类社会本质上是生产实践基础上形成的物质体系}%
\label{subsub:人类社会本质上是生产实践基础上形成的物质体系}
人类社会是物质世界发展到一定阶段的产物,是物质世界的一部分,是物质存在的一种特定形态

世界的物质统一性原理是\textbf{辩证唯物主义最基础、最核心的观点,是马克思主义哲学的基石}
\begin{notation}
    党章规定,党的思想路线的基本内容是:一起从实际出发,理论联系实际,实事求是,在实践中检验真理和发展真理
\end{notation}

