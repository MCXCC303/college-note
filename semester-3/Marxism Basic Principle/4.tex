\lecture{4}{09.11}
\section{世界的物质性和发展规律}%
\label{sec:世界的物质性和发展规律}
\subsection{世界多样性和物质统一性}%
\label{sub:世界多样性和物质统一性}
\subsubsection*{物质及其存在方式}%
\label{subsub:物质及其存在方式}
\begin{question}
    世界是什么?世界的本质是什么?人与世界的关系是怎么样的?
\end{question}
由“世界是什么”讨论出世界观:
\begin{notation}
    个体的世界观没有体系性
\end{notation}

最开始人们用神话故事解释世界的来源
\begin{notation}
    哲学在古希腊时期作为认知世界的工具

    恩格尔指出全部哲学,特别是近代哲学的重大基本问题,是思维和存在的关系问题
\end{notation}
哲学的基本问题:
\[
    \text{思维和存在的关系问题}
    \begin{cases}
        \text{思维和存在谁决定谁}\begin{cases}
            \text{存在决定思维:唯物主义}\\ 
            \text{思维决定存在:唯心主义}\\ 
        \end{cases}\\ 
        \text{思维能不能正确反映存在}\begin{cases}
            \text{能:可知论}\\ 
            \text{不能:不可知论}
        \end{cases}
    \end{cases}
.\] 
哲学的第一阶段:古希腊哲学:世界的本原及其统一性
\begin{notation}
    泰勒斯观点:水是万物的始基

    毕达哥拉斯观点:万物皆数、数是存在由之构成的原则

    柏拉图观点:世界由理念世界和现象世界组成,理念世界真实存在,永恒不变,人类接触的仅是现象世界

    亚里士多德:事物四因说:质料因、形式因、动力因、目的因,推动事物从潜能变为现实
\end{notation}

中世纪哲学时期:
\begin{notation}
    培根,唯名论:否认共相具有客观实在性,认为共相后于事物

    安瑟尔谟,唯实论:认为一般概念是真实、独立的
\end{notation}

第二阶段:近代哲学:思维与存在是否具有统一性
\begin{notation}
    培根、贝克莱为代表的经验论哲学:与唯理论相对,主张感性经验是知识的唯一来源

    笛卡尔、莱布尼茨为代表的唯理论哲学:具有普遍必然性的可靠知识来自于从先天的自明之理出发、经过严密的逻辑推理得到的
\end{notation}
\[
    \text{世界的本原}
    \begin{cases}
        \text{一元论}\begin{cases}
            \text{唯物主义一元论:物质为本原}\\ 
            \text{唯心主义一元论:意识为本原}
        \end{cases}\\ 
        \text{二元论}\begin{cases}
            \text{德谟克利特:原子与虚空}\\ 
            \text{笛卡尔:物质与意识}
        \end{cases}
    \end{cases}
.\] 
\begin{notation}
    二元论主张物质与意识彼此独立,互不依赖均为世界的本原
\end{notation}
\[
    \text{物质与意识的统一性}\begin{cases}
        \text{可知论}\begin{cases}
            \text{彻底的唯物主义}\\ 
            \text{彻底的唯心主义}
        \end{cases}\\ 
        \text{不可知论}\begin{cases}
            \text{休谟:世界一切联系都是心理联想}\\ 
            \text{康德:存在不可认知的物自体}\\ 
            \text{中世纪:上帝只能被仰望} 
        \end{cases}
    \end{cases}
.\] 
\begin{notation}
    唯心主义的两种表现方式:

    1. 主观唯心主义:世界万物由个体的主观意识派生

    2. 客观唯心主义:在物质世界和人类产生之前独立存在一种物质精神
\end{notation}
\begin{eg}
    判断主观/客观唯心主义

    1. 人是万物的尺度:主观

    2. 宇宙便是吾心:主观

    3. 理念:客观

    4. 绝对精神(黑格尔):客观

    5. 天理(程朱理学):客观
\end{eg}
唯物主义:物质第一性、意识第二性
\[
    \text{唯物主义的三种形态}
    \begin{cases}
        \text{古代朴素唯物主义(古希腊自然哲学)}\\ 
        \text{近代形而上学唯物主义(机械唯物主义)}\\ 
        \text{辩证唯物主义(马克思主义哲学)}
    \end{cases}
.\] 
\begin{notation}
    古代朴素唯物主义:用某一种或某几种具体的物质形态解释世界

    泰勒斯/阿那克西米尼/赫拉克利特/恩培多克勒:水/气/火/土、气、火与水是万物的本原

    中国五行说:金、木、水、火、土

    德谟克利特:万物的本原是原子和虚空
\end{notation}

古代朴素唯物主义具有先进性和局限性

先进性:资源有限的条件下坚持物质第一性

局限性:缺乏科学依据
\begin{notation}
    近代形而上学唯物主义:世界的本质是物质的(自然科学的原子),原子是在一切化学变化不可再分的最小单元

    随着科技进步,微观结构层次不断进化
    \[
        \text{原子}\implies\text{原子核}\implies\text{夸克、轻子}\implies\text{原子非物质化,矛盾}
    .\]

    局限性:把极其复杂而多样的物质世界仅仅归结为某种特殊的、简单的粒子
\end{notation}
\begin{notation}
    辩证唯物主义:批判旧唯物主义对物质世界的直观、消极的理解,强调要从能动的实践出发去把握世界

    马克思主义中物质定义的科学化:所谓物质,就是不依赖于人类的意识而存在,并能为人类的意识所反映的客观存在
\end{notation}



\subsubsection*{物质与意识的辩证关系}%
\label{subsub:物质与意识的辩证关系}


\subsubsection*{世界的物质统一性}%
\label{subsub:世界的物质统一性}

