\lecture{16}{11.12}
\subsubsection*{价值评价及其特点}%
\label{subsub:价值评价及其特点}
价值评价:主题对客体的价值以及价值大小所作的评判或判断,含\textbf{肯定或否定、喜欢或反感、善或恶};正确进行价值评价需要以\textbf{真理}为依据,要\textbf{与社会历史发展的趋势和人民的需求相一致}
\subsubsection*{真理与价值在实践中的辩证统一}%
\label{subsub:真理与价值在实践中的辩证统一}
$\circ$ 对立性:
\begin{description}
    \item[真理尺度] 实践中必须遵循真理
    \item [价值尺度] 按照自己的尺度和需要改变世界
\end{description}
$\circ$ 统一性:
\begin{enumerate}
    \item 真理本身有价值
    \item 价值的实现过程离不开对真理的认识和运用
    \item 规律性和目的性统一
\end{enumerate}
$\circ$ 价值尺度必须以真理尺度为前提
\subsection{一切从实际出发、实事求是}%
\label{sub:一切从实际出发、实事求是}
\begin{notation}
    实事求是是中国共产党思想路线的核心:一切从实际出发、理论联系实际、实事求是、在实践中检验和发展真理
\end{notation}
\section{人类社会历史唯物主义}%
\label{sec:人类社会历史唯物主义}
\subsection{社会存在和社会意识}%
\label{sub:社会存在和社会意识}
\begin{notation}
    两种根本对立的历史观:
    \begin{enumerate}
        \item 唯物史观
        \item 唯心史观
    \end{enumerate}
\end{notation}
\begin{question}
    人类社会历史的创造者是谁?
\end{question}
$\circ$ 唯心主义:人类历史是思想的外化
\begin{eg}
    梁启超:历史者英雄之舞台也,舍英雄几无历史
\end{eg}
$\circ$ 唯物主义:人民是人类历史的创造者
\begin{notation}
    人口因素是加速或延缓社会发展的基本条件,对社会发展具有影响和制约作用
\end{notation}
