\lecture{18}{11.26}
考试题型:

第三章:大题少,辨证关系(生产力和生产关系、)

第四章:无大题

\begin{notation}
生产资料所有制:以生产资料公有制为基础的生产关系

生产力+生产关系=生产方式

生产力决定生产关系
\end{notation}
\begin{enumerate}
    \item 生产力状况决定生产关系性质
    \item 生产力的发展决定生产关系的变化
\end{enumerate}
\subsubsection*{生产关系反作用于生产力}%
\label{subsub:生产关系反作用于生产力}
当生产关系适应生产力的发展状况时促进生产力的发展
\begin{notation}
    不适应有两种情况:落后、超越
\end{notation}
\subsection{经济基础与上层建筑的矛盾运动及规律}%
\label{sub:经济基础与上层建筑的矛盾运动及规律}
\begin{notation}
    经济基础与经济制度之间有内在联系:经济制度是生产关系的具体实现形式
\end{notation}
社会主义市场经济体制:国家的力量可以参与到市场调控中
\begin{notation}
    我国的国体:人民民主专政

    我国的政体:人民代表大会制度

    人民民主专政决定人民代表大会制度,人民代表大会制度保障人民民主专政,社会主义民主的本质和核心是人民当家作主
\end{notation}
