\lecture{10}{10.18}
\subsection{对立统一规律是事物发展的根本规律}%
\label{sub:对立统一规律是事物发展的根本规律}
\begin{align*}
    \begin{cases}
        \text{对立统一规律}\\
        \text{量变质变规律}\\
        \text{否定之否定规律}
    \end{cases}
.\end{align*}
\begin{notation}
    对立统一规律:

    唯物辩证法的实质和核心
\end{notation}
\begin{align*}
    \text{矛盾就是对立统一}
    \begin{cases}
        \text{矛盾的同一性和斗争性}\\
        \text{矛盾的普遍性与特殊性}\\
        \text{矛盾的}
    \end{cases}
.\end{align*}
\begin{notation}
    形式逻辑的矛盾律:非此即彼

    辩证法的矛盾规律:亦此亦彼
\end{notation}
\subsubsection{同一性与斗争性在事物发展的作用}%
\label{subsub:同一性与斗争性在事物发展的作用}
矛盾是反映事物内部和事物之间对立统一关系的哲学范畴
\begin{notation}
    矛盾的斗争性:

    矛盾着的对立面之间相互排斥、相互分离的性质和趋势

    哪里有压迫,哪里就有反抗
\end{notation}
\begin{notation}
    人民内部的矛盾是非对抗性的
\end{notation}
\begin{notation}
    矛盾的对立统一是同时存在的,矛盾的同一性和斗争性相互结合,共同推动事物的发展

    在不同的条件下,矛盾的同一性和斗争性占比不同

    方法论:应正确把握和谐对事物发展的作用
\end{notation}
\subsubsection{普遍性和特殊性}%
\label{subsub:普遍性和特殊性}
矛盾的特殊性:

1. 同一事物在不同阶段有不同矛盾

2. 不同事物的矛盾不同

3. 构成事物的诸多矛盾和每一矛盾的不同方面有不同的性质、地位和作用
\begin{notation}
    矛盾的特殊性决定了事物的不同性质
\end{notation}
社会主义是矛盾的主要方面,矛盾的主要方面决定社会的性质

矛盾的共性和个性可以体现矛盾的普遍性和特殊性
\begin{notation}
    没有离开个性的共性,也没有离开共性的个性
\end{notation}
