\lecture{14}{11.01}
\subsubsection{实践对认识的决定作用}%
\label{subsub:实践对认识的决定作用}
$\circ$ 实践是认识的来源
\[
    \text{认识}
    \begin{cases}
        \text{直接经验:亲身实践}\\
        \text{间接经验:教育学习}
    \end{cases}
.\] 

$\circ$ 实践是认识发展的动力

$\circ$ 实践是认识的目的

$\circ$ 实践是检验真理的唯一标准
\subsection{认识的本质与过程}%
\label{sub:认识的本质与过程}
\begin{defi}
    认识论:
    \begin{description}
        \item[唯物主义认识论] 主体对客体的反映
        \item[唯心主义认识论] 认识是先于人的实践
    \end{description}
\end{defi}
\begin{notation}
    洛克“白板说”:心灵是一块白板,一切知识来源于经验
\end{notation}
辩证唯物主义的突出特点:
\begin{itemize}
    \item 把实践的观点引入认识论
    \item 把辩证法用于反映论考察认识的发展过程
\end{itemize}
辨证主义认识论的反映具有反映的\textbf{摹写性}和\textbf{创造性} 
\subsubsection*{从实践到认识}%
\label{subsub:从实践到认识}
\begin{defi}
    感性认识:在实践的基础上由感觉器官直接感受到的关于事物的现象
\end{defi}
感性认识包括三种形式:
\begin{description}
    \item[感觉] 对事物个别属性的反映
    \item[知觉] 对事物外部特征整体形象的反映
    \item[表象] 对过去的感觉和知觉的回忆和再现
\end{description}
\begin{defi}
    理性认知:对同类事物共同的一般特性和本质属性的概况和反映
\end{defi}

\begin{notation}
    感性认识和理性认识的辨证关系:
    \begin{itemize}
        \item 没有感性认识就没有理性认识
        \item 感性认识有待于发展为理性认识
    \end{itemize}

    割裂两者会犯错误:
    \begin{description}
        \item[经验论] 经验主义错误
        \item[唯理论] 教条主义错误
    \end{description}
\end{notation}
\begin{notation}
    人们对事物的认识受到主客观条件的限制

    客观现实世界的变化会影响主客观认识的变化
\end{notation}


