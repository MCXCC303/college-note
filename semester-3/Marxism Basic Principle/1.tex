\lecture{1}{09.02}
\section*{课程介绍}%
\label{subsub:课程介绍}
24节课(1-18周,48学时)

3学分

\subsection*{要求}%
\label{sub:要求}
闭卷考试

成绩组成:平时(40\%)+期末(60\%)

平时成绩:点名(10\%,一次缺勤-2\%)+读书报告(10\%,学期内阅读一本经典文献)+课堂作业(20\%,共2-3次,随堂写作)

请假时未在课堂上完成的作业与假条一同提交

不提交假条的补交作业较平时评分低

教材:马克思主义基本原理

\subsection*{期末考试范围}%
\label{sub:期末考试范围}
导论,第一到三章,第四章1,2节

总分(100\%)=单选(20\%,20道)+多选(10\%,5道,选多与选少都不得分)+辨析(30\%,需写出判断和正误原因)+简答(20\%,题干上存在题目要点)+材料分析(20\%)
\subsection*{课堂要求}%
\label{sub:课堂要求}

\section{导论}%
\label{sec:导论}
马克思主义基本原理

马克思:千年第一思想家,千年伟人
\subsection{什么是马克思主义}%
\label{sub:什么是马克思主义}
\begin{notation}
    马克思主义是由马克思与恩格斯创立并为后继者所不断发展的科学理论体系,是关于自然、社会和人类思维发展一般规律的学说,是关于社会主义必然代替资本主义,最终实现共产主义的学说,是关于无产阶级解放、全人类解放和每个人资源而全面发展的学说,是指引人民创作美好生活的行动指南
\end{notation}
马克思主义理论是一个博大精深的理论体系:
\[
    \text{马克思主义}
    \begin{cases}
        \text{马克思主义哲学}\\ 
        \text{马克思主义政治经济学}\\ 
        \text{科学社会主义}
    \end{cases}
.\] 
还包括其他如历史学,政治学,法学,文化学,新闻学,军事学等,并随着实践和科学的发展而不断丰富自身的内容

党中央成立了中央编译局,系统地编译马克思主义经典著作,至今已有2版翻译,全集共30卷

\[
    \text{马克思主义基本原理}
    \begin{cases}
        \text{马克思主义基本立场}\\ 
        \text{马克思主义基本观点}\\ 
        \text{马克思主义基本方法}
    \end{cases}
.\] 
\begin{notation}
    马克思主义的基本立场是\textbf{人民立场},以人民为中心,一切为了人民,一切依靠人民
\end{notation}
\begin{notation}
    马克思主义的基本方法:
\end{notation}
\begin{notation}
    马克思主义的基本观点:
\end{notation}

\subsection{马克思主义的创立}%
\label{sub:马克思主义的创立}
马克思主义产生于19世纪40年代,创始人是马克思与恩格斯

马克思主义的产生与创立具有深刻的社会根源、阶级基础和思想渊源

\begin{notation}
    社会根源:18世纪60年代到19世纪,工业革命和科技进步极大地提高了劳动生产率,促进生产力的发展

    工业革命一方面带来了社会化大生产的迅猛发展,另一方面造成了深重的社会灾难

    财富增加伴随着贫困的扩散,生产的发展却引发经济危机

    阶级基础:无产阶级反对资产阶级的斗争,19世纪30-40年代爆发了“三大工人运动”:丝织工人起义(1831,法国里昂)、宪章运动(1836,英国)、纺织工人起义(1841,德国)

    工人阶级作为独立的政治力量已经登上了历史舞台,这种斗争必须有革命理论的知道和无产阶级政党的领导

    思想渊源:
    \[
        \text{直接理论来源}\\ 
        \begin{cases}
            \text{德国古典哲学(黑格尔):辩证法}\\ 
            \text{英国古典政治经济学(亚当斯密):劳动价值论}\\ 
            \text{空想社会主义}
        \end{cases}
    .\] 
\end{notation}

