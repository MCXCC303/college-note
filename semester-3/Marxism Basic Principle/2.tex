\lecture{2}{09.06}
\begin{notation}
    古典政治经济学认为资本是永恒的自然关系,掩饰资本主义的基本矛盾,否定资本主义经济危机的可能性
\end{notation}
\begin{notation}
    空想社会主义主要的缺陷:其描述的世界运作状态尚未提出合理的实现方式

    1516年:《乌托邦》确定了空想社会主义

    柏拉图著有《理想国》

    1848年:科学社会主义建立
\end{notation}
\begin{eg}
    马克思主义创立的直接理论来源:(BCD)

    A. 中国古典哲学

    B. 德国古典哲学

    C. 英国古典政治经济学

    D. 空想社会主义
\end{eg}
\begin{notation}
    马克思:律师家庭出身,17岁写下《青年在选择职业时的考虑》,1841年就读波恩、柏林大学,在耶拿大学获得PhD;1842年5月,任《莱茵报》编辑工作;1843年4月,因工作问题离开祖国终身漂泊

    1844年2月,马克思和恩格斯统一了唯物主义思想;1844年8月,马克思与恩格斯相遇

    马克思和恩格斯合作了《神圣家族》;1845-1846,合作《德意志意志形态》;1848年,《共产党宣言》正式发表,标志着马克思主义的公开问世;后来马克思回到德国创建了《新莱茵报》

    1867年9月,发表《资本论》第一卷,系统阐述了剩余价值学说;1871年代表第一国际著《法兰西内战》;1876-1878年,恩格斯著《反杜林论》,全面阐述马克思主义理论体系
    
    马克思一生流离各地,途经德国、法国、比利时等地区,于1883年去世

    马克思去世后,恩格斯整理了《资本论》二、三卷
\end{notation}
\subsection{马克思主义的发展}%
\label{sub:马克思主义的发展}
\begin{notation}
    时代背景

    客观:战争加剧了资本主义国家的内部矛盾(第一次世界大战)

    主观:俄国成为帝国主义各种矛盾的焦点与集合点(无产阶级/封建地主阶级/资产阶级,自助游/农奴制,民族矛盾等),因此成为了帝国主义提醒等中最薄弱的环节
\end{notation}

