\lecture{5}{09.24}
\[
    \text{哲学的物质范畴}
    \begin{cases}
        \text{古代朴素唯物主义物质观:物质=具体的物质形态}\\
        \text{近代形而上学唯物主义物质观:物质=原子}\\
        \text{辩证唯物主义物质观:物质=客观实在性}
    \end{cases}
.\] 
由上至下:个性$\to $ 部分共性$\to $ 共性

马克思主义的物质范畴抽象出了现实中的自然与社会存在:客观实在性
\[
    \begin{cases}
        \text{坚持了唯物主义一元论}\\
        \text{坚持能动的反映论和可知论}\\
        \text{体现了唯物论和辩证法的统一}\\
        \text{体现了唯物主义自然观和唯物主义历史观的统一}
    \end{cases}\Leftrightarrow\begin{cases}
        \text{批判了二元论}\\
        \text{批判了不可知论}
    \end{cases}
.\]
\begin{notation}
    哲学史的两个对立:
    
    1. 世界的本原是什么

    1.1. 唯物主义

    1.2. 唯心主义

    2. 世界的存在状态是什么

    2.1. 形而上学

    2.2. 辩证法
\end{notation}
形而上学:用孤立、静止、片面的思维方式看问题

辩证法:用联系的、发展的观点看世界
\begin{notation}
    形而上学代表:柏拉图
\end{notation}
\begin{notation}
    物质与运动的关系:

    物质的根本属性是运动,物质和运动不可分割:运动是物质的运动,物质是运动着的物质
    
    如果没有运动的物质,将变为形而上学
    
    如果没有物质的运动,将变为唯心主义
\end{notation}
\begin{defi}
    相对静止:物质运动在一定条件下、一定范围内处于暂时稳定和平衡
\end{defi}
\begin{notation}
    运动和静止的关系:

    运动是绝对的、无条件的;静止是相对的、有条件的

    静中有动、动中有静
\end{notation}
\begin{eg}
    赫拉克利特:人不能两次踏入同一条河流(绝对运动、相对静止)

    克拉底鲁:人连一次也不能踏入同一条河流(否认相对静止)
\end{eg}
\subsection{物质和意识的辩证关系}%
\label{sub:物质和意识的辩证关系}
\subsubsection*{物质决定意识}%
\label{subsub:物质决定意识}
物质对意识的决定作用表现在意识的起源和本质上

从意识的起源:一方面,意识是自然界长期发展的产物;另一方面,意识是社会历史的产物:社会实践,特别是劳动,在意识形成中起到决定性作用

\begin{notation}
    意识在内容上是客观的,在形式上是主观的

    意识是客观内容和主观形式的统一
\end{notation}

