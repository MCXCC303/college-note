列宁认为:资本主义发达国家已经发展到帝国主义阶段,经济政治发展的不平衡已经成为资本主义发展的绝对规律,并提出社会主义革命可能在一国或数国首先发生并取得胜利的论断

\begin{notation}
    十月革命的胜利确定了列宁的正确,列宁把马克思主义基本原理和俄国本地情况相结合,成立了列宁主义(布尔什维克主义),后来出版列宁全集
\end{notation}
李大钊、陈独秀等人受马列主义影响,以《新青年》和《每周评论》为阵地传播马克思主义

中共一大成立了中国共产党,并印制了共产党宣言

\begin{notation}
    马克思主义中国化的伟大成果:

    1. 毛泽东思想

    2. 邓小平理论

    3. “三个代表”重要思想

    4. 科学发展观

    5. 习近平新时代中国特色社会主义思想
\end{notation}
\begin{notation}
    现存的社会主义国家:中国、朝鲜、越南、古巴、老挝
\end{notation}

\subsection{马克思主义的基本特征}%
\label{sub:马克思主义的基本特征}
\[
    \begin{cases}
        \text{科学的理论}\\ 
        \text{人民的理论}\\ 
        \text{实践的理论}\\ 
        \text{不断发展的开发的理论}
    \end{cases}
.\] 
\subsubsection{科学的理论}%
\label{subsub:科学的理论}
马克思主义是对自然、社会、人类思维发展本质和规律的\textbf{正确反映}

1. 马克思主义在社会实践和科学发展的基础上产生,并在发展过程中不断总结经验,\textbf{吸取自然科学和社会科学发展的最新成就}

2. 马克思主义具有科学的世界观和方法论基础

3. 马克思主义理论是一个逻辑严密的有机整体

\subsubsection{人民的理论}%
\label{subsub:人民的理论}
“人民至上”是马克思主义的政治立场:人民群众是历史的创造者,是社会主义事业的依靠力量
\begin{notation}
    中国共产党人的初心和使命就是\textbf{为中国人民谋幸福},为中华民族谋复兴
\end{notation}
\subsubsection{实践的理论}%
\label{subsub:实践的理论}
1. 马克思主义的使命和作用决定了它是直接服务于无产阶级和人民群众改造世界的实践活动的科学理论

2. 马克思主义的内容反映了实践观点是马克思主义首要的和基本的观点,始终强调理论和实践统一
\subsubsection{发展的理论}%
\label{subsub:发展的理论}
马克思主义具有与时俱进的理论品质,是时代的产物

1. 马克思主义理论体系是开放的

2. 当今世界和我们所处的新时代在不断变化
\begin{notation}
    马克思主义的鲜明特征是\textbf{科学性与革命性的统一}
\end{notation}
