\lecture{15}{11.05}
\textit{Review:}

新事物代替旧事物是事物发展的总趋势
\subsection{真理与价值}%
\label{sub:真理与价值}
关于真理的认识:
\begin{table}[htpb]
    \centering
    \caption{真理的认识}
    \label{tab:真理的认识}
    \begin{tabular}{cc}
    \toprule
    哲学家 & 观点\\
    \midrule
    柏拉图 & 真理是某种超验的、永恒的理念\\
    休谟 & 真理是观念与主题感觉相符合\\
    贝克莱 & 真理存在于理念之中\\
    康德 & 真理是死亡而与他的先验形式一致\\
    黑格尔 & 真理是绝对理念的自我显现\\
    \bottomrule
    \end{tabular}
\end{table}
其中柏拉图、黑格尔为客观唯心主义的代表,其他为主观唯心主义的代表
\begin{notation}
    真理的客观性:真理的内容是对客观事物及其规律的正确反映(内容的客观性)

    客观性是真理的本质属性,但是真理的形式是主观的
\end{notation}
\begin{notation}
    真理的客观性决定了真理的一元性(针对真理的客观内容而言)

    即:真理的内容唯一,形式多样
\end{notation}
\subsubsection*{真理的绝对性和相对性}%
\label{subsub:真理的绝对性和相对性}
\begin{notation}
    独断论:绝对主义真理观

    绝对主义真理观否定真理的相对性,夸大真理的绝对性,认为真理永恒不变

    这样的观点是错误的
\end{notation}
\subsubsection*{真理和谬误}%
\label{subsub:真理和谬误}
真理和谬误的区别:认识是否正确反映了客观事物及其规律

谬误是对客观事物及其发展规律的歪曲反映
\begin{notation}
    真理和谬误的关系:

    $\circ$ 真理和谬误相互独立:在确定的条件下,一种认识不能既是真理又是谬误

    $\circ$ 真理和谬误的对立是相对的,真理在超出一定范围的时候会转为谬误,同理谬误也可以转为真理
\end{notation}
\subsection{真理的检验标准}%
\label{sub:真理的检验标准}
过去的观点:
\begin{table}[htpb]
    \centering
    \caption{代表性人物和观点}
    \label{tab:代表性人物和观点}
    \begin{tabular}{ccc}
    \toprule
    真理标准 & 代表性观点 & 代表人物\\
    \midrule
    权威 & 君子有三畏 & 孔子\\
    自我意见 & 鄙夫自知的是非便是他本来天则 & 王阳明\\
    \multirow{2}{*}{群体意见} & 集体的知觉就是实在性的证据 & 贝克莱\\
                              & 真理可能在少数人一边 & 柏拉图\\
    形式 & 凡是清晰明确被人认知的观点都是真理 & $\ldots $ \\
    \bottomrule
    \end{tabular}
\end{table}
\begin{notation}
    \textbf{实践是检验真理的唯一标准}
\end{notation}
\[
    \text{实践是检验真理的唯一标准}
    \begin{cases}
        \text{真理的本质:主管和客观相结合}\\
        \text{实践的特点:直接现实性}
    \end{cases}
.\] 
\begin{eg}
    伽利略比萨斜塔落球实验
\end{eg}
\subsubsection*{实践标准的不确定性}%
\label{subsub:实践标准的不确定性}
\subsection{真理与价值的辩证统一}%
\label{sub:真理与价值的辩证统一}
过去的价值论:

$\circ$ 客观主义价值论:物体的价值来源于其本身的作用

$\circ$ 主观主义价值论:物体的价值是主体的主观论断,与客体无关
\begin{notation}
    $\circ$ (马克思)关系说:物体价值的体现是主客体的一种关系,具有积极的意义
\end{notation}
\subsubsection*{价值的基本特征}%
\label{subsub:价值的基本特征}
\begin{itemize}
    \item 主体性
    \item 客观性
    \item 多维性
    \item 社会历史性
\end{itemize}
