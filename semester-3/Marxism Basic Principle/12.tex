\lecture{12}{10.25}

\subsection{唯物辩证法}%
\label{sub:唯物辩证法}
\[
    \text{唯物辩证法总论}
    \begin{cases}
        \text{特征}\\
        \text{规律}\begin{cases}
            \text{对立统一}\\
            \text{量变质变}\\
            \text{否定之否定}
        \end{cases}\\
        \text{结果}
    \end{cases}
.\] 
\section{实践与认识及发展规律}%
\label{sec:实践与认识及发展规律}
\[
    \begin{cases}
        \text{实践与认识}\\
        \text{真理与价值}\\
        \text{认识和改造世界}
    \end{cases}
.\] 
\subsection{科学的实践观和意义}%
\label{sub:科学的实践观和意义}
\begin{notation}
    马克思强调全部的社会生活在本质上是实践的
\end{notation}
\begin{defi}
    实践:能动地改变物质活动
\end{defi}


