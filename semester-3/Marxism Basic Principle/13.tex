\lecture{13}{10.29}
实践的基本特征:
\begin{itemize}
    \item 客观实在性
    \item 自觉能动性
    \item 社会历史性
\end{itemize}
\begin{notation}
    客观实在性:

    1. 构成实践活动的诸要素客观实在

    2. 受到客观条件制约

    3. 能够引起客观世界的某种变化
\end{notation}
\begin{notation}
    自觉能动性:人的实践活动是一种有意识、有目的的活动
\end{notation}
\subsubsection{实践的基本结构}%
\label{subsub:实践的基本结构}
\begin{defi}
    实践主体:具有一定主体能力,从事现实社会实践活动的人
\end{defi}
实践的主体、客体和终结是不断发展的
\begin{notation}
    客体主体化:客体失去客体性的形式,成为主体的一部分
\end{notation}
\begin{notation}
    主题客体化
\end{notation}
\subsubsection{实践形式的多样性}%
\label{subsub:实践形式的多样性}
物质生产实践在马克思主义基本原理中被纳入实践中
\[
    \begin{cases}
        \text{社会政治实践:具有鲜明的阶级性}\\
        \text{物质生产实践}\\
        \text{科学文化实践:科学、艺术、教育}\\
        \text{(虚拟实践:三种类型在网络空间中的实践)}
    \end{cases}
.\] 
\begin{notation}
    物质生产实践是最基础的实践
\end{notation}



