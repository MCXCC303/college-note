\lecture{11}{10.22}
\subsection{量变质变规律和否定之否定规律}%
\label{sub:量变质变规律和否定之否定规律}
\begin{notation}
    度:两个临界点之间的幅度
\end{notation}
1. 量变是质变的必要准备

2. 质变是量变的必然结果

3. 量变和质变是相互渗透的
\begin{notation}
    否定之否定规律揭示了事物发展的前进性和曲折性的统一
\end{notation}
\subsection{联系与发展的基本环节}%
\label{sub:联系与发展的基本环节}
形式具有相对独立性,同一内容可以通过多种形式来体现
\subsubsection{现实与可能}%
\label{subsub:现实与可能}
现实:相互联系的实际存在的事物的综合

可能:包含在事物中预示事物发展前途的信息

