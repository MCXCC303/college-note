\lecture{17}{11.19}
\begin{notation}
    物质生产方式是社会历史发展的决定力量
\end{notation}
\subsubsection*{社会意识}%
\label{subsub:社会意识}
艺术和宗教都具有阶级属性,马克思认为宗教是一种精神压迫(马克思主义宗教观),批判作为社会意识形态的宗教;当代中国应该深入开展马克思主义宗教观教育,引导宗教与社会主义社会相适应

哲学是系统化、理论化的世界观,与其他相比较,哲学有思辨的特点,用间接、抽象的方式反映社会存在的意识形态

\subsubsection*{社会存在和社会意识的辨证关系}%
\label{subsub:社会存在和社会意识的辨证关系}
社会意识具有相对独立性:
\begin{notation}
社会意识在从根本上受到社会存在决定的通水还具有自身特有的发展规律
\end{notation}
表现:
\begin{enumerate}
    \item \textbf{不完全同步性}和不平衡状态
    \item 内部形式之间相互影响且各具\textbf{历史继承性}
    \item 社会意识对社会存在具有能动的\textbf{反作用}(突出表现)
\end{enumerate}
\begin{notation}
    社会基本矛盾:
    \begin{itemize}
        \item 生产力和生产关系之间
        \item 经济基础与上层建筑的矛盾
    \end{itemize}
\end{notation}
1. 生产力和生产关系:生产力是物质力量

生产力的结构:
\[
    \text{实体性要素}
    \begin{cases}
        \text{劳动资料/劳动手段:生产工具,决定生产力发展水平}\\
        \text{劳动对象}\\
        \text{劳动者}
    \end{cases}
.\]
\begin{notation}
    劳动资料是人类劳动力发展的测量工具,经济时代的区别不在于生产什么,而在于怎么生产
\end{notation}
\begin{notation}
劳动者:生产力中最活跃的因素,包括脑力劳动者和体力劳动者wq
\end{notation}
