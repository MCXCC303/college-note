\lecture{2}{}
\begin{notation}
    课程有闭卷考试(60\%),9-10次作业和2次报告(40\%)

    考试基于课上内容
\end{notation}
\section{人工智能发展历程}%
\label{sec:人工智能发展历程}
人工智能发展开始:1956年
\subsection*{孕育期:1956年前}%
\label{sub:孕育期:1956年前}
\begin{notation}
    1943年麦克洛奇和皮兹建成第一个神经网络模型(MP模型)

    1949年提出了Hebb规则(激发函数规则)
\end{notation}
神经网络的一些标准:神经元层数、个数,激发函数,连接方式(全连接/非全连接),\textbf{权重},$\ldots\ldots$
\subsection*{第一次低谷期:1957-1973}%
\label{sub:第一次低谷期:1957-1973}

\subsection*{形成期:1974-1980}%
\label{sub:形成期:1974-1980}

\subsection*{黄金期:1980-1987}%
\label{sub:黄金期:1980-1987}
专家系统出现:MYCIN,PROSPECTOR,XCON等

AI被引入市场:Rumelhart提出BP(反向传播)算法,实现多层神经网络学习

\subsection*{第二次低谷期:1987-1993}%
\label{sub:第二次低谷期:1987-1993}
专家系统难以使用、升级、维护,AI未能完成既定目标

\subsection*{平稳期:1993-2011}%
\label{sub:平稳期:1993-2011}

\subsection*{蓬勃期:2012至今}%
\label{sub:蓬勃期:2012至今}

\subsection*{小结}%
\label{sub:小结}

\begin{notation}
    图灵测试:在封闭的房间中,一个人分别对两个对象询问并获得答案,两个对象分别是AI和人类,判断AI是否具备人类的特征
\end{notation}

\begin{notation}
    人工智能三大学派:

    1. 符号学派

    2. 连接主义

    3. 行为主义
\end{notation}
