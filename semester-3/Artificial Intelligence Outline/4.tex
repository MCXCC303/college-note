\lecture{4}{}
\begin{table}[htpb]
	\centering
	\caption{命题的五种连结词}
	\label{tab:命题的五种连结词}
	\begin{tabular}{ccccc}
	\toprule
	$\land$(and) & $\lor$(or) & $\neg$(not) & $\Rightarrow$ (implies) & $\Leftrightarrow$(forth and come)\\
	\bottomrule
	\end{tabular}
\end{table}

\begin{table}[htpb]
    \centering
    \caption{命题真值表}
    \label{tab:命题真值表}
    \begin{tabular}{ccccccc}
    \toprule
    $p$ & $q$& $\neg p$& $p\land q$& $p\lor q$& $p\Rightarrow q$& $p\Leftrightarrow q$\\
    \midrule
    F&F&T&F&F&T&T\\
    F&T&T&F&T&T&F\\
    T&F&F&F&T&F&F\\
    T&T&F&T&T&T&T\\
    \bottomrule
    \end{tabular}
\end{table}

\begin{notation}
	充分条件:$p\subset q\text{即} p\Rightarrow q$
\end{notation}

\begin{eg}
	符号化:

	1. 铁和氧化合但铁和氮不化合:
	\[
		p= \text{铁和氧化合,}q=\text{铁和氮化合}
	.\] 
	\[
		\text{Org: }p \lor \left( \neg q \right) 
	.\] 

	2. 小张或小明通过了CET6
	\[
		p=\text{小张通过CET6,}q=\text{小明通过CET6}
	.\] 
	\[
		\text{Org: }p \lor q
	.\] 

	3. 如果我下班早,就去商城看看,除非我很累
	\[
		p=\text{我下班早,}q=\text{我很累,}r=\text{我会去商城}
	.\] 
	\[
		\text{Org: }\left( p\land \left( \neg q \right)  \right) \Rightarrow r
	.\] 
\end{eg}

\begin{notation}
	命题逻辑的优劣:

	优点:能把客观世界的各种事实转化为逻辑命题

	缺点:不适合表达复杂问题、细节缺省
\end{notation}

\subsection{谓词逻辑}%
\label{sub:谓词逻辑}
\begin{defi}
	谓词逻辑:一种形式语言,接近自然语言,方便计算机处理
\end{defi}
谓词:用于刻画个体的性质、状态和个体之间关系的成分
\begin{eg}
	$x$ 是$A$ 类型的命题使用$A\left( x \right) $ 表达

	$x$ 大于$y$ 可表达为$B\left( x,y \right) $ 

	$A\left( x \right) $ 称为一元谓词,$B\left( x,y \right) $ 称为二元谓词

	用$P\left( x_1,x_2,\ldots,x_{n} \right) $ 表示一个$n$ 元谓词公式,$P$ 为$n$ 元谓词,$x_1,x_2,\ldots,x_{n}$ 为客体变量或变元

	定义谓词$U\left( x \right) $表示$x$ 为大学生,该谓词可以记录相关的属性
\end{eg}
语法元素:

1. 常量(个体符号):通常是对象的名称

2. 变量符号:小写字母

3. 函数符号:小写英文字母$f,g$ 等

\begin{eg}
	我喜欢音乐和绘画:
	\[
		\text{Like}\left( \text{I},\text{Music} \right) \land \text{Like}\left( \text{I},\text{Painting} \right) 
	.\] 
\end{eg}
连词:

1. 与/合取:$\text{Like}\left( \text{I},\text{Music} \right) \land \text{Like}\left( \text{I},\text{Painting} \right) $ 

2. 或/析取

\begin{notation}
	全称量词$\forall $ 
	
	\begin{eg}
		所有机器人都是灰色的:
		\[
			\forall \left( x \right) [\text{Robot}\left( x \right) \to \text{Color}\left( x,\text{gray} \right) ]
		.\] 
	\end{eg}
	
\end{notation}
\begin{notation}
	存在量词$\exists $ 
	\begin{eg}
		1号房间有一个物品:
		\[
			\exists \left( x \right) \text{InRoom}\left( x,R_1 \right) 
		.\] 
	\end{eg}
	
\end{notation}
函数和命题的区别:

函数是定义域到值域的映射

命题是定义域到$\left\{ \text{True},\text{False} \right\} $ 的映射
\begin{eg}
	符号化“所有数的平方是非负的”:

	1. 个体$x$

	2. 函数符号$f$ :某数的平方

	3. 谓词$Q$ :某个数是非负的

	4. 符号化:$\left( \forall x \right) Q\left( f\left( x \right)  \right) $ 

	第二种:

	1. 个体$z$ :表达一个数

	2. 谓词$R$ :$x$ 是一个实数

	3. 函数符号$f$ 

	4. 谓词$Q$ 

	5. 符号化:$\left( \forall z \right) [R\left( z \right) \to Q\left( f\left( z \right)  \right)] $
\end{eg}

\subsubsection*{谓词逻辑推理形式化}%
\label{subsub:谓词逻辑推理形式化}
\begin{eg}
	所有人都要死,孔子是人,所以孔子会死:
	\[
		\left( \forall x \right) \left( A\left( x \right) \to B\left( x \right) \right) \land A\left( \text{Confucious} \right) \to B\left( \text{Confucious} \right) 
	.\] 
\end{eg}

\begin{notation}
	谓词逻辑的优点:自然性、精确性、易实现

	缺点:不能表示不确定性知识,过于自由而兼容性差

	应用:

	1. 自动问答系统

	2. 机器人行动规划系统

	3. 机器博弈系统

	4. 问题求解系统
\end{notation}
作业:第二章 1-18题
\subsection{产生式知识表示法}%
\label{sub:产生式知识表示法}
\begin{notation}
	确定性规则知识产生式:
	\[
		P\to Q
	.\] 
	
	不确定性规则知识产生式:
	\[
		P\to Q \left( \text{Conf} \right) 
	.\] 

	确定性规则知识产生式表示:
	\[
		\left( \text{Relate},a,b \right) 
	.\] 

	不确定性规则知识产生式表示:
	\[
		\left( \text{Relate},a,b,\text{Conf} \right) 
	.\] 
\end{notation}


