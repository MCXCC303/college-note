\lecture{9}{10.21}
\begin{center}
    \begin{tikzpicture}
        \draw [->] (0,0)--(1,3) node [anchor=south west] {A};
        \draw [->] (0,0)--(5,1) node [anchor=south west] {B};
        \draw [->] (0,0)--(2,-4) node [anchor=north west] {C};
        \node at(0,2) {5};
        \node at(3,1) {4};
        \node at(2,-2) {7};
        \node at(-0.7,-0.5) {$g\left( s \right) =0$};
        \node at(-2,-2) {OPEN: $\left\{ B:4,A:5,C:7 \right\} $};
        \node at(-2,-2.5) {CLOSED: $\left\{ s:0 \right\} $};
        \filldraw [color=white] (0,0) circle [radius=2pt];
        \draw [] (0,0) circle [radius=2pt] node [anchor=east] {start};
    \end{tikzpicture}
    \begin{tikzpicture}
        \draw [->] (0,0)--(0.5,1.5) node [anchor=south west] {A};
        \draw [->] (0,0)--(2.5,0.5) node [anchor=south] {B};
        \draw [->] (0,0)--(1,-2) node [anchor=north] {C};
        \draw [->] (2.5,0.5)--(5.5,1) node [anchor=west] {D};
        \draw [->] (2.5,0.5)--(4,0) node [anchor=west] {E};
        \draw [->] (2.5,0.5)--(3.5,-1) node [anchor=north west] {F};
        \draw [->] (0.5,1.5)--(3.5,-1) node at(1.6,1) {9};
        \node at(4,1) {4};
        \node at(3.6,0.4) {2};
        \node at(3.3,-0.3) {3};
        \node at(-3.5,2) {OPEN=$\left\{ A:5,E:6,C:7,F:7,D:8 \right\} $};
        \node at(-2,-1) {CLOSED=$\left\{ s:0,B:4 \right\} $};
        \filldraw [color=white] (2.5,0.5) circle [radius=2pt];
        \filldraw [color=white] (0,0) circle [radius=2pt];
        \draw [] (0,0) circle [radius=2pt] node [anchor=east] {start};
        \draw [] (2.5,0.5) circle [radius=2pt];
    \end{tikzpicture}
    \begin{align*}
        g\left( D \right) =g\left( B \right) +g\left( B,D \right) =4+4=8
    .\end{align*}
\end{center}
\subsection{启发式搜索}%
\label{sub:启发式搜索}
\begin{notation}
    盲目搜索的不足:效率低、组合爆炸、产生大量无用节点
\end{notation}
\begin{notation}
    启发式信息:与具体问题求解过程有关的,指导搜索过程朝最可能前进方向的数据
\end{notation}
\begin{notation}
    A算法:

    引入估价函数:$f\left( n \right) =g\left( n \right) +h\left( n \right) $ 

    $g\left( n \right) $ :从起始状态到当前状态已实际付出的代价

    $g\left( n \right) $ 从当前状态到目标状态的估计代价(启发函数)
\end{notation}
\begin{eg}
    错位个数:与目标状态的比较差别
    \begin{table}[htpb]
        \centering
        \begin{tabular}{|c|c|c|}
        \hline
        2 & 8 & 3 \\
        \hline
        1 & 6 & 4 \\
        \hline
        7 &   & 5 \\
        \hline
        \end{tabular}
        $\ce{->[\text{目标状态}][\text{错位个数:4(不包含空格)}]}$
        \begin{tabular}{|c|c|c|}
        \hline
        1 & 2 & 3 \\
        \hline
        8 &   & 4 \\
        \hline
        7 & 6 & 5 \\
        \hline
        \end{tabular}
    \end{table}
     \[
         g\left( n \right) =0 \hspace{0.5cm} h\left( n \right) =4
     .\] 
     可得$f\left( n \right) =g\left( n \right) + h\left( n \right) =4 $

     类似于等代价算法,通过比较估价函数值即可减少遍历节点数
\end{eg}
\begin{notation}
    $A^\star$算法:对函数进行限定,使其一定可以找到最优解
\end{notation}
\[
    A^\star=g\left( n \right) +h\left( n \right) 
.\] 

$g\left( n \right) $ 为起点到$n$ 点已走过距离

$g^\star\left( n \right) $ 是起点到$n$点的最短路径

$g\left( n \right) $是对$g^\star\left( n \right) $的估计

$h\left( n \right) $ 为引导从$n$ 点到目的地的参照距离,一般为欧氏距离$L_2\left( \bm{x}_i,\bm{x}_j \right) $

$h^\star\left( n \right) $ 为从$n$ 点到目的地的实际最短距离,$h\left( n \right) \le h^\star\left( n \right) $
\begin{eg}
    百度地图:一直有一条红线引导方向,该红线即是$h\left( n \right) $ 

    确定的路线为绿色,为$h^\star\left( n \right) $
\end{eg}
\begin{eg}
    八数码难题:$h_1\left( n \right) $ 表示不在位置上的数字数量

     $h_2\left( n \right) $ 表示节点$n$ 到目标位置的曼哈顿距离之和

     易得$0\le h\left( n \right) \le h_1\left( n \right) \le h_2\left( n \right) $
\end{eg}

