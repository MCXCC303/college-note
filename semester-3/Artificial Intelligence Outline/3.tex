\begin{notation}
    行为主义的代表性成果:蚁群算法、粒子群算法
\end{notation}
比较三种主流方法:

\begin{table}[htpb]
    \centering
    \caption{学习模式}
    \label{tab:学习模式}
    \begin{tabular}{ccc}
    \toprule
    符号主义 & 连接主义 & 行为主义\\
    \midrule
    与人类逻辑类似 & 直接从数据中学习 & 从经验中持续学习\\
    \bottomrule
    \end{tabular}
\end{table}

第一章作业:1-19题
\section{人工智能的知识表示}%
\label{sec:人工智能的知识表示}
\subsection{概述}%
\label{sub:概述}
研究人工智能的目的:使其得以模拟、延伸、扩展,
\begin{notation}
    人是一个物理符号系统
\end{notation}
为使人工智能达到相应的功能:将知识破译、重新编码、建立相应的符号系统
\begin{notation}
    知识的层次:

    现象$\implies$ 数据$\implies$ 信息$\implies$ 知识$\implies$ 智慧

    数据:一些无关联的现象

    数据$\to$ 信息:组织、分析

    信息$\to$ 知识:解释、评价

    知识$\to$ 智慧:理解、归纳
\end{notation}
\begin{eg}
    数据:下雨了,温度下降至15度

    信息:地面水蒸发,遇冷暖峰过境

    知识:理解下雨、蒸发、空气状况、地形、风向等及其中的作用机理

    智慧:模拟天气变化,人工天气可控化
\end{eg}
\begin{notation}
    知识的特性:

    1. 相对正确性

    2. 不确定性

    3. 可表示性和可利用性
\end{notation}
\begin{question}
    如何将人类知识形式化/模型化
\end{question}
对知识的一种描述或约定:转化为机器可接受描述的形式
\[
    \text{知识表示}
    \begin{cases}
        \text{符号主义}\begin{cases}
            \text{谓词逻辑}\\
            \text{产生式系统}\\
            \text{框架系统}
        \end{cases}\\
        \text{经验主义}\begin{cases}
            \text{状态表示}\\
            \text{特征表示}
        \end{cases}\\
        \text{连接主义}\begin{cases}
            \text{语义变量}\\
            \text{网络权重}
        \end{cases}
    \end{cases}
.\] 
\begin{notation}
    亚里士多德提出了“三段论”演绎推理方法

    莱布尼茨在17世纪提出二进制,乔治贝尔提出用简单符号表示逻辑命题,产生了“贝尔代数”:适于机器使用的数学规律
\end{notation}
概念理论由概念名、概念内涵和概念外延组成
\begin{notation}
    命题:一个非真即假的陈述句
    \[
        \text{命题分类}
        \begin{cases}
            \text{真命题}\\
            \text{假命题}\\
            \text{在一定条件下为真,一定条件下为假}
        \end{cases}
    .\] 

    对于$R: x<8$ 由于$R$的真假依赖于 $x$ 的取值,因此无法判断
\end{notation}
\subsection{命题逻辑}%
\label{sub:命题逻辑}
\begin{notation}
    蕴含连结词:“若$p$ 则$q$”称为$p$ 对$q$ 的蕴含式:$p\to q$
\end{notation}

