\lecture{12}{10.31}
\[
    \text{机器学习}\begin{cases}
        \text{计算机视觉}
        \begin{cases}
            \text{手写识别}\\
            \text{行人再识别}
        \end{cases} \\
        \text{自然语言处理}\\
        \text{社交媒体计算}\\
        \text{经济金融}
    \end{cases}
.\] 

机器学习的四个部分:
\[
    \begin{cases}
        \text{T: Task}\\
        \text{A: Algorithm}\\
        \text{E: Experience}\\
        \text{P: Performance}
    \end{cases}
.\] 
\begin{eg}
    人脸识别:

    A:线性回归

    E:以标定身份的人脸图片数据

    P:人脸识别准确率
\end{eg}
\subsubsection*{机器学习的基本过程}%
\label{subsub:机器学习的基本过程}
从给定的数据中学习规律$\to $ 学习方法,建立模型$\to $ 预测$\to $ 测试匹配度
\subsubsection*{机器学习分类}%
\label{subsub:机器学习分类}
\[
    \begin{cases}
        \text{半监督学习}\begin{cases}
            \text{监督学习}\\
            \text{无监督学习}
        \end{cases}\\
        \text{强化学习}
    \end{cases}
.\] 
\subsubsection{监督学习}%
\label{subsub:监督学习}
\begin{defi}
    根据已知的输入和输出训练模型,预测未来输出
\end{defi}
监督学习的数据存在样本标签,有训练集和测试集
\begin{eg}
    学习书籍内容,设定标签:艺术/政治/科学等,找出训练文字和标签的映射关系
\end{eg}
\begin{notation}
    分类方法:\textit{K-nearest neighbour},决策树,支持向量机,朴素贝叶斯

    回归方法:线性、树、支持向量回归,集成方法
\end{notation}
../../semester-0/Medicine AI/KNN.tex

