\lecture{8}{10.17}
\begin{notation}
    DFS: Deep-First Search

    优先朝长节点扩展:先进入开放表的节点先扫描

    优点:搜索空间可以远小于宽度优先

    缺点:忽略深度

    修正:加入深度界限,在已知目标节点的深度范围时限制搜素深度

    最坏情况:$o\left( n \right) $
\end{notation}
应用:状态表=树状图 
\begin{notation}
    等代价搜索/Dijkstra算法:

    BFS的一种推广

    $g\left( n \right) $代表从初始节点到节点$n$ 的代价

    $c\left( n_1,n_2 \right) $ 表示从$n_1$ 到$n_2$ 的代价

    $g\left( n_2 \right) =g\left( n_1 \right) +c\left( n_1,n_2 \right) $ 

    优点:加入了状态图中的路径长短元素(走一步看一步),以等代价选择下一节点的选择
\end{notation}


