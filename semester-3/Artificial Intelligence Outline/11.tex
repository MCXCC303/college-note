\lecture{11}{10.28}
\subsection{遗传算法}%
\label{sub:遗传算法}
\begin{notation}
    如何模拟物种繁殖:

    1. 种群中选择两个个体

    2. 随机确定编码序列断裂点

    3. 交换编码片段

    基因发生突变的概率称为遗传算法的\textbf{突变算子}
\end{notation}
\begin{notation}
    如何模拟竞争与选择:\textbf{适应度}
\end{notation}
\begin{eg}
    求函数$f\left( x \right) =x^2$ 的最大值,$x\in [0,31],x\in \mathbb{Z}$

    使用5个二进制码表示取值:$0\sim 31\Rightarrow \text{0b00000} \sim \text{0b11111}$ 

    定义32条染色体:
     \begin{table}[htpb]
        \centering
        \caption{染色体}
        \label{tab:染色体}
        \begin{tabular}{|c|c|}
        \hline
        $X_o$ & $X_b$ \\
        \hline
        0 & 0b00000\\
        1 & 0b00001\\
        $\ldots $ & $\ldots $ \\
        31 & 0b11111\\
        \hline
        \end{tabular}
    \end{table}
    
    假设初始种群数量$N=4$ ,随机产生20位的二进制串,每5个一组,得到4个初始个体

    如:0b\textbf{00110}10010\textbf{10011}01010

    适应度函数为给定问题$f\left( x \right) =x^2$
    \begin{table}[htpb]
        \centering
        \caption{选择算子}
        \label{tab:选择算子}
        \begin{tabular}{|c|c|c|c|c|c|c|}
        \hline
        编号 & 个体编码 & 个体 & 适应度 & $\displaystyle{\frac{f}{\sum f}} $ & $\displaystyle{\frac{4f}{\sum f}} $ & 生存数 \\
        \hline
        $S_1$ & 00110 & 6 & 36 & 0.043 & 0.175 & 0\\
        $S_2$ & 10010 & 18 & 324 & 0.394 & 1.578 & 2\\
        $S_3$ & 10011 & 19 & 361 & 0.439 & 1.759 & 2\\
        $S_4$ & 01010 & 10 & 100 & 0.121 & 0.487 & 0\\
        \hline
        \multicolumn{3}{|c|}{适应度总和} & 821 & \multicolumn{2}{|c|}{平均适应度} & 205.25\\
        \hline
        \end{tabular}
    \end{table}

    得到第一代种群:(0b10010, 0b10011), (0b10010, 0b10011):避免近亲相交

    令交叉概率$P_s=1$ ,变异概率$P_m=0.01$ (每五代变异一个基因),通过生存数生成新的种群,配对后随机选择断裂点位交叉配对,完成第一代遗传:第二代种群的适应度更高
    \begin{table}[htpb]
        \centering
        \caption{第一次交叉}
        \label{tab:第一次交叉}
        \begin{tabular}{|c|c|c|c|c|}
        \hline
        交叉前 & 交叉后 & 个体 & 适应度 & 生存数 \\
        \hline
        0b100\textbf{10} & 0b10011 & 19 & 361 & 1 \\
        0b100\textbf{11} & 0b10010 & 18 & 324 & 1 \\
        \hline
        0b1\textbf{0010} & 0b10011 & 19 & 361 & 1 \\
        0b1\textbf{0011} & 0b10010 & 18 & 324 & 1 \\
        \hline
        \multicolumn{3}{|c|}{适应度总和} & \multicolumn{2}{|c|}{821 $\Rightarrow $ 1370}\\
        \hline
        \end{tabular}
    \end{table}

    生成第五代种群后,通过变异算子随机挑选一个基因进行改变

    经过数代遗传后,种群趋于稳定,适应度不再提升:$X=31$
\end{eg}
\subsection{蚁群算法}%
\label{sub:蚁群算法}
\begin{notation}
    蚂蚁的智能程度非常低,单个觅食随机性很大;但组合成群体后可以完成复杂的任务,且可以适应环境变化
\end{notation}
模拟蚂蚁的选择路线时:贪心原则
\begin{eg}
    旅行商问题
\end{eg}
\subsection{机器学习}%
\label{sub:机器学习}
\begin{defi}
    学习:

    系统改进其性能的过程(西蒙)

    获取知识的过程(专家系统)

    技能的获取(心理学家)

    事物规律的发现过程
\end{defi}


