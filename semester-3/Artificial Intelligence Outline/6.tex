
% 
{\centering{\section*{AIGC}%
\label{sec:AIGC}}}
\begin{defi}
    AIGC: Artificial Intelligence Generated Content(人工智能生成内容),包含文本、图像、音视频等
\end{defi}
按模态分类:文本回答、视频生成、图像生成、多模态等
\subsection*{主要应用领域}%
\label{sub:主要应用领域}
\begin{eg}
    文本生成:文本理解(交互)、结构化写作(写作)、语音克隆(地图)
\end{eg}
\subsection*{GPT}%
\label{sub:GPT}
\begin{defi}
    GPT: Generatice Pre-trained Transformer

    通过以基于注意力机制的transformer为核心架构,让AI学习文本接龙(马尔可夫链)
\end{defi}
\subsection*{部分模型}%
\label{sub:部分模型}
1. GPT Series(Access by VPN)

2. 文心一言

3. 天工

4. 豆包

5. 可灵(多模态)

\subsection{框架式表达方法}%
\label{sub:框架式表达方法}
\begin{defi}
    框架:对某种知识的整体认识,描述所论对象属性的数据结构
\end{defi}
通过框架可以生成表格
\begin{eg}
    框架:课程

    \begin{table}[htpb]
        \centering
        \caption{上课}
        \label{tab:上课}
        \begin{tabular}{|c|c|}
        \hline
        XXX课程 & \\
        \hline
        \multirow{3}{*}{课程需求} & 需求1 \\
                                  & 需求2 \\
                                  & 需求3 \\
                                  \hline
        \multirow{3}{*}{课程内容} & 内容1 \\
                                  & 内容2 \\
                                  & 内容3 \\
                                  \hline
        $\ldots$ & $\ldots$ \\
        \hline
        \end{tabular}
    \end{table}
\end{eg}
框架理论使用层次化结构表达知识
\newpage
\begin{eg}
    \begin{table}[htpb]
        \centering
        \caption{框架}
        \label{tab:框架}
        \begin{tabular}{|c|c|c|c|}
        \hline
        框架名 &  &  & \\
        \hline
        \multirow{3}{*}{槽名1} & 侧面名11 & 值111,值112,$\ldots$ & 约束条件$\ldots$\\
                               & 侧面名12 & 值121,值122,$\ldots$ & \\
                               & 侧面名13 & 值131,值132,$\ldots$ & \\
                               \hline
        \multirow{3}{*}{槽名2} & 侧面名21 & 值211 & \\
                               & 侧面名22 & 值212 & \\
                               & 侧面名23 & 值213 & \\
                               \hline
        $\ldots$ & $\ldots$ & $\ldots$ & \\
        \hline
        关联框架 & \multicolumn{3}{c|}{<框架名1,关系>,<框架名2,关系>$\ldots$} \\
        \hline
        \end{tabular}
    \end{table}

    例:教师

    框架名:教师

    1. 姓名(VARCHAR(12))

    2. 年龄(INT)

    3. 性别(男、女)

    4. 职称(教授、副教授、讲师、助教)

    5. 部门:(系、教研室)

    6. 住址:(VARCHAR(64))

    7. 工资(INT)

    8. 开始工作时间(DATETIME)

    9. 截止时间(DATETIME,DEFAULT DATE(CURRENT\_TIMESTAMP))

    10. 框架关联:教职工,教师
\end{eg}
\begin{notation}
    框架表达的特点:

    1. 结构性

    2. 继承性

    3. 自然性
\end{notation}
作业:习题24,25

