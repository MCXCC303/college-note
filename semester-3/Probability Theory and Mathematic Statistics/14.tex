\lecture{14}{11.05}
\subsection{密度函数}%
\label{sub:密度函数}
\begin{notation}
    密度函数和分布函数的关系:
    \begin{align*}
        F_X\left( x \right) &=\int_{-\infty}^{+\infty} f_X\left( x \right)  \mathrm{d}x\\
        f_X\left( x \right) &=\frac{\mathrm{d}F_X\left( x \right) }{\mathrm{d}x} 
    .\end{align*}
\end{notation}
对于直方图:将中点光滑连接=密度函数

或:核密度
\subsubsection*{直方图}%
\label{subsub:直方图}
\begin{notation}
    直方图的面积代表频率:\[
        \text{高度}h_{i}=\frac{\text{面积}f_{i}}{\text{区间长度}\Delta x_{i}} 
    .\] 
\end{notation}
直方图的高度代表密度,直方图的横坐标的取值范围为观测值的取值范围,直方图分块的区间来源一般为经验公式:$m\approx 1.87\left( n-1 \right) ^{0.4}$,其中$m$ 为区间分组数量

计算直方图频率:\[
    \text{频率}f_{i}=\frac{\text{落入区间的个数}y_{i}}{\text{总个数}y} 
.\] 
\subsection{统计量}%
\label{sub:统计量}
统计量(statistic),统计学(statistics)
\begin{defi}
    统计量:关于样本的函数,不含任何未知参数

    完整定义:
\end{defi}
\begin{eg}
    $X_1,X_2$ 来自正态总体$N\left( \mu,\sigma^2 \right) $的样本(这两个任意抽出一个都属于一个样本),其中$\mu,\sigma$ 均未知,以下表达式:
    \begin{itemize}
        \item $\frac{1}{4} \left( X_1+X_2 \right) -\mu$
        \item $\frac{X_1}{\sigma} $
    \end{itemize}
    均不是统计量(使用了未知的数),以下表达式都是统计量
    \begin{itemize}
        \item $3X_1$
        \item $X_1-8$
        \item $X_1^2+X_2^2$
    \end{itemize}
\end{eg}
提出统计量的目的:通过样本估计或检测未知量,因此统计量不能含未知量

常见统计量:
\begin{itemize}
    \item 样本均值(算术平均数):$\bar{X}=\frac{1}{n} \sum_{i=1}^{n} X_{i}$
    \item 样本方差:$S^2=\frac{1}{n-1} \sum_{i=1}^{n} \left( X_{i}-\bar{X} \right) ^2=\frac{1}{n-1} \left( \sum_{i=1}^{n} X_{i}^2-n\bar{X}^2 \right) $
    \item 样本标准差:$S=\sqrt{\frac{1}{n-1} \sum_{i=1}^{n} \left( X_{i}-\bar{X} \right) ^2} $
    \item 样本阶原点矩
    \item 样本阶中心距
\end{itemize}
\begin{notation}
    样本均值:若$X_1,X_2,\ldots X_{n}$ \textit{i.i.d}:根据辛钦大数定律:$\bar{X}\xrightarrow[n \to +\infty]{P} E\bar{X}=EX$
\end{notation}
\begin{notation}
    样本阶中心矩:
     \[
        B_2=\frac{1}{n} \sum_{i=1}^{n} \left( X_{i}-\bar{X} \right) ^{k}\xrightarrow[n\to \infty]{P} DX
    .\] 
    或:$S^2=B_2\times \frac{n}{n-1} \Rightarrow E\left( S^2 \right) =DX$
\end{notation}
\begin{proof}
    \begin{align*}
        S^2&=\frac{1}{n-1} \sum_{i=1}^{n} \left( X_{i}-\bar{X} \right) ^2=\frac{1}{n-1} \left( \sum_{i=1}^{n} X_{i}^2-n\bar{X}^2 \right) \\
        &= \frac{1}{n-1} \sum_{i=1}^{n} \left( X_{i}^2-2X_{i}\bar{X}+\bar{X}^2 \right) \\
        &= \frac{1}{n-1} \left( \sum_{i=1}^{n} Xi^2 \right) -\sum_{i=1}^{n} \bar{X}^2 \\
        ES^2&= E\left[ \frac{1}{n-1} \left( \sum_{i=1}^{n} X_{i}^2-n\bar{X}^2 \right)  \right]  \\
        &= \frac{1}{n-1} \left( \sum_{i=1}^{n} EX_{i}^2-nE\bar{X}^2 \right)  \\
        &= \frac{1}{n-1} \left[ \sum_{i=1}^{n} \left( DX_{i}+\left( EX_{i} \right) ^2 \right) -n\left( D\bar{X}+\left( E\bar{X} \right) ^2   \right)  \right]  \\
        &= \frac{1}{n} \left[ \sum_{i=1}^{n} \left( DX+\left( EX \right) ^2 \right) -n\left( \frac{DX}{n} +\left( EX \right) ^2 \right)  \right]  \\
        &= \frac{1}{n-1} \left[ nDX+n\left( EX \right) ^2-DX-n\left( EX \right) ^2 \right] =\frac{1}{n-1} \left( n-1 \right) DX =DX
    .\end{align*}
    即:$EB_2=E\left( \frac{n-1}{n} S^2 \right) =\frac{n-1}{n}DX $
\end{proof}
用样本均值估计总体均值:
\[
    \sum_{i=1}^{n} \left( X_{i}-\bar{X} \right) ^2\le \sum_{i=1}^{n} \left( X_{i}-x \right) ^2
.\] 
\subsubsection*{顺序统计量}%
\label{subsub:顺序统计量}
令$X_{\left( 1 \right) }=\min\left\{ X_1,X_2,\ldots ,X_{n} \right\} $ 为最小顺序统计量,最大同理

要求第几小的顺序统计量:$R$ 成为样本极差,$\widetilde{X}$ 称为样本中位数
\subsection{样本均值的分布}%
\label{sub:样本均值的分布}
\begin{thm}
    $X_1,X_2,\ldots ,X_{n}$ 来自$N\left( \mu,\sigma^2 \right) $ ,则\[
        \frac{\bar{X}-\mu}{\sigma} \sqrt{n} \sim N\left( 0,1 \right) 
    .\] 
\end{thm}
定义$\bar{X}$ 为$X_1,X_2,\ldots ,X_{n}$的线性函数,$\bar{X}\sim N\left( \mu,\frac{\sigma^2}{n}  \right) $,计算期望和方差,将$\bar{X}$ 标准化
\begin{thm}
    标准化后的线性函数$\frac{\bar{X}-\mu}{\sigma} \sqrt{n} $ :
    \[
        \frac{\bar{X}-\mu}{\sigma} \sqrt{n} \xrightarrow[n\to \infty]{L} N\left( 0,1 \right) 
    .\] 
\end{thm}
\begin{eg}
    总体:$X\sim N\left( 20,9 \right) $ ,求样本容量$n$ 多大时使样本均值与总体均值的绝对值之差$\le 0.3$ 的概率$> 95\%$
\end{eg}
\subsubsection{三大抽样分布}%
\label{subsub:三大抽样分布}
\begin{itemize}
    \item 卡方分布:$\chi^2\left( n \right) $
    
\end{itemize}
\begin{notation}
    卡方分布实际上为$\alpha=\frac{1}{2} ,\lambda=\frac{n}{2} $ 的Gamma分布
    
    当$n=2$ 时为参数为$\frac{1}{2} $ 的指数分布
    
    一般称$f\left( x \right) =\frac{\lambda^{\alpha}}{\Gamma\left( \alpha \right) } x^{\alpha-1}\mathrm{e}^{-\lambda x} $ 为伽马分布族
\end{notation}
\begin{defi}
    设$X_1,X_2,\ldots ,X_{n}\sim N\left( 0,1 \right) $ \textit{i.i.d} ,令$\chi^2=\sum_{i=1}^{n} X_{i}^2$ ,称$\chi^2$ 为自由度为$n$ 的卡方分布
\end{defi}
\begin{notation}
    卡方分布具有可加性:\[
        Y_1\sim \chi^2\left( m \right) ,Y_2\sim \chi^2\left( n \right) : Y_1+Y_2\sim \chi^2\left( m+n \right) 
    .\] 
\end{notation}
从$n=3$ 开始,卡方分布出现最大值,且$n$ 越大卡方分布的方差越大

卡方分布的性质:
\begin{itemize}
    \item $E\left( \chi^2 \right) =n,D\left( \chi^2 \right) =2n$
    \item 可加性
    \item 分位点:
\end{itemize}
对性质1:
\begin{proof}
    \begin{align*}
        E\left( \sum_{i=1}^{n} X_{i}^2 \right) &= \sum_{i=1}^{n} E(X_{i}^2) = nEX^2\\
                                               &=n\left( DX+\left( EX \right) ^2 \right)  \\
        D\left( \sum_{i=1}^{n} X_{i}^2 \right) &= nDX^2 =n\left( E\left( X^2 \right) ^2-\left( EX^2 \right) ^2 \right) \\
        &= n\left( EX^4-1 \right)  
    .\end{align*}
\end{proof}


