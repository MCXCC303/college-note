\begin{notation}
	马尔可夫分布也具有无记忆性
\end{notation}
回忆:正态分布(高斯分布)
\[
	f\left( x \right) =\frac{1}{\sqrt{2\pi} \sigma}\text{e}^{-\frac{ (x-\mu)^2 }{2\sigma^2}}
.\] 

标准正态分布\[
	\phi\left( x \right) =\frac{1}{\sqrt{2\pi} }\text{e}^{-\frac{x^2}{2\sigma^2}}
.\] 
\subsection{标准化}%
\label{sub:标准化}
\begin{defi}
	标准化将随机变量转化为另一组随机变量

	对于$X\sim N\left( \mu,\sigma^2 \right) $ ,移除前者的中心(将中心变为0),除以标准差,即得到符合标准正态分布的$Y\sim N\left( 0,1 \right) $
\end{defi}

1. 去中心化

2. 除以标准差

即:
\[
	Y=\frac{X-E\left( X \right) }{\sqrt{D\left( X \right) } }
.\] 

标准化后的随机变量期望为0,方差为1

\[
	P\left\{ \left| X-\mu \right| <k\sigma \right\} =P\left\{ \left| \frac{x-\mu}{\sigma} \right| <k \right\} =P\left\{ Y<k \right\} 
.\] 

此时$Y\sim N\left( 0,1 \right) $ ,即符合标准正态分布

对于原来的$3\sigma$ 原则,转化为$P\left\{ -3<Y<3 \right\} $

\[
	P\left\{ -3<Y<3 \right\} = F_{Y}\left( 3 \right) -F_{Y}\left( -3 \right)
.\] 
\[
	=\phi\left( 3 \right) -\phi\left( -3 \right) 
.\] 
\begin{notation}
	 \[
		\phi\left( -x \right) =1-\phi\left( x \right)
	.\] 
\end{notation}
\[
	=2\phi\left( 3 \right) -1
.\] 

\begin{eg}
	设人的高度符合正态分布$X\sim N\left( 170,49 \right) $,问在公共设施处的门需要设计多高才能使至少90\%的人通过

	求门高$H$ 使得$P\{X\le H\}\ge 0.9$

	\[
		P\left\{ X\le H \right\} \implies P\left\{ \frac{X-170}{49}\le \frac{H-170}{49} \right\} 
	.\]
	即:
	\[
		\phi\left( \frac{H-170}{49} \right) \ge 0.9
	.\] 
	 \[
		\frac{H-170}{49}\ge 1.28 \implies H\ge 1.80
	.\] 
\end{eg}

\subsection{随机变量函数的分布}%
\label{sub:随机变量函数的分布}
\begin{eg}
	 \[
		Y=\frac{X-E\left( X \right) }{\sqrt{D\left( X \right) } }
	.\] 
	\[
		Y=g\left( X \right) \sim F_{Y}\left( Y \right) 
	.\] 

	求解$F_{Y}\left( Y \right) $
\end{eg}
\begin{eg}
	$D\sim U[a,b]$且
	 \[
		S=\pi\left( \frac{b}{2} \right) ^2 = \frac{\pi\rho^2}{4}
	.\] 
\end{eg}
1. $X$ 离散:$Y$ 一般是离散的

2. $X$ 连续:$Y$ 可能连续,可能分段连续(离散)
\begin{eg}
	$X$ 的分布律:
	\[
		\begin{pmatrix}
			-2&-1&0&1&2&3\\
			0.15&0.1&0.1&0.2&0.3&0.15
		\end{pmatrix}
	.\] 

	求$Y=X^2$ 的分布律

	列举$Y$ 的分布律:
	\begin{table}[htpb]
		\centering
		\caption{Y Func}
		\label{tab:Y-Func}
		\begin{tabular}{cccccc}
		\toprule
		0.15 & 0.1 & 0.1 & 0.2 & 0.3 & 0.15\\
		\midrule
		4 & 1 & 0 & 1 & 4 & 9\\
		\bottomrule
		\end{tabular}
	\end{table}
	合并后:
	\[
		\begin{pmatrix}
			0 & 1 & 4 & 9\\
			0.1 & 0.3 & 0.45 & 0.15\\
		\end{pmatrix}
	.\] 
\end{eg}

\begin{eg}
	分析法:$X\sim G\left( 0.5 \right) $ (几何分布),求$Y=\sin\left( \displaystyle{\frac{\pi}{2}}X \right) $ 的分布律

	易得:$Y$ 可以取得:0,1,-1

	\[
		Y=\sin\left( \frac{\pi}{2}X \right) =\begin{cases}
			-1, X=4n-1\\
			0,X=4n\text{且}X=4n-2\\
			1,X=4n-3
		\end{cases}
	.\] 
	\[
		P\left\{ Y=-1 \right\} =\sum_{n=1}^{+\infty} P\left\{ X=4n-1 \right\} =\sum_{n=1}^{+\infty} 0.5\times 0.5^{4n-1-1}
	.\] 

	同理:
	\[
		P\left\{ Y=0 \right\} =\sum_{n=1}^{+\infty} 0.5\times 0.5^{2n-1}
	.\] 
	\[
		P\left\{ Y=1 \right\} =\sum_{n=1}^{+\infty} 0.5\times 0.5^{4n-4}
	.\] 

	求得$Y$ 的分布律:
	\[
		\begin{pmatrix}
			-1 & 0 & 1\\
			\displaystyle{\frac{2}{15}} & \displaystyle{\frac{1}{3}} & \displaystyle{\frac{8}{15}}\\
		\end{pmatrix}
	.\] 
\end{eg}

