\lecture{13}{10.31}
\begin{notation}
    偏度$r_1$:三阶标准化随机变量的矩,用于描述对称性

    峰度$r_2$:四阶标准化随机变量的矩,一般使正态分布的峰度$r_2=0$,描述分布的陡峭程度
\end{notation}
\begin{table}[htpb]
    \centering
    \caption{常见分布的数字特征}
    \label{tab:常见分布的数字特征}
    \begin{tabular}{ccccc}
    \toprule
    分布 & $EX$ & $DX$ & $r_1$ & $r_2$\\
    \midrule
    $B\left( 1,p \right) $ & $p$ & $p\left( 1-p \right) $ & $\frac{1-2p}{\sqrt{p\left( 1-p \right) } } $ & $\frac{1}{p\left( 1-p \right) -6} $\\
    $B\left( n,p \right) $ & $np$ & $np\left( 1-p \right) $ & $\frac{1-2p}{\sqrt{np\left( 1-p \right) } } $ & $\frac{1-6p\left( 1-p \right) }{np\left( 1-p \right) } $ \\
    $P\left( \lambda \right) $ & $\lambda$ & $\lambda$ & $\frac{1}{\sqrt{\lambda} } $ & $\frac{1}{\lambda} $ \\
    $G\left( p \right) $ & $\frac{1}{p} $ & $\frac{1-p}{p^2} $ & $\frac{2-p}{\sqrt{1-p} } $ & $6+\frac{p^2}{1-p} $ \\
    $U[a,b]$ & $\frac{a+b}{2} $ & $\frac{\left( b-a \right) ^2}{12} $ & $0$ & $\frac{9}{5}-3 $ \\
    $\Gamma\left( 1,\lambda \right) $ & $\frac{1}{\lambda} $ & $\frac{1}{\lambda^2} $ & 2 & 6\\
    $N\left( \mu,\sigma^2 \right) $ & $\mu$ & $\sigma^2$ & 0 & 0\\
    \bottomrule
    \end{tabular}
\end{table}
\section{数理统计基本概念}%
\label{sec:数理统计基本概念}
随机变量引入:使样本空间映射到实数轴上

分布函数:任意随机变量的概率

大数定律和中心极限定理:由概率论过渡到数理统计
\[
    \begin{cases}
        \text{描述统计学:过去的实验数据/相关分析图}\\
        \text{推断统计学:根据现有的实验数据决策}\begin{cases}
            \text{参数估计:第七章}\\
            \text{假设检验:第八章}\\
            \text{回归分析:第九章}
        \end{cases}
    \end{cases}
.\] 
\begin{defi}
    总体:全部研究对象,可以用分布描述(随机变量组)
\end{defi}
\begin{defi}
    个体:组成总体的成员,符合总体分布(每一个个体都是一个随机变量)
\end{defi}
\begin{eg}
    从总体中抽取$n$ 个样本

    对数据记录:$x_1,x_2,\ldots ,x_n$ 称为$n$ 维随机变量$X_1,X_2,\ldots ,X_n$ 对应的观测值,$X_1,X_2,\ldots ,X_n$ 为来自总体$X$ 的一个样本
\end{eg}
\begin{notation}
    简单样本:$X_1,X_2,\ldots ,X_n$ \textit{i.i.d},且与总体分布相符 

    特点:

    $\circ$ 独立性

    $\circ$ 代表性
\end{notation}
\begin{defi}
    样本空间:$\bm{\varOmega}=\left\{ \left( x_1,x_2,\ldots ,x_n \right) |x_i\in \mathbb{R},\mathrm{i}=1,2,\ldots ,n \right\} $
\end{defi}
\begin{notation}
    样本联合分布和总体分布的关系(\textit{i.i.d}):
    \begin{align*}
        F\left( x_1,x_2,\ldots ,x_n \right) &=P\left\{ X_1\le x_1,X_2\le x_2,\ldots ,X_n\le x_n \right\} \\
        &=\prod_{i=1}^{n} P\left\{ X_i\le x_i \right\} \\
        &= \prod_{i=1}^{n} F\left( x_i \right) 
    .\end{align*}
\end{notation}
扩展:$X$ 为连续型,密度函数的关系:
\begin{align*}
    f\left( x_1,x_2,\ldots ,x_n \right) &= \prod_{i=1}^{n} f_{X_i}\left( x_i \right)   \\
    &= \prod_{i=1}^{n} f\left( x_i \right) \quad x_i\in \mathbb{R},i=1,2,\ldots ,n
.\end{align*}
\subsection{经验分布函数}%
\label{sub:经验分布函数}
经验分布函数:$F_n\left( x \right) $

将样本观测值$x_1,x_2,\ldots ,x_n$ 按大小分类为$x_{\left( 1 \right)} ,x_{\left( 2 \right)} ,\ldots ,x_{\left( n \right) }$
\begin{align*}
    F_n\left( x \right) &=f_n\left\{ X\le x \right\} \\
    &= \begin{cases}
        0,&x<x_{\left( 1 \right) }\\
        \frac{k}{n} ,&x\in [x_{\left( k \right) },x_{\left( k+1 \right) })\\
        1,&x\ge x_{\left( n \right) }
    \end{cases}\\
    &\approx F\left( X \right) 
.\end{align*}
\begin{cor}
    格利文科定理:\[
        P\left\{ \limsup_{n \to \infty,x\in \mathbb{R}} \left| F\left( x \right) -F_n\left( x \right) =0 \right|  \right\} =1
    .\] 
\end{cor}
根据格利文科定理:可以使用经验分布函数来估计理论分布函数
