\lecture{6}{}
\begin{cor}
    $X$ 是连续性随机变量,密度函数为$f_{X}\left( x \right) $ ,随机变量$Y=g\left( X \right) $,且$\exists D, P\left\{ Y\in D \right\} =1$,$g\left( x \right) $ 存在反函数$h \left( y \right) $且严格单调可导,则:
    \[
        f_{Y}\left( y \right) =\begin{cases}
            \left| h'\left( y \right)  \right| f_{X}\left( h\left( y \right)  \right) ,y\in D\\
            0, \text{Others}
        \end{cases}
    .\] 
\end{cor}

\begin{notation}
    指数分布$X\sim \varGamma\left( 1,\lambda \right) $的数学期望$E\left( X \right) ={\frac{1}{\lambda}}$
\end{notation}

\section{多维随机变量函数及其分布}%
\label{sec:多维随机变量函数及其分布}
在实际问题中,试验结果有时需要使用两个或两个以上的随机变量(random value, r.v.)来描述
\begin{eg}
    天气预报:温度、湿度、风力、降水等
\end{eg}
\subsection{二维随机变量及其分布}%
\label{sub:二维随机变量及其分布}
\begin{defi}
    设 $\Omega$ 为随机试验的样本空间,则\[
    \forall \omega\in \Omega \ce{->[\text{某种变化}][]} \exists \left( X\left( \omega \right) ,Y\left( \omega \right)  \right) \in \mathbb{R}^2
    .\] 

    或:\[
        \left\{ X\le x,Y\le y \right\} =\left\{ \omega|X\left( \omega \right) \le x,Y\left( \omega \right) \le y ,\omega\in \Omega\right\} \in \mathscr{F}
    .\] 
    
    称$\left( X,Y \right) $为概率空间$\left( \Omega,\mathscr{F},P \right) $ 上的二维随机变量
\end{defi}
\begin{notation}
    $\left\{ X\le x,Y\le y \right\} =\left\{ X\le x \right\} \cap \left\{ Y\le y \right\} $
\end{notation}

性质:

1. $F\left( x,y \right) \in [0,1]$

2. 关于每个变量单调不减,即固定$x$ ,对$\forall y_1<y_2$, \[
    F\left( x,y_1 \right) \le F(x,y_2)
.\] 

3. 对每个变量右连续,即:\[
    F\left( x_0,y_0 \right) =F\left( x_0+0^{+},y_0 \right) =F\left( x_0,y_0+0^{+} \right) 
.\] 

4. 对$\forall a<b,c<d$ ,有:\[
    F\left( b,d \right) -F\left( b,c \right) -F\left( a,d \right) +F\left( a,c \right) \ge 0
.\] 

即:在任意地方框一个矩形,内部区域的概率必须大于等于0
\begin{eg}
    性质4例题:设 \[
        F\left( x,y \right) =\begin{cases}
            0,x+y<1\\
            1,x+y\ge 1
        \end{cases}
    .\] 

    讨论$F\left( x,y \right) $ 能否成为二维随机变量的分布函数

    \begin{figure}[htbp]
        \centering
        \caption{分布函数图像}
        \label{分布函数图像}
        \begin{tikzpicture}[yscale=1]
            % Axis
            \draw [->] (-2,0)--(2,0) node [right] {$x$};
            \draw [->] (0,-2)--(0,2) node [above] {$y$};
            \node [anchor=north east] at(0,0) {$O$};
        
            \draw [color=blue,semithick,domain=-1:2] plot (\x, {\x-1});
        \end{tikzpicture}
    \end{figure}
\end{eg}
\begin{notation}
    \[
        P\left\{ X>a,Y>c \right\} \neq 1-F\left( a,c \right) 
    .\] 
\end{notation}

\subsection{二维随机变量的边缘分布函数}%
\label{sub:二维随机变量的边缘分布函数}
边缘分布:降一维
\begin{align*}
    F_{X}\left( x \right) &= P\left\{ X\le x \right\}  \\
    &= P\left\{ X\le x,Y<+\infty \right\}  \\
    &= F\left( x,+\infty \right)  \\
.\end{align*}
\begin{eg}
    设随机变量$\left( X,Y \right) $ 的联合分布函数为\[
        F\left( x,y \right) =A\left( B+\arctan \frac{x}{2} \right) \left( C+\arctan \frac{y}{2} \right) , x,y\in \left( -\infty,+\infty \right) 
    .\] 

    求$A,B,C$ 

    解:$\arctan x$ 的性质:\[
        \lim_{x \to \pm\infty} \arctan x=\pm\frac{\pi}{2}
    .\] 

    则\[
        F\left( +\infty,+\infty \right) =A\left( B+\frac{\pi}{2} \right) \left( C+\frac{\pi}{2} \right)=1 
    .\] 
    \[
        F\left( -\infty,+\infty \right) =A\left( B-\frac{\pi}{2} \right) \left( C+\frac{\pi}{2} \right) =0
    .\] 
    \[
        F\left( -\infty,-\infty \right) =A\left( B-\frac{\pi}{2} \right) \left( C-\frac{\pi}{2} \right) =1
    .\] 

    联立解出$A,B,C$
\end{eg}
\subsection{联合分布律}%
\label{sub:联合分布律}
\begin{table}[htpb]
    \centering
    \caption{联合分布律}
    \label{tab:联合分布律}
    \begin{tabular}{c|cccc|c}
    \toprule
    $_X \hspace{1pt} ^Y$ & $b_1$ & $b_2$ & $\ldots$ & $b_{j}$ & $p_{a\cdot }$ \\
    \hline
    $a_1$ & $p_{11}$ & $p_{12}$ & $\ldots$ & $p_{1j}$ & ${\sum_{n=1}^{j} p_{nj}}$\\
    $a_2$ & $p_{21}$ & $p_{22}$ & $\ldots$ & $p_{2j}$ & ${\sum_{n=2}^{j} p_{nj}}$ \\
    $\vdots$ & $\vdots$ & $\vdots$ & $\ddots$ & $\vdots$ & $\vdots$\\
    $a_{i}$ & $p_{i1}$ & $p_{i2}$ & $\ldots$ & $p_{ij}$ & ${\sum_{n=i}^{j} p_{nj}}$\\
    \hline
    $p_{b\cdot }$ & ${\sum_{m=1}^{i} p_{im}}$ & ${\sum_{m=2}^{i} p_{im}}$ & $\ldots$ & ${\sum_{m=1}^{i} p_{im}}$ & 1\\
    \bottomrule
    \end{tabular}
\end{table}

\subsubsection*{二维离散随机变量的联合分布函数}%
\label{subsub:二维离散随机变量的联合分布函数}
\[
    F\left( x,y \right) =\sum_{x_{i}\le x} \sum_{y_{j}\le y} p_{ij}
.\] 

如何求$p_{ij}$ :

1. 古典概型

2. 乘法公式:\[
    p_{ij}=P\left\{ X=x_{i} \right\} P\left\{ Y=y_{i}|X=x_{i} \right\} =P\left\{ X=x_{i},Y=y_{i} \right\} 
.\] 
\subsection{二维连续性随机变量及其概率特性}%
\label{sub:二维连续性随机变量及其概率特性}
\begin{defi}
    若\[
        F\left( x,y \right) =\int_{-\infty}^{x} \int_{-\infty}^{y} f\left( u,v \right)  \mathrm{d}v \mathrm{d}u
    .\] 

    则称$f\left( x,y \right) $为二维随机变量$(X,Y)$ 的联合密度函数,称$\left( X,Y \right) $ 为二维连续型随机变量
\end{defi}
\begin{notation}
    联合密度与联合分布函数的性质:

    1. $f\left( x,y \right) \ge 0$ 

    2. \[
        \int_{-\infty}^{+\infty} \int_{-\infty}^{+\infty} f\left( x,y \right)  \mathrm{d}y \mathrm{d}x = 1 = F\left( -\infty,+\infty \right) 
    .\] 
    
    3. 对每个边缘连续,在$f\left( x,y \right) $ 的连续点处:\[
        \frac{\partial^2 F}{\partial x\partial y} = f\left( x,y \right) 
    .\] 

    从而有:$P\left( x<X\le x+\Delta x,y<Y\le y+\Delta y \right) \approx f\left( x,y \right) \Delta x \Delta y$
\end{notation}

