\lecture{21}{11.28}
\subsubsection*{两个总体的参数估计}%
\label{subsub:-两个总体的参数估计}
假设种类:
\begin{itemize}
    \item $H_0:\mu_1=\mu_2,H_1:\mu_1\neq \mu_2$
    \item $H_0:\mu_1\ge \mu_2,H_1:\mu_1<\mu_2$
    \item $H_0:\mu_1\le \mu_2,H_2:\mu_1>\mu_2$
    \item $^\star H_0:\mu_1-\mu_2\ge c,H_1:\mu_1-\mu_2<c$
    \item $^\star H_0:\mu_1-\mu_2\le c,H_1:\mu_1-\mu_2>c$
\end{itemize}
一般使用点估计:$\overline{X}=\hat{\mu_1},\overline{Y}=\hat{\mu_2}$,将假设转为:$\mu_1-\mu_2\ge c\Rightarrow \overline{X}-\overline{Y}\ge c$

原因:$\overline{X}-\overline{Y}$ 是$\mu_1-\mu_2$ 的最小方差无偏估计
\begin{notation}
    两个总体匹配/不独立
\end{notation}
\begin{eg}
    一种马达正常工作的平均电流不超过0.8A,抽取16台马达,测得$\overline{X}=0.92,S^2 =0.32$ ,假设电流符合正态分布$X\sim N\left( \mu,\sigma^2  \right)$ ,取$\alpha=0.05$ ,求厂家的话是否可信
\end{eg}
\begin{sol}
    确定假设:有两种可能的假设:
    \begin{enumerate}
        \item $H_0:\mu\le 0.8,H_1:\mu>0.8$ 
        \item $H_0:\mu\ge 0.8,H_1:\mu<0.8$
    \end{enumerate}
    确定假设统计量:由于$\sigma$ 未知,因此使用$t$ 统计量:$\frac{\overline{X}-\mu}{S}\sqrt{n}>t_{1-\frac{\alpha}{2}}\left( n-1 \right)$

    分别带入数据后发现:对于$H_0:\mu<0.8$ 和$H_0:\mu\ge 0.8$ ,都不拒绝原假设
\end{sol}
\begin{notation}
    对于假设检验,任何假设都有犯错误的可能,拒绝原假设的可能是充分的($\alpha$ 一般较小),不拒绝原假设有较大的可能犯错误($\beta$ 可能更大),因此不拒绝原假设的结论需要加大样本量继续验证
\end{notation}
\subsubsection*{正态总体的方差的检验}%
\label{subsub:正态总体的方差的检验}
检验统计量不使用$\mu$ :$\chi^2 =\frac{\left( n-1 \right)S^2 }{\sigma_0^2 }$

继续化简:\[
    \chi^2 =\frac{\left( n-1 \right)S^2 }{\sigma_0^2 }=\frac{\left( n-1 \right)\cdot \frac{1}{n-1}\sum_{i=1}^{n} \left( X_{i}-\overline{X} \right)^2 }{\sigma_0^2 } =\frac{\sum_{i=1}^{n} \left( X_{i}-\overline{X} \right)^2 }{\sigma_0^2 }
.\]
此时使用$S^2 =\hat{\sigma}^2 $ 来估计$\sigma$ ,如果$\mu$ 已知则可以使用$\frac{1}{n}\sum_{i=1}^{n} \left( X_{i}-\mu \right)^2 =\hat{\sigma}^2 $ 来估计$\sigma$
\subsubsection*{总体分布的卡方拟合优度检验}%
\label{subsub:总体分布的卡方拟合优度检验}
根据样本预测总体的分布种类(假设)
\begin{eg}
    某建筑工地发生事故的记录:
    \begin{table}[htpb]
        \centering
        \caption{工地事故}
        \label{tab:工地事故}
        \begin{tabular}{cc}
        \toprule
        事故数 & 天数\\
        \midrule
        0 & 102\\
        1 & 59 \\
        2 & 30 \\
        3 & 8 \\
        4 & 0 \\
        5 & 1 \\
        $\ge 6$ & 0 \\
        合计 & 200\\
        \bottomrule
        \end{tabular}
    \end{table}
    求$\alpha=0.05$ 下,数据是否符合泊松分布$P\left( \lambda \right)$
\end{eg}
\begin{notation}
    泊松分布:
    \[
        X\sim P\left( \lambda \right)\quad p=P\left( X\le x \right)=\sum_{k=1}^{n} \frac{\lambda^{k}}{k!}\mathrm{e}^{-\lambda}
    .\]
\end{notation}
\begin{sol}
    设每天发生事故$i$ 次为事件$A_{i}$,确定假设:
    \begin{itemize}
        \item 原假设$H_0:\forall i,P\left( A_{i} \right)=p_i$
        \item 被则假设$H_1:\exists i,P\left( A_{i} \right)\neq p_i$
    \end{itemize}
    $\lambda$ 可以使用$\overline{X}=\hat{\lambda}$ 估计,即$\hat{\lambda}=\overline{x}=0.74$ ,使用$P\left( 0.74 \right)$ 可以计算$\hat{p_i}$

    确定假设统计量:
    \begin{align*}
        \chi^2 &=\sum_{i=1}^{m} \frac{n_{i}}{np_i}-n\\
        \hat{\chi}^2 &= \sum_{i=1}^{m} \frac{n_{i}}{n\hat{p}_i}-n \\
    .\end{align*}
\end{sol}
