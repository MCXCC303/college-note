\lecture{19}{11.21}
\begin{eg}
    参数假设检验:主场优势:NBA某球队进行82场比赛,41场为主场,统计过去15年主场胜率$P_1$ 和客场胜率$P_2$ ,判断该球队是否存在主场优势
\end{eg}
通过胜率差$\Delta P=P_1-P_2$,当$\Delta P=0$时不存在,当$\Delta P>0$时存在
\begin{notation}
    非参数假设检验:住房面积和家庭生活幸福感的关系,学历程度和年均收入的关系;非参数假设检验的两个变量独立
\end{notation}
\begin{eg}
    规定工业废水中$\text{Cr(VI)}$ 排放浓度不超过0.5,即$c_\text{Cr(VI)}\le 0.5$,设$X$ 为工业废水有害物质排放浓度总体,抽取16份废水$X_1,X_2,\ldots ,X_{16}$,测得物质浓度$\bar{x}=0.52,s^2 =0.09$,假设该物质浓度分布为$X\sim N\left( \mu,\sigma^2  \right)$ ,求排放浓度是否符合规定($\alpha=0.5$)
\end{eg}
解:\[
    \bar{x}=\frac{1}{16}\sum_{i=1}^{16} x_{i}\quad s^2 =\frac{1}{15}\sum_{i=1}^{16} \left( x_{i}-\bar{x} \right)^2 
.\]
判断是否超标:通过检测$\mu\le 0.5$ 达标,$\mu>0.5$ 超标:一般把带等号的假设设为原假设$H_0:\mu\le 0.5$ ,被则假设$H_1:\mu>0.5$ ;

检验水平$\alpha=0.05$ :通常不超过$0.1$ ,表示犯第一类错误的概率($\alpha=P\left( \bar{H_0} \right)$);对比$\beta$ :犯第二类错误

选择检验统计量:将最大似然估计标准化:$\frac{\bar{X}-\mu}{s /\sqrt{n}}=T$ ,当$H_0$ 成立时:$T=\frac{\bar{X}-0.5}{s /\sqrt{n}}\sim t\left( 15 \right)$(分布为$t$ 分布,该检验方法为$t$ 检验法)

确定拒绝域$\mathscr{X}_0$:确定拒绝域的形式,由于被则假设为$H_1:\mu>0.5$ ,假设拒绝域为$\left\{ \bar{X}-0.5>c \right\}$ ,$c$ 为未知量

犯第一类错误的概率$P\left\{ \bar{X}-0.5>c \mid \mu\le 0.5 \right\}\le 0.05$,原式放缩:
\begin{align*}
    P\left\{ \bar{X}-0.5>c \mid \mu\le 0.5 \right\}&= P\left\{ \frac{\bar{X}-0.5}{\frac{s}{\sqrt{16}}}>\frac{c}{\frac{s}{\sqrt{16}}} \mid \mu\le 0.5\right\} \\
    &= P\left\{ \frac{\bar{X}-\mu}{\frac{s}{4}}>\frac{c+0.5-\mu}{\frac{s}{4}}\mid \mu\le 0.5 \right\} \\
    &\le P\left\{ \frac{X-\mu}{\frac{s}{4}}>\frac{c}{\frac{s}{4}} \right\}=\alpha \quad \left( 0.5-\mu\ge 0 \right)
.\end{align*}
由题$\frac{c}{s /4}=t_{0.95}\left( 15 \right)$,则检验统计量在拒绝域中可以表示:$\left\{ \frac{\bar{X}-0.5}{s /4}>t_{0.95}\left( 15 \right) \right\}$ 

判断:$\bar{x}=0.52,s=0.3$ ,查表得$t_{0.95}\left( 15 \right)=1.753$ ,带入原式得:$\frac{\bar{X}-0.5}{s /4}=\frac{0.52-0.5}{0.3 /4}<1.753$ ,因此并未落在拒绝域中,接受原假设$H_0$

下结论:认为$t\not\in \mathscr{X}_0$ ,因此不拒绝原假设$H_0$ ,\textbf{在显著性水平}$\bm{\alpha=0.05}$\textbf{下,可以认为该区域有害物质排放浓度符合规定}
\subsubsection*{假设检验的两类错误}%
\label{subsub:假设检验的两类错误}
假设$H_0$ 正确,可以认为$H_0$ 正确或错误,类似二分类问题
\begin{table}[htpb]
    \centering
    \caption{假设检验的两类错误}
    \label{tab:假设检验的两类错误}
    \begin{tabular}{|c|c|c|}
    \hline
    $_\text{真实情况}\hspace{1pt}^\text{判断}$ & 接受$H_0$ & 拒绝$H_0$ \\
    \hline
    $H_0$ 为真 & 正确 & 第一类错误$\alpha$ \\
    \hline
    $H_0$ 为假 & 第二类错误$\beta$ & 正确\\
    \hline
    \end{tabular}
\end{table}

对于上题:$\beta=\left\{ \frac{X-0.5}{s /4}\le t_{0.95}\left( 15 \right) \mid \mu>0.5 \right\}$
\begin{align*}
    \beta&=P\left\{ \frac{X-0.5}{\frac{s}{4}}<t_{0.95}\left( 15 \right) \mid \mu>0.5 \right\} \quad \left( \text{We assume that: } \mu=0.6\right) \\
         &= P\left\{ \frac{X-0.5}{\frac{s}{4}}<t_{0.95\left( 15 \right)}+\frac{0.6-0.5}{\frac{s}{4}} \right\} \approx 60\%
.\end{align*}
使用被则假设检验一般有较大的概率犯错误,因此一般使用原假设检验

假设检验的内容:
\[
    \text{理论依据:小概率原理}
    \begin{cases}
        \text{参数检验}\begin{cases}
            \text{总体}\mu,\Delta\mu \text{的检验}\\
            \text{总体}\sigma,\partial\sigma \text{的检验}
        \end{cases}\\
        \text{非参数检验}\begin{cases}
            \text{分布拟合检验}\\
            \text{符号检验}\\
            \text{秩和检验:两个检验的分布是否是同一分布}
        \end{cases}
    \end{cases}
.\]
\begin{notation}
小概率原理:犯第一类错误的概率$\alpha$ 为小概率;如果认为$H_0$ 正确,但是数据观测得到的是$H_1:\bar{x}-0.5>c$ ,则应该反过来认为$H_1$ 正确
\end{notation}
