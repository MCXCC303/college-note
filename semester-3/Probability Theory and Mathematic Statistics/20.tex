\lecture{20}{11.26}
\begin{eg}
    假设产品出厂时要求次品率$p\le 0.04$ ,从10000件产品中抽取12件产品发现3件次品,问该批产品能否出厂;若抽出1件次品,能否出厂
\end{eg}
解:不使用假设检验:$p_0=\frac{3}{12}=0.25>0.04$ ,明显不能出厂(不严谨),即使$p_0=\frac{1}{12}\approx 0.08>0.04$ 也不能出厂
\begin{notation}
    问题在于:12件产品和10000件产品中出现次品数量的分布不一样
\end{notation}
使用假设检验:假设$p\le 0.04$ ,抽查率$\frac{12}{10000}\approx 0.001$,认为分布符合$B\left( 12,0.04 \right)$ ,即12重伯努利试验,计算得发生事件“在12个产品中抽到3个次品的概率为”: \[
    P_{12}\left( 3 \right)= \mathrm{C}_{12}^{3}p^3 \left( 1-p \right)^{9}\approx0.0097
.\]
因此“在12个产品中抽到3个次品的概率”事件是小概率事件,但一次试验就发生,因此推翻原假设,即$p>0.04$ ,所以不能出厂

同理,\[
    P_{12}\left( 1 \right)\approx 0.306
.\]
概率并不算小,因此可以认为原假设$p\le 0.04$ 成立
\begin{notation}
一致最优势假设检验:使第二类错误尽可能小
\end{notation}
将上述问题建模:

1. 提出假设:$H_0:p\le 0.04, H_1:p>0.04$

2. 样本观测值:$\left( x_1,x_2,\ldots ,x_{12} \right)$

3. 设定小概率阈值:$\alpha=0.01$ 
\begin{question}
    白糖打包机:包装得到的糖重量为一个服从正态分布的随机变量$X\sim N\left( \mu,\sigma^2  \right)$ ,运行正常时$\mu=100\text{ kg},\sigma^2 =0.05$ ,为检验运行是否正常,随机抽取9袋糖测得净重:\[
        \left( 99.3,98.7,100.5,101.2,98.3,99.7,99.5,102.1,100.5 \right)
    .\]
求机器是否运作正常($H_0:\mu=100,H_1:\mu\neq 100$)
\end{question}
\begin{sol}
    由于$\sigma^2 =0.05$ ,且工艺不改变,因此可以认为方差不变,即$X\sim N\left( \mu,0.05 \right)$ ;

假设:$\mu=\mu_0=100$ ,首先求出样本均值和方差:$\overline{X}=99.978$

由于小概率事件原理以及:$\overline{X}$ 是$\mu$ 的一致最小方差无偏估计量:$\left| \overline{X}-\mu \right|$ 应该很小;易得$\overline{X}$ 是$\mu$ 的点估计(矩估计、极大似然估计),则当$H_0$ 成立时,$\left| \overline{X}-\mu_0 \right|$ 也应该很小

设定拒绝域,即当\[
    \left| \overline{X}-\mu_0 \right|>c
.\]
时就拒绝原假设

由于$\alpha=P\left( \text{犯第一类错误} \right)=P\left( \text{拒绝}H_0\mid H_0\text{成立} \right)$,且$\overline{X}\sim N\left( \mu_0,\frac{\sigma}{n} \right)$(上一章总体的结论),标准化后为$U=\frac{\overline{X}-\mu_0}{\sigma /\sqrt{n}}\sim N\left( 0,1 \right)$

根据标准化后的随机变量转换后:
\begin{align*}
    P\left( \left| \overline{X}-\mu_0 \right| >c\right)&= P\left( \frac{\left| \overline{X}-\mu_0 \right|}{\sigma /\sqrt{n}}>\frac{c}{\sigma /\sqrt{n}} \right) \\
    &= 1-P\left( \frac{\left| \overline{X} -\mu_0\right|}{\sigma /\sqrt{n}}\le \frac{c}{\sigma /\sqrt{n}} \right) \\
    &= 1-P\left( \frac{-c}{\sigma /\sqrt{n}}\le U \le \frac{c}{\sigma /\sqrt{n}} \right) =\alpha
.\end{align*}
这里的显著性水平$\alpha$ 可以随意定,但一般使用$0.01,0.025,0.05$ 这几个值,假定$\alpha=0.05$ ,则对于标准正态分布:$P\left( U\le \frac{c}{\sigma /\sqrt{n}} \right)=\frac{1-\alpha}{2}$,查表得到$u_{\frac{1-\alpha}{2}}=u_{0.475}=\frac{c}{\sigma /\sqrt{n}}$,即:\[
    c=\frac{u_{0.475}\cdot \sigma}{\sqrt{n}}
.\]

已知$\overline{X}=99.98,\sigma=\sqrt{0.05},\mu_0=100,\alpha=0.05$,带入后

计算可得:\[
    \left| \overline{X}-\mu_0 \right|=0.02
.\]
\end{sol}
\subsection{假设检验的种类}%
\label{sub:假设检验的种类}
\begin{enumerate}
    \item $H_0:\mu=\mu_0,H_1:\mu\neq \mu_0$:双侧检验(简单原假设/简单备择假设)
    \item $H_0:\mu\le \mu_0,H_1:\mu>\mu_0$:右侧检验(拒绝域在右边)
    \item $H_0:\mu\ge \mu_0,H_1:\mu<\mu_0$:左侧检验(拒绝域在左边)
\end{enumerate}
\begin{notation}
    期末考试规范性:10分,每步2分
    \begin{enumerate}
        \item 提出假设$H_0,H_1$
        \item 选择统计检验量$\frac{\overline{X}-\mu}{\sigma /\sqrt{n}} \text{ or } \frac{\overline{X}-\mu}{S /\sqrt{n}}$
        \item 确定拒绝域$\mathscr{X}$
        \item 计算统计检验量的样本值,观察是否在拒绝域内
        \item 下结论
    \end{enumerate}
\end{notation}
\begin{eg}
书上例题8.2.1:慢性铅中毒
\end{eg}

