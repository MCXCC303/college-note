\lecture{18}{11.19}
\textit{Review:}

$\circ$ 对参数(完全可观测数据)进行点估计:极大似然估计、矩估计

两种方法得到的结果可能一样或不一样

$\circ$ 点估计的评价标准:
\begin{description}
    \item [(渐进)无偏] 估计的参数的期望$E \hat{\theta}$求极限为$\theta$
    \item [有效性] 在无偏的前提下:$D \hat{\theta}$ 越小越有效
    \item [相合性(一致性)] $\text{MSE}\left( \hat{\theta},\theta \right)$:均方误差
\end{description}
\begin{eg}
$\bar{X}$ 和$\hat{W}=\sum_{i=1}^{n} a_{i}X_{i}$ 可以证明都是$\mu$ 的无偏估计,称$\bar{X}$ 为算术均值,$\hat{W}$ 为加权均值,且算术均值比加权均值更有效(均值不等式)
\end{eg}
\begin{notation}
均方误差:$\text{MSE}\left( \hat{\theta},\theta \right)=E\left( \hat{\theta}-\theta \right)^2 $
\end{notation}
\subsection{区间估计}%
\label{sub:区间估计}
\begin{defi}
    置信区间:$\alpha$ 为给定值,总体的分布函数为$F\left( x,\theta \right)$ ,有两个从总体中抽取后构造的统计量$T_1,T_2$ ,当:\[
        P\left\{ T_1<\theta<T_2 \right\}=1-\alpha
    .\]
    时:称$P$ 为置信度,区间长度$T_2-T_1$ 的数学期望$E\left( T_2-T_1 \right)$为精度
\end{defi}
纽曼提出的准则:先确定$\alpha$ 来确定置信度,再确定置信上下限
\begin{notation}
    已知标准正态分布$X\sim N\left( \mu,\sigma^2  \right)$ 中的$\sigma^2 $ ,置信度为$1-\alpha$ 时参数$\mu$ 的置信区间:\[
        \left( T_1,T_2 \right)=\left( \bar{X}-\frac{\sigma}{\sqrt{n}}u_{1-\frac{\alpha}{2}},\bar{X}+\frac{\sigma}{\sqrt{n}}u_{1-\frac{\alpha}{2}} \right)
    .\]
    如果$\sigma^2 $ 未知:通过$t$ 分布可得:$T=\frac{\bar{X}-\mu}{S}\sqrt{n}\sim t\left( n-1 \right)$ ,置信区间为:\[
        \left( T_1,T_2 \right)=\left( \bar{X}-\frac{S}{\sqrt{n}}t_{1-\frac{\alpha}{2}}\left( n-1 \right),\bar{X}+\frac{S}{\sqrt{n}}t_{1-\frac{\alpha}{2}}\left( n-1 \right) \right)
    .\]
\end{notation}
\begin{notation}
卡方分布下:方差为$\sigma^2 $ ,置信度为$1-\alpha$ 的置信区间:\[
    \left( T_1,T_2 \right)=\left( \frac{\left( n-1 \right)S^2 }{\chi^2 _{1-\frac{\alpha}{2}}\left( n-1 \right)},\frac{\left( n-1 \right)S^2 }{\chi^2 _{\frac{\alpha}{2}}\left( n-1 \right)} \right)
.\]
对应的标准差为$\sigma$ ,置信度为$1-\alpha$ 的置信区间:$T_1'=\sqrt{T_1},T_2'=\sqrt{T_2}$

如果$\mu$ 已知,$\sigma^2 $ 未知,令$S_1^2 =\frac{1}{n}\sum_{i=1}^{n} \left( X_{i}-\mu \right)^2 $ ,因为$\chi^2 =\frac{nS_1^2 }{\sigma^2 }\sim \chi^2 \left( n \right)$ ,可得置信度为$1-\alpha$ 的方差的置信区间为:\[
    \left( \frac{nS_1^2 }{\chi^2 _{1-\frac{\alpha}{2}}\left( n \right)},\frac{nS_1^2 }{\chi^2 _{\frac{\alpha}{2}}\left( n \right)} \right)
.\] 
\end{notation}
\section{假设检验}%
\label{sec:假设检验}
\begin{notation}
    在医药研发大量应用
\end{notation}
\begin{itemize}
    \item 参数假设检验:假设效果
    \item 非参数假设检验
\end{itemize}
\begin{notation}
    下节课之前准备一个本专业的假设检验问题
\end{notation}
