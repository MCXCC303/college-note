\subsection{公理化}%
\label{sub:公理化}
\[
    \left( \Omega,\mathscr{F},p \right) 
.\] 
\begin{defi}
    $\Omega$:随机试验所产生的所有样本点的集合

    $\mathscr{F}$:集合内所有子集为元素的集合

    $P\left( X \right) $:概率函数
\end{defi}

\begin{axiom}
    非负性:\[
        P\left( A \right) \ge 0,A\in \mathscr{F}
    .\] 
\end{axiom}
\begin{axiom}
    规范性:\[
        P\left( \Omega \right) =1
    .\] 
\end{axiom}
\begin{axiom}
    可列可加性:对两两互斥的事件$A_1,A_2,\ldots$,\[
        P\left( \sum_{i=1}^{+\infty} A_i \right) =\sum_{i=1}^{+\infty} P\left( A_i \right) 
    .\] 
\end{axiom}
从三条公理得出的性质:
\begin{notation}
    1. $P\left( \varnothing \right) =0$ 

    2. 有限可加性: \[
        \sum_{i=1}^{n} P\left( A_i \right) =P\left( \sum_{i=1}^{n} A_i \right) 
    .\] 

    3. $P\left( \bar{A} \right) =1-P\left( A \right) $

    4. $A\subset B\implies P\left( B-A \right) =P\left( B \right) -P\left( A \right) $

    5. $A\subset B\implies P\left( A \right) \le P\left( B \right) $

    6. $P\left( A\cup B \right) =P\left( A \right) +P\left( B \right) -P\left( AB \right) $
\end{notation}
\begin{notation}
    6.1. \[
    P\left( A\cup B\cup C \right) =P\left( A \right) +P\left( B \right) +P\left( C \right) -P\left( AB \right) -P\left( BC \right) 
    \]
    \[
    -P\left( AC \right) +P\left( ABC \right) 
    .\] 
    6.2. \[
        P\left( \bigcup_{i=1}^{n}A_i \right) =\sum_{i=1}^{n} P\left( A_i \right) -\sum_{i=1}^{n-1} \sum_{j=i+1}^{n} P\left( A_{i}A_{j} \right)+\sum_{i=1}^{n-2} \sum_{j=i+i}^{n-1} \sum_{k=j+1}^{n} P\left( A_{i}A_{j}A_{k} \right) 
    \]
    \[
        \ldots+\left( -1 \right) ^{n-1}P\left( \prod_{i=1}^{n} A_{i}\right)  
    .\] 
\end{notation}

\begin{eg}
    从有号码$1,2,\ldots,n$的$n$个球中有放回地取$m$个球,求取出的$m$个球中最大号码为$k$的概率

    \[
        P\left\{ k=1 \right\} =\left( \frac{1}{n} \right) ^m
    .\]  

    逐个列举计算较复杂,记事件$B_k$ 为取出的$m$个球最大号码不超过$k$,只需保证每次摸出的球都不超过$k$即可:
    \[
        P\left( B_{k} \right) =\frac{k^m}{n^m}
    .\] 

    又有$P\left( A_{k} \right) =P\left( B_{k} \right) -P\left( B_{k-1} \right) $,且$B_{k-1}\subset B_{k}$ 

    所以:\[
        P\left( A_k \right) =\frac{k^m}{n^m}-\frac{\left( k-1 \right) ^m}{n^m}
    .\] 
\end{eg}
 
\begin{eg}
    匹配问题:$n$个学生各带有一个礼品,随机分配礼品,设第$i$个人抽到自己的礼品称为一个配对,求至少有一个配对的概率

    设$A_{i}$ 是第$i$个人抽到自己的礼品,$A$为目标事件,则: \[
        A=\bigcup_{i=1}^{n}A_{i}
    .\] 
    \[
        P\left( A_{i} \right) =\frac{\left( n-1 \right) !}{n!}=\frac{1}{n}
    .\] 
    \[
        P\left( A_iA_j \right) =\frac{\left( n-2 \right) !}{n!}=\frac{1}{\mathrm{P}_{n}^{2}}
    .\] 
    \[
        P\left( A_iA_jA_k \right) =\frac{1}{\mathrm{P}_{n}^{3}}
    .\] 
    \[
        \ldots\ldots\ldots
    .\] 
    \[
        P\left( \prod_{i=1}^{n} A_i  \right) =\frac{1}{n!}
    .\] 
    \[
        P\left( A \right) =P\left( \sum_{i=1}^{n} A_{i} \right) = \sum_{i=1}^{n} P\left( A_{i} \right) -\sum_{i=1}^{n-1} \sum_{j=i+1}^{n} P\left( A_iA_j \right) +\ldots
    .\] 
\end{eg}
\subsection{条件概率与乘法公式}%
\label{sub:条件概率与乘法公式}
\begin{defi}
    \[
        P\left( A \right) >0,P\left( B | A \right) =\frac{P\left( AB \right) }{P\left( A \right) }
    .\] 
    即:在\textbf{A发生的条件下},B发生的概率
\end{defi}
\begin{defi}
    乘法公式:\[
        P\left( AB \right) =P\left( A \right) P\left( B|A \right) =P\left( B \right) P\left( A|B \right) 
    .\] 
\end{defi}
\begin{notation}
    $A,B$独立:$P\left( AB \right) =P\left( A \right) P\left( B \right) $

    结合乘法公式:\[
        P\left( B \right) =P\left( B|A \right) 
    .\] 
    \[
        P\left( A \right) =P\left( A|B \right) 
    .\] 
\end{notation}
\subsection{全概率公式}%
\label{sub:全概率公式}
\begin{cor}
    事件$A_1,A_2,\ldots,A_n$为完备事件组,事件$B\subset \Omega=\bigcup_{i=1}^{n}A_i$,则:\[
        P\left( B \right) =\sum_{i=1}^{n} P\left( A_i \right) P\left( B|A_i \right) 
    .\] 
\end{cor}
\begin{notation}
    此时完备事件组的情况应该已知,通过完备事件组$A$的辅助可以求得较复杂事件$B$的概率
\end{notation}
\subsection{贝叶斯公式}%
\label{sub:贝叶斯公式}
\begin{cor}
    \[
        P\left( A_{k}|B \right) =\frac{P\left( A_{k}B \right) }{P\left( B \right) }=\frac{P\left( A_{k} \right) P\left( B|A_k \right) }{\displaystyle{\sum_{i=1}^{n}} P\left( A_i \right) P\left( B|A_i \right)}
    .\] 
\end{cor}
贝叶斯公式被称为“逆概率公式/后验公式”,其中事件$B$更可能是事件的结果,将事件组$A$看作结果出现的原因,则贝叶斯公式是一个从“结果”推“原因”的可能性的公式

\begin{notation}
    对比一般公式:事件A导致B,求B发生的概率

    贝叶斯公式:事件A导致B,A中的一个事件$A_i$ 导致B发生的概率
\end{notation}

\begin{axiom}
    条件概率的公理:

    1. 非负性:$P\left( A \right) \in \left[ 0,1 \right] $

    2. 规范性:$P\left( \Omega|A \right) =1$

    3. 可列可加性: \[
        P\left( \sum_{i=1}^{\infty} B_i|A \right) =\sum_{i=1}^{\infty} P\left( B_i|A \right) 
    .\] 
\end{axiom}
\begin{cor}
    \[
        P\left( \bar{B}|A \right) =P\left( \Omega-B|A \right) =P\left( \Omega|A \right) -P\left( B|A \right)
    .\] 
\end{cor}
\begin{cor}
    \[
        P\left( B_1\cup B_2 \right) =P\left( B_1 \right) +P\left( B_2 \right) -P\left( B_1B_2 \right) 
    .\] 
    \[
        \implies P\left( B_1\cup B_2|A \right) =P\left( B_1|A \right) +P\left( B_2|A \right) -P\left( B_1B_2|A \right) 
    .\] 
\end{cor}
\begin{cor}
    乘法公式:
    \[
        P\left( ABC \right) =P\left( A\left( BC \right)  \right) =P\left( A \right) P\left( BC|A \right) =P\left( A \right) P\left( B|A\right) P\left( C|AB \right)  
    .\] 
\end{cor}
\begin{eg}
    8个红球2个白球,求前三次结果是“红红白”的概率:

    1. 不放回取3个(和一次取三个球相同)

    所有可能性:$10\times 9\times 8$ 

    目标事件:$8\times 7\times 2$

    或使用乘法公式:设$A_i$ 为第$i$次取到红球,目标事件可表示为$A_1A_2\bar{A_3}$

    概率:\[
        P\left( A_1A_2\bar{A_3} \right) =P\left( A_1 \right) P\left( A_2|A_1 \right) P\left( \bar{A_3}|A_1A_2 \right) =\frac{8}{10}\times \frac{7}{9}\times \frac{2}{8}=\frac{7}{45}
    .\] 

    2. 每次取后放回,并加入两个同色的球,取3次(不能使用古典概型)

    概率:

    \[
        P\left( A_1A_2\bar{A_3} \right) =\frac{8}{10}\times \frac{8}{12}\times \frac{2}{14}=\frac{8}{105}
    .\] 
\end{eg}
\begin{eg}
    某疾病的发病率为0.0004,患病检测呈阳性的概率为0.99,误诊为阴性的概率为0.01,误诊为阳性的概率为0.05,不患病检测呈阴性概率为0.95,一个人检测呈阳性,求其患病的概率

    设阳性为$A$,患病为$B$

    则: \[
        P\left( A|B \right) =0.99,P\left( A|\bar{B} \right) =0.05,P\left( B \right) =0.0004
    .\] 
    要求:$P\left( B|A \right) $ 

    使用贝叶斯公式:

    \[
        P\left( B|A \right) =\frac{P\left( AB \right) }{P\left( A \right) }=\frac{P\left( B \right) P\left( A|B \right) }{P\left( AB \right) +P\left( A\bar{B} \right) }
    .\] 
    \[
        =\frac{P\left( B \right) P\left( A|B \right) }{P\left( B \right) P\left( A|B \right) +P\left( A|\bar{B} \right) P\left( \bar{B} \right) }=0.0079
    .\] 
\end{eg}
\subsection{独立性}%
\label{sub:独立性}
\begin{defi}
    $A,B$ 独立,则:$P\left( A|B \right) =P\left( A \right) $
\end{defi}
\begin{notation}
    证明独立性:

    1. $P\left( A \right) P\left( B \right) =P\left( AB \right) $ 
\end{notation}
\begin{notation}
    独立事件的特点:
    
    1. $A,B$独立有:$A,B$所有的组合(包含补集)均独立

    2. $A,B$独立的充要条件:$P\left( A|B \right) =P\left( A \right) \text{ or } P\left( B|A \right) =P\left( B \right) $

    3. $\varnothing$ 与任何随机事件独立,$\Omega$ 与任何随机事件独立
\end{notation}
对于三个事件相互独立:
\[
    \begin{cases}
        P\left( AB \right) =P\left( A \right) P\left( B \right) \\
        P\left( AC \right) =P\left( A \right) P\left( C \right) \\
        P\left( BC \right) =P\left( C \right) P\left( C \right) \\
        P\left( ABC \right) =P\left( A \right) P\left( B \right) P\left( C \right) 
    \end{cases}
.\] 

对比乘法公式:
$P\left( ABC \right) =P\left( A \right) P\left( B|A \right) P\left( C|AB \right) $

\begin{defi}
    相互独立:

    有$A_1,A_2,\ldots,A_n$事件组,对$\forall s \in \left[ 2, n\right] $个事件$A_{k_1},A_{k_2},\ldots,A_{k_s}$ 均有:\[
        P\left( \prod_{n=1}^{s} A_{k_n}  \right) =\prod_{n=1}^{s} P\left( A_{k_n} \right)  
    .\] 

    称事件$A_1,A_2,\ldots,A_{n}$相互独立
\end{defi}

\begin{defi}
    两两独立:对事件$A_1,A_2,\ldots,A_n$ ,若任意两个事件独立,则称为两两独立
\end{defi}

\begin{notation}
    相互独立一定两两独立,反之不一定
\end{notation}

\begin{notation}
    相互独立事件组的性质:

    1. 事件$A_1,A_2,\ldots,A_n$ 相互独立,将其中任意部分改为对立事件,事件组仍为相互独立

    2. 事件相互独立,将事件组任意分为两组(或多组),对组内事件进行“并、交、差、补”操作后,事件间依然相互独立
\end{notation}
\subsubsection*{独立重复实验}%
\label{subsub:独立重复实验}
\begin{defi}
    $E_1,E_2$中一个试验的任何结果和另一个试验的任何结果相互独立,则试验相互独立;若$n$个独立试验相互独立且试验相同,称$E_1,E_2,\ldots,E_n$为$n$次独立重复实验,或$n$重独立试验

    \begin{eg}
        扔硬币和掷骰子为独立试验,其中扔硬币为伯努利试验(只有两个结果)
    \end{eg}
    
\end{defi}
\begin{defi}
    $n$重独立试验$E$中,每次试验都是伯努利试验(可能结果只有两个),称$E$为$n$重伯努利试验
\end{defi}
    1. 二项概率公式:成功$k$次的概率记为$P_n\left( k \right) $,假定前$k$次成功,后$n-k$次失败,则\[
        P_i =p^k\left( 1-p \right) ^{n-k}
    .\] 

    指定事件$A$发生的位置有$\mathrm{C}_{n}^{k}$ 种,则:\[
        P_n\left( k \right) =\mathrm{C}_{n}^{k}p^{k}\left( 1-p \right) ^{n-k}
    .\] 

    称为二项概率公式

    2. 几何概率公式:首次成功恰好发生在第$k$次的概率记为$G\left( k \right) $ ,设前$k-1$次失败,则:\[
        G\left( k \right)= q^{k-1}p
    .\] 

    可以验证:$\sum G\left( k \right) =1$ 

    3. 负二项概率:需要成功$r$次,第$r$次成功恰好发生在第$k$次的概率记为$G_r\left( k \right) $ ,设前$k-1$次试验有$r-1$次成功,则:\[
        G_r\left( k \right) =\mathrm{C}_{k-1}^{n-1}p^rq^{k-r}
    .\] 

    同样有:$\sum G_r\left( k \right) =1$
