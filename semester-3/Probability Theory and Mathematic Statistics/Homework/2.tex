%─────────────────%
% Header Settings %
%─────────────────%
\def\lecturer{LiManMan}
\def\noter{THF}
\def\className{Statistics}
\def\term{III-B}
%─────────────────────%
% Undefined Variables %
%─────────────────────%
\ifx\noter\undefined
    \def\noter{THF}
\else
\fi
\ifx\lecturer\undefined
    \def\lecturer{None}
\else
\fi

%───────────────────%
% Document Settings %
%───────────────────%
\documentclass[12pt,a4paper]{article}

%─────────────────%
% Package Imports %
%─────────────────%
\usepackage[]{amsmath}
\usepackage[]{amssymb}
\usepackage[]{amsthm}
\usepackage[]{array}
\usepackage[]{bm}
\usepackage[]{booktabs}
\usepackage[UTF8]{ctex}
\usepackage[]{fancyhdr}
\usepackage[]{float}
\usepackage[]{geometry}
\usepackage[]{graphicx}
\usepackage[]{hyperref}
\usepackage[]{import}
\usepackage[]{inputenc}
\usepackage[]{mathrsfs}
\usepackage[]{multirow}
\usepackage[]{pdfpages}
\usepackage[]{pgfplots}
\usepackage[]{stmaryrd}
\usepackage[]{tabu}
\usepackage[]{tcolorbox}
\usepackage[]{textcomp}
\usepackage[]{thmtools}
\usepackage[]{tikz}
\usepackage[]{tkz-euclide}
\usepackage[]{url}
\usepackage[]{wrapfig}
\usepackage[dvipsnames]{xcolor}
\usepackage[]{xifthen}
\usepackage[]{yhmath}

%────────────────%
% Fancy Settings %
%────────────────%
\fancypagestyle{CustomStyle}{%
    \fancyhf{}
    \setlength{\headheight}{14.49998pt}
    \fancyhead[R]{\thepage}
    \fancyhead[L]{\lecturer: \className}
}

%──────────────────%
% pgfplot Settings %
%──────────────────%
\usepgfplotslibrary{external}
\pgfarrowsdeclarecombine{twolatex'}{twolatex'}{latex'}{latex'}{latex'}{latex'}
\pgfplotsset{compat=1.12}

%───────────────%
% Tikz Settings %
%───────────────%
\usetikzlibrary{arrows.meta}
\usetikzlibrary{decorations.markings}
\usetikzlibrary{decorations.pathmorphing}
\usetikzlibrary{positioning}
\usetikzlibrary{fadings}
\usetikzlibrary{intersections}
\usetikzlibrary{cd}
\tikzset{->/.style = {decoration={markings,mark=at position 1 with {\arrow[scale=2]{latex'}}},postaction={decorate}}}
\tikzset{<-/.style = {decoration={markings,mark=at position 0 with {\arrowreversed[scale=2]{latex'}}},postaction={decorate}}}
\tikzset{<->/.style = {decoration={markings,mark=at position 0 with {\arrowreversed[scale=2]{latex'}},mark=at position 1 with {\arrow[scale=2]{latex'}}},postaction={decorate}}}
\tikzset{->-/.style = {decoration={markings,mark=at position #1 with {\arrow[scale=2]{latex'}}},postaction={decorate}}}
\tikzset{-<-/.style = {decoration={markings,mark=at position #1 with {\arrowreversed[scale=2]{latex'}}},postaction={decorate}}}
\tikzset{->>/.style = {decoration={markings,mark=at position 1 with {\arrow[scale=2]{latex'}}}postaction={decorate}}}
\tikzset{<<-/.style = {decoration={markings,mark=at position 0 with {\arrowreversed[scale=2]{twolatex'}}},postaction={decorate}}}
\tikzset{<<->>/.style = {decoration={markings,mark=at position 0 with {\arrowreversed[scale=2]{twolatex'}},mark=at position 1 with {\arrow[scale=2]{twolatex'}}},postaction={decorate}}}
\tikzset{->>-/.style = {decoration={markings,mark=at position #1 with {\arrow[scale=2]{twolatex'}}},postaction={decorate}}}
\tikzset{-<<-/.style = {decoration={markings,mark=at position #1 with {\arrowreversed[scale=2]{twolatex'}}},postaction={decorate}}}
\tikzset{circ/.style = {fill, circle, inner sep = 0, minimum size = 3}}
\tikzset{scirc/.style = {fill, circle, inner sep = 0, minimum size = 1.5}}
\tikzset{mstate/.style={circle, draw, blue, text=black, minimum width=0.7cm}}
\tikzset{eqpic/.style={baseline={([yshift=-.5ex]current bounding box.center)}}}
\tikzset{commutative diagrams/.cd,cdmap/.style={/tikz/column 1/.append style={anchor=base east},/tikz/column 2/.append style={anchor=base west},row sep=tiny}}

%──────────────────────%
% Theorem Environments %
%──────────────────────%
\theoremstyle{definition}
\declaretheoremstyle[
    headfont=\bfseries\sffamily\color{ForestGreen!70!black}, bodyfont=\normalfont,
    mdframed={
        linewidth=2pt,
        rightline=false, topline=false, bottomline=false,
        linecolor=ForestGreen, backgroundcolor=ForestGreen!5,
    }
]{thmgreenbox}
\declaretheoremstyle[
    headfont=\bfseries\sffamily\color{NavyBlue!70!black}, bodyfont=\normalfont,
    mdframed={
        linewidth=2pt,
        rightline=false, topline=false, bottomline=false,
        linecolor=NavyBlue, backgroundcolor=NavyBlue!5,
    }
]{thmbluebox}
\declaretheoremstyle[
    headfont=\bfseries\sffamily\color{NavyBlue!70!black}, bodyfont=\normalfont,
    mdframed={
        linewidth=2pt,
        rightline=false, topline=false, bottomline=false,
        linecolor=NavyBlue
    }
]{thmblueline}
\declaretheoremstyle[
    headfont=\bfseries\sffamily\color{RawSienna!70!black}, bodyfont=\normalfont,
    mdframed={
        linewidth=2pt,
        rightline=false, topline=false, bottomline=false,
        linecolor=RawSienna, backgroundcolor=RawSienna!5,
    }
]{thmredbox}
\declaretheoremstyle[
    headfont=\bfseries\sffamily\color{RawSienna!70!black}, bodyfont=\normalfont,
    numbered=no,
    mdframed={
        linewidth=2pt,
        rightline=false, topline=false, bottomline=false,
        linecolor=RawSienna, backgroundcolor=RawSienna!1,
    },
    qed=\qedsymbol
]{thmproofbox}
\declaretheoremstyle[
    headfont=\bfseries\sffamily\color{NavyBlue!70!black}, bodyfont=\normalfont,
    numbered=no,
    mdframed={
        linewidth=2pt,
        rightline=false, topline=false, bottomline=false,
        linecolor=NavyBlue, backgroundcolor=NavyBlue!1,
    },
]{thmexplanationbox}
\declaretheorem[style=thmblueline,numbered=no,name=Notation]{notation}
\declaretheorem[style=thmgreenbox,numbered=no,name=Definition]{defi}
\declaretheorem[style=thmproofbox,numbered=no,name=Proof]{replacementproof}
\newtheorem*{aim}{Aim}
\newtheorem*{assumption}{Assumption}
\newtheorem*{axiom}{Axiom}
\newtheorem*{claim}{Claim}
\newtheorem*{conjecture}{Conjecture}
\newtheorem*{cor}{Corollary}
\newtheorem*{eg}{Example}
\newtheorem*{exercise}{Exercise}
\newtheorem*{ex}{Exercise}
\newtheorem*{fact}{Fact}
\newtheorem*{law}{Law}
\newtheorem*{lemma}{Lemma}
\newtheorem*{prop}{Proposition}
\newtheorem*{question}{Question}
\newtheorem*{remark}{Remark}
\newtheorem*{rrule}{Rule}
\newtheorem*{thm}{Theorem}
\newtheorem*{warning}{Warning}
% \newtheorem{ncor}[nthm]{Corollary}
% \newtheorem{nlemma}[nthm]{Lemma}
% \newtheorem{nprop}[nthm]{Proposition}
% \newtheorem{nthm}{Theorem}[section]
\renewenvironment{proof}[1][\proofname]{\vspace{-10pt}\begin{replacementproof}}{\end{replacementproof}}

%─────────────%
% Math Symbol %
%─────────────%

%────────────%
% Beginnings %
%────────────%
\title{\textbf{Part III-B: \className}}
\author{Lecture by \lecturer\\Note by \noter}
\pagestyle{CustomStyle}
%────────────────────────────────%
% Page Settings (set when print) %
%────────────────────────────────%
% \addtolength{\parskip}{-1mm}
% \addtolength{\parindent}{-2mm}
% \geometry{left=0.5cm,right=0.5cm,top=0.5cm,bottom=0.5cm}


%──────────%
% Document %
%──────────%
\begin{document}
\section*{1-A}%
\label{sec:1-A}
\subsection*{14}%
\label{sub:14}
设“第一次比赛取到$k$个新球”为$A_k$,“第二次全取到新球”为 $B$ 
 \[
    P\left( A_k \right) =\frac{C_{9}^{k}C_{6}^{3-k}}{C_{15}^{3}}
.\] 
\[
    P\left( B|A_k \right) =\frac{C_{9-k}^{3}C_{6+k}^{0}}{C_{15}^{3}}
.\] 
\[
    P\left( B \right) =\sum_{k=0}^{3} P\left( B|A_k \right)P\left( A_k \right)  \approx 0.089
.\] 
\subsection*{22}%
\label{sub:22}
射击为伯努利试验:
\begin{table}[htpb]
    \centering
    \caption{射击}
    \label{tab:射击}
    \begin{tabular}{cc}
    \toprule
    $X$ & $p$\\
    \midrule
    0 & 0.8\\
    1 & 0.2\\
    \bottomrule
    \end{tabular}
\end{table}

假设n次射击都不击中,概率:
\[
    P\left( \bar{A} \right) =0.8^n
.\] 
\[
    P\left( A \right) =1-P\left( \bar{A} \right) \ge 0.9
.\] 

求得:$n\ge 11$,即至少要11次独立射击
\section*{2-B}%
\label{sec:2-B}
\subsection*{6}%
\label{sub:6}
若采用三局两胜制,甲胜利概率:
\[
    P\left( A_1 \right) =0.6\times 0.4\times 0.6+0.4\times 0.6\times 0.6+0.6\times 0.6=0.648
.\] 

若采用五局三胜制:

1. 甲全胜:$p_1=0.6^{5}$ 

2. 甲胜4场:$p_2=0.6^{4}\cdot 0.4\cdot C_{5}^{1}$ 

3. 甲胜3场:$p_3=0.6^{3}\cdot 0.4^{2}\cdot C_{5}^{2}$ 
\[
    P\left( A_2 \right) =p_1+p_2+p_3=0.68256>P\left( A_1 \right) 
.\] 

即五局三胜制有利
\subsection*{7}%
\label{sub:7}
抽取一次,令:产品为正品为$A$ ,检测得正品为$B$ 

有:\[
    P\left( A \right) =0.96,P\left( \bar{A} \right) =0.04,P\left( B|A \right) =0.99,P\left( B|\bar{A} \right) 0.05
.\]

则:\[
    P\left( B \right) =P\left( B|A \right)P\left( A \right)  +P\left( B|\bar{A} \right)P\left( \bar{A} \right) =0.9524
.\] 

抽取三次:$P\left\{ X=3 \right\} =0.9524^{3}\approx 0.864$
\section*{1. }%
\label{sec:1. }
事件发生的情况:第一次、第二次不合格,第三次合格

设“第$k$ 次抽到不合格”为$A_k$,求:$P\left( A_1A_2\bar{A_3} \right) $
\[
    P\left( A_1A_2\bar{A_3} \right) =P\left( A_1 \right) P\left( A_2|A_1 \right) P\left( \bar{A_3}|A_1A_2  \right) 
.\] 
\[
    =\frac{5}{20}\cdot \frac{4}{19}\cdot \frac{15}{18}\approx 0.044
.\] 
\section*{2. }%
\label{sec:2. }
\[
    P\left( A|C \right) =0.95,P\left( C \right) =0.0004
.\] 

可得:\[
    P\left( AC \right) =P\left( A|C \right) P\left( C \right) ,P\left( \bar{C} \right) =0.9996
.\] 
\[
    P\left( \bar{A}|\bar{C} \right) =0.9\implies P\left( A|\bar{C} \right) =0.1
.\] 

由贝叶斯公式:\[
    P\left( C|A \right) =\frac{P\left( AC \right) }{P\left( A \right) }=\frac{P\left( A|C \right) P\left( C \right) }{P\left( A|\bar{C} \right) P\left( \bar{C} \right) +P\left( A|C \right) P\left( C \right) }
.\] 

计算得:$P\left( C|A \right) \approx 0.0038$
\section*{3. }%
\label{sec:3. }
$\displaystyle{P\left( A \right) =\frac{1}{2^2}=\frac{1}{4}}$
\section*{4. }%
\label{sec:4. }
题目更正:载客量5人

目标事件:5个人分到7层楼中,即$N\left( A \right) =P_{7}^{5}$

基本事件总数:$N\left( \Omega \right) =7^5$

概率: \[
    P\left( A \right) =\frac{N\left( A \right) }{N\left( \Omega \right) }\approx 0.150
.\] 

\section*{5. }%
\label{sec:5. }
不被所有的雷达捕捉到的概率:
\[
    P\left( \bar{A} \right) =\left( 1-0.4 \right) \cdot \left( 1-0.3 \right) \cdot \left( 1-0.2 \right) \cdot \left( 1-0.3 \right) =0.2352
.\] 

被至少一个雷达捕捉到即收到超速罚单,即:
\[
    P\left( A \right) =1-P\left( \bar{A} \right) =0.7648
.\] 
\section*{6. }%
\label{sec:6. }
令:“完成任务”为$A$,“通过测试”为$B$ ,则:\[
    P\left( B|A \right) =0.65,P\left( B|\bar{A} \right) =0.25,P\left( A \right) =0.9
.\] 

求:$P\left( \bar{A}|B \right) $ :使用贝叶斯公式:
\begin{align*}
    P\left( \bar{A}|B \right) &=\frac{P\left( \bar{A}B \right) }{P\left( B \right) } =\frac{P\left( B|\bar{A} \right) P\left( \bar{A} \right) }{P\left( B|\bar{A} \right)P\left( \bar{A} \right) +P\left( B|A \right) P\left( A \right)  }\\
                              &=\frac{0.25\cdot 0.1}{0.215\cdot 0.1+0.65\cdot 0.9}\approx 0.041
.\end{align*}
\end{document}
