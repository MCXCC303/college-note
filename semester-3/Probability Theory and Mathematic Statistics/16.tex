\lecture{16}{11.12}
\textit{Review:}

1. 抽样分布定理:定理6.4.3

2. 表达式$\frac{\left( n-1 \right)S^2 }{\sigma^2 }$ 什么时候是统计量:$\sigma^2 $ 已知

当$\sigma^2 $ 未知时:表达式符合$\chi^2 \left( n-1 \right)$ ,称为\textbf{枢轴量}
\begin{cor}
    $X_1,X_2,\ldots X_{n}$ 来自总体$X\sim N\left( \mu,\sigma^2  \right)$ ,样本均值和方差记为$\bar{X},S^2 $ ,则:
    \begin{enumerate}
        \item 求$ES^2 $ :$ES^2 =\sigma^2 $ 
        \item 求$DS^2 $ :$DS^2 =\frac{2\sigma^4 }{n-1}$
        \item 构造$t$ 分布:$\frac{\bar{X}-\mu}{\sigma /\sqrt{n}}/\sqrt{\frac{S^2 }{\sigma^2 }}=\frac{\bar{X}-\mu}{S}\sqrt{n}\sim t\left( n-1 \right)$
    \end{enumerate}
    (上述$\sqrt{\left( n-1 \right)S^2 /\sigma^2} \sim \chi^2 \left( n-1 \right)$ 且$(\bar{X}-\mu) /\frac{\sigma}{\sqrt{n}}\sim \Phi\left( x \right)$)
\end{cor}
重点题目:例6.4.4
\begin{cor}
    两个正态总体的抽样分布定理:两个总体记为$X,Y$ ,样本容量分别为$m,n$ ,样本分布分别符合$\mu=\mu_1,\sigma^2 =\sigma_1^2 $和$\mu=\mu_2,\sigma^2 =\sigma_2^2 $ 的正态分布,$\ldots $
\end{cor}
\section{参数估计}%
\label{sec:参数估计}
参数:\textit{param}
\begin{itemize}
    \item 点估计
    \begin{itemize}
        \item 矩估计
        \item 极大似然估计
    \end{itemize}
    \item 区间估计
\end{itemize} 
\subsection{矩估计}%
\label{sub:矩估计}
使用样本矩$\bar{X}$替代总体矩$\hat{\mu}$ 
\begin{notation}
    总体矩不存在(无穷)时不能使用矩估计
\end{notation}
\begin{eg}
    总体$X\sim U[a,b]$,$X_1,X_2,\ldots ,X_{n}$ 为总体的样本,求$a,b$ 的矩估计量
\end{eg}
解:\[
    \begin{cases}
        EX&= \frac{a+b}{2} \\
        DX&= \frac{\left( b-a \right)^2 }{12}
    \end{cases}
.\]
求解得:$\begin{cases}
    a&= EX-\sqrt{3DX} \\
    b&= EX+\sqrt{3DX}
\end{cases}$ ,替代后为:$\begin{cases}
    \hat{a}&= \bar{X}-\sqrt{3M_2^\star } \\
    \hat{b}&= \bar{X}+\sqrt{3M_2^\star }
\end{cases}$ ,$\hat{a}$ 代表对$a$ 的估计
\subsection{极大似然估计}%
\label{sub:极大似然估计}
似然函数:样本的联合概率分布函数(P167)
\begin{eg}
    同矩估计例题,求$a,b$ 的极大似然估计量
\end{eg}
解:写出联合密度函数:$L\left( \ldots  \right)=\prod_{i=1}^{n} f_{X_{i}}\left( x_{i} \right)$,由于$f_X\left( x \right)=\begin{cases}
    \frac{1}{b-a},&x\in [a,b]\\
    0,&\text{Others}
\end{cases}$,则联合密度函数为:
\begin{align*}
    L&=\prod_{i=1}^{n} \frac{1}{b-a}I_{[a,b]}\left( x_{i} \right)\\
     &= \frac{1}{\left( b-a \right)^n} \prod_{i=1}^{n} I_{[a,b]}\left( x_{i} \right)
.\end{align*}
