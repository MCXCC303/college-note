\lecture{12}{10.29}
\begin{notation}
    相关系数/Pearson相关系数:描述两个随机变量之间的线性相关性

    只能描述数值性的变量
\end{notation}
$|\rho\left( X,Y \right)|=1 $ 时:正相关

$\left| \rho\left( X,Y \right) >0.8 \right| $ :强相关

$\left| \rho\left( X,Y \right) \in \left( 0,0.5 \right)  \right| $:弱相关

$\rho=0$ :不相关/非线性关系
\begin{notation}
    相关系数本质上描述:
    \[
        P\left\{ Y=aX+b \right\} 
    .\] 
\end{notation}
\begin{eg}
    $f\left( x,y \right) =\begin{cases}
        \frac{1}{\pi} ,&x^2+y^2\le 1\\
        0,&x^2+y^2>1
    \end{cases}$ ,求:

    1. $X,Y$ 的相关性;2. $X,Y$ 的独立性
\end{eg}
解:1. 
\begin{align*}
    EX&=\int_{-\infty}^{+\infty} \int_{-\infty}^{+\infty} xf\left( x,y \right)  \mathrm{d}x \mathrm{d}y\\
    &= \int_{-1}^{1}  \mathrm{d}x \int_{-\sqrt{1-x^2} }^{\sqrt{1-x^2} } \frac{x}{\pi}  \mathrm{d}y \\
    &= 0
.\end{align*}

同理$EY=0$,即不相关

2. 
\begin{align*}
    f_X\left( x \right) &= \int_{D}^{} f\left( x,y \right)  \mathrm{d}y \\
    &=\int_{-\sqrt{1-x^2} }^{\sqrt{1-x^2} } \frac{1}{\pi}  \mathrm{d}y\\
                        &=\frac{2}{\pi} \sqrt{1-x^2} 
.\end{align*}

同理$f_Y\left( y \right) ={\frac{2}{\pi} \sqrt{1-y^2}}$ ,易得$f\left( x,y \right) \neq f_X\left( x \right) f_Y\left( y \right) $ ,即不独立

{\centering{\section*{数理统计部分}%
\label{sec:数理统计部分}
}}
\section{大数定律和中心极限定理}%
\label{sec:大数定律和中心极限定理}
\begin{defi}
    大数定律:\[
        \bar{X}\xrightarrow[P]{n\to +\infty }EX
    .\] 
    % From here I tested some arrows from package chemfig
    % I will soon add them to snippet

    即:以某事件发生的频率估计该事件的概率
\end{defi}
\begin{defi}
    中心极限定理: \[
        \bar{X}=\frac{1}{n} \sum_{i=1}^{n} X_{i}
    .\] 

    其中$X_1,X_2,\ldots X_{i}$ 独立同分布

    该随机变量序列存在分布,中心极限定理提出不论$\bar{X}$ 的分布是什么,该序列的分布为正态分布
    \[
        \bar{X}\ce{->[$L$][$n\to \infty $]}N\left( E\bar{X},D\bar{X} \right) 
    .\] 
\end{defi}
如何判断随机变量的敛散性:
\begin{cor}
    依概率收敛:

    对$\forall \varepsilon$ 有:
    \[
        \lim_{n \to \infty} P\left\{ \left| X_n-X \right| <\varepsilon \right\} =1
    .\]

    代表序列 $\left\{ X_n \right\} $ 收敛于随机变量$X$ ,记为$X_n\ce{->[$P$][$n\to \infty $]}X$
\end{cor}
\begin{cor}
    依分布收敛:

    序列的分布函数为$F_n\left( x \right) $ ,随机变量的分布函数$F\left( x \right) $ ,对$\forall x$ ,有: \[
        \lim_{n \to \infty} F_n\left( x \right) =F\left( x \right) 
    .\] 

    则$\left\{ X_n \right\} $ 依分布收敛于$X$ ,记为$X_n\ce{->[$L$][$n\to \infty $]}X$
\end{cor}
\begin{notation}
    测度变换:通过将问题映射到另一个空间简化计算

    依分布收敛要求更弱, 即:依概率收敛$\Rightarrow $ 依分布收敛

    当收敛对象为常数时二者可互推
\end{notation}
\begin{notation}
    撞骗:只要发出的短信足够多,成功率符合大数定律
\end{notation}
三大大数定律:
\[
    \begin{cases}
        \text{\textbf{切比雪夫大数定律}:最根本}\\
        \text{伯努利大数定律:例子}\\
        \text{辛钦大数定律}
    \end{cases}
.\] 
\subsection{大数定律}%
\label{sub:大数定律}
\begin{notation}
    切比雪夫大数定律:
\end{notation}

\begin{defi}
    $\left\{ X_i \right\} i.i.d$ ,$\exists EX_i,DX_i$ ,且$\exists C$ ,使得$DX_i\le C$ (方差有界),则对$\forall \varepsilon>0$ 当:
    \[
        \lim_{n \to \infty} P\left\{ \left| \bar{X_n}-E\bar{X_n} \right| <\varepsilon \right\} =1
    .\] 

    时:
    \[
        \bar{X_n}\ce{->[$P$][$n\to +\infty $]}E\bar{X_n}
    .\] 
\end{defi}
\begin{proof}
    $\bar{X_n}={\frac{1}{n} \sum_{i=1}^{n} X_i}$,有:
    \begin{align*}
        E\bar{X_n}&=\frac{1}{n} \sum_{i=1}^{n} EX_i\\
        D\bar{X_i}&=\frac{1}{n^2} \sum_{i=1}^{n} DX_i\\
                  &\le \frac{C}{n}
    .\end{align*}

    由切比雪夫不等式:
    \begin{align*}
        P\left\{ \left| \bar{X_n}-E\bar{X_n} \right| <\varepsilon \right\} &\ge 1-\frac{D\bar{X_n}}{\varepsilon^2}\\
                                                                        &\ge 1-\frac{C}{n\varepsilon^2}
    .\end{align*}

    当$n\to \infty $ 时原式收敛于1
\end{proof}
\begin{notation}
    辛钦大数定律:序列中的随机变量独立同分布
\end{notation}
\begin{notation}
    伯努利大数定律:序列中$X_i\sim B\left( 1,p \right) $(已知分布),记$\mu_s$ 为随机变量序列之和,有:
    \[
        \lim_{n \to \infty} P\left\{ \left| \frac{\mu_s}{n} -p  \right| <\varepsilon \right\} =1
    .\] 
    
    即:${\frac{\mu_s}{n}} $ 依概率收敛于$p$
\end{notation}
\subsection{中心极限定理}%
\label{sub:中心极限定理}
\begin{eg}
    高尔顿钉板
\end{eg}
\begin{cor}
    \textit{i.i.d}的中心极限定理:

    \[
        \lim_{n \to \infty} P\left\{ \frac{\bar{X_n}-\mu}{\sigma /\sqrt{n}  }\le x  \right\} =\Phi\left( x \right) 
    .\] 
\end{cor}
\begin{cor}
    棣莫弗-拉普拉斯定理:$X_i$ 独立同分布,$X_i\sim B\left( 1,p \right) $,令$Y=\sum_{i=1}^{n} X_i$,对$\forall x$ 有:
    \[
        \lim_{n \to \infty} P\left\{ \frac{Y-np}{\sqrt{np\left( 1-p \right) } } \le x \right\} =\Phi\left( x \right) 
    .\]
\end{cor}
