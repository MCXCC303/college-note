\[
    \text{随机变量}\begin{cases}
        \text{离散型}\begin{cases}
            \text{有限型}\\
            \text{无限型}
        \end{cases}\\
        \text{连续型}
    \end{cases}
.\] 
\begin{notation}
    回忆:分布函数有以下特征:

    1. 非负性

    2. 规范性

    3. 右连续性

    4. $\forall x<y\in \mathbb{R}, F\left( x \right) \le F\left( y \right) $ 

    计算随机变量的概率可以用分布函数表达:
    \[
        P\left\{ a\le X\le b \right\} =F\left( b \right) -F\left( a \right) 
    .\]
\end{notation}
\subsection{常见分布律}%
\label{sub:常见分布律}
1. 退化分布

2. 两点分布

2.1. $0 \sim 1$ 分布($X\sim B\left( 1,p \right) $):\[
    P\left\{ X=k \right\} =p^{k}\left( 1-p \right) ^{1-k}, k=0,1
.\] 

3. 二项分布($X\sim B\left( n,p \right) $):\[
    P\left\{ X=k \right\} =\text{C}_{n}^{k}p^{k}\left( 1-p \right) ^{k}, k=0,1,2,\ldots,n, p \in \left( 0,1 \right) 
.\] 

4. 几何分布($X\sim G\left( p \right) $):\[
    P\left\{ X=k \right\} =p\left( 1-p \right) ^{k-1}, k=1,2,3,\ldots, p \in\left( 0,1 \right) 
.\] 

5. 泊松分布($X\sim P\left( \lambda \right) $):用于描述稀有事件的发生\[
    P\left\{ X=k \right\} =\text{e}^{-\lambda}\frac{\lambda^{k}}{k!}, k=0,1,2,\ldots,\lambda>0
.\] 
\begin{notation}
    由\[
        \text{e}^{x}=\sum_{i=0}^{+\infty} \frac{x^{i}}{i!}
    .\] 
    
    可得:
    \[
        \sum_{k=0}^{+\infty} P\left\{ X=k \right\} =\text{e}^{-\lambda}\cdot \text{e}^{\lambda}=1
    .\] 
\end{notation}

\begin{notation}
    分布律的基本性质:

    1. 非负性:$p_{i}\ge 0$

    2. 正则性:$\displaystyle{\sum_{i=1}^{+\infty} p_{i}=1}$,即每一个点的概率都应该知道
\end{notation}
\begin{eg}
    保险问题

    若一年中某类保险受保人死亡的概率为0.005,现有10000人参加保险,求未来一年中:

    1. 40人死亡的概率

    设$X$ 为未来一年中死亡的人数,有$X\sim B\left( 10000,0.005 \right) $ ,计算:
    \[
        P\left\{ X=40 \right\} =\text{C}_{10000}^{40}0.005^{40}\cdot 0.995^{9960}\approx 2.143\times 10^{-2}
    .\] 

    直接计算较为复杂,可以使用近似计算

    有两种近似计算方法:泊松定理、中心极限定理

    \begin{notation}
        泊松定理:二项分布有时可以转化为泊松分布:

        如果$\displaystyle{\lim_{n \to +\infty} np_n=\lambda>0}$(极小但不为0),则:
        \[
            \lim_{n \to +\infty} \text{C}_{n}^{k}p_n^{k}\left( 1-p_n \right) ^{n-k}=\text{e}^{-\lambda}\frac{\lambda^{k}}{k!},k=0,1,2\ldots
        .\] 
        
        前提:$n$ 大$p$ 小
    \end{notation}
    
    将保险问题转换为泊松分布:\[
        \lambda=np=50
    .\] 
    \[
        P\left\{ X=40 \right\} =\frac{50^{40}}{40!}\text{e}^{-50}\approx 0.02
    .\] 

    2. 死亡人数不超过70的概率

    \[
        P\left\{ X\le 70 \right\} =\sum_{k=0}^{70} \text{C}_{10000}^{k}0.005^{k}\cdot 0.995^{(10000-k)}
    .\] 
    \[
        =\sum_{k=0}^{70} \frac{50^{k}}{k!}\text{e}^{-50} \left( \lambda=np=50 \right) 
    .\] 
\end{eg}
\begin{notation}
    几何分布具有无记忆性:当前试验对过去的试验无任何影响,即:\[
        P\left\{ X=k+1|X>k \right\} =P\left\{ X=1 \right\} 
    .\] 

    可以使用条件概率证明:
    \[
        P\left\{ X=k+1|X>k \right\} =\frac{P\left\{ X=k+1,X>k \right\} }{P\left\{ X>k \right\} }
    .\] 

    由于:\[
        P\left\{ X>k \right\} =\sum_{j=k}^{+\infty} p\left( 1-p \right) ^{j}=p \sum_{j=k}^{+\infty} \left( 1-p \right) ^{j}=p\left( 1-p \right) ^{k}\cdot \frac{1}{1-1+p}=\left( 1-p \right) ^{j}
    .\] 
\end{notation}


\subsection{连续型随机变量}%
\label{sub:连续型随机变量}
\begin{defi}
    \[
        F\left( x \right) =\int_{-\infty}^{x} f\left( x \right) \text{d}x,x\in \mathbb{R}
    .\] 

    则$X$ 为连续型随机变量,$f\left( x \right) $ 称为$X$ 的密度函数
\end{defi}
连续型随机变量的性质:

1. 非负性:$f\left( x \right) \ge 0, x\in \mathbb{R}$

2. 规范性: \[
    \int_{-\infty}^{+\infty} f\left( x \right) \text{d}x =1
.\]

3. \[
    P\left\{ a<X\le b \right\} =\int_{b}^{a}f\left( x \right) \text{d}x, a<b
.\] 

4. $F$连续

5. $F'\left( x \right) =f\left( x \right) $

\begin{notation}
    由于连续性随机变量的分布函数$F$ 处处连续,所以$\forall x\in \mathbb{R}$ ,有$P\left\{ X=x \right\} =F\left( x \right) -F\left( x-0 \right) =0$,即:概率为0的事件不一定是不可能事件
\end{notation}
\begin{eg}
    
\end{eg}
常见的连续型密度函数:

1. 均匀分布($X\sim U\left[ a,b \right] $):\[
    f\left( x \right) =\begin{cases}
        \displaystyle{\frac{1}{b-a}},x\in \left[ a,b \right] \\
        0,x \not\in \left[ a,b \right] 
    \end{cases}
.\] 
对应的分布函数图像:
\begin{center}
    \begin{tikzpicture}
        \draw [->] (0,0)--(0,3) node at (-0.5,3) {$F\left( x \right) $};
        \draw [->] (-2,0)--(3,0) node at (3,-0.5) {$x$};
        \draw [] (0,0)--(1,1);
        \draw [] (1,1)--(3,1);
        \draw [dashed] (1,1)--(1,0);
        \node at (1,-0.5) {$b$};
        \node at (0,-0.5) {$a$};
    \end{tikzpicture}
\end{center}

2. 指数分布($X\sim \Gamma\left( 1,\lambda \right) $):\[
    f\left( x \right) =\begin{cases}
        \lambda\text{e}^{-\lambda x},x>0\\
        0,x\le 0
    \end{cases}
.\] 
\begin{notation}
    指数分布大多数与等待时间有关

    指数分布的充分必要条件为\[
        \forall s,t\ge 0,P\left\{ X>s+t|X>s \right\} =P\left\{ X>t \right\} 
    .\] 

    即指数分布有无记忆性/无后效型(指数分布的特点)
\end{notation}

3. 正态分布($X\sim N\left( \mu,\sigma^{2} \right) $):\[
    f\left( x \right) =\frac{1}{\sqrt{2\pi} \sigma}\text{e}^{-\frac{\left( x-\mu \right) ^{2}}{2\sigma^{2}}}
.\] 

\begin{notation}
    $\sigma^{2}$ :方差,$\mu$ :数学期待
\end{notation}

3.1. 标准正态分布($X\sim N\left( 0,1 \right) $) :\[
\Phi\left( x \right) =\frac{1}{\sqrt{2\pi}} \text{e}^{-\frac{x}{\sqrt{2} \sigma}}
.\] 
