%─────────────────%
% Header Settings %
%─────────────────%
\def\lecturer{李漫漫}
\def\noter{THF}
\def\className{Probability Theory and Mathematical Statistics}
\def\term{III-B}
%─────────────────────%
% Undefined Variables %
%─────────────────────%
\ifx\noter\undefined
    \def\noter{THF}
\else
\fi
\ifx\lecturer\undefined
    \def\lecturer{None}
\else
\fi

%──────────────%
% New Commands %
%──────────────%
\newcommand{\lecture}[2]{
    \def\@lecture{Lecture #1}%
    \subsection*{\@lecture}
    \marginpar{\small\textsf{\text{#2}}}
}

%───────────────────%
% Document Settings %
%───────────────────%
\documentclass[10pt,a4paper]{article}

%─────────────────%
% Package Imports %
%─────────────────%
\usepackage[]{amsmath}
\usepackage[]{amssymb}
\usepackage[]{amsthm}
\usepackage[]{array}
\usepackage[]{bm}
\usepackage[]{booktabs}
\usepackage[]{chemfig}
\usepackage[]{enumitem}
\usepackage[]{fancyhdr}
\usepackage[]{float}
\usepackage[]{geometry}
\usepackage[]{graphicx}
\usepackage[]{hyperref}
\usepackage[]{import}
\usepackage[]{inputenc}
\usepackage[]{mathrsfs}
\usepackage[]{multirow}
\usepackage[]{pdfpages}
\usepackage[]{pgfplots}
\usepackage[]{stmaryrd}
\usepackage[]{tabu}
\usepackage[]{tcolorbox}
\usepackage[]{textcomp}
\usepackage[]{thmtools}
\usepackage[]{tikz}
\usepackage[]{tkz-euclide}
\usepackage[]{url}
\usepackage[]{wrapfig}
\usepackage[]{xifthen}
\usepackage[]{yhmath}
\usepackage[UTF8]{ctex}
\usepackage[dvipsnames]{xcolor}
\usepackage[notransparent]{svg}
\usepackage[version=4]{mhchem}

%───────────────────%
% Inkscape Settings %
%───────────────────%
\newcommand{\incfig}[2][1]{%
    \def\svgwidth{#1\columnwidth}
    \import{./figures/}{#2.pdf_tex}
}

%────────────────%
% Fancy Settings %
%────────────────%
\fancypagestyle{CustomStyle}{%
    \fancyhf{}
    \setlength{\headheight}{14.49998pt}
    \fancyhead[R]{\thepage}
    \fancyhead[L]{\lecturer: \className}
    \fancyfoot[C]{\@lecture}
}

%──────────────────%
% pgfplot Settings %
%──────────────────%
\usepgfplotslibrary{external}
\pgfarrowsdeclarecombine{twolatex'}{twolatex'}{latex'}{latex'}{latex'}{latex'}
\pgfplotsset{compat=1.12}

%───────────────%
% Tikz Settings %
%───────────────%
\usetikzlibrary{arrows.meta}
\usetikzlibrary{decorations.markings}
\usetikzlibrary{decorations.pathmorphing}
\usetikzlibrary{positioning}
\usetikzlibrary{fadings}
\usetikzlibrary{intersections}
\usetikzlibrary{cd}
\tikzset{->/.style = {decoration={markings,mark=at position 1 with {\arrow[scale=2]{latex'}}},postaction={decorate}}}
\tikzset{<-/.style = {decoration={markings,mark=at position 0 with {\arrowreversed[scale=2]{latex'}}},postaction={decorate}}}
\tikzset{<->/.style = {decoration={markings,mark=at position 0 with {\arrowreversed[scale=2]{latex'}},mark=at position 1 with {\arrow[scale=2]{latex'}}},postaction={decorate}}}
\tikzset{->-/.style = {decoration={markings,mark=at position #1 with {\arrow[scale=2]{latex'}}},postaction={decorate}}}
\tikzset{-<-/.style = {decoration={markings,mark=at position #1 with {\arrowreversed[scale=2]{latex'}}},postaction={decorate}}}
\tikzset{->>/.style = {decoration={markings,mark=at position 1 with {\arrow[scale=2]{latex'}}}postaction={decorate}}}
\tikzset{<<-/.style = {decoration={markings,mark=at position 0 with {\arrowreversed[scale=2]{twolatex'}}},postaction={decorate}}}
\tikzset{<<->>/.style = {decoration={markings,mark=at position 0 with {\arrowreversed[scale=2]{twolatex'}},mark=at position 1 with {\arrow[scale=2]{twolatex'}}},postaction={decorate}}}
\tikzset{->>-/.style = {decoration={markings,mark=at position #1 with {\arrow[scale=2]{twolatex'}}},postaction={decorate}}}
\tikzset{-<<-/.style = {decoration={markings,mark=at position #1 with {\arrowreversed[scale=2]{twolatex'}}},postaction={decorate}}}
\tikzset{circ/.style = {fill, circle, inner sep = 0, minimum size = 3}}
\tikzset{scirc/.style = {fill, circle, inner sep = 0, minimum size = 1.5}}
\tikzset{mstate/.style={circle, draw, blue, text=black, minimum width=0.7cm}}
\tikzset{eqpic/.style={baseline={([yshift=-.5ex]current bounding box.center)}}}
\tikzset{commutative diagrams/.cd,cdmap/.style={/tikz/column 1/.append style={anchor=base east},/tikz/column 2/.append style={anchor=base west},row sep=tiny}}

%──────────────────────%
% Theorem Environments %
%──────────────────────%
\theoremstyle{definition}
% \declaretheoremstyle[
%     headfont=\bfseries\sffamily\color{ForestGreen!70!black}, bodyfont=\normalfont,
%     mdframed={
%         linewidth=2pt,
%         rightline=false, topline=false, bottomline=false,
%         linecolor=ForestGreen, backgroundcolor=ForestGreen!5,
%     }
% ]{thmDefi} % Defi = Green box
% \declaretheoremstyle[
%     headfont=\bfseries\sffamily\color{RawSienna!70!black}, bodyfont=\normalfont,
%     numbered=no,
%     mdframed={
%         linewidth=2pt,
%         rightline=false, topline=false, bottomline=false,
%         linecolor=RawSienna, backgroundcolor=RawSienna!1,
%     },
%     qed=\qedsymbol
% ]{thmProof} % Proof = Red box
% \declaretheoremstyle[
%     headfont=\bfseries\sffamily\color{NavyBlue!70!black}, bodyfont=\normalfont,
%     mdframed={
%         linewidth=2pt,
%         rightline=false, topline=false, bottomline=false,
%         linecolor=NavyBlue
%     }
% ]{thmNotation} % Notation = Blue line
% \declaretheoremstyle[
%     headfont=\bfseries\sffamily\color{NavyBlue!70!black}, bodyfont=\normalfont,
%     mdframed={
%         linewidth=2pt,
%         rightline=false, topline=false, bottomline=false,
%         linecolor=NavyBlue, backgroundcolor=NavyBlue!5,
%     }
% ]{thmbluebox}
% \declaretheoremstyle[
%     headfont=\bfseries\sffamily\color{RawSienna!70!black}, bodyfont=\normalfont,
%     mdframed={
%         linewidth=2pt,
%         rightline=false, topline=false, bottomline=false,
%         linecolor=RawSienna, backgroundcolor=RawSienna!5,
%     }
% ]{thmredbox}
% \declaretheoremstyle[
%     headfont=\bfseries\sffamily\color{NavyBlue!70!black}, bodyfont=\normalfont,
%     numbered=no,
%     mdframed={
%         linewidth=2pt,
%         rightline=false, topline=false, bottomline=false,
%         linecolor=NavyBlue, backgroundcolor=NavyBlue!1,
%     },
% ]{thmexplanationbox}
% \declaretheorem[style=thmNotation,numbered=no,name=Notation]{notation}
% \declaretheorem[style=thmDefi,numbered=no,name=Definition]{defi}
% \declaretheorem[style=thmProof,numbered=no,name=Proof]{replacementproof}
\newtheorem*{aim}{Aim}
\newtheorem*{assumption}{Assumption}
\newtheorem*{axiom}{Axiom}
\newtheorem*{claim}{Claim}
\newtheorem*{conjecture}{Conjecture}
\newtheorem*{cor}{Corollary}
\newtheorem*{defi}{Definition}
\newtheorem*{eg}{Example}
\newtheorem*{exercise}{Exercise}
\newtheorem*{ex}{Exercise}
\newtheorem*{fact}{Fact}
\newtheorem*{law}{Law}
\newtheorem*{lemma}{Lemma}
\newtheorem*{notation}{Notation}
\newtheorem*{prop}{Proposition}
\newtheorem*{question}{Question}
\newtheorem*{remark}{Remark}
\newtheorem*{sol}{Solve}
\newtheorem*{rrule}{Rule}
\newtheorem*{thm}{Theorem}
\newtheorem*{warning}{Warning}
\newtheorem{nthm}{Theorem}[section]
\newtheorem{ncor}[nthm]{Corollary}
\newtheorem{nlemma}[nthm]{Lemma}
\newtheorem{nprop}[nthm]{Proposition}
% \renewenvironment{proof}[1][\proofname]{\vspace{-10pt}\begin{replacementproof}}{\end{replacementproof}}

%─────────────%
% Math Symbol %
%─────────────%

%────────────%
% Enum items %
%────────────%
\setlist[description]{noitemsep}
\setlist[enumerate,1]{noitemsep,leftmargin=4em,label=\emph{\alph*}.}
\setlist[enumerate,2]{noitemsep,leftmargin=1em,label=$\circ$}
\setlist[itemize,1]{noitemsep,leftmargin=3em,label=$\bullet$}
\setlist[itemize,2]{noitemsep,leftmargin=1em,label=$\circ$}

%────────────%
% Beginnings %
%────────────%
\title{\textbf{Part III-B: \className}}
\author{Lecture by \lecturer\\Note by \noter}
\pagestyle{CustomStyle}

%────────────────────────────────%
% Page Settings (set when print) %
%────────────────────────────────%
% \addtolength{\parskip}{-1mm}
% \addtolength{\parindent}{-2mm}
% \geometry{left=0.5cm,right=0.5cm,top=0.5cm,bottom=0.5cm}


%──────────%
% Document %
%──────────%
\begin{document}
\maketitle
\tableofcontents
\section*{概述}%
\label{sec:概述}
\subsection*{资源}%
\label{sub:资源}
公众号:狗熊会、大数据文摘,好玩的数学

MOOC:爱课程,Coursera,Edx,网易公开课等

\subsection*{教师要求}%
\label{sub:教师要求}

教材:概率论与数理统计第二版

参考:The Lady Tasting Tea,程序员数学之概率统计,\ldots

学习目的:自问自答,自言自语

考核及成绩组成:\\期中(10)\\作业与考勤(10)\\期末(70)\\MOOC(10)

\subsection*{课程简介}%
\label{sub:课程简介}
概率:Probability

统计:Statistics

概率论与数理统计:Probability theory and Mathematical statistics

\begin{notation}
第一章重要但不突出
\end{notation}
从概率到概率论:新增时间(随机事件、样本空间变化)

从统计到数理统计:统计最开始为记录性质,后来衍生出预测,通过数学模型引入数理统计

类似的还有政府统计、经济统计等

2000-2015年间,IT时代逐渐转换为DT(Data Technology)时代,大数据逐渐占时代主体
\section{第一章}%
\label{sec:第一章}

\subsection{随机事件}%
\label{sub:随机事件}
\subsubsection{现象}%
\label{subsub:现象}
确定性现象:一定条件下必然发生

随机现象强调统计规律性

\begin{notation}
统计规律性:

1. 每次试验前不能预测结果

2. 结果不止一个

3. 大量试验下有一定规律
\end{notation}
\begin{eg}
    星际旅行时宇航员看到的现象不是随机现象:

    对星际旅行的人而言,无法完成大量试验

    宇航员观测到的结果无规律,只能称为不确定现象(Uncertain)
\end{eg}
\begin{eg}
    扔一个骰子不能预测结果,但可以知道结果是$1,2,3,4,5,6$的一个,因此观察扔骰子是随机现象(Random)
\end{eg}
\subsubsection{随机试验}%
\label{subsub:随机试验}
随机试验(E):研究随机现象时进行的实验或观察等

\begin{notation}
    随机试验的特性:

    1. 可以在完全相同的条件下重复进行

    2. 试验的可能结果在试验前已知

    3. 试验的结果不可预测
\end{notation}
\subsubsection{样本}%
\label{subsub:样本}
在随机试验中,不可再分的最简单结果成为样本点$\omega$,全体样本点组成样本空间$\Omega$
\begin{notation}
    随机事件是基本事件的集合
\end{notation}
\begin{eg}
    扔骰子存在6个基本事件,可以产生$2^6$ 个随机事件,其中样本空间$\Omega=\{x|x\in \left[ 1,6 \right] ,x\in \mathbb{R}\}$
\end{eg}
\begin{eg}
    1. 射击时用$\omega_i$ 表示击中$i$ 环,样本空间为:\[
        \Omega=\{\omega_0,\omega_1,\omega_2,\ldots,\omega_{10}\}
    .\]

    2. 微信用户每天收到信息条数的取值范围是$\left[ 0,+\infty \right) $,样本空间为无限集:\[
        \Omega=\left\{ N|N\ge 0 ,N\in \mathbb{R}\right\} 
    .\] 

    3. 电视机的寿命样本空间为$\Omega=\left\{ t|t>0 \right\} $,为连续的非负实数集

    4. 投掷两枚硬币,样本空间为$\Omega=\left\{ \left( x,y \right) |x,y=0,1 \right\} $,其中$0,1$ 分别代表正面和背面
\end{eg}
\begin{notation}
    1. 样本点可以不是数

    2. 样本空间可以是无限集
\end{notation}
\subsubsection{随机事件}%
\label{subsub:随机事件}
\subsection{事件关系与运算}%
\label{sub:事件关系与运算}
1.1. $A\subset B$:$A$ 发生必然$B$ 发生

1.2. $A=B$ :$A\subset B, B\subset A$ 

2. $A\cup B$:$A$ 和$B$ 至少有一个发生

2.1 $A_1\cup A_2\cup \ldots\cup A_n=\bigcup_{i=1}^{n}A_1$

3. $A\cap B$:$A$ 和$B$ 只发生一个

4.1. $A,B$互斥:不能同时发生: $AB=\varnothing$

4.2. $A,B$对立:非此即彼:$A\cup B=\Omega$

5. $A-B$:$A \bar{B}$或$A(\Omega-B)$,或$A$ 发生但$B$ 不发生
\begin{notation}
    $A-B=A\bar{B}\subset A$,$B-A=B\bar{A}\subset B$

    当$AB=\varnothing$ 时,$A-B= A,B-A=B$
\end{notation}

\begin{notation}
    $P\left( \Omega \right) =1,P\left( \varnothing \right) =0$,且$P\left( \Omega \right) +P\left( \varnothing \right) =1$ ,即$\Omega$ 与$\varnothing$ 互斥
\end{notation}

6. 结合律:$\left( A\cup B \right) \cup C = A\cup \left( B\cup C \right) $

7. 分配律:$\left( AB \right) \cup C=\left( A\cup C \right) \left( B\cup C \right) $ ,$\left( AUB \right) C =AC\cup BC$

8. 交换律:$A\cup B=B\cup A,AB=BA$
\begin{notation}
    德摩根律:\[
    \overline{\bigcup_{i=1}^{n}A_i} = \bigcap_{i=1}^{n}\overline{A_i}
    .\] 
    \[
        \overline{\bigcap_{i=1}^{n}A_i}=\bigcup_{i=1}^{n}\overline{A_i}
    .\] 
\end{notation}
\begin{eg}
     \[
        \overline{A\cup B}=\bar{A}\bar{B}
    .\]
    \[
        \overline{\left( A\cup B \right) \cup C} = \overline{A\cup B}\bar{C}=\ldots
    .\] 
    \[
        \overline{A\cap B}=\bar{A}\cup \bar{B}
    .\] 
\end{eg}
\subsection{事件的概率}%
\label{sub:事件的概率}
概率分类:
\begin{align*}
    \begin{cases}
        \mbox{主观概率}\\ 
        \mbox{统计概率}\\
        \mbox{古典概型}\\
        \mbox{几何概型}
    \end{cases}
.\end{align*}
\begin{notation}
    德摩根、蒲丰、皮尔逊、维纳均进行过投掷硬币的试验,随着试验次数的增加,出现正面的频率逐渐接近0.5

    大数定律说明,该事件的概率为0.5
\end{notation}
\begin{defi}
    统计概率:$A$为试验$E$的一个事件,随着重复次数$n$的增加,$A$的频率接近于某个常数$p$,定义事件$A$的概率为$p$,记为$P\left( A \right) =p$
\end{defi}
频率的特性:

1. 非负性:$f_n\left( A \right) \in \left[ 0,1 \right] $

2. 规范性:$f_n\left( \Omega \right) =1$

3. 有限可加性:$A_i$ 两两互斥,则$f_n\left( \sum_{i=1}^{n} A_i \right) =\sum_{i=1}^{n} f_n\left( A_i \right) $
\begin{defi}
    主观概率:人对某个事件发生与否的可能性的估计
\end{defi}

\begin{defi}
    完备事件组:$A_1,A_2,\ldots,A_n$两两互斥,且\[
        P\left( \sum_{i=1}^{n} A_i \right) =1
    .\] 或\[
        \sum_{i=1}^{n} A_i=\Omega
    .\] 则称$A_1\to A_n$ 为完备事件组(不重不漏)
\end{defi}
\begin{eg}
    $A,\bar{A}$ 是完备事件组
\end{eg}
\subsubsection{古典概型}%
\label{subsub:古典概型}
古典概型特点:有限等可能性(基本事件数有限,基本事件发生的可能性相等)
\begin{notation}
    概率计算:\[
        P\left( A \right) =\frac{m}{n} = \frac{n\left( A \right) }{n\left( \Omega \right) }
    .\] 
\end{notation}
\begin{eg}
    某年级有6人在9月份出生,求6个人中没有人同一天过生日的概率

    基本事件总数:$30^6$

    目标事件: $30\cdot 29\cdot 28\cdot 27\cdot 26\cdot 25=P_{30}^{6}$ 

    概率:\[
        P\left( A \right) =\frac{P_{30}^{6}}{30^6}
    .\] 
\end{eg}
\begin{eg}
    有$N$个乒乓球中有$M$个白球、$N-M$个白球,任取$n(n<N)$个球,分有放回和不放回,求取到$m$个黄球的概率

    1. 不放回:

    基本事件总数: $C_{N}^{n}$ 

    目标事件:$C_{M}^{m}C_{N-M}^{n-m}$ 

    概率: \[
        P=\frac{C_{M}^{m}C_{N-M}^{n-m}}{C_{N}^{n}},n=\max\left\{ 0,n-\left( N-M \right)  \right\}, \ldots,\min\left\{ n,M \right\} 
    .\] 

    2. 有放回:

    \[
    P=\frac{C_{n}^{m}M^m\left( N-M \right) ^{n-m}}{N^n}=C_{n}^{m}\left( \frac{M}{N} \right) ^m\left( 1-\frac{M}{N} \right) ^{n-m},m\in \left[ 0,n \right]
    .\] 
    注意到该概率为伯努利分布$C_{n}^{m}B\left( n,\frac{M}{N} \right) $
\end{eg}
匹配问题:
\begin{eg}
    麦克斯韦-玻尔兹曼统计问题:

    $n$个质点随机落入$N\left( N>n \right) $ 个盒子,盒子容量不限,设$A$ 表示指定的$n$个盒子各有一个质点,$B$ 表示恰好有$n$个盒子装一个质点

    基本事件总数:$N^n$ 

    $A$考虑顺序,即:\[
        P\left( A \right) =\frac{n!}{N^n}
    .\] 

    同理:\[
        P\left( B \right) =\frac{C_{N}^{n}}{N^n}
    .\] 
\end{eg}        
\subsubsection{几何概型}%
\label{subsub:几何概型}
几何概型特点:使用事件所对应的\textbf{几何度量}计算
\[
    P\left( A \right) =\frac{m\left( A \right) }{m\left( \Omega \right) }
.\] 
\begin{notation}
    度量:面积、体积、长度等描述几何量大小的测度方式
\end{notation}
\begin{eg}
    地面铺满2 dm的地砖,向地面投掷一个$r=0.5$ dm的光盘,求光盘不与边线相交的概率
\end{eg}
如图:
\begin{center}
    \begin{tikzpicture}
        \draw [] (-4,4) rectangle (4,-4);
        \draw [] (-2,2) circle [radius=2] node at(-2,2) {};
        \draw [dashed] (-2,2) rectangle (2,-2);
    \end{tikzpicture}
\end{center}
课后习题:A组8题,B组3题
\begin{eg}
    两人相约8-9点间在某地相见,先到的人等待20分钟后离去,求二人会面的概率

    设$\left( x,y \right) $ 分别表示两人到达的时刻

    设G为样本空间,绘制样本空间:
    \begin{center}
        \begin{tikzpicture}
            \draw [->] (0,0)--(0,7);
            \draw [->] (0,0)--(7,0);
            \draw [] (0,0) rectangle (6,6) node at(5,5) {$G$};
            
        \end{tikzpicture}
    \end{center}

    由题:两人到达的时间之差的绝对值小于20分钟($\frac{1}{3}$ 小时),即:\[
        |x-y|\le \frac{1}{3}
    .\] 
    将事件绘制:
    \begin{center}
        \begin{tikzpicture}
            \draw [->] (0,0)--(0,7);
            \draw [->] (0,0)--(7,0);
            \draw [] (0,0) rectangle (6,6) node at(5,1) {$G$};
            \draw [dashed] (2,0)--(6,4);
            \draw [dashed] (0,2)--(4,6);
            \node[] (g) at (3,3) {$g$};
            
        \end{tikzpicture}
    \end{center}
    \[
        P\left( g \right) =\frac{m\left( g \right) }{m\left( G \right) }=\frac{S\left( g \right) }{S\left( G \right) }=\frac{1-\left( \frac{2}{3} \right) ^2}{1}=\frac{4}{9}
    .\] 
\end{eg}
\begin{notation}
    几何概型的特点:

    1. 非负性:\[
        P\left( A \right) \in \left[ 0,1 \right] 
    .\] 

    2. 规范性:\[
        P\left( \Omega \right) =1
    .\] 

    3. 可列可加性: \[
        P\left( \sum_{i=1}^{n} A_i \right) =\sum_{i=1}^{n} P\left( A_i \right) 
    .\] 
\end{notation}

\subsection{公理化}%
\label{sub:公理化}
\[
    \left( \Omega,\mathscr{F},p \right) 
.\] 
\begin{defi}
    $\Omega$:随机试验所产生的所有样本点的集合

    $\mathscr{F}$:集合内所有子集为元素的集合

    $P\left( X \right) $:概率函数
\end{defi}

\begin{axiom}
    非负性:\[
        P\left( A \right) \ge 0,A\in \mathscr{F}
    .\] 
\end{axiom}
\begin{axiom}
    规范性:\[
        P\left( \Omega \right) =1
    .\] 
\end{axiom}
\begin{axiom}
    可列可加性:对两两互斥的事件$A_1,A_2,\ldots$,\[
        P\left( \sum_{i=1}^{+\infty} A_i \right) =\sum_{i=1}^{+\infty} P\left( A_i \right) 
    .\] 
\end{axiom}
从三条公理得出的性质:
\begin{notation}
    1. $P\left( \varnothing \right) =0$ 

    2. 有限可加性: \[
        \sum_{i=1}^{n} P\left( A_i \right) =P\left( \sum_{i=1}^{n} A_i \right) 
    .\] 

    3. $P\left( \bar{A} \right) =1-P\left( A \right) $

    4. $A\subset B\implies P\left( B-A \right) =P\left( B \right) -P\left( A \right) $

    5. $A\subset B\implies P\left( A \right) \le P\left( B \right) $

    6. $P\left( A\cup B \right) =P\left( A \right) +P\left( B \right) -P\left( AB \right) $


\end{notation}
\begin{notation}
    6.1. \[
    P\left( A\cup B\cup C \right) =P\left( A \right) +P\left( B \right) +P\left( C \right) -P\left( AB \right) -P\left( BC \right) 
    \]
    \[
    -P\left( AC \right) +P\left( ABC \right) 
    .\] 
    6.2. \[
        P\left( \bigcup_{i=1}^{n}A_i \right) =\sum_{i=1}^{n} P\left( A_i \right) -\sum_{i=1}^{n-1} \sum_{j=i+1}^{n} P\left( A_{i}A_{j} \right)+\sum_{i=1}^{n-2} \sum_{j=i+i}^{n-1} \sum_{k=j+1}^{n} P\left( A_{i}A_{j}A_{k} \right) 
    \]
    \[
        \ldots+\left( -1 \right) ^{n-1}P\left( \prod_{i=1}^{n} A_{i}\right)  
    .\] 
\end{notation}

\begin{eg}
    从有号码$1,2,\ldots,n$的$n$个球中有放回地取$m$个球,求取出的m个球中最大号码为$k$的概率

    \[
        P\left\{ k=1 \right\} =\left( \frac{1}{n} \right) ^m
    .\]  

    逐个列举计算较复杂,记事件$B_k$ 为取出的$m$个球最大号码不超过k,只需保证每次摸出的球都不超过$k$即可:
    \[
        P\left( B_{k} \right) =\frac{k^m}{n^m}
    .\] 

    又有$P\left( A_{k} \right) =P\left( B_{k} \right) -P\left( B_{k-1} \right) $,且$B_{k-1}\subset B_{k}$ 

    所以:\[
        P\left( A_k \right) =\frac{k^m}{n^m}-\frac{\left( k-1 \right) ^m}{n^m}
    .\] 
\end{eg}
 
\begin{eg}
    匹配问题:$n$个学生各带有一个礼品,随机分配礼品,设第$i$个人抽到自己的礼品称为一个配对,求至少有一个配对的概率

    设$A_{i}$ 是第$i$个人抽到自己的礼品,$A$为目标事件,则: \[
        A=\bigcup_{i=1}^{n}A_{i}
    .\] 
    \[
        P\left( A_{i} \right) =\frac{\left( n-1 \right) !}{n!}=\frac{1}{n}
    .\] 
    \[
        P\left( A_iA_j \right) =\frac{\left( n-2 \right) !}{n!}=\frac{1}{P_{n}^{2}}
    .\] 
    \[
        P\left( A_iA_jA_k \right) =\frac{1}{P_{n}^{3}}
    .\] 
    \[
        \ldots\ldots\ldots
    .\] 
    \[
        P\left( \prod_{i=1}^{n} A_i  \right) =\frac{1}{n!}
    .\] 
    \[
        P\left( A \right) =P\left( \sum_{i=1}^{n} A_{i} \right) = \sum_{i=1}^{n} P\left( A_{i} \right) -\sum_{i=1}^{n-1} \sum_{j=i+1}^{n} P\left( A_iA_j \right) +\ldots
    .\] 
\end{eg}


\subsection{条件概率与乘法公式}%
\label{sub:条件概率与乘法公式}
\begin{defi}
    \[
        P\left( A \right) >0,P\left( B | A \right) =\frac{P\left( AB \right) }{P\left( A \right) }
    .\] 
    即:在\textbf{A发生的条件下},B发生的概率
\end{defi}
\begin{defi}
    乘法公式:\[
        P\left( AB \right) =P\left( A \right) P\left( B|A \right) =P\left( B \right) P\left( A|B \right) 
    .\] 
\end{defi}
\begin{notation}
    $A,B$独立:$P\left( AB \right) =P\left( A \right) P\left( B \right) $

    结合乘法公式:\[
        P\left( B \right) =P\left( B|A \right) 
    .\] 
    \[
        P\left( A \right) =P\left( A|B \right) 
    .\] 
\end{notation}
\subsection{全概率公式}%
\label{sub:全概率公式}
\begin{cor}
    事件$A_1,A_2,\ldots,A_n$为完备事件组,事件$B\subset \Omega=\bigcup_{i=1}^{n}A_i$,则:\[
        P\left( B \right) =\sum_{i=1}^{n} P\left( A_i \right) P\left( B|A_i \right) 
    .\] 
\end{cor}
\begin{notation}
    此时完备事件组的情况应该已知,通过完备事件组$A$的辅助可以求得较复杂事件$B$的概率
\end{notation}


\subsection{贝叶斯公式}%
\label{sub:贝叶斯公式}
\begin{cor}
    \[
        P\left( A_{k}|B \right) =\frac{P\left( A_{k}B \right) }{P\left( B \right) }=\frac{P\left( A_{k} \right) P\left( B|A_k \right) }{\sum_{i=1}^{n} P\left( A_i \right) P\left( B|A_i \right)}
    .\] 
\end{cor}
贝叶斯公式被称为“逆概率公式/后验公式”,其中事件$B$更可能是事件的结果,将事件组$A$看作结果出现的原因,则贝叶斯公式是一个从“结果”推“原因”的可能性的公式

\begin{notation}
    对比一般公式:事件A导致B,求B发生的概率

    贝叶斯公式:事件A导致B,A中的一个事件$A_i$ 导致B发生的概率
\end{notation}

\begin{axiom}
    条件概率的公理:

    1. 非负性:$P\left( A \right) \in \left[ 0,1 \right] $

    2. 规范性:$P\left( \Omega|A \right) =1$

    3. 可列可加性: \[
        P\left( \sum_{i=1}^{\infty} B_i|A \right) =\sum_{i=1}^{\infty} P\left( B_i|A \right) 
    .\] 
\end{axiom}
\begin{cor}
    \[
        P\left( \bar{B}|A \right) =P\left( \Omega-B|A \right) =P\left( \Omega|A \right) -P\left( B|A \right)
    .\] 
\end{cor}
\begin{cor}
    \[
        P\left( B_1\cup B_2 \right) =P\left( B_1 \right) +P\left( B_2 \right) -P\left( B_1B_2 \right) 
    .\] 
    \[
        \implies P\left( B_1\cup B_2|A \right) =P\left( B_1|A \right) +P\left( B_2|A \right) -P\left( B_1B_2|A \right) 
    .\] 
\end{cor}
\begin{cor}
    乘法公式:
    \[
        P\left( ABC \right) =P\left( A\left( BC \right)  \right) =P\left( A \right) P\left( BC|A \right) =P\left( A \right) P\left( B|A\right) P\left( C|AB \right)  
    .\] 
\end{cor}
\begin{eg}
    8个红球2个白球,求前三次结果是“红红白”的概率:

    1. 不放回取3个(和一次取三个球相同)

    所有可能性:$10\times 9\times 8$ 

    目标事件:$8\times 7\times 2$

    或使用乘法公式:设$A_i$ 为第$i$次取到红球,目标事件可表示为$A_1A_2\bar{A_3}$

    概率:\[
        P\left( A_1A_2\bar{A_3} \right) =P\left( A_1 \right) P\left( A_2|A_1 \right) P\left( \bar{A_3}|A_1A_2 \right) =\frac{8}{10}\times \frac{7}{9}\times \frac{2}{8}=\frac{7}{45}
    .\] 

    2. 每次取后放回,并加入两个同色的球,取3次(不能使用古典概型)

    概率:

    \[
        P\left( A_1A_2\bar{A_3} \right) =\frac{8}{10}\times \frac{8}{12}\times \frac{2}{14}=\frac{8}{105}
    .\] 
\end{eg}
\begin{eg}
    某疾病的发病率为0.0004,患病检测呈阳性的概率为0.99,误诊为阴性的概率为0.01,误诊为阳性的概率为0.05,不患病检测呈阴性概率为0.95,一个人检测呈阳性,求其患病的概率

    设阳性为$A$,患病为$B$

    则: \[
        P\left( A|B \right) =0.99,P\left( A|\bar{B} \right) =0.05,P\left( B \right) =0.0004
    .\] 
    要求:$P\left( B|A \right) $ 

    使用贝叶斯公式:

    \[
        P\left( B|A \right) =\frac{P\left( AB \right) }{P\left( A \right) }=\frac{P\left( B \right) P\left( A|B \right) }{P\left( AB \right) +P\left( A\bar{B} \right) }
    .\] 
    \[
        =\frac{P\left( B \right) P\left( A|B \right) }{P\left( B \right) P\left( A|B \right) +P\left( A|\bar{B} \right) P\left( \bar{B} \right) }=0.0079
    .\] 
\end{eg}
\subsection{独立性}%
\label{sub:独立性}
\begin{defi}
    $A,B$ 独立,则:$P\left( A|B \right) =P\left( A \right) $
\end{defi}
\begin{notation}
    证明独立性:

    1. $P\left( A \right) P\left( B \right) =P\left( AB \right) $ 
\end{notation}
\begin{notation}
    独立事件的特点:
    
    1. $A,B$独立有:$A,B$所有的组合(包含补集)均独立

    2. $A,B$独立的充要条件:$P\left( A|B \right) =P\left( A \right) \text{ or } P\left( B|A \right) =P\left( B \right) $

    3. $\varnothing$ 与任何随机事件独立,$\Omega$ 与任何随机事件独立
\end{notation}
对于三个事件相互独立:
\[
    \begin{cases}
        P\left( AB \right) =P\left( A \right) P\left( B \right) \\
        P\left( AC \right) =P\left( A \right) P\left( C \right) \\
        P\left( BC \right) =P\left( C \right) P\left( C \right) \\
        P\left( ABC \right) =P\left( A \right) P\left( B \right) P\left( C \right) 
    \end{cases}
.\] 

对比乘法公式:
$P\left( ABC \right) =P\left( A \right) P\left( B|A \right) P\left( C|AB \right) $

\begin{defi}
    相互独立:

    有$A_1,A_2,\ldots,A_n$事件组,对$\forall s \in \left[ 2, n\right] $个事件$A_{k_1},A_{k_2},\ldots,A_{k_s}$ 均有:\[
        P\left( \prod_{n=1}^{s} A_{k_n}  \right) =\prod_{n=1}^{s} P\left( A_{k_n} \right)  
    .\] 

    称事件$A_1,A_2,\ldots,A_{n}$相互独立
\end{defi}

\begin{defi}
    两两独立:对事件$A_1,A_2,\ldots,A_n$ ,若任意两个事件独立,则称为两两独立
\end{defi}

\begin{notation}
    相互独立一定两两独立,反之不一定
\end{notation}

\begin{notation}
    相互独立事件组的性质:

    1. 事件$A_1,A_2,\ldots,A_n$ 相互独立,将其中任意部分改为对立事件,事件组仍为相互独立

    2. 事件相互独立,将事件组任意分为两组(或多组),对组内事件进行“并、交、差、补”操作后,事件间依然相互独立
\end{notation}
\subsubsection*{独立重复实验}%
\label{subsub:独立重复实验}
\begin{defi}
    $E_1,E_2$中一个试验的任何结果和另一个试验的任何结果相互独立,则试验相互独立;若$n$个独立试验相互独立且试验相同,称$E_1,E_2,\ldots,E_n$为$n$次独立重复实验,或$n$重独立试验

    \begin{eg}
        扔硬币和掷骰子为独立试验,其中扔硬币为伯努利试验(只有两个结果)
    \end{eg}
    
\end{defi}
\begin{defi}
    $n$重独立试验$E$中,每次试验都是伯努利试验(可能结果只有两个),称$E$为$n$重伯努利试验
\end{defi}
    1. 二项概率公式:成功$k$次的概率记为$P_n\left( k \right) $,假定前$k$次成功,后$n-k$次失败,则\[
        P_i =p^k\left( 1-p \right) ^{n-k}
    .\] 

    指定事件$A$发生的位置有$C_{n}^{k}$ 种,则:\[
        P_n\left( k \right) =C_{n}^{k}p^{k}\left( 1-p \right) ^{n-k}
    .\] 

    称为二项概率公式

    2. 几何概率公式:首次成功恰好发生在第$k$次的概率记为$G\left( k \right) $ ,设前$k-1$次失败,则:\[
        G\left( k \right)= q^{k-1}p
    .\] 

    可以验证:$\sum G\left( k \right) =1$ 

    3. 负二项概率:需要成功$r$次,第$r$次成功恰好发生在第$k$次的概率记为$G_r\left( k \right) $ ,设前$k-1$次试验有$r-1$次成功,则:\[
        G_r\left( k \right) =C_{k-1}^{n-1}p^rq^{k-r}
    .\] 

    同样有:$\sum G_r\left( k \right) =1$





























\end{document}
