%─────────────────%
% Header Settings %
%─────────────────%
\def\lecturer{李漫漫}
\def\noter{THF}
\def\className{Math Homework}
\def\term{III-B}
%─────────────────────%
% Undefined Variables %
%─────────────────────%
\ifx\noter\undefined
    \def\noter{THF}
\else
\fi
\ifx\lecturer\undefined
    \def\lecturer{None}
\else
\fi

%───────────────────%
% Document Settings %
%───────────────────%
\documentclass[12pt,a4paper]{article}

%─────────────────%
% Package Imports %
%─────────────────%
\usepackage[]{amsmath}
\usepackage[]{amssymb}
\usepackage[]{amsthm}
\usepackage[]{array}
\usepackage[]{bm}
\usepackage[]{booktabs}
\usepackage[UTF8]{ctex}
\usepackage[]{fancyhdr}
\usepackage[]{float}
\usepackage[]{geometry}
\usepackage[]{graphicx}
\usepackage[]{hyperref}
\usepackage[]{import}
\usepackage[]{inputenc}
\usepackage[]{mathrsfs}
\usepackage[]{multirow}
\usepackage[]{pdfpages}
\usepackage[]{pgfplots}
\usepackage[]{stmaryrd}
\usepackage[]{tabu}
\usepackage[]{tcolorbox}
\usepackage[]{textcomp}
\usepackage[]{thmtools}
\usepackage[]{tikz}
\usepackage[]{tkz-euclide}
\usepackage[]{url}
\usepackage[]{wrapfig}
\usepackage[dvipsnames]{xcolor}
\usepackage[]{xifthen}
\usepackage[]{yhmath}

%────────────────%
% Fancy Settings %
%────────────────%
\fancypagestyle{CustomStyle}{%
    \fancyhf{}
    \setlength{\headheight}{14.49998pt}
    \fancyhead[R]{\thepage}
    \fancyhead[L]{\lecturer: \className}
}

%──────────────────%
% pgfplot Settings %
%──────────────────%
\usepgfplotslibrary{external}
\pgfarrowsdeclarecombine{twolatex'}{twolatex'}{latex'}{latex'}{latex'}{latex'}
\pgfplotsset{compat=1.12}

%───────────────%
% Tikz Settings %
%───────────────%
\usetikzlibrary{arrows.meta}
\usetikzlibrary{decorations.markings}
\usetikzlibrary{decorations.pathmorphing}
\usetikzlibrary{positioning}
\usetikzlibrary{fadings}
\usetikzlibrary{intersections}
\usetikzlibrary{cd}
\tikzset{->/.style = {decoration={markings,mark=at position 1 with {\arrow[scale=2]{latex'}}},postaction={decorate}}}
\tikzset{<-/.style = {decoration={markings,mark=at position 0 with {\arrowreversed[scale=2]{latex'}}},postaction={decorate}}}
\tikzset{<->/.style = {decoration={markings,mark=at position 0 with {\arrowreversed[scale=2]{latex'}},mark=at position 1 with {\arrow[scale=2]{latex'}}},postaction={decorate}}}
\tikzset{->-/.style = {decoration={markings,mark=at position #1 with {\arrow[scale=2]{latex'}}},postaction={decorate}}}
\tikzset{-<-/.style = {decoration={markings,mark=at position #1 with {\arrowreversed[scale=2]{latex'}}},postaction={decorate}}}
\tikzset{->>/.style = {decoration={markings,mark=at position 1 with {\arrow[scale=2]{latex'}}}postaction={decorate}}}
\tikzset{<<-/.style = {decoration={markings,mark=at position 0 with {\arrowreversed[scale=2]{twolatex'}}},postaction={decorate}}}
\tikzset{<<->>/.style = {decoration={markings,mark=at position 0 with {\arrowreversed[scale=2]{twolatex'}},mark=at position 1 with {\arrow[scale=2]{twolatex'}}},postaction={decorate}}}
\tikzset{->>-/.style = {decoration={markings,mark=at position #1 with {\arrow[scale=2]{twolatex'}}},postaction={decorate}}}
\tikzset{-<<-/.style = {decoration={markings,mark=at position #1 with {\arrowreversed[scale=2]{twolatex'}}},postaction={decorate}}}
\tikzset{circ/.style = {fill, circle, inner sep = 0, minimum size = 3}}
\tikzset{scirc/.style = {fill, circle, inner sep = 0, minimum size = 1.5}}
\tikzset{mstate/.style={circle, draw, blue, text=black, minimum width=0.7cm}}
\tikzset{eqpic/.style={baseline={([yshift=-.5ex]current bounding box.center)}}}
\tikzset{commutative diagrams/.cd,cdmap/.style={/tikz/column 1/.append style={anchor=base east},/tikz/column 2/.append style={anchor=base west},row sep=tiny}}

%──────────────────────%
% Theorem Environments %
%──────────────────────%
\theoremstyle{definition}
\declaretheoremstyle[
    headfont=\bfseries\sffamily\color{ForestGreen!70!black}, bodyfont=\normalfont,
    mdframed={
        linewidth=2pt,
        rightline=false, topline=false, bottomline=false,
        linecolor=ForestGreen, backgroundcolor=ForestGreen!5,
    }
]{thmgreenbox}
\declaretheoremstyle[
    headfont=\bfseries\sffamily\color{NavyBlue!70!black}, bodyfont=\normalfont,
    mdframed={
        linewidth=2pt,
        rightline=false, topline=false, bottomline=false,
        linecolor=NavyBlue, backgroundcolor=NavyBlue!5,
    }
]{thmbluebox}
\declaretheoremstyle[
    headfont=\bfseries\sffamily\color{NavyBlue!70!black}, bodyfont=\normalfont,
    mdframed={
        linewidth=2pt,
        rightline=false, topline=false, bottomline=false,
        linecolor=NavyBlue
    }
]{thmblueline}
\declaretheoremstyle[
    headfont=\bfseries\sffamily\color{RawSienna!70!black}, bodyfont=\normalfont,
    mdframed={
        linewidth=2pt,
        rightline=false, topline=false, bottomline=false,
        linecolor=RawSienna, backgroundcolor=RawSienna!5,
    }
]{thmredbox}
\declaretheoremstyle[
    headfont=\bfseries\sffamily\color{RawSienna!70!black}, bodyfont=\normalfont,
    numbered=no,
    mdframed={
        linewidth=2pt,
        rightline=false, topline=false, bottomline=false,
        linecolor=RawSienna, backgroundcolor=RawSienna!1,
    },
    qed=\qedsymbol
]{thmproofbox}
\declaretheoremstyle[
    headfont=\bfseries\sffamily\color{NavyBlue!70!black}, bodyfont=\normalfont,
    numbered=no,
    mdframed={
        linewidth=2pt,
        rightline=false, topline=false, bottomline=false,
        linecolor=NavyBlue, backgroundcolor=NavyBlue!1,
    },
]{thmexplanationbox}
\declaretheorem[style=thmblueline,numbered=no,name=Notation]{notation}
\declaretheorem[style=thmgreenbox,numbered=no,name=Definition]{defi}
\declaretheorem[style=thmproofbox,numbered=no,name=Proof]{replacementproof}
\newtheorem*{aim}{Aim}
\newtheorem*{assumption}{Assumption}
\newtheorem*{axiom}{Axiom}
\newtheorem*{claim}{Claim}
\newtheorem*{conjecture}{Conjecture}
\newtheorem*{cor}{Corollary}
\newtheorem*{eg}{Example}
\newtheorem*{exercise}{Exercise}
\newtheorem*{ex}{Exercise}
\newtheorem*{fact}{Fact}
\newtheorem*{law}{Law}
\newtheorem*{lemma}{Lemma}
\newtheorem*{prop}{Proposition}
\newtheorem*{question}{Question}
\newtheorem*{remark}{Remark}
\newtheorem*{rrule}{Rule}
\newtheorem*{thm}{Theorem}
\newtheorem*{warning}{Warning}
% \newtheorem{ncor}[nthm]{Corollary}
% \newtheorem{nlemma}[nthm]{Lemma}
% \newtheorem{nprop}[nthm]{Proposition}
% \newtheorem{nthm}{Theorem}[section]
\renewenvironment{proof}[1][\proofname]{\vspace{-10pt}\begin{replacementproof}}{\end{replacementproof}}

%─────────────%
% Math Symbol %
%─────────────%

%────────────%
% Beginnings %
%────────────%
\title{\textbf{Part III-B: \className}}
\author{Lecture by \lecturer\\Note by \noter}
\pagestyle{CustomStyle}
%────────────────────────────────%
% Page Settings (set when print) %
%────────────────────────────────%
% \addtolength{\parskip}{-1mm}
% \addtolength{\parindent}{-2mm}
% \geometry{left=0.5cm,right=0.5cm,top=0.5cm,bottom=0.5cm}


%──────────%
% Document %
%──────────%
\begin{document}
%\maketitle
%\tableofcontents
\section{1-A}%
\label{sec:1-A}
\subsection{7.}%
\label{sub:7.}
设事件$A$为一条船卸货时另一条船等待,$x$ 为船甲到达的时间,$y$ 为船乙到达的时间,$A$ 可能以如下方式发生:

1. 甲先到,乙后到,且乙不在甲到达后3小时后才到达,即:$x<y$, $y<x+3$,即:\[
    x\in \left( y-3,y \right) 
.\] 

2. 乙先到,甲后到,且甲不在乙到达后4小时后才到达,即:$y<x$, $x<y+4$,即:\[
    y\in \left( x-4,x \right) 
.\] 

合并后:\[
    \begin{cases}
        y<x+3\\
        y>x-4
    \end{cases}
.\] 
绘制样本空间如下:
\begin{center}
    \begin{tikzpicture}
        \draw [->] (0,0)--(0,7);
        \draw [->] (0,0)--(7,0);
        \node[] (x) at (7,0.5) {$x$};
        \node[] (y) at (0.5,7) {$y$};
        \draw [] (0,6) rectangle (6,0) node at(3,5) {$\Omega$};
        \draw [] (1,0)--(6,5);
        \draw [] (0,0.75)--(5.25,6);
        \node[] (A) at (5,5) {$A$};
        \node[] (4) at (1,-0.5) {$4$};
        \node[] (3) at (-0.5,0.75) {$3$};
        \node[] (24) at (6,-0.5) {$24$};
        \node[] (24) at (-0.5,6) {$24$};
    \end{tikzpicture}
\end{center}
\[
    P\left( A \right) =\frac{S\left( A \right) }{S\left( \Omega \right) }
.\] 
\[
    S\left( \Omega \right) =24\cdot 24=576
.\] 
\[
    S\left( A \right) =S\left( \Omega \right) -\frac{1}{2}\cdot 20^2-\frac{1}{2}\cdot 21^2=155.5
.\] 
\[
    P\left( A \right) =\frac{311}{1152}
.\] 

\subsection{8.}%
\label{sub:8.}
构成三角形的条件:$a+b+c=L$, $a+b>c$(设c为最大边)
 
$c=L-\left( a+b \right)$,$\left( a+b \right) =L-c$

即$L-c>c$,$c<\frac{L}{2}$

即\[
    l\left( A \right) =\frac{L}{2}
.\]
\[
    l\left( \Omega \right) =L
.\] 
\[
    P\left( A \right) =\frac{l\left( A \right) }{l\left( \Omega \right) }=\frac{1}{2}
.\] 
\section{1-B}%
\label{sec:1-B}
\subsection{2.}%
\label{sub:2.}
六次摸球中第四次是黑球:前5次中有3次是黑球,2次是白球

目标事件数:\[
    N\left( A \right) =5\cdot \left( 2^4\cdot 3^2 \right) 
.\]

基本事件总数:$N\left( \Omega \right) =5^6$

事件概率:\[
    P\left( A \right) =\frac{N\left( A \right) }{N\left( \Omega \right) }=\frac{5\cdot \left( 2^4\cdot 3^2 \right) }{5^6} = \frac{144}{3125}
.\] 
\subsection{3.}%
\label{sub:3.}
画单位圆:
\begin{center}
    \begin{tikzpicture}
        \draw [] (0,0) circle [radius=2] node at(-1,1) {$O$};
        \draw [->] (-3,0)--(3,0) node at(3,-0.5) {$x$};
        \draw [->] (0,-3)--(0,3) node at(0.5,3) {$y$};
        \draw [dashed] (-2,0)--(1.42,-1.42)--(1,1.73)--(-2,0);
        \node[] (A) at (-2.5,0.5) {$A$};
        \node[] (B) at (1.7,-1.7) {$B$};
        \node[] (C) at (1.5,2) {$C$};
    \end{tikzpicture}
\end{center}
设$x=\wideparen{AB}$,$y=\wideparen{BC}$ ,$\wideparen{AC}=2\pi-x-y$

$\Delta_{ABC}$ 是锐角三角形的条件:\[
    \begin{cases}
        x>0\\
        y>0\\
        2\pi-\left( x+y \right) >0\\
        x<\pi\\
        y<\pi\\
        2\pi-\left( x+y \right) <\pi
    \end{cases}
    \Rightarrow
    \begin{cases}
        x\in \left( 0,\pi \right) \\
        y\in \left( 0,\pi \right)\\
        x+y>\pi\\
        x+y<2\pi
    \end{cases}
.\] 
绘制样本空间:
\begin{center}
    \begin{tikzpicture}
        \draw [->] (0,0)--(3,0);
        \node[] (x) at (3,-0.5) {$x$};
        \draw [->] (0,0)--(0,3);
        \node[] (y) at (0.5,3) {$y$};
        \draw [] (2.5,0)--(0,2.5);
        \node[] (2pi) at (2.5,-0.5) {$2\pi$};
        \node[] (2pi) at (-0.5,2.5) {$2\pi$};
        \draw [dashed] (1.25,0)--(1.25,1.25)--(0,1.25)--(1.25,0);
        
    \end{tikzpicture}
\end{center}
由图可得:\[
    S\left( A \right) =\frac{3}{4}S
.\] ($\Delta_{ABC}$为钝角三角形)
\[
    S\left( \Omega \right) =S
.\] 
\[
    P\left( A \right) =\frac{S\left( A \right) }{S\left( \Omega \right) }=\frac{3}{4}
.\] 

\section{1.}%
\label{sec:1.}
基本事件总数:$N\left( \Omega \right) =P_{5}^{5}=5\cdot 4\cdot 3\cdot 2\cdot 1=5\cdot P_{4}^{4}$

目标事件:五封信错排的事件数$N\left( A \right) =44$

概率:\[
    P\left( A \right) =\frac{N\left( A \right) }{N\left( \Omega \right) }=\frac{11}{30}
.\] 

\section{2.}%
\label{sec:2.}
\[
    P\left( A \right) =\frac{C_{4}^{1}}{C_{5}^{2}}=\frac{2}{5}
.\] 
求B事件的逆事件:两台电机均来自于供应商I:
\[
    P\left( \bar{B} \right) =\frac{C_{2}^{2}}{C_{5}^{2}}=\frac{1}{10}
.\] 
\[
    P\left( B \right) =1-P\left( \bar{B} \right) =\frac{9}{10}
.\] 

\section{3.}%
\label{sec:3.}
\[
    P\left( A \right) =\frac{C_{4}^{4}}{C_{54}^{4}}=\frac{1}{316251}
.\] 
\section{4.}%
\label{sec:4.}
\subsection{4.1}%
\label{sub:1}
\[
    \Omega=\left\{ \left( x,y,z \right) |x,y,z=0,1 \right\} 
.\] 
\subsection{4.2}%
\label{sub:2}
X可以取到的值有0,1,2,3

令$P\left( X=X_{i} \right) =p_i$:\[
    \begin{cases}
        p_0=\frac{1}{2^3}\\
        p_1=\frac{3}{2^3}\\
        p_2=p_1\\
        p_3=p_0
    \end{cases}
.\] 
\[
    F_X\left( X \right) =\sum_{X_i\le X} p_i
.\] 
\subsection{4.3}%
\label{sub:4.3}
 \[
    F_X\left( 2.5 \right) =p_0+p_1+p_2=\frac{7}{8}
.\]
\section{5.}%
\label{sec:5.}
\subsection{5.1}%
\label{sub:5.1}
\subsection{5.2}%
\label{sub:5.2}
\subsection{5.3}%
\label{sub:5.3}







\end{document}
