\lecture{1}{02.21}
\begin{notation}
    课堂分享回答问题:为什么说没有民主就没有社会主义?
\end{notation}
五大背景:
\begin{itemize}
    \item 时代背景:世界百年之未有之大变局加速演进
    \item \ldots 
    
\end{itemize}
\begin{defi}
    习近平新时代中国特色社会主义思想是:党和人民的实践经验和集体智慧的结晶;
    \begin{itemize}
        \item 坚定的政治信仰和朴素的人民情怀为其注入了强大的精神基因
        \item 丰富的文化积淀和良好的素养
        \item \ldots 
        
    \end{itemize}
\end{defi}
历史选择了毛泽东:毛泽东是忠诚的马克思主义者,结合理论和实际带领人民取得了成功;历史选择了邓小平:邓小平个人能力素质强大,是最合适的少壮派;历史选择了习近平:习近平是人民的领袖,从群众中来,到群众中去
\begin{question}
    中国的领导人和外国的领导人有哪些区别?

    看三段视频,总结习近平为什么是人民的领袖
\end{question}

\textit{1. 初心$\cdot $ 梁家河篇}

习近平出身于革命家庭,长期与劳动人民打交道;曾自费去学习沼气技术,回乡造福村民;村民推举他参加了上山下乡知识大会。1975年习近平考上清华大学。习近平认为,对他帮助最大的是陕北革命前辈和老乡,让他见识到了劳动人民的力量

\textit{2. 初心$\cdot $ 正定篇}

在正定工作期间,习近平骑着自行车跑遍了正定的所有村;正定当时是全国的“高产县”,但很多人无法解决温饱问题,关键问题是高征购问题。习近平向上提出问题,使得上级对正定的征购降低。习近平推动了对知名学者、专家、教授的招收进度,使得大量专业人员来到了正定发展;后来,正定实现了“半城郊型”发展。

\textit{3. 初心$\cdot $ 宁德篇}

习近平喜欢读书。习近平心中始终装着人民
\begin{notation}
江山就是人民,人民就是江山。“我将无我,不负人民”
\end{notation}
毛泽东系统阐述了马克思主要基本原理同中国具体实际相结合的一系列重大问题;马克思主义理论与中国的具体情况有共同点:体现了中国人民宇宙观、天下观、社会观、道德观;习近平提出的“人类命运共同体”理论与这一观点相符。

习近平的思想是马克思主义思想和中国特色传统文化的有机结合结合
