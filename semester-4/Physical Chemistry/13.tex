\lecture{13}{03.28}
回顾:
\[
    \mu\left( T,p \right) = \mu^\ominus\left( T,p^\ominus \right)+RT\ln \frac{p}{p^\ominus}
.\]
\[
    \mu_{\ce{B(l)}}\left( T,p \right) = \mu^\ominus_{\ce{B(l)}}\left( T \right)+RT\ln x_\text{B}
.\]
其中$\mu^\ominus_{\ce{B(l)}}$ 一般已知
\begin{itemize}
    \item 对于单组分的理想气体,$p_\text{B} = p$ ;
    \item 对于多组分的混合理想气体,$p_\text{B} = p_\text{总}\cdot x_\text{B}$ ;
    \item 对于实际气体,$p_\text{B} = f_\text{B}$ (逸度);
    \item 对于\textbf{理想液态化合物}:$p_\text{B} = p_\text{B}^\star \cdot x_\text{B}$
\end{itemize}
即:\[
    \mu_\text{B}\left( T,p \right) = \mu_\text{B}^\star \left( T,p^\ominus \right)+RT\ln \frac{p_\text{B}^\star x_\text{B}}{p^\ominus}
.\]
\begin{notation}
    理想稀溶液:在一定的温度、压强和浓度的范围内,溶剂服从拉乌尔定律,溶质服从亨利定律

    溶质B的化学势(摩尔分数):由于$p_\text{B} = k_{x}\cdot x_\text{B}$ ,带入:\[
        \mu_\text{B}\left( T,p \right) = \mu_{\ce{B(g)}}\left( T,p \right) = \mu_\text{B}^\ominus\left( T \right)+RT\ln \frac{p_\text{B}}{p^\ominus}
    .\]
    得:\[
        \mu_\text{B}\left( T,p \right) = \mu_\text{B}^\ominus\left( T \right) + RT\ln \frac{k_{x}}{p^\ominus}+RT\ln x_\text{B} = \mu_\text{B}^\star \left( T,p \right)+RT\ln x_\text{B}
    .\]
\end{notation}
不同的浓度表达方式:
\begin{align*}
    \mu_\text{B} &= \mu_\text{B}^\star \left( T,p \right)+RT\ln x_\text{B} \\
    &= \mu_\text{B}^\square\left( T,p \right)+RT\ln \frac{m_\text{B}}{m^\ominus} \\
    &= \mu_\text{B}^\triangle\left( T,p \right)+RT\ln \frac{c_\text{B}}{c^\ominus}
.\end{align*}
\begin{notation}
    活度: \[
        \alpha_{x,\text{B}} = \gamma_{x,\text{B}}\cdot x_\text{B}
    .\]
    称$\gamma_{x,\text{B}}$ 为活度系数,可以表示任意组分中的化学势:\[
        \mu_\text{B}\left( T,p \right) = \mu_\text{B}^\star \left( T,p^\ominus \right) + RT\ln \alpha_{x,\text{B}}
    .\]
    带入其他的浓度表达:
    \begin{align*}
        \mu_\text{B}\left( T,p \right) &= \mu_\text{B}^\star \left( T,p^\ominus \right) + RT\ln \alpha_{x,\text{B}}\\
        \mu_\text{B}\left( T,p \right) &= \mu_\text{B}^\square \left( T,p^\ominus \right) + RT\ln \alpha_{m,\text{B}} \\
        \mu_\text{B}\left( T,p \right) &= \mu_\text{B}^\triangle \left( T,p^\ominus \right) + RT\ln \alpha_{c,\text{B}}
    .\end{align*}
\end{notation}
\subsection{稀溶液的依数性}%
\label{sub:稀溶液的依数性}
\begin{defi}
    依数性:在溶剂中加入非挥发性溶质后,使得溶液的沸点升高、凝固点降低及渗透压等性质
\end{defi}
\subsubsection*{蒸汽压下降}%
\label{subsub*:蒸汽压下降}
溶剂中溶剂的蒸汽压$p_\text{A}$ 低于同温度下纯溶剂的饱和蒸汽压$p_\text{A}^\star $ 称为蒸汽压下降
\subsubsection*{凝固点降低}%
\label{subsub*:凝固点降低}
在大气压力下,纯物质固态和液态的蒸汽压相等,固液两相平衡共存时的温度称为凝固点

对于稀溶液:\[
    \mu_\text{A} = \mu_\text{A}^\star +RT\ln x_\text{A}\quad \left(\frac{\partial \mu_\text{B}}{\partial T}\right)_{p,n_{i}} = -S_{\text{B},m}
.\]
结合:\[
    -S_{\ce{A(l)},m}\mathrm{d}T+\frac{RT}{x_\text{A}}\mathrm{d}x_\text{A}=-S_{\ce{A(s),m}}\mathrm{d}T \implies \ldots 
.\]
得到:\[\boxed{
    \Delta T_\text{f} = k_\text{f}m_\text{B}
.}\]
为\textbf{稀溶液凝固点降低公式},即浓度越大,凝固点降低的越多
\subsubsection*{沸点升高}%
\label{subsub*:沸点升高}
\textbf{压强越小,沸点越低;压强越大,沸点越高}
\subsubsection*{渗透压}%
\label{subsub*:渗透压}
使用\textbf{半透膜}
\begin{notation}
    纯水的化学势大于溶液中的化学势
\end{notation}
\subsection{分配定律}%
\label{sub:分配定律}
定温定压下,溶质同时可以溶解在两种互不相溶的溶剂中,该溶质在两相中的浓度比值为定值:\[
    K = \frac{m_\text{B}\left( \alpha \right)}{m_\text{B}\left( \beta \right)} = \frac{c_\text{B}\left( \alpha \right)}{c_\text{B}\left( \beta \right)}
.\]
可以利用分配定律来进行萃取:
\[
    W_\text{萃取} = W\left[ 1-\left( \frac{KV_1}{KV_1+V_2} \right)^{n} \right]
.\]
其中$W$ 为原溶质的质量,$V_1$ 为溶液体积,$V_2$ 为萃取剂体积,得到:\textbf{萃取次数越大,得到的萃取物越多}
