\lecture{7}{03.07}
常用表示:
\[
    \Delta_\text{f}H_{\text{m}}^\ominus \text{(物质、相态、温度)}
.\]
\begin{notation}
    生成焓是一个相对值(相对于稳定单质的焓),最稳定的单质标准摩尔生成焓为0
\end{notation}
使用标准摩尔生成焓计算热效应:
\begin{eg}
    从始态$M\left( T,p^\ominus \right)$ 到终态$g\ce{G}+h\ce{H}\left( T,p^\ominus \right)$,通过一个中间态$a\ce{A}+d \ce{D}\left( T,p^\ominus \right)$
\end{eg}
任意一个反应的标准摩尔焓变$\Delta_\text{r}H_{\text{m}}^\ominus $ 等于产物的标准摩尔生成焓乘反应系数后求和,即:\[
    \Delta_\text{r}H_{\text{m}}^\ominus = -\sum_\text{B}^{} \nu_\text{B}\Delta_\text{f}H_{\text{m}}^\ominus 
.\]
\subsection{燃烧热}%
\label{sub:燃烧热}
定义:在$p^\ominus,T$的标准状态,1 mol物质完全燃烧的定压反应热成为标准摩尔燃烧焓,表示为$\Delta_\text{c}H_{\text{m}}^\ominus $,要求燃烧后变为\textbf{最稳定的产物}

同理,从始态到终态,经过中间态,一个燃烧反应的反应热:\[
    \Delta_\text{r}H_{\text{m}}^\ominus = \sum_\text{B}^{} \nu_\text{B}\Delta_\text{c}H_{\text{m}}^\ominus 
.\]
\subsection{溶解热}%
\label{sub:溶解热}
定温定压,一定量的物质溶于一定量的溶剂中所产生的热效应
\begin{eg}
    NaOH溶解放热
\end{eg}
原理为晶体晶格的破坏、电离等产生的能量。溶解热使用$\Delta_\text{s}H_{\text{m}} $或$\Delta_\text{sol}H_{\text{m}} $ 表示,注意此时\textbf{可以不是标准状态}
\begin{defi}
摩尔积分溶解热:记为:\[
    \Delta_\text{isol}H_{\text{m}} = \frac{\Delta_\text{isol}H_{}}{n_\text{B}}
.\]

摩尔微分溶解热:朝大量水中放入溶质的热量,即对溶解热求导\[
    \Delta_\text{dsol}H_{\text{m}} = \left(\frac{\partial H}{\partial n_\text{B}}\right)_{T,p,n_\text{B}} 
.\]
\end{defi}
\subsection{基尔霍夫定律}%
\label{sub:基尔霍夫定律}
\begin{equation}
    \label{eq:基尔霍夫定律}
    \left(\frac{\partial \Delta_\text{}H }{\partial T}\right)_{p} = \Delta C_{p}
\end{equation}
作业:$\text{P}_{33},6. \text{P}_{34}, 6. \text{P}_{35}, 10. \text{P}_{36}, 27. \text{P}_{37}, 31$, 3月21日之前提交
\section{热力学第二定律}%
\label{sec:热力学第二定律}
学习重点:
\begin{itemize}
    \item 第二定律
    \item 卡诺循环
    \item 熵
    \item 克劳修斯不等式
    \item 吉布斯自由能
    \item \ldots 
    
\end{itemize}
\begin{notation}
    热力学第二定律的特点:
    \begin{itemize}
        \item 只考虑变化前后
        \item 只考虑可能性
        
    \end{itemize}
\end{notation}
\subsection{第二定律}%
\label{sub:第二定律}
目标:\textbf{判断各种自发过程的方向和限度}
\begin{defi}
    自发过程:无需外力作用,任其自然发生的过程
\end{defi}
焦耳实验在完成后发现:气体不能自发地回到原来的瓶中,因此这个过程是不可逆的。自发过程的共同特征为\textbf{不可逆性}。任何自发变化的逆过程都是不能自动进行的。
\begin{notation}
\textbf{第二类永动机不可能造成}。

第二类永动机:从单一热源吸热使其完全变为功而不留下任何影响
\end{notation}
\subsection{卡诺循环、卡诺定律}%
\label{sub:卡诺循环,卡诺定律}
卡诺将热机理想化,设计一个4步可逆过程组成的循环:
\begin{itemize}
    \item 等温膨胀
    \item 绝热膨胀
    \item 等温压缩
    \item 绝热压缩
\end{itemize}
卡诺指出:
\begin{itemize}
    \item 热机必须有2个热源
    \item 效率与介质无关
    \item 效率有极限值
    \item 可逆卡诺热机的效率最高
\end{itemize}
\begin{eg}
    汽车发动机:4冲程活塞发动机
\end{eg}
克劳修斯看了这个循环,绘制了一个图像:
\begin{figure}[ht!]
    \centering
    \incfig[0.5]{卡诺循环}
    \caption{卡诺循环}
    \label{fig:卡诺循环}
\end{figure}

循环方向:A $\to $ B $\to $ C $\to $ D

回忆:等温可逆的做功:曲线下的部分\[
    W = nRT \ln \frac{V_2}{V_1}
.\]

绝热可逆:对$C_{V}$ 积分:\[
    W = \int_{T_1}^{T_2} C_{V} \mathrm{d}T
.\]
\subsubsection*{热机效率}%
\label{subsub*:热机效率}
    克劳修斯如何推出熵的:由$W = nR\left( T_{h}-T_c \right)\ln \frac{V_1}{V_2}$,要求热机的效率:
    \[
        \eta = \frac{-W}{Q_{h}} = \frac{-nR\left( T_{h}-T_c \right)\ln \frac{V_1}{V_2}}{-nRT_{h}\ln \frac{V_1}{V_2}} = 1-\frac{T_{c}}{T_h}
    .\]
即热机的效率与两热源的温度有关,温差越大,效率越大
