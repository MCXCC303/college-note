\lecture{1}{02.19}
\section{课程信息}%
\label{sec:课程信息}
总学分3,总学时48

考试为\textbf{闭卷考试},成绩占70\%

平时成绩包括:
\begin{itemize}
    \item 考勤
    \item 课堂回答
    \item 作业
\end{itemize}
请假需要请假单,并告知助教

使用教材:物理化学第八版,人民卫生出版社

推荐教材:
\begin{itemize}
    \item 物理化学第六版,傅献彩等
    \item 物理化学简明教程,印永嘉
    \item 物理化学讲义,彭笑刚
    \item Physical Chemistry
\end{itemize}
\section{绪论}%
\label{sec:绪论}
17世纪末,\textbf{燃素说}流行,氧气的发现将主流观点转为\textbf{能量守恒与转化定律};1752年罗蒙诺索夫提出\textbf{物理化学}的概念,1760年《轮物体的固态与液态》提出\textbf{物质和能量不灭};1887年物理化学正式形成,第一位诺贝尔化学奖的获得主题:溶液中的\textbf{化学动力学}法则和\textbf{渗透压}规律以及对立体化学和化学平衡理论做出的贡献

20世纪初物理化学以化学热力学为特征(热力学定律、电离学说、热定理、多相平衡、逸度和活度、活化能、晶体结构等)

物理化学的其他分支:
\begin{itemize}
    \item 胶体化学
    \item 化学动力学
    \item 计算化学
    \item 热化学
    \item 电化学
    \item 催化化学
    \item 溶液化学
    \item 结构化学
    \item 界面化学
    \item \ldots 
    
\end{itemize}
\subsection{物理化学建立与发展}%
\label{sub:物理化学建立与发展}
\subsubsection*{从宏观到微观}%
\label{subsub:从宏观到微观}

宏观$\to $ 介观$\to $ 微观
\begin{notation}
    $10^{-9}$ :纳米尺度
\end{notation}
\subsubsection*{从体相到表相}%
\label{subsub:从体相到表相}
气固界面的键如何切断
\begin{eg}
    单个金属离子催化活性位的结构调控:大连化物所
\end{eg}
\subsubsection*{从静态到动态}%
\label{subsub:从静态到动态}

\begin{eg}
激光技术和分子束技术
\end{eg}
\begin{notation}
\textbf{分子反应动力学}已成为活跃学科
\end{notation}
\subsubsection*{从定性到定量}%
\label{subsub:从定性到定量}

\begin{eg}
    IR, FT-IR, ESR, NMR, ESCA, 计算化学
\end{eg}
\subsubsection*{从单一到边缘}%
\label{subsub:从单一到边缘}
\begin{eg}
    计算+化学=计算化学,医学+化学=医用化学,化学+天文=天体化学
\end{eg}
\begin{eg}
    3D打印钠离子微型电池:高容量、高倍率、柔性化
\end{eg}
\subsubsection*{从平衡到非平衡}%
\label{subsub:从平衡到非平衡}

平衡态热力学主要研究\textbf{封闭系统}、\textbf{孤立系统}
\begin{eg}
    Prigogine:非平衡态热力学
\end{eg} 
\subsection{物理化学任务和内容}%
\label{sub:物理化学任务和内容}
物理化学是运用物理学原理和方法,从化学现象和物理现象的联系入手,探求化学变化基本规律的学科,物理化学是化学中的哲学

主要内容:化学热力学、化学动力学(化学反应的速率和机理)、物质结构和性能(结构化学、量子化学)
\subsection{物理化学的研究方法}%
\label{sub:物理化学的研究方法}
\textbf{归纳法、演绎法、热力学方法、统计力学方法、量子力学方法}
\subsection{学习方法}%
\label{sub:学习方法}
\begin{itemize}
    \item 重点
    \item 主要公式的意义和条件
    \item 课前自学
    \item 章节联系
    \item \textbf{习题}
\end{itemize}
\subsection{物理化学在医学的应用}%
\label{sub:物理化学在医学的应用}
电泳可以用来诊断血液疾病,药物剂型需要考虑表面现象、崩解速率等

屠呦呦通过《肘后备急方》提取青蒿素:使用\textbf{乙醚}提取
\section{热力学第一定律}%
\label{sec:热力学第一定律}
\subsection{理想气体状态方程}%
\label{sub:理想气体状态方程}
波义耳(波义耳-马略特定律)、阿伏伽德罗(阿伏伽德罗假说/定律)、吕萨克(盖吕萨克定律)
\begin{defi}
    理想气体状态方程:\[
        pV=nRT
    .\]
    理想气体:凡在任何温度、压力下均\textbf{服从理想气体状态方程}的气体
\end{defi}
\subsection{摩尔气体常数}%
\label{sub:摩尔气体常数}
 \[
    R=8.314\text{J} \cdot \text{mol}^{-1} \cdot \text{K}^{-1}
.\]
\begin{eg}
    制氧机每小时生产100kPa,298.15K的纯氧6000 $\text{m}^{3}$ ,求一天生产氧气量
\end{eg}
\begin{sol}
    由理想气体方程:\[
        pV=nRT
    .\]可得:$n=\frac{pV}{RT}$,已知:$T=298.15, V=6000, p=100\times 10^{3}$,带入计算得每小时产量$7.74\text{mol}$ ,
\end{sol}
\begin{notation}
    道尔顿分压定律: \[
        p=\sum_{i}P_i
    .\]
\end{notation}
\begin{notation}
    实际气体的范德华方程:
    \[
        \left( P+\frac{a}{V_{m}^2 } \right)\left( V_{m}-b \right)=RT
    .\]
    $a$ 表示分子之间的吸引强度;$b$ 表示每个分子占据的体积
\end{notation}
