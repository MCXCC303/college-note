\lecture{6}{03.05}
\subsection{恒压热容和恒容热容的差}%
\label{sub:恒压热容和恒容热容的差}
\begin{align*}
    C_{p}-C_{V} &= \left(\frac{\partial U}{\partial T}\right)_{p} - \left(\frac{\partial U}{\partial T}\right)_{V} \\
    &= \left[ \left(\frac{\partial U}{\partial V}\right)_{T}+p \right]\left(\frac{\partial V}{\partial T}\right)_{p} \quad \left( H=U+pV \right)\\
    &= 0\quad \left( \text{固体} \right)\\
    \text{or: }&= nR \quad \left( \text{理想气体} \right)\\
    \text{or: }C_{p,m}-C_{V,m} &= R 
.\end{align*}
\begin{notation}
在常温下,$C_{p}$ 和$C_{V}$ 都为常数
\end{notation}
\begin{eg}
证明:\[
    \left(\frac{\partial U}{\partial T}\right)_{p} = C_{p} - p\left(\frac{\partial V}{\partial T}\right)_{p}
.\]
\end{eg}
\subsection{定温过程}%
\label{sub:定温过程}
对于定温过程:$\Delta_\text{}U =0,\Delta_\text{}H=0, \Delta T =0 ,W=-Q$

前面有证明,气体做的最大功:\[
    Q_R = -W_R = nRT \ln \frac{V_2}{V_1} = nRT \ln \frac{p_1}{p_2}
.\]
对于不可逆过程:$Q_R = -W_R \neq 0$
\subsection{绝热过程}%
\label{sub:绝热过程}
在绝热过程,$Q=0$ ,即$\mathrm{d}U = \delta W$ ,积分:\[
    \delta W = -p_\text{e}\mathrm{d}V = -p\mathrm{d}V = -\frac{nRT }{V}\mathrm{d}V
.\]
又因为:$\delta W = \mathrm{d}U = C_{V}\mathrm{d}T$ ,即:\[
    -\frac{nRT }{V}\mathrm{d}V = C_{V}\mathrm{d}T \quad -\frac{nR}{V}\mathrm{d}V = \frac{C_{V}}{T}\mathrm{d}T
.\]积分后可得:\[
    W = \int \frac{nR }{V} \mathrm{d}V = \int -\frac{C_{V}}{T} \mathrm{d}T = nR \ln \frac{V_2}{V_1}
.\]
由于$C_{p}-C_{V} = nR$ (理想气体),带入后:\[
    \left( C_{p}-C_{V} \right)\ln \frac{V_2}{V_1}=C_{V}\ln \frac{T_1}{T_2}
.\]
令$C_{p}/C_{V}=\gamma$,写为:
\[
    \left( \gamma-1 \right)\ln \frac{V_2}{V_1} = \ln \frac{T_1}{T_2}
.\]
即:\[
    T_1V_1^{\gamma-1} = T_2V_2^{\gamma-1}
.\]
即得到一个常数$K$:
\begin{equation}
    \label{eq:TV-K}
    TV^{\gamma-1} = K
\end{equation}
如果带入$T=\frac{pV}{nR}$ 可以得到另一个常数$K'=KnR$:
\begin{equation}
    \label{eq:pV-K'}
    pV^{\gamma} = K'
\end{equation}
带入$V=\frac{nRT }{p}$ 得到另一个常数$K''=\frac{K}{n^{\gamma-1}R^{\gamma-1}}$:
\begin{equation}
    \label{eq:Tp-K''}
    T^{\gamma}p^{1-\gamma}=K''
\end{equation}
绝热和定温的比较:\textbf{等温可逆过程做的功大于绝热可逆过程}
\begin{figure}[ht!]
    \centering
    \incfig[0.5]{绝热可逆和等温可逆比较}
    \caption{绝热可逆和等温可逆比较}
    \label{fig:绝热可逆和等温可逆比较}
\end{figure}
\section{热效应}%
\label{sec:热效应}
\subsection{定容和定压热效应}%
\label{sub:定容和定压热效应}
\begin{eg}
    从反应物到产物,温度和压力都变化($\Delta_\text{r}H_1 $),可以设计一个等温等容($\Delta_\text{r}U_2, \Delta_\text{r}H_2  $)的过程,然后在通过等压($\Delta_\text{}H_3 $)的过程达到末态
\end{eg}
这一过程:\[
    \Delta_\text{r}H_1 = \Delta_\text{r}H_2 + \Delta_\text{}H_3 = \Delta_\text{r}U_2 + \Delta\left( pV \right)_2 + \Delta_\text{}H_3    
.\]
对于理想气体,恒温过程3的$\Delta_\text{}H_3=0 $ ,且固体和液体的$\Delta_\text{}H_3 $ 都可以忽略,则:
\begin{equation}
    \label{eq:Qp->QV}
    Q_{p} = Q_{V} + \left( \Delta n_\text{gas} \right)RT
\end{equation}
\begin{eg}
    298.2K时1mol正庚烷至于量热计中燃烧,测得定容反应热为$-4.807\times 10^{6}\ce{J}$ ,求定压反应热
\end{eg}
注意此时正庚烷为液体,\textbf{反应的总气体变化量为4mol}
\subsection{热化学方程式}%
\label{sub:热化学方程式}
\begin{defi}
    表示化学反应和热效应的化学方程式
\end{defi}
如:\[
    \ce{H_2}\left( \text{g},p^\ominus \right) + \ce{I_2}\left( \text{g},p^\ominus \right) = \ce{2HI}\left( \text{g},p^\ominus \right) \Delta_\text{r}H_{\text{m}}^\ominus
.\]
\begin{notation}
    反应进度:\[
        \sum_{\text{B}} \nu_\text{B}\text{B} = 0
    .\]
    则:\[
        \xi = \frac{\Delta n_\text{B}}{\nu_\text{B}} \quad \mathrm{d}\xi = \frac{\mathrm{d}n_\text{B}}{\nu_\text{B}}
    .\]
    计算反应进度:找到一个成分的$\Delta n_\text{B}$ ,找到这个成分的反应系数$\nu_\text{B}$ ,计算$\xi = \frac{\Delta n_\text{B}}{\nu_\text{B}}$
\end{notation}
易得:同一个反应的不同成分的反应进度是一样的
\subsection{赫斯定律}%
\label{sub:赫斯定律}
又名\textbf{反应热加成定律} 
\begin{defi}
    反应的热效应只与反应的始态和末态有关,与途径无关。
\end{defi}
若一个反应是两个反应式的代数和时,其\textbf{反应热也为两个反应热的代数和},可以看作是两个反应分步进行
\subsection{反应热}%
\label{sub:反应热}
定义标准态:$p^\ominus=100\ce{kPa}$,固体/液体为纯固体/纯液体,气体为纯气体或混合气体,注意\textbf{并没有确定标准温度},但常见的温度为$T=298.15\ce{K}=25^\circ\text{C}$
\subsubsection*{生成热}%
\label{subsub*:生成热}
元素的单质(最稳定单质)化合成单一化合物1mol的反应物的焓变,记为$\Delta_\text{f}H_{\text{m}}^\ominus $,可查表
