\lecture{14}{04.02}
\section{化学平衡}%
\label{sec:化学平衡}
重点:
\begin{itemize}
    \item 化学反应等温方程式
    \item 平衡常数
    \item 反应限度
    \item 计算$\Delta_\text{r}G_{m}^\ominus $
    \item 反应偶合
\end{itemize}
\subsection{化学平衡的条件}%
\label{sub:化学平衡的条件}
\begin{defi}
    反应进度:$\xi = \frac{\Delta n_\text{B}}{\nu_\text{B}}$
\end{defi}
对于任意一个封闭体系发生的化学反应,系统不做非膨胀功,反应为:\[
    a\ce{A}+d \ce{D}\ce{<=>[][]}g\ce{G}+h\ce{H}
.\]
由\[\boxed{
    \mathrm{d}G = -S\mathrm{d}T + V\mathrm{d}p + \sum_{\text{B}}\mu_\text{B}\mathrm{d}n_\text{B}
.}\]
得:\[
    \left(\frac{\partial G}{\partial \xi}\right)_{T,p} = \sum_{\text{B}}\nu_\text{B}\mu_\text{B}
.\]
当$\xi=1\ce{mol}$ 时:
\[
    \left( \Delta_\text{r}G_{m}  \right)\left.\right|_{T,p}^{} = \sum_{\text{B}}\nu_\text{B}\mu_\text{B}
.\]
\begin{notation}
    适用于:1. 等温等压,2. 反应中化学势不变
\end{notation}
\begin{notation}
    由上:使用以下三种条件判断等价:
    \[
        \left.\left( \Delta_\text{r}G_{m} \right)\right|_{T,p}^{}\quad \sum_{\text{B}}\nu_\text{B}\mu_\text{B} \quad \left(\frac{\partial G}{\partial \xi}\right)_{T,p}
    .\]
    当:
    \begin{enumerate}
        \item $\Delta_\text{r}G_{m} <0$时,反应向右进行
        \item $\Delta_\text{r}G_{m} >0$ 时:反应向左进行
        \item $\Delta_\text{r}G_{m} =0$ 时反应平衡
    \end{enumerate}
\end{notation}
\begin{figure}[ht!]
    \centering
    \incfig[0.5]{系统吉布斯自由能和反应进度的关系}
    \caption{系统吉布斯自由能和反应进度的关系}
    \label{fig:系统吉布斯自由能和反应进度的关系}
\end{figure}
\subsection{等温方程式}%
\label{sub:等温方程式}
回顾:
\[\boxed{
    \mu_\text{B}\left( T,p \right) = \mu_\text{B}^\ominus\left( T,p^\ominus \right) + RT\ln \frac{p_\text{B}}{p^\ominus}
.}\]
由于$\Delta_\text{r}G_{m} =\sum_{\text{B}}\nu_\text{B}\mu_\text{B}$ ,带入:
\begin{align*}
    \Delta_\text{r}G_{m} &= \sum_{\text{B}}\left( \mu^\ominus\left( T,p^\ominus \right) +RT\ln \frac{p_\text{B}}{p^\ominus}\right)\nu_\text{B} \\
    &= \sum_{\text{B}}\mu_\text{B}^\ominus\left( T,p^\ominus \right)\nu_\text{B} + \sum_{\text{B}}\left( RT\ln \frac{p_\text{B}}{p^\ominus} \right)\nu_\text{B}
.\end{align*}
由定义\[\boxed{
    \sum_{\text{B}}\mu_\text{B}^\ominus\left( T,p \right)\nu_\text{B} = \Delta_\text{r}G_{m}^\ominus 
.}\]
则原式:
\begin{align*}
    \Delta_\text{r}G_{m} &= \Delta_\text{r}G_{m}^\ominus + \sum_{\text{B}}RT\ln \frac{p_\text{B}}{p^\ominus}\\
    &= \Delta_\text{r}G_{m}^\ominus +RT\ln \frac{\left( \frac{p_\text{G}}{p^\ominus} \right)^{g}\left( \frac{p_\text{H}}{p^\ominus} \right)^{h}}{\left( \frac{p_\text{A}}{p^\ominus} \right)^{a}\left( \frac{p_\text{D}}{p^\ominus} \right)^{d}} \\
    &= \Delta_\text{r}G_{m}^\ominus +RT\ln Q_{p}
.\end{align*}
称$Q_{p}$ 为压力商,可以通过求各物质的逸度$f$ 来算。当平衡时,$\Delta_\text{r}G_{m} = 0 $ ,则:\[
    \Delta_\text{r}G_{m}^\ominus \left( T \right)= -RT\ln K^\ominus
.\]
称$K^\ominus$ 为平衡常数。$K^\ominus = K^\ominus\left( T \right)$,且为无量纲量
\subsection{平衡常数表示法}%
\label{sub:平衡常数表示法}
\subsubsection*{理想气体反应}%
\label{subsub*:理想气体反应}
物质的量分数表示:\[
    K_{x} = \frac{x_\text{G}^{g}x_\text{H}^{h}\ldots }{x_\text{A}^{a}x_\text{D}^{d}\ldots } = \prod_{\text{B}}^{} x_\text{B}^{\nu_{\text{B}}}
.\]
\subsubsection*{实际气体}%
\label{subsub*:实际气体}
使用逸度代替:\[
    K_{f}^\ominus = K_{p}^\ominus\cdot K_{\gamma}
.\]
仅了解
\subsubsection*{复相化学反应}%
\label{subsub*:复相化学反应}
\begin{eg}
     \[
        \ce{CaCO_3(s)}=\ce{CaO(s)}+\ce{CO_2(g)}
    .\]
\end{eg}
\[
    K^\ominus = \left( \frac{p_\text{\ce{CO2}}}{p^\ominus} \right)_{\ce{eq.}}^{}
.\]
\begin{notation}
    分解压力
\end{notation}
\subsection{反应限度计算}%
\label{sub:反应限度计算}
\begin{notation}
    平衡转化率:\[
        n_\text{trans} = \frac{n_\text{dis}}{n_\text{org}}\times 100\%
    .\]
\end{notation}
\subsection{标准反应的自由能计算}%
\label{sub:标准反应的自由能计算}
已知:\[
    \Delta_\text{r}G_{m}^\ominus \left( T \right) = -RT\ln K^\ominus
.\]\ldots 
\begin{notation}
    近似估计反应的可能性:
    \[
        \left( \Delta_\text{r}G_{m}  \right)_{T,p} = \Delta_\text{r}G_{m}^\ominus +RT\ln Q_{a}
    .\]
    当$\Delta_\text{r}G_{m}^\ominus >42\ce{kJ*mol^{-1}}$ 时反应不太可能正向进行。
\end{notation}
其他计算方法:
\begin{notation}
    根据公式:\[
        \Delta_\text{r}G_{m}^\ominus = \Delta_\text{r}H_{m}^\ominus -T\Delta_\text{r}S_{m}^\ominus 
    .\]
\end{notation}
\begin{notation}
    通过已知的$\Delta_\text{r}G_{m}^\ominus $通过吉布斯亥姆霍兹公式计算
\end{notation}


