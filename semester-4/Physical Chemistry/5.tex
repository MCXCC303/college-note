\lecture{5}{02.28}
回顾:可逆恒温压缩、膨胀过程
\begin{eg}
    1mol乙醇在沸点351.5K下蒸发为气体,求这个过程的做功
\end{eg}
\begin{sol}
    由$pV=nRT $ :$pV=1\times 351.5\times 8.314$ ,由于一开始$V=0$ ,因此\[
        W = -p\Delta V = -p\left( V-0 \right)=-pV = -2.92\ce{kJ}
    .\]
\end{sol}
\subsection{热力学第一定律}%
\label{sub:热力学第一定律}
热力学第一定律有经验叙述和数学形式
\begin{notation}
    三位奠基人:焦耳、迈耶、亥姆霍兹
\end{notation}
文字表达:能量守恒定律,热力学能、热、功可以相互转换,总能量不变

系统总能量通常有三部分:整体运动的动能、在外力场中的位能、热力学能(内能),热力学能包含了系统中的一切能量,系统热力学能的绝对值无法测定,改变值只与始末态有关

数学表达:\[
    \Delta_\text{}U = Q+W \quad \mathrm{d}U = \delta Q + \delta W
.\]
热力学能的单位为焦耳
\subsubsection*{定压热}%
\label{subsub*:定压热}
用$Q_{p}$ 表示,将热力学第一定律改写:
\begin{align*}
    \Delta_\text{}U &= Q_{p}-p_\text{e}\left( V_2-V_1 \right)\\
    U_2-U_1&= Q_{p}-p_2V_2+p_1V_1 \\
    Q_{p} &= \left( U_2+p_2V_2 \right)-\left( U_1+p_1V_1 \right)
.\end{align*}
记$U+pV=H$ ,则:
\begin{equation}
    \label{eq:定压热}
    Q_{p} = \Delta_\text{}H \quad W= 0 \quad p_1=p_2=p_\text{e} 
\end{equation}
\begin{eg}
    100kPa,1173K下,1mol $\ce{CaCO_3}\text{(s)}$ 分解为$\ce{CaO}\text{(s)}$ 和$\ce{CO_2}\text{(g)}$ ,吸热178kJ,求$W,Q,\Delta_\text{}U ,\Delta_\text{}H $
\end{eg}
\begin{sol}
    由于压力恒定,即$Q_{p} = 178\ce{kJ}$ ,且$W=-p_\text{e}V = -nRT = -9.75\ce{kJ}$ ,$\Delta_\text{}H = Q_{p} = 178\ce{kJ} $ ,$\Delta_\text{}U = Q_{p} + W = 178-9.75 =168.25\ce{kJ}$
\end{sol}
\section{热容}%
\label{sec:热容}
平均热容$\overline{C}$ :\[
    \overline{C} = \frac{Q}{T_2-T_1} = \frac{Q}{\Delta T}
.\]

比热容:物质的数量为1kg

摩尔热容:物质的数量为1mol

不同物质有不同的比热容和摩尔热容,比热容值是一个强度性质,但和物质状态有关
\subsection{定容热容}%
\label{sub:定容热容}
\begin{equation}
    \label{eq:定容热容}
    C_{V} = \frac{\delta Q}{\mathrm{d}T} = \left(\frac{\partial U}{\partial T}\right)_{p} \quad \left( Q_{V} = \Delta_\text{}U  \right) 
\end{equation}
移项后积分:\[
    \delta Q = C_{V}\mathrm{d}T \implies \Delta_\text{}U  = \int_{T_1}^{T_2} C_{V} \mathrm{d}T
.\]
如果使用摩尔热容也可以使用:\[
    \Delta_\text{}U = \int_{T_1}^{T_2} nC_{V,m} \mathrm{d}T 
.\]
如果在温度范围内$C_{V,m}$ 为常数,则:\[
    \Delta_\text{}U = nC_{V,m}\left( T_1-T_2 \right) 
.\]
\subsection{定压热容}%
\label{sub:定压热容}
\begin{equation}
    \label{eq:定压热容}
    C_{p} = \frac{\delta Q_{p}}{\mathrm{d}T} = \left(\frac{\partial H}{\partial T}\right)_{p}
\end{equation}
同理:\[
    \Delta_\text{}H = \int_{T_1}^{T_2} C_{p} \mathrm{d}T = n \int_{T_1}^{T_2} C_{p,m} \mathrm{d}T 
.\]
热容有一个常用的经验公式:
\begin{equation}
    \label{eq:热容的经验公式}
    C_{p,m} = a+bT+cT^2 +\ldots 
\end{equation}
通过查找热力学数据表可以得到$a,b,c\ldots $ 的值,再积分会方便很多
\subsection{热力学第一定律应用}%
\label{sub:热力学第一定律应用}
\begin{notation}
    焦耳实验:理想气体扩散到真空。
    \[
        \textcolor{red}{\mathrm{d}U} = \left(\frac{\partial U}{\partial T}\right)_{V}\textcolor{red}{\mathrm{d}T} + \left(\frac{\partial U}{\partial V}\right)_{T}\textcolor{blue}{\mathrm{d}V}
    .\]
    两个部分为0,而$\mathrm{d}V \neq 0$ ,因此必有
    \begin{equation}
        \label{eq:焦耳实验内压}
        \left(\frac{\partial U}{\partial V}\right)_{T}=0
    \end{equation}
    即理想气体的内压为0,如果$U = f\left( T,p \right)$ ,同样有:
    \begin{equation}
        \label{eq:焦耳实验热效应}
        \left(\frac{\partial U}{\partial p}\right)_{T} = 0
    \end{equation}
    即理想气体的内能只与温度有关:$U = f\left( T \right)$

    事实上,焦耳实验在当时是不准确的。只有当气体接近理想气体时这些结论才成立,而这个实验的结果事实上有十分微小的温度变化
\end{notation}
