\lecture{9}{03.14}
\subsection{化学反应的熵变}%
\label{sub:化学反应的熵变}
\subsection{熵的物理意义}%
\label{sub:熵的物理意义}
1. 能斯特:\[
    \lim_{T \to 0} -\left(\frac{\partial \Delta G}{\partial T}\right)_{p} = \lim_{T \to 0} \left( \Delta S \right)_T = 0
.\]

2. 热力学第三定律:0K时任何纯物质完美晶体的熵为0
\begin{notation}
    完美晶体:晶体中的质点排列只有一种方式
\end{notation}
\section{自由能}%
\label{sec:自由能}
\subsection{亥姆霍兹自由能}%
\label{sub:亥姆霍兹自由能}
\begin{defi}
    由热力学第一定律:\[
        \delta Q = \mathrm{d}U -\delta W
    .\]
    由克劳修斯不等式:\[
        \mathrm{d}S \ge \frac{\delta Q}{T}
    .\]
    带入:\[
        \mathrm{d}S \ge \frac{\mathrm{d}U - \delta W}{T} \implies T\mathrm{d}S -\mathrm{d}U \ge -\delta W
    .\]
    即:\[
        \mathrm{d}\left( U-TS \right)\le \delta W \Rightarrow -\mathrm{d}\left( U-TS \right) \le -\delta W
    .\]
    定义亥姆霍兹自由能\[\boxed{
        A = U-TS
    .}\] 
    则:\[
        \begin{cases}
            -\mathrm{d}A \ge -\delta W\\
            -\Delta A \ge -W
        \end{cases}
    .\]
    此时反应可以自发进行
\end{defi}
在定温定容的过程中$W=0$ ,则判断:\[
    -\Delta A \ge 0
.\]
即:$\Delta A\le 0$ 时反应自动发生。使用$T,V$ 一定$A$ 的判断方法称为亥姆霍兹自由能判据
\subsection{吉布斯自由能}%
\label{sub:吉布斯自由能}
\begin{defi}
    由热力学第二定律:\[
        \delta Q = \mathrm{d}U -\delta W
    .\]
    同亥姆霍兹自由能,结合体积功$\mathrm{d}\left( p_\text{e}V \right)$ 和非体积功$W'$ 得到:\[
        -\mathrm{d}\left( U-TS \right) \ge -\delta W' + \mathrm{d}\left( p_\text{e}V \right)
    .\]
    即:
    \[
        -\mathrm{d}\left( U+p_\text{e}V-TS \right) \ge -\delta W'
    .\]
    记\[\boxed{
        U+p_\text{e}V-TS=H-TS=G
    .}\]
    为吉布斯自由能,即当:\[
        \begin{cases}
            -\mathrm{d}G \ge -\delta W'\\
            \Delta G \le W'
        \end{cases}
    .\]
    时反应自动进行
\end{defi}
$G$ 为状态函数,只与始末态有关;吉布斯自由能可以通过可逆过程计算。
\begin{notation}
    在定温定容的条件下$W'=0$,即当
    \[\boxed{
        \Delta G \le 0
    .}\]
    时反应自动发生,且在定温定压和非体积功为0的条件下$\mathrm{d}G>0$ 的反应不可能发生
\end{notation}
使用$G$ 代替$S$ 判断的理由:一个系统下:\[
    -\frac{\Delta G_\text{sys}}{T} = -\frac{\Delta H_\text{sys}}{T}+\Delta_\text{}S_{\text{sys}} 
.\]
当$W'=0$ 时代表:\[
    \Delta_\text{}H_{\text{sys}} = Q_{p} = -Q_\text{env} 
.\]
带入:\[
    -\frac{\Delta_\text{}H_{\text{env}} }{T} = \frac{-Q_{p}}{T} = \frac{Q_\text{env}}{T} = \Delta_\text{}S_\text{env} 
.\]
即:\[
    -\frac{\Delta_\text{}G_{\text{env}} }{T} = \Delta_\text{}S_{\text{env}} + \Delta_\text{}S_{\text{sys}} = \Delta_\text{}S_{\text{iso}} \Rightarrow \Delta_\text{}G_{\text{sys}} = -T\Delta_\text{}S_\text{iso}
.\]
因此可以使用$\Delta_\text{}G_\text{sys} $ 来代替$\Delta_\text{}S_{\text{iso}} $ 来判断
\begin{notation}
    由于\[
        G = U + p_\text{e}V -TS\quad A = U-TS
    .\]
    因此$G$ 也可以定义为$A+p_\text{e}V$
\end{notation}
\subsection{吉布斯自由能变的计算}%
\label{sub:吉布斯自由能变的计算}
\subsubsection*{定温过程}%
\label{subsub*:定温过程}
仅有体积功$p_\text{e}\mathrm{d}V$ ,则:\[
    T\mathrm{d}S = \delta Q_\text{R} = \mathrm{d}U -\delta W_\text{R} = \mathrm{d}U + p_\text{e}\mathrm{d}V
.\]
对$G$ 微分:\[
    \mathrm{d}G = \mathrm{d}U + \mathrm{d}\left( p_\text{e}V \right) - \mathrm{d}\left( TS \right) = \mathrm{d}U +V\mathrm{d}p_\text{e} + p_\text{e}\mathrm{d}V -T\mathrm{d}S -S\mathrm{d}T
.\]
即:$\mathrm{d}U = T\mathrm{d}S -p_\text{e}\mathrm{d}V$ ,带入$\mathrm{d}G = \mathrm{d}U +p_\text{e}\mathrm{d}V + V\mathrm{d}p_\text{e} -T\mathrm{d}S -S\mathrm{d}T$ 得到:\[
    \mathrm{d}G = -S\mathrm{d}T + V\mathrm{d}p_\text{e}
.\]
由于是等温过程,$\mathrm{d}T = 0$ ,积分:\[
    \mathrm{d}G = V\mathrm{d}p_\text{e} = \frac{nRT }{p_\text{e}}\mathrm{d}p_\text{e} \implies \Delta G = \int_{p_1}^{p_2} V\mathrm{d}p_\text{e} = nRT \ln \frac{p_2}{p_1}
.\]
即:\[\boxed{
    \begin{cases}
    \Delta_\text{}G = nRT \ln \frac{p_2}{p_1} \\
    \Delta A = -nRT \ln \frac{V_2}{V_1}
    \end{cases}
.}\]
\subsubsection*{凝聚态自由能变}%
\label{subsub*:凝聚态自由能变}

