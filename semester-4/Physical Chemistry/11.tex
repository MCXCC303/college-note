\lecture{11}{03.21}
\section{多组分系统热力学}%
\label{sec:多组分系统热力学}
主要内容:
\begin{itemize}
    \item \textbf{偏摩尔量}
    \item \textbf{化学势}
    \item \textbf{稀溶液中的两个经验定律}
    \item 稀溶液的依数性
    \item 分配定律
\end{itemize}
\begin{eg}
    冰面撒盐加速溶解,乙醚萃取青蒿素,可乐气泡打开时释放
\end{eg}
\subsection{混合物}%
\label{sub:混合物}
\begin{notation}
    溶液:含有一种以上组分的固相或液相
\end{notation}
本章考虑\textbf{非电解质溶液}。溶液中如果含有极少的溶质,称为稀溶液,表示为$\infty $
\begin{notation}
    基本的浓度计算:\[
        c_\text{B} = \frac{n_\text{B}}{V}
    .\]
    单位:$\ce{mol}\cdot \ce{m}^{3}$,可以表示为$c_\text{B}$ 或[B]
\end{notation}
\begin{notation}
    质量摩尔浓度:\[
        m_\text{B} = \frac{n_\text{B}}{m\left( \text{A} \right)}
    .\]
    其中B为溶质,A为溶剂,可以表示$m_\text{B}$ 或$b_\text{B}$ ,单位为$\ce{mol}\cdot \text{kg}^{-1}$;可以使用称量法计算
\end{notation}
常用的其他浓度表示:
\begin{itemize}
    \item 摩尔分数:$x_\text{B}\% = \frac{x_\text{B}}{\sum_{A}^{} x_\text{A}}$
    \item \ldots 
\end{itemize}
单组分系统的广度性质具有\textbf{加和性}
\begin{eg}
    1mol单组分B的体积为$V_\text{B}$ ,则2mol单组分B的体积为$2V_\text{B}$
\end{eg}
\subsection{偏摩尔量}%
\label{sub:偏摩尔量}
不符合严格的加和性的部分
\begin{eg}
    在$25^\circ\text{C}, 100\ce{kPa}$时:
    \begin{itemize}
        \item 100mL水+100mL乙醇=192mL
        \item 150mL水+50mL乙醇=195mL
        \item 50mL水+150mL乙醇=193mL
    \end{itemize}
\end{eg}
即:
\begin{align*}
    n_\text{A}+n_\text{B} &= n_\text{A}+n_\text{B}\\
    n_\text{A}V^\star_{m,\text{A}} + n_\text{B}V^\star_{m,\text{B}} &\neq  V_1+V_2
.\end{align*}
实验发现,在混合体积为1:1时这个差值$\Delta V = \left( V_1+V_2 \right)-\left( nV^\star_{m,\text{A}}+nV^\star_{m,\text{B}} \right)$ 达到最大
\begin{notation}
    同是1mol物质,在混合物中对体积的贡献不等于单独存在时对体积的贡献
\end{notation}
Lewis定义偏摩尔体积$V_\text{B}$ :混合物总体积随B组分物质的量的变化率\[
    \mathrm{d}V = \left(\frac{\partial V}{\partial T}\right)_{p,n_\text{B};n_\text{C}\ldots }\mathrm{d}T + \left(\frac{\partial V}{\partial p}\right)_{T,n_\text{B},n_\text{C}\ldots }+\left(\frac{\partial V}{\partial n_\text{B}}\right)_{T,p,n_\text{C};n_\text{D}\ldots }\mathrm{d}n_\text{B}+\left(\frac{\partial V}{\partial n_\text{C}}\right)_{T,p,n_\text{B};n_\text{D}\ldots }\mathrm{d}n_\text{C}+\ldots 
.\]
则定义\[\boxed{
    V_\text{B}=\left(\frac{\partial V}{\partial n_\text{B}}\right)_{T,p,n_\text{C};n_\text{D}\ldots }
.}\]
\begin{notation}
    偏摩尔体积的物理意义为:在一定温度、压力下,1mol组分B在确定组成的混合物中对体积的贡献值。一般认为$V_\text{B}\neq V^\star_{m,\text{B}}$
\end{notation}
同理,对于多组分系统的$V,U,H,S,A,G$ 等其他性质,其中任意一个广延性质都可以写为:$X = f\left( T,p,n_\text{B},n_\text{C}\ldots  \right)$;定义:\[
\mathrm{d}X = \left(\frac{\partial X}{\partial T}\right)_{p,\ldots }\mathrm{d}T+\left(\frac{\partial X}{\partial p}\right)_{T,\ldots }\mathrm{d}p+\left(\frac{\partial X}{\partial n_\text{B}}\right)_{T,p,\ldots }\mathrm{d}n_\text{B}+\left(\frac{\partial X}{\partial n_\text{C}}\right)_{T,p,n_\text{B},\ldots}\mathrm{d}n_\text{C}+\ldots 
.\]
则定义:\[\boxed{
    X_\text{B} = \left(\frac{\partial X}{\partial n_\text{B}}\right)_{T,p,n_\text{C};\ldots }
.}\]
也可以令$n_\text{C}$ 为所有其他组分,定义为:\begin{equation}
    \label{eq:偏摩尔量}
    \boxed{
        X_\text{B} = \left(\frac{\partial X}{\partial n_\text{B}}\right)_{T,p,n_\text{C}}
    .}
\end{equation}
\begin{eg}
    偏摩尔焓:
    \[
        H_\text{B} = \left(\frac{\partial H}{\partial n_\text{B}}\right)_{T,p,n_\text{C}}
    .\]
\end{eg}
\begin{notation}
    必须强调恒$T,p$ 条件,即:\[
        U_\text{B}\neq \left(\frac{\partial U}{\partial n_\text{B}}\right)_{T,\textcolor{red}{V},n_\text{C}}
    .\]
\end{notation}
偏摩尔量带入微分式得:\[
    \mathrm{d}X = \left(\frac{\partial X}{\partial T}\right)_{p,n_\text{B},n_\text{C}} + \left(\frac{\partial X}{\partial p}\right)_{T,n_\text{B},n_\text{C}}+X_\text{B}\mathrm{d}n_\text{B}+X_\text{C}\mathrm{d}n_\text{C}+\ldots 
.\]
如果恒压恒温(即$\mathrm{d}T=\mathrm{d}p=0$),则:\[\boxed{
    \mathrm{d}X_{T,p} = \sum_{B}^{} X_\text{B}\mathrm{d}n_\text{B}
.}\]
积分后:\[\boxed{
    X_{T,p} = \sum_{B}^{} X_B\cdot n_\text{B}
.}\]
\begin{notation}
    对这个公式求全微分:
    \[
        \mathrm{d}X = \sum_{\text{B}}^{} X_\text{B}\mathrm{d}n_\text{B} + \sum_{\text{B}}^{} n_\text{B}\mathrm{d}X_\text{B}
    .\]
    吉布斯-杜亥姆方程:\[
        \sum_{\text{B}}n_\text{B}\mathrm{d}X_\text{B} = 0 \Rightarrow \sum_{\text{B}}x_\text{B}\mathrm{d}X_\text{B} =0
    .\]
\end{notation}
