\lecture{19}{04.25}
\begin{notation}
    科尔劳施发现:在很稀的溶液中,强电解质的摩尔电导率与浓度的平方根呈线性关系,$A$ 在一定温度下为常数;可以通过实验不断降低溶液的浓度来回归$\Lambda_{m}^{\infty }$
\end{notation}
对于弱电解质不符合该公式,科尔劳施提出离子独立移动定律
\subsubsection{离子独立移动定律}%
\label{ssub*:离子独立移动定律}
\begin{defi}
    在无限稀释的溶液中,每种离子的$\Lambda_{m}^{\infty }$ 是独立移动的,不受其他离子的影响
\end{defi}
\begin{eg}
    A,B物质有相同的阴离子,测定得到$\Lambda_{m_\text{A}}^{\infty }-\Lambda_{m_\text{B}}^{\infty }=C$
\end{eg}
在无限稀释时,每一种离子都是独立移动的,不受其他离子影响,每一种离子对$\Lambda_{m}^{\infty }$ 有恒定的贡献
\begin{equation}
    \label{eq:kolhrausch}
    \Lambda_{m}^{\infty } = \Lambda_{m,+}^{\infty }+\Lambda_{m,-}^{\infty }
\end{equation}
根据该定律:在无限稀释的HCl和HAc中,$\Lambda_{m}^{\infty }\left( \ce{H+} \right)$ 为常数,即将弱电解质转为强电解质的加和
\begin{eg}
    对于弱电解质HAc,可以转为HCl、KAc、KCl的组合:
    \[
        \Lambda_{m}^{\infty } \left( \ce{HAc} \right) = \Lambda_{m}^{\infty }\left( \ce{HCl} \right)+\Lambda_{m}^{\infty }\left( \ce{KAc} \right)-\Lambda_{m}^{\infty }\left( \ce{KCl} \right)
    .\]
\end{eg}
\subsubsection{检验水的纯度}%
\label{ssub*:检验水的纯度}
纯水本身有微弱的电离,使用标准数据计算得:\[
    \Lambda_{m}^{\infty }\left( \ce{H_2O} \right) = 5.5\times 10^{-6}\ce{S*m^{-1}}
.\]
当水的电导率$\Lambda<1\times 10^{-4}$ 时就可以认为是较纯的水,称为\textbf{电导水};普通蒸馏水的电导率为$1\times 10^{-3}\ce{S*m^{-1}}$
\subsubsection{弱电解质的解离度和解离常数}%
\label{ssub*:弱电解质的解离度和解离常数}
\begin{notation}
    在无限稀释弱电解质溶液中,可以认为弱电解质能全部解离;一定浓度下的若电解质的$\Lambda_{m}$ 除了与浓度有关外,还与\textbf{电解质的解离程度}有关
\end{notation}
\begin{defi}
    解离度:\[
        \alpha = \frac{\Lambda_{m}}{\Lambda_{m}^{\infty }}
    .\]
\end{defi}
\subsubsection{测定难溶盐的溶解度}%
\label{ssub*:测定难溶盐的溶解度}
对于难溶盐饱和溶液,由于浓度极低,可以认为$\Lambda_{m} \approx \Lambda_{m}^{\infty }$,此时水的电导率不能忽略,即:\[
    \kappa\left( \ce{AB} \right) = \kappa\left( \ce{AB(aq)} \right)-\kappa\left( \ce{H_2O} \right)
.\]
计算摩尔电导率:\[
    \Lambda_{m} = \frac{\kappa\left( \ce{AB(aq)} \right)-\kappa\left( \ce{H_2O} \right)}{c}
.\]
\subsection{强电解质溶液的活度(因子)}%
\label{sub:强电解质溶液的活度(因子)}
对任意强电解质:
\begin{align*}
    a &= a_-^{v_-}+a_+^{v_+}\\
    \gamma &= \gamma_-^{v_-}+\gamma_+^{v_+}\\
    m &= m_-^{v_-}+m_+^{v_+}
.\end{align*}
\begin{notation}
    德拜-休克尔极限定理:\[
        \ln \gamma_\text{B} = -AZ_+\left| Z_- \right|\sqrt{I}
    .\]
\end{notation}
\subsection{可逆电池}%
\label{sub:可逆电池}
热力学与电化学的桥梁:\begin{equation}
    \label{eq:drgmtpzef}
    \left( \Delta_\text{r}G_{m} \right)_{T,p} = -zEF
\end{equation}
可以计算化学能转变为电能的最高限度
\begin{notation}
    可逆电池的条件:
    \begin{itemize}
        \item 化学反应可逆
        \item 能量转换可逆(充放电时电流无穷小保证化学反应在无限接近平衡态反应)
        \item 其他过程可逆(离子迁移等)
    \end{itemize}
\end{notation}
可逆电池书写的方式:
\begin{enumerate}
    \item 左边:氧化作用的负极
    \item 右边:还原作用的正极
    \item |:相界面
    \item ||:盐桥
    \item ,:两种溶液的接界面
    \item 注明物态(压力、浓度、温度,默认为$298.15\ce{K}, p^\ominus$)
\end{enumerate}
\begin{eg}
    铜锌原电池:\[
        \ce{Zn(s)} | \ce{ZnSO_4}\left( a_1 \right)\text{||}\ce{CuSO_4}\left( a_2 \right) | \ce{Cu(s)}
    .\]
\end{eg}
\begin{eg}
    将反应设计为电池:\[
        \ce{Zn(s)}+\ce{H_2SO_4(aq)}\to \ce{H_2}\left( p_{\ce{H_2}} \right) + \ce{ZnSO_4(aq)}
    .\]
    易得Zn(s)为负极,选用Pt(s)为正极:
    \[
        \ce{Zn(s)} | \ce{ZnSO_4(aq)} \text{||}\ce{H_2SO_4(aq)} | \ce{H_2}\left( p_{\ce{H_2}} \right) | \ce{Pt(s)}
    .\]
\end{eg}
\begin{eg}
    写出电池中的化学反应:
    \begin{align*}
        &\ce{Zn(s)} | \ce{ZnSO_4}\text{||}\ce{HCl}|\ce{AgCl(s)}|\ce{Ag(s)}\\
        \Rightarrow &\begin{cases}
            \oplus&: \ce{Zn(s)}\to \ce{Zn^2+}+\ce{2e-}\\
            \ominus&: \ce{2AgCl(s)}+\ce{2e-}\to \ce{2Ag(s)}+\ce{2Cl-}
        \end{cases}\\
            \Rightarrow &\ce{Zn(s)}+\ce{2AgCl(s)}\to \ce{2Ag(s)}+\ce{2Cl-}+\ce{Zn^2+}
    .\end{align*}
    如果作电解池:\[
        \ce{Zn^2}+\ce{2Ag(s)} + \ce{2Cl-} \to \ce{2AgCl(s)}+\ce{Zn(s)}
    .\]
    即二者成可逆反应
\end{eg}
\subsubsection{可逆电极}%
\label{ssub*:可逆电极}
\begin{enumerate}
    \item 第一类电极
        \begin{itemize}
            \item 金属阳离子电极
            \item 氢电极
            \item 氧电极
            \item \ldots 
        \end{itemize}
    \item 第二类电极:金属-难溶盐电极
    \item 第三类电极:氧化还原电极
    \item 第四类电极:气体电极和离子选择性电极
\end{enumerate}
电化学和热化学的两座桥梁:
\begin{align*}
    \left( \Delta_\text{r}G  \right)_{T,p,R} &= -nEF\\
    \left( \Delta_\text{r}G_{m}  \right)_{T,p,R} &= -\frac{-nEF}{\xi} = -zEF
.\end{align*}
\subsection{可逆电池热力学}%
\label{sub:可逆电池热力学}
\subsubsection{计算熵变和可逆热效应}%
\label{ssub*:计算熵变和可逆热效应}
已知等式\ref{eq:drgmtpzef},由基本表达式:
\begin{align*}
    \mathrm{d}G &= S\mathrm{d}T + V\mathrm{d}p \\
    \left(\frac{\partial G}{\partial T}\right)_{p} &= S \\
    \left(\frac{\partial -zEF}{\partial T}\right)_{p} &= \Delta_\text{r}S_{m} 
.\end{align*}
\begin{align*}
    \Delta_\text{r}S_{m} &= zF\left(\frac{\partial E}{\partial T}\right)_{p}\\
    \Delta_\text{r}H_{m} &= \Delta_\text{r}G_{m} + T\Delta_\text{r}S_{m} = -zEF + zFT\left(\frac{\partial E}{\partial T}\right)_{p}  
.\end{align*}
\subsubsection{电池电动势能斯特方程}%
\label{ssub*:电池电动势能斯特方程}
对反应:\[
    a\text{A}\left( a_\text{A} \right) + d \text{D}\left( a_\text{D} \right) \to g\text{G}\left( a_\text{G} \right) + h\text{H}\left( a_\text{H} \right)
.\]
由于:\[
    \Delta_\text{r}G_{m} = \Delta_\text{r}G_{m}^\ominus  + RT\ln \prod_{\text{B}}^{} a_\text{B}^{\nu_\text{B}}
.\]
两边同除以$-zF$ ,由于$\Delta_\text{r}G_{m} = -zEF $ ,可得能斯特方程:
\[
    E = E^\ominus - \frac{RT}{zF}\ln \frac{a^{g}_\text{G}\cdot a^{h}_\text{H}}{a^{a}_\text{A}\cdot a^{d}_\text{D}}
.\]
