\lecture{22}{05.09}
\[
    \ln k = \ln A - \frac{E_\text{a}}{RT}
.\]
假定指前因子、活化能与温度无关:
\[
    \frac{\mathrm{d}\ln k}{\mathrm{d}T} = \frac{E_\text{a}}{RT^2 }
.\]
实际上有关,在速率理论中指出:\[
    k = A_0T^{m}\mathrm{e}^{-\frac{E_\text{a}}{RT}}
.\]
\begin{notation}
    活化能:反应分子需要一定的能量发生碰撞后才能反应

    表现为达到最高峰的能量
\begin{figure}[ht!]
    \centering
    \incfig[0.5]{活化能与反应热的关系}
    \caption{活化能与反应热的关系}
    \label{fig:活化能与反应热的关系}
\end{figure}

图\ref{fig:活化能与反应热的关系}所示$\Delta_\text{}H_{m}>0 $ ,为放热反应
\end{notation}
\subsection{复杂反应}%
\label{sub:复杂反应}
\begin{defi}
    复杂反应:两个或两个以上的基元反应的组合
\end{defi}
\subsubsection{对峙反应}%
\label{ssub*:对峙反应}
相当于可逆反应
\subsubsection{平行反应}%
\label{ssub*:平行反应}
相同反应物同时进行若干个不同反应
\begin{eg}
    \[
        \ce{CH_4 + O_2 \to 2CO_2 + H_2O}
    .\]
    该反应存在很多副反应,其中$\ce{CO_2 + H_2O}$ 为主反应产物
\end{eg}
特点:浓度之比等于速率常数之比\[
    \frac{k_1}{k_2} = \frac{x_1}{x_2}
.\]
可以通过加入催化剂加快某个反应的进行速度来增大选择性
\subsubsection{连串反应}%
\label{ssub*:连串反应}
反应产生的物质继续参与下一步的反应
\begin{eg}
\begin{figure}[ht!]
    \centering
    \incfig[0.5]{连串反应示例}
    \caption{连串反应示例}
    \label{fig:连串反应示例}
\end{figure}
\end{eg}
\subsubsection{链反应}%
\label{ssub*:链反应}
\begin{eg}
    自由基反应
\end{eg}
\subsection{碰撞理论}%
\label{sub:碰撞理论}

