\lecture{4}{02.27}
\section*{预习2}%
\label{sec*:预习2}
回忆:热力学第一定律的数学表达:
\begin{equation}
    \label{eq:热力学第一定律}
    \Delta_\text{}U = Q+W 
\end{equation}
前面已经完成了对$Q_{p},Q_{V},W$ 的求解,这一节进入热力学第一定律在不同环境的应用

\begin{notation}
    回顾:
    \begin{align*}
        \mathrm{d}U &= C_{V}\mathrm{d}T + \left(\frac{\partial U}{\partial V}\right)_{T}\mathrm{d}V\\
        \mathrm{d}H &= C_{p}\mathrm{d}T + \left(\frac{\partial H}{\partial p}\right)_{T}\mathrm{d}p
    .\end{align*}
    使用条件:只要能用$T,V$ 表达$U$ ,能用$T,p$ 表示$H$ 即可使用(双变量系统)
\end{notation}
\begin{notation}
    回顾:
    \begin{align*}
        \Delta_\text{}U &= \int_{T_1}^{T_2} C_{V} \mathrm{d}T \\
        \Delta_\text{}H &= \int_{T_1}^{T_2} C_{p} \mathrm{d}T 
    .\end{align*}
    第一个公式是上面的公式的一个限制:$dV=0$ ,使用条件为气液固的恒容过程(初态和末态的体积相等即可:定容);第二个公式同理,使用条件为恒压过程(初态和末态的压力相等:定压)
\end{notation}
除此之外还提到了一个概念:内压\[
    \left(\frac{\partial U}{\partial V}\right)_{T}
.\]
且在理想气体的条件下内压为0:
\begin{equation}
    \label{eq:理想气体内压为0}
    \left(\frac{\partial U}{\partial V}\right)_{T} = 0\quad \text{理想气体} 
\end{equation}
从数学表达来看表示温度恒定时,内能随体积的变化率;由于内能和分子的动能和势能有关,因为恒温即动能不变,即体积的变化导致了分子间势能的变化。这个量数值越大,表示分子间作用力越大

热力学第一定律可以用于:
\begin{itemize}
    \item 理想气体
    \item 相变
    \item 化学反应
    \item 真实气体
    \item \ldots 
\end{itemize}
回顾:理想气体:分子之间无相对作用且没有体积,符合$pV=nRT $ 
\begin{notation}
    1843年焦耳进行了焦耳实验:两个容器,一个有100kPa的氢气,另一个是真空,整个装置置于水中,温度为$T_1$ ,将两个容器融合,得到一个均一的氢气系统,水温变为$T_2$ 

    焦耳记录了水温度的变化,发现温度没有变化,意味着$Q=0$,这个过程可以看为是自由膨胀过程,则 $W=0$ ,因此$\Delta_\text{}U = Q+W=0 $。

    系统的内能不变,但是体积发生变化,由于$U=f\left( T,V \right)$ ,可以得出对于理想气体$U= f\left( T \right) $,即理想气体的内能与体积无关

    焦耳做了多种试验,发现温度都不变,直到换为氯气,温度发生了变化。由于氯气的分子间相互作用和氯气的分子大小都大于氢气,不能看作理想气体
\end{notation}
从数学角度,理想气体的内能与体积无关:\[
    \mathrm{d}U = C_{V}\mathrm{d}T + \left(\frac{\partial U}{\partial V}\right)_{T}\mathrm{d}V
.\]
由于分子之间无相互作用,因此内压为0,即:\[\boxed{
    \mathrm{d}U = C_{V}\mathrm{d}T
.}\]
积分后:\[
.\]
\begin{equation}
    \label{eq:使用恒容热容计算内能变}
    \Delta_\text{}U = \int_{T_1}^{T_2} C_{V} \mathrm{d}T 
\end{equation}
这个公式的使用条件:气液固的定容过程或\textbf{理想气体的任意过程}

在理想气体的条件下,焓只是温度的函数$H=f\left( T \right)$。

由于焓的定义:$H=U+pV=U+nRT $,由于$U+nRT $ 都只和温度有关,因此$H=f\left( T \right)$,即:\[
    \mathrm{d}H = C_{p}\mathrm{d}T + \left(\frac{\partial H}{\partial p}\right)_{T}\mathrm{d}p \implies \mathrm{d}H = C_{p}\mathrm{d}T
.\]
即:
\begin{equation}
    \label{eq:h/p_t=0}
    \left(\frac{\partial H}{\partial p}\right)_{T}=0
\end{equation}

再来看热容$C_{V},C_{p}$ :前面提到,热容是$T,V$ 的函数,在理想气体的条件下是否只和温度有关?来证明一下:

只需要证明$\left(\frac{\partial C_{V}}{\partial V}\right)_{T}=0$ ,即热容和体积无关,带入恒容热容的定义:
\[
    \left(\frac{\partial C_{V}}{\partial V}\right)_{T} = \left(\frac{\left(\frac{\partial U}{\partial T}\right)_{V}}{\partial V}\right)_{T}
.\]
事实上可以看作:\[
    \frac{\partial^2 U}{\partial T\partial V}
.\]
由欧拉倒易关系:\[
    \left(\frac{\partial C_{V}}{\partial V}\right)_{T} = \left(\frac{\textcolor{red}{\left(\frac{\partial U}{\partial V}\right)_{T}}}{\partial T}\right)_{V}
.\]
注意到红色部分为\textbf{内压},而前面等式\ref{eq:理想气体内压为0}已得\textbf{理想气体的内压为0},因此$\left(\frac{\partial C_{V}}{\partial V}\right)_{T}=0$ ,恒压热容同理,即热容只与温度有关
\begin{sol}
    由恒压热容的定义:\[
        C_{p} = \left(\frac{\partial H}{\partial T}\right)_{p}
    .\]
    要证明$\left(\frac{\partial C_{p}}{\partial p}\right)_{T}=0$ ,带入:\[
        \left(\frac{\partial C_{p}}{\partial p}\right)_{T} = \left(\frac{\left(\frac{\partial H}{\partial T}\right)_{p}}{\partial p}\right)_{T} = \left(\frac{\textcolor{red}{\left(\frac{\partial H}{\partial p}\right)_{T}}}{\partial T}\right)_{p}
    .\]
    前面等式\ref{eq:h/p_t=0}已经提到这一部分为0,因此
    \begin{equation}
        \left(\frac{\partial C_{p}}{\partial p}\right)_{T}=0
    \end{equation}
\end{sol}
