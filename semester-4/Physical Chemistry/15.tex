\lecture{15}{04.09}
\subsection{温度对K的影响}%
\label{sub:温度对K的影响}
由Gibbs-Helmheltz公式:
\[
    \frac{\mathrm{d}\ln K_{p}^\ominus}{\mathrm{d}T} = \frac{\Delta_\text{f}H_{m}^\ominus }{RT^2 }
.\]
设在一定区间内$\Delta_\text{f}H_{m}^\ominus $与温度无关,积分:\[
    \int_{K_{p}^\ominus\left( T_1 \right)}^{K_{p}^\ominus\left( T_2 \right)}  \mathrm{d}\ln K_{p}^\ominus = \frac{\Delta_\text{r}H_{m}^\ominus }{R}\int_{T_1}^{T_2} \frac{1}{T_1^2 } \mathrm{d}T
.\]
不定积分为:
\[
    \ln K_{p}^\ominus = -\frac{\Delta_\text{r}H_{m}^\ominus }{RT}+C
.\]
定积分为:
\[\boxed{
    \ln \frac{K_{p}^\ominus\left( T_2 \right)}{K_{p}^\ominus\left( T_1 \right)} = \frac{\Delta_\text{r}H_{m}^\ominus }{R} \left( \frac{1}{T_1}-\frac{1}{T_2} \right)
.}\]
\subsection{压力对G的影响}%
\label{sub:压力对G的影响}
根据基本关系式:\[
    \left(\frac{\partial G}{\partial p}\right)_{T} = V
.\]
得:\[
    \left(\frac{\partial \Delta_\text{f}G_{m}}{\partial p}\right)_{T} = \Delta_\text{r}V_{m}
.\]
凝聚相反应$\Delta V$ 变化很小,视为常数积分:\[
    \int_{\Delta_\text{r}G_{m,1} }^{\Delta_\text{r}G_{m,2} }  \mathrm{d}\Delta_\text{r}G_{m} = \int_{p_1}^{p_2}  \mathrm{d}p \cdot \Delta_\text{r}V_{m}
.\]
即:
\[\boxed{
    \Delta_\text{r}G_{m,2} - \Delta_\text{r}G_{m,1} = \Delta_\text{r}V_{m}\left( p_2-p_1 \right) 
.}\]
\section{相平衡}%
\label{sec:相平衡}
重点:
\begin{itemize}
    \item 相律
    \item 单组分
    \item 完全互溶双液系统
\end{itemize}
\subsection{相律}%
\label{sub:相律}
\begin{defi}
    系统内部物理和化学性质完全均匀的部分:相
\end{defi}
使用$\Phi$ 表示一个系统中的相的个数
\begin{notation}
    液体可以一相、两相、三相混合,每种固体各为一个相(除固体溶液),不论有多少气体,只有一个气相
\end{notation}
