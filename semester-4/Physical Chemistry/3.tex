\lecture{3}{02.26}
\subsection{引言}%
\label{sub:引言}
第一类永动机,发电机:能量的来源是什么?

热力学研究内容:
\begin{itemize}
    \item 定义
    \item 研究内容:方向和限度、能量效应、宏观和微观的联系
    \item 理论基础:第一、第二定律
\end{itemize}
物理化学常用严格的数理逻辑推理方法,只研究大量例子组成系统的宏观性质,不需要知道物质的微观结构和反应机理,只研究可能性和限度,不涉及时间、细节

热力学的作用:
\begin{eg}
    石墨加压(15000个大气压)可以转为金刚石,这是一个相变过程
\end{eg}
通过计算反应限度可以预测一个反应是否能够进行,计算药物制剂辅料是否能够稳定存在,利用相变来提取中药成分
\subsection{系统和环境}%
\label{sub:系统和环境}
\begin{defi}
    系统:一部分与其他物质分开的研究对象
\end{defi}
\begin{defi}
    环境:与系统密切相关、有相互作用条件或影响的部分
\end{defi}
系统可以分为三类:
\begin{itemize}
    \item 敞开系统(物质+能量交换,经典热力学不研究)
    \item 封闭系统(只有能量交换,研究最多)
    \item 孤立系统/隔离系统(无交换)
    
\end{itemize}
有时可以把系统和影响所涉及的环境一起作为孤立系统来考虑(小环境$\Rightarrow $ 大环境),对于这个孤立系统的熵:\[
    \Delta_\text{}S_{\text{iso}} = \Delta_\text{}S_{\text{sys}} + \Delta_\text{}S_{\text{env}}   
.\]
这个在热力学第二定律再提。
\subsection{系统的性质}%
\label{sub:系统的性质}
\begin{defi}
    性质:能描述系统状态的物理量
\end{defi}
分为广度性质(与物质量成正比,如质量、体积)和强度性质(与物质的量无关,如温度、密度)
\begin{notation}
    广度性质有加和性,强度性质无加和性,且\[
        \text{广度性质}\times \text{强度性质}=\text{广度性质}
    .\]
\end{notation}
\begin{eg}
    \[
        \rho = \frac{m}{V}
    .\]
    其中:$m,V$ 均为广度性质,$\rho$ 为强度
\end{eg}
如果一个系统的性质不随时间而改变,称系统处于热力学平衡态
\subsection{状态函数}%
\label{sub:状态函数}
\begin{defi}
    状态:系统一切性质的综合表现
\end{defi}
当系统处于某一个确定的状态时,性质有一个确定的值;当系统的一个性质发生变化,系统的状态也发生变化,成单值对应的关系(状态函数)。

系统的性质之间有关联,如$pV=nRT $ ,因此不需要罗列所有的状态

\begin{defi}
    状态函数:系统的性质都是状态函数
\end{defi}
一般容易测定的为状态变量,不易测定的为状态函数
\begin{notation}
    状态函数的变化\textbf{仅取决于系统的始态和终态},状态函数的微小变化为\textbf{全微分},状态函数的集合(和、差、积、商)也是状态函数
\end{notation}
\begin{eg}
    若$z=f\left( x,y \right)$ ,则对$z$ 全微分:\[
        \mathrm{d}z = \left(\frac{\partial z}{\partial x}\right)_{y}\mathrm{d}x + \left(\frac{\partial z}{\partial y}\right)_{x}\mathrm{d}y
    .\]
    如果体积描述为$V=f\left( T,p \right)$ ,则:\[
        \mathrm{d}V = \left(\frac{\partial V}{\partial T}\right)_{p}\mathrm{d}T + \left(\frac{\partial V}{\partial p}\right)_{T}\mathrm{d}p
    .\]
\end{eg}
\subsection{状态方程}%
\label{sub:状态方程}
\begin{defi}
    状态方程:系统状态函数之间的定量关系式
\end{defi}
\begin{eg}
     \[
        T = f\left( p,V \right)\quad V = f\left( T,p \right)\quad \ldots 
    .\]
\end{eg}
\subsection{过程和途径}%
\label{sub:过程和途径}
\begin{defi}
    过程:系统状态发生的一切变化
\end{defi}
常见的变化过程见lecture 2
\begin{defi}
    途径:某一过程的具体步骤
\end{defi}
\begin{eg}
    始态:$273\ce{K}, 1\times 10^{5}\ce{Pa}$,终态:$373\ce{K}, 5\times 10^{5}\ce{Pa}$,其中还有两种途径,经历两种中间态:

    1. 第一步定温:$273\ce{K}, 5\times 10^{5}\ce{Pa}$ 

    2. 第一步定压:$373\ce{K}, 1\times 10^{5}\ce{Pa}$
\end{eg}
\subsection{热和功}%
\label{sub:热和功}
\begin{defi}
    热和功:系统的状态变化时与环境交换和传递能量的形式

    热:系统和环境之间的温度差引起的能量交换
    
    功:除了热以外就是功
\end{defi}
系统吸热:$Q>0$ ,系统放热:$Q<0$

系统对环境做功:$W<0$ ,环境对系统做功:$W>0$

功的计算:\[
    \delta W = F_\text{广义} \cdot \mathrm{d}x_\text{广义}
.\]

如何计算体积功:
如果$p_\text{sys}-p_\text{env}=\mathscr{O}\left( p \right)$,活塞移动$\mathrm{d}l$,则系统对环境做的体积功:\[
    \sigma W = -F\mathrm{d}l = -p_\text{e}A\mathrm{d}l = -p_\text{e}\mathrm{d}V
.\]
积分后:\[
    W = -p_\text{e}\int_{V_1}^{V_2} \mathrm{d}V = -p\Delta V
.\]
\begin{notation}
    自由膨胀:$p_\text{e}=0$
\end{notation}
\subsection{功的过程}%
\label{sub:功的过程}
\begin{eg}
    自由膨胀:$p_\text{e}=0$ ,则:$W=-p_\text{e}\Delta V = 0$
\end{eg}
非自由膨胀:由于$pV=nRT $ ,可得:$V = \frac{nRT }{p}$

假如始态:$4\ce{kPa}, V_1$ ,在外压变化后终态:$p_\text{e}=1\ce{kPa}, V_2$ ,在外压变化后,系统缓慢膨胀到末态,则:\[
    W_1 = -p_\text{e}\left( V_2-V_1 \right)
.\]
如果在中间加入一个中间态:$p_\text{e}=2\ce{kPa}, V'$ :\[
    W_2 = -p_\text{e}'\left( V'-V_1 \right)-p_\text{e}\left( V_2-V' \right)
.\]
如果继续增加中间态的数量,逐渐把$V\left( p \right) = \frac{nRT }{p}$ 填满,即:\[
    W = -\int_{V_1}^{V_2} p \mathrm{d}V = -\int_{V_1}^{V_2} \frac{nRT }{V} \mathrm{d}V = -nRT \ln \frac{V_2}{V_1}
.\]
这个过程称为可逆膨胀或准静态膨胀

压缩同理,由于每次压缩的$p_\text{e}>p_\text{sys}$ ,因此做功比膨胀的做功大,但是积分后准静态压缩的做功达到最小,和准静态膨胀一致
\subsection{可逆过程}%
\label{sub:可逆过程}
体系可以恢复到原始状态,如标准气体恒温可逆膨胀

由热力学第一定律:\[
    \Delta_\text{}U = Q+W 
.\]
因为$W=0$ ,$\Delta_\text{}U = 0 $ ,同样$Q=0$ 
\begin{eg}
    在298K时,2mol  $\ce{O_2}$ 的体积为$1.8\times 10^{-2}\ce{m}^{3}$ ,在恒温下经历以下过程:
    \begin{itemize}
        \item 自由膨胀
        \item 抗外压$100\ce{kPa}$ 膨胀
        \item 可逆膨胀
        
    \end{itemize}
    到$5\times 10^{-2}\ce{m}^{3}$ ,求做功
\end{eg}
\begin{sol}
    1. 自由膨胀:$W=0$ 

    2. $p_\text{e}=100\ce{kPa}$ :\[
        W = -p_\text{e}\Delta V = -100\ce{kPa}\times (3.2\times 10^{-2})\ce{m}^{3} = -3.2\ce{kJ}
    .\]

    3. \[
        W = -nRT \ln \frac{V_2}{V_1} = -2\times 8.314\ce{J*K^{-1}*mol^{-1}} \times 298\ce{K} \times \ln \frac{(5\times 10^{-2}) \ce{m}^{3}}{(1.8\times 10^{-2}) \ce{m}^{3}} \approx -5062\ce{J}
    .\]
\end{sol} 

