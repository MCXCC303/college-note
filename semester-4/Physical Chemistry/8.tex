\lecture{8}{03.12}
回顾:
\begin{notation}
    克劳修斯描述热力学第二定律:不可能把热从低温物体传到高温物体而不引起其他变化

    Kelvin描述:不可能从单一热源取出热使之完全变为功而不损失其他能量
\end{notation}
\begin{notation}
    卡诺定理:可逆热机的效率最大
\end{notation}
卡诺循环引入:\[
    \eta_{I} \le \eta_R
.\]
将可逆循环和不可逆循环定量区分开
\begin{notation}
可以通过将任意一个可逆循环切割为无数个小的卡诺循环
\end{notation}
\begin{figure}[ht!]
    \centering
    \incfig[0.5]{卡诺循环微积分}
    \caption{卡诺循环微积分}
    \label{fig:卡诺循环微积分}
\end{figure}
由\begin{equation}
    \label{eq:nwq}
    \eta = \frac{-W}{Q_{h}} = 1-\frac{T_c}{T_{h}}
\end{equation}
结合两个效率可得:
\begin{equation}
    \label{eq:qt}
    \frac{Q_c}{T_c} + \frac{Q_{h}}{T_{h}} = 0
\end{equation}
当无限个循环时:\begin{equation}
    \label{eq:ointqt}
    \oint\limits_{R} {\frac{\delta Q_R}{T}} = 0
\end{equation}
这个式子被克劳修斯定义为\textbf{熵}:
\[
    S = \frac{Q_R}{T}
.\]
\begin{eg}
    把任意一个可逆循环拆分为两个过程:\[
        \int_{A}^{B} \left(\frac{\delta Q}{T}\right)_{R_1} = \int_{B}^{A} \left(\frac{\delta Q}{T}\right)_{R_2}
    .\]
    即定义为:\[
        \Delta_\text{}S = S_{B} -S_A = \int_{A}^{B} \left(\frac{\delta Q}{T}\right)_{R}
    .\]
    多个过程合成:\[
        \Delta_\text{}S = \sum_{i}\left(\frac{\delta Q_{i}}{T_{i}}\right)_{R}
    .\]
    微小变化:\[
        \mathrm{d}S = \left(\frac{\delta Q}{T}\right)_{R}
    .\]
\end{eg}
如果过程不可逆,则:\[
    \oint\limits_{R} {\frac{\delta Q}{T}} \le 0
.\]
\begin{notation}
    克劳修斯不等式:\begin{equation}
        \label{eq:cla}
        \Delta_1^2 S \ge \int_{1}^{2} \frac{\delta Q}{T}\quad \text{or}\quad \mathrm{d}S \ge \frac{\delta Q}{T}
    \end{equation}
    熵增原理的表达:\textbf{隔离系统的熵不可能减小},即:\[
        \Delta_\text{}S_{\text{隔离}} = \Delta_\text{}S_{\text{系统}} + \Delta_\text{}S_{\text{环境}} \ge 0
    .\]
    同时可以认为$\Delta_\text{}S > 0 $,即不可逆过程为\textbf{自发过程},利用隔离系统的熵变化来判断反应的方向又称为\textbf{熵判据}
\end{notation}
\subsection{熵变的计算}%
\label{sub:熵变的计算}
\textbf{熵是系统的状态函数},与过程的可逆与否无关。基本计算式:\[
    \Delta_\text{}S_{\text{e}} = \int_{A}^{B} \frac{\delta Q}{T} 
.\]
环境熵变的计算:
\begin{notation}
    由于环境很大,可认为处于热力学平衡,吸热和放热可以看作可逆,和系统的放热、吸热相反
\end{notation}
\[
    \Delta_\text{}S_{\text{e}} = -\frac{Q_\text{real}}{T_\text{e}} 
.\]
\begin{notation}
    单纯的状态变化过程包括:
    \begin{enumerate}
        \item 理想气体定温:$\Delta T = 0$
        \item 简单变温:$\Delta p$ 或$\Delta V$ 为0
        \item 理想气体的$p,V,T$ 均变化
        \item 理想气体混合
    \end{enumerate}
\end{notation}
\subsubsection*{理想气体定温过程熵变}%
\label{subsub*:理想气体定温过程熵变}
由于$\Delta_\text{}U =0,\Delta_\text{}H=0,Q_R = -W_R$,带入熵变计算式:\[
    \Delta_\text{}S = \frac{Q_R}{T} = \frac{-W_{\max }}{T} = \frac{\int_{V_1}^{V_2} p \mathrm{d}V}{T} = \frac{\int_{V_1}^{V_2} \frac{nRT }{V}  \mathrm{d}V}{T} 
.\]
\subsubsection*{简单变温过程熵变}%
\label{subsub*:简单变温过程熵变}
对于定压热容,由于$\delta Q = C\mathrm{d}T$,因此:\[
    \mathrm{d}S = \frac{C\mathrm{d}T}{T}
.\]
积分可得:\[
    \Delta_\text{}S = \int_{T_1}^{T_2} \frac{C_{p}}{T} \mathrm{d}T 
.\]
同理定容:\[
    \Delta_\text{}S = \int_{T_1}^{T_2} \frac{C_{V}}{T} \mathrm{d}T 
.\]
如果$C$ 为常数,可以写为:
\begin{align*}
    \Delta_\text{}S = \overline{C_{p}}\ln \frac{T_2}{T_1} =n \overline{C_{p,m}}\ln \frac{T_2}{T_1}\\
    \Delta_\text{}S = \overline{C_{V}}\ln \frac{T_2}{T_1} =n \overline{C_{V,m}}\ln \frac{T_2}{T_1}
.\end{align*}
当$T_2>T_1$ 时$\Delta_\text{}S>0 $
\subsubsection*{\textit{pVT}都变化}%
\label{subsub*:-textit-pVT-都变化}
过程:\[
    \text{A}\left( p_1,V_1,T_1 \right) \implies \text{B}\left( p_2,V_2,T_2 \right)
.\]
分为两步进行:等温可逆+等压可逆或等温可逆+等容可逆。
\begin{figure}[ht!]
    \centering
    \incfig[0.5]{三变量的变化图}
    \caption{三变量的变化图}
    \label{fig:三变量的变化图}
\end{figure}
