\lecture{21}{05.07}
\section{化学动力学}%
\label{sec:化学动力学}
化学反应速率:
\[
    v = \frac{\mathrm{d}[\text{R/P}]}{\mathrm{d}t}
.\]
反应进度:
\[
    \mathrm{d}\xi = \frac{\mathrm{d}n_\text{B}}{\nu_\text{B}}
.\]
\begin{notation}
    $\nu_\text{B}$ 为化学计量数
\end{notation}
转化速率:
\[
    \gamma = \frac{\mathrm{d}\xi}{\mathrm{d}t} = \frac{1}{\nu_\text{B}}\frac{\mathrm{d}n_\text{B}}{\mathrm{d}t}
.\]
反应速率(定容反应):
\begin{align*}
    r &= \frac{1}{V}\frac{\mathrm{d}\xi}{\mathrm{d}t} \\
    &= \frac{1}{\nu_\text{B}}\frac{\mathrm{d}n_\text{B} / V}{\mathrm{d}t} \\
    &= \frac{1}{\nu_\text{B}}\frac{\mathrm{d}c_\text{B}}{\mathrm{d}t} = \frac{1}{\nu_\text{B}}\frac{\mathrm{d}[\text{B}]}{\mathrm{d}t}
.\end{align*}
当反应为$\ce{a R \to b P}$ 时:
\begin{align*}
    r &= \frac{1}{a}\frac{\mathrm{d}[\text{R}]}{\mathrm{d}t} \\
      &= \frac{1}{b}\frac{\mathrm{d}[\text{R}]}{\mathrm{d}t}
.\end{align*}
\begin{defi}
    基元反应:由反应物微粒一步直接生成产物的反应

    一个反应要经过若干个基元反应才能完成,动力学上称其为反应机理
\end{defi}
    已知的基元反应的分子数只有1,2,3
\begin{eg}
    酯化反应:1分子乙酸和一分子乙醇参加反应
\end{eg}
反应速率方程:$r = f\left( c \right)$ (微分式),$c = f\left( t \right) = \int r \mathrm{d}t$ (积分式)
\begin{notation}
    对于基元反应\[
        \ce{A + 2D \to G}
    .\]
    其速率为:
    \[
        r = kc_\text{A}c_\text{D}^2 
    .\]
    其中$k$ 为反应速率常数,与反应浓度无关,与反应条件有关;对特定的反应$k$ 为定值
\end{notation}
\subsection{阿伦尼乌斯公式}%
\label{sub:阿伦尼乌斯公式}
\[
    k = A\mathrm{e}^{-\frac{E_\text{a}}{RT}}
.\]
其中$E_\text{a}$ 为反应的活化能
