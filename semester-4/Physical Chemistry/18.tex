\lecture{18}{04.23}
本章作业:
\begin{itemize}
    \item P148. 1, 3
    \item 绘制水的相图
\end{itemize}
5月9日提交。
\section{电化学}%
\label{sec:电化学}
研究内容包含电解和电池
\begin{notation}
    重点:
    \begin{itemize}
        \item 电解质溶液的电导和测定
        \item 可逆电池
        \item 标准氢电极
    \end{itemize}
\end{notation}
\subsection{电解质溶液的导电机理}%
\label{sub:电解质溶液的导电机理}
\begin{defi}
    导体:能导电的物质
\end{defi}
分为两类:电子导体和离子导体;电子导体的导电由自由电子定向运动产生,温度升高电阻升高;离子导体通过正负离子反方向移动而导电,温度升高电阻下降
\begin{notation}
    能实现电解质溶液连续导电的装置称为电化学装置,包含原电池和电解池;装置由电极和电解质溶液组成,电极上的得失电子反应来源于溶液的电荷转移
\end{notation}
电池的电极为\textbf{阴极、阳极},电解的电极为\textbf{正极、负极}
\begin{notation}
    电势高的电极为正极,电流从正极流向负极,电子由负极流向正极

    阴极发生还原反应,(阳离子)得到电子;阳极发生氧化反应,(阴离子)失去电子
\end{notation}
\begin{eg}
    电解$\ce{CuCl_2}$ :
    \begin{align*}
        \oplus:& \ce{2Cl-}-\ce{2e-} \to \ce{Cl_2}\\
        \ominus:& \ce{Cu^2+}+\ce{2e-} \to \ce{Cu}
    .\end{align*}
\end{eg}
\subsubsection{法拉第定律}%
\label{ssub:法拉第定律}
\begin{itemize}
    \item 在电极界面上发生化学变化物质的质量与通入的电荷量成正比:$m\propto Q$ 
    \item 当相同的电量通过不同电解质时,各电极上析出(或溶解)的物质的质量与其化学当量($M/z$)成正比。
\end{itemize}
法拉第常数:$F=96485$,或约为$F\approx 96500$,代表1mol电子的总电荷量。可以由元电荷$e$ 和阿伏伽德罗常数$N_\text{A}$ 直接计算:
\[
    F = e\cdot N_\text{A}
.\]

如果在电解池中发生: \[
    \ce{M^{Z+}} + Z\ce{e-} \to M
.\]
在阴极上生成$n\ce{mol}$金属M所需电子数量为$N =n\cdot N_\text{A}Z$ ,转移的电荷量为: \[
    Q = N\cdot e
.\]
带入法拉第常数:$e = \frac{F}{N_\text{A}}$ \[
    Q = nZF
.\]
即发生$n\ce{mol}$ 反应:
\[
    n = \frac{Q}{ZF}
.\]
为法拉第电解定律的数学表达式,由于$I = \frac{\mathrm{d}Q}{\mathrm{d}t}\Rightarrow Q = \int_{0}^{t} I \mathrm{d}t$,当电流恒定时$Q = It$ ,即:\[
    It = nZF
.\]
\subsubsection{离子电迁移}%
\label{ssub:离子电迁移}
假定电化学装置的阴阳极之间有假想的A, B平面,在通电时会有离子迁移过A, B平面,物质的量变化与迁移速率有关
\[
    \frac{\Delta n_{+}}{\Delta n_{-}} = \frac{Q_+}{Q_-} = \frac{r_{+}}{r_{-}}
.\]
\subsubsection{离子迁移数}%
\label{ssub:离子迁移数}
\begin{defi}
在电解质溶液中,某种离子所承担的电流占总电流的比例
\[
    t_i = \frac{I_i}{I_\text{total}} \implies t_+ = \frac{Q_+}{Q_++Q_-}\quad t_- = \frac{Q_-}{Q_++Q_-}
.\]
\end{defi}
\[
    t_++t_- = 1
.\]
\subsection{电导、电导率}%
\label{sub:电导、电导率}
\begin{defi}
    电导:电阻的倒数\[
        R = \frac{V}{I} = \frac{1}{L} \implies L = \frac{I}{V}
    .\]
    单位:西门子
\end{defi}
\begin{defi}
    电导率:电阻率$\rho$ 的倒数\[
        \kappa = \frac{1}{\rho} = \frac{1}{R}\cdot \frac{l}{A} = L\cdot \frac{l}{A}
    .\]
\end{defi}
\begin{defi}
    摩尔电导率:在相距为单位距离的两个平行电导电极之间放置含有1mol电解质溶液,此时溶液所具有的电导称为摩尔电导率\[
        \Lambda_{m} = \kappa V_{m} = \frac{\kappa}{c}
    .\]
    其中,$V_{m}$ 是含有1mol电解质溶液的体积,$c$ 为电解质溶液的浓度;摩尔电导率的单位为$\ce{S*m^2*mol^{-1}}$
\end{defi}
\subsubsection{电导、摩尔电导和浓度的关系}%
\label{ssub*:电导、摩尔电导和浓度的关系}
\begin{notation}
    电导率:

强电解质:电导率随浓度增加而升高,到达一定程度后,电导率降低

中性:升高但不能太高(饱和)

弱电解质:与浓度没有明显关联
\end{notation}
\begin{notation}
    在无限稀释后,强电解质的摩尔电导率趋近一个极限值,称为无限稀释时的摩尔电导率$\Lambda_{m}^{\infty }$ ;科尔劳施发现:\[
        \Lambda_{m} = \Lambda_{m}^{\infty } -A\sqrt{c}
    .\]
\end{notation}
