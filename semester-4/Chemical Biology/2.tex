\lecture{2}{03.04}
\section{现代生物基因技术}%
\label{sec:现代生物基因技术}
\begin{notation}
    DNA在碱性环境下稳定性更好:RNA在五元环的2位上多一个羟基,因此DNA约比RNA稳定100倍;但是DNA在酸性条件下更容易离去质子,因此在酸性条件下RNA比DNA稳定约1000倍
\end{notation}
\begin{notation}
    把人类的DNA拉长后约为2m长,约有$10^{8}\sim 10^{9}$ 个碱基
\end{notation}
\subsection{电泳分离DNA}%
\label{sub:电泳分离DNA}
一般使用:琼脂糖凝胶,从石花菜中提取的胶状物

\begin{notation}
    DNA带负电荷,弱碱性
\end{notation}
电泳的显色剂可以插入到DNA的碱基序列中,如EB(溴化乙锭),SYBR Green
\section{RNA的种类和功能}%
\label{sec:RNA的种类和功能}
\begin{itemize}
    \item rRNA: 核糖体
    \item tRNA:转录
    \item mRNA:转录基因,有三个基本区域:编码区、5'端和3'端
    \item microRNA
\end{itemize}
