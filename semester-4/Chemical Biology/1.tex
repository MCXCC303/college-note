\lecture{1}{02.18}
\section{第一课}%
\label{sec:第一课}
课程安排:2学分,32学时,周二6-9节,教学内容共7章

内容安排:
\begin{itemize}
    \item 导论
    \item DNA
    \item RNA
    \item 多肽和蛋白质
    \item 非模板依赖的分子
    \item 概念性技术
    \item 药物研发
\end{itemize}
成绩组成:
\begin{itemize}
    \item 课堂参与:30\%
    \item 课堂展示:30\%,4月8日,5人一组,13分钟,分享一篇近期的化学生物学领域研究性论文
    \item 课程报告:30\%,4月22日前提交电子版报告,包括:研究内容,科学问题,创新点,启发;1500-2000字
    \item 学习心得:10\%,发至jinluliu@cqu.edu.cn
\end{itemize}
\begin{notation}
    文献选择
\end{notation}
\subsection{概论}%
\label{sub:概论}
\begin{question}
    什么是中心法则?
\end{question}
\begin{notation}
中心法则:生命科学的基础,遗传信息从DNA传递到RNA,再从RNA传递到蛋白质
\end{notation}
\[
    \text{DNA}\xrightarrow[]{\text{Transcript}}\text{RNA}\xrightarrow[]{\text{Translate}}\text{Prot}
.\]
\begin{question}
    不同个体中序列相同的基因,功能是否一样?
\end{question}
\begin{sol}
    不一定
\end{sol}
\begin{question}
所有DNA序列都是基因吗?
\end{question}
\begin{sol}
    不是
\end{sol}
\begin{question}
    生命体是如何调控基因表达水平的?
\end{question}
\begin{notation}
    目前已知具有最少基因的生物为\textbf{支原体}
\end{notation}
\begin{notation}
    \textbf{基因组具有多样性}
\end{notation}
\subsection{多样性的体现}%
\label{sub:多样性的体现}
1. 从DNA到RNA:转录,DNA到RNA的转录并非绝对线性,不同基因对RNA的转录时间、空间、数量都有不同

2. RNA:并非所有转录出来的序列都会翻译为蛋白质(高等生物),还要先进行RNA加工(切除、拼接等)

3. RNA翻译为蛋白质:较稳定,没有功能的蛋白质会被降解;蛋白质完成翻译后还会进行化学修饰(组氨酸转为白喉酰胺,乙酰化等)

4. 非模板合成(糖类)
\begin{notation}
    多样性来源:\textbf{转录、RNA加工、翻译、非模板}

    现代生命科学在过去二十年的基础:人类基因组计划
\end{notation}
\begin{notation}
    点击化学(Click Chemistry)的诞生:一般化学反应的效率较低,点击化学不仅效率较高,还可以在生命体系/活细胞中进行
\end{notation}
\subsection{化学和生物学交合}%
\label{sub:化学和生物学交合}
\begin{notation}
第一次交叉:尿素的人工合成
\end{notation}
Using biology to advance chemistry: 生物用于发现氧气

Using chemistry to advance chemistry: 笑气(\ce{N2O})的应用,现在笑气可以用于牙医麻醉
\begin{defi}
    化学生物学是\textbf{化学、医学、生物学}的高度交叉,是利用\textbf{外源的}化学物质,通过介入式化学方法和途径,在分子层面上对生命体系进行精准修饰或调控的一门学科
\end{defi}
\begin{notation}
    化学生物学是\textbf{外源化学},使用\textbf{化学探针}(有别于药物化学)
\end{notation}
\textit{对比:}生物化学使用\textbf{内源}的化学物质(生物体内有的物质,如糖的合成,三羧酸循环)
\subsection{研究方向}%
\label{sub:研究方向}
\textit{利用外源物质调控解析生命体系}

Struart L. Schreiber,雷帕霉素
\begin{eg}
    分子胶水:某些小分子的化学结构可以将两个蛋白质粘在一起
\end{eg}
\textit{提供全新方法和手段}

Peter G. Schultz,蛋白质密码子扩展技术(可以创建非天然的氨基酸,引入含叠氮集团的氨基酸可以在体内进行click反应)
\begin{eg}
    在斑马鱼的胚胎中使用非天然氨基酸标记,使某些部位发光
\end{eg}
\textit{天然产物化学生物学}
\begin{eg}
    青霉素的杀菌作用机理
\end{eg}
\subsection{模块组装}%
\label{sub:模块组装}
\textbf{线性生物大分子可极大地丰富多样性}
\begin{eg}
    组合合成可用于DNA和多肽的文库合成
\end{eg}
\subsection{分子探针工具盒}%
\label{sub:分子探针工具盒}
\begin{eg}
    发色团:使分子级别的现象可见
\end{eg}
\subsection{基因操作工具盒}%
\label{sub:基因操作工具盒}
\subsection{活性分子工具盒}%
\label{sub:活性分子工具盒}
\subsection{活性分子与靶标研究}%
\label{sub:活性分子与靶标研究}
\begin{eg}
    青蒿素结构确定,结构改变(双氢青蒿素等)
\end{eg}
\begin{eg}
    Karl Paul Link发现某种植物中的活性物质可以抑制动物血液凝固
\end{eg}
\begin{question}
    生物体系中\textbf{分子相互作用}(如蛋白-蛋白相互作用)的本质?$^\star$
\end{question}
\subsection{非键合相互作用的主要类型}%
\label{sub:非键合相互作用的主要类型}
\begin{itemize}
    \item 色散
    \item 芳香
    \item 偶极
    \item 氢键
    \item 盐桥
\end{itemize}
\begin{notation}
    用于判断一定物质量中所包含的能量或破坏这些物质所需的能量的单位:$\text{kcal}\cdot \text{mol}^{-1}$
\end{notation}
其中盐桥的作用最强,色散作用最弱
\begin{notation}
    疏水相互作用
\end{notation}
\begin{question}
为什么生物大分子的结构单元是嘌呤、嘧啶、核糖、氨基酸?
\end{question}
\begin{notation}
$\text{pK}_\text{a}\left( \text{HCN} \right)=9.2$

最开始的生命分子由氢氰酸等小分子组合而成
\end{notation}
\begin{question}
DNA、RNA、蛋白质、酯类的稳定性排序如何,是否与其功能匹配?
\end{question}
作业:
\begin{itemize}
    \item 简要解释中心法则
    \item 如何理解模块组装带来多样性
\end{itemize}
