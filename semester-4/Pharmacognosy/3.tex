\lecture{3}{03.06}
\begin{notation}
    种子植物才有种子、果实等,被子植物没有果实
\end{notation}
根据果实的质地分为\textbf{肉果和干果}
\begin{eg}
    桃子的种子:由子房发育,种子包在中间,内果皮较硬

    相比之下,苹果、冬瓜和梨的内果皮较软
\end{eg}
干果在发育一段时间后可能开裂,也可以不开裂
\begin{eg}
    不开裂干果:瘦果/葵花子、大部分坚果
\end{eg}
\subsubsection*{果实的结构}%
\label{subsub*:果实的结构}
由外到内:外果皮、果肉、内果皮、种子
\subsection{种子}%
\label{sub:种子}
由雌蕊的胚珠发育而成,是种子植物特有的繁殖器官
\begin{notation}
    有胚乳的种子可以使用胚乳供应生长,没有胚乳的种子一般种叶肥大
\end{notation}
\section{植物分类}%
\label{sec:植物分类}
\subsection{分类系统}%
\label{sub:分类系统}
人为分类:\textbf{形态、习性、用途}

林奈:\textbf{24纲分类}
\subsection{命名系统}%
\label{sub:命名系统}
林奈“双名法”:\textbf{属名+种加名+命名人}

亚种:subsp.、变种:var.、变型:forma.

\begin{eg}
    山楂:\textit{Crataegus pinnatifida} Bge.

    山楂:\textit{C. pinnatifida} Bge. var. \textit{major} N. E. Br
\end{eg}
植物分为孢子植物/隐花植物和种子植物/显花植物,孢子植物中除了苔藓和蕨类植物之外都是低等植物,蕨类、裸子和被子植物是维管植物。
\section{藻类植物}%
\label{sec:藻类植物}
\begin{itemize}
    \item 最原始的低等植物
    \item 无根茎叶、无维管束,无胚
    \item \ldots 
\end{itemize}
藻类分布极为广泛
\begin{notation}
    螺旋藻被称为是21世纪最佳保健品
\end{notation}
常见蓝藻:
\begin{itemize}
    \item 铜绿微蓝藻:会污染水源,导致人蓝藻中毒
    \item 裙带菜:可食用、可药用
    \item \ldots 
    
\end{itemize}
