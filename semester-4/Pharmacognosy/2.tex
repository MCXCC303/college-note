\lecture{2}{02.27}
回顾:植物细胞的主要特征:
\begin{itemize}
    \item 有色体
    \item 液泡
    \item 质体
    \item \ldots 
    
\end{itemize}
\subsection{细胞壁}%
\label{sub:细胞壁}
细胞壁共三层,最外层为胞间层(果胶,可以相互粘连),中间为初生壁(纤维素、半纤维素,$1-3\mu$m),最里面为次生壁(含少量木质素,$5-10\mu$m,较坚硬)
\begin{notation}
    直链淀粉的连接方式:1-4 $\alpha$ 糖苷键(直立键)

    支链淀粉的连接除了直链以外还含有1-6 $\alpha$ 糖苷键,由于结构更加密集因此常常更黏

    纤维素也是1-4糖苷键,但是是1-4 $\beta$ 糖苷键(平伏键),人体内只含有能分解1-4 $\alpha$ 糖苷键的糖苷酶
\end{notation}
细胞壁上有胞间连丝和纹孔,胞间连丝可以用于物质传输和信息传递;纹孔可以组成纹孔对,共三种:\textbf{单纹孔(有纹孔膜)、具缘纹孔(有拱形边缘)、半圆纹孔(一边有纹孔缘,另一边无纹孔塞)},可以用于信号传递

细胞壁在其他环境的条件下可能发生其他的特化现象:
\begin{itemize}
    \item 木质化:木质素较多
    \item 木栓化:增加了脂肪性的木栓质,是原生质体坏死
    \item 角质化:产生脂肪性角质,如柑橘表面
    \item 黏液质化:产生果胶质纤维素的结构变化,如芦荟
    \item 矿物质化:产生硅质等,可以用于打磨
\end{itemize}
\subsection{植物细胞的分裂、生长和分化}%
\label{sub:植物细胞的分裂、生长和分化}
常见的分裂方式:

\textit{无丝分裂}:直接分裂,简单快速,不出现纺锤体,容易出错

\textit{有丝分裂}:复制$\Rightarrow $ 纺锤体$\Rightarrow $ 细胞壁生成$\Rightarrow $ 分裂,保证遗传物质稳定传递

\textit{减数分裂}:生成生殖细胞
\subsubsection*{分化}%
\label{subsub:分化}
有一些细胞具有干细胞的特性,可以分化为不同的细胞结构
\begin{defi}
    细胞分化:在个体发育过程中,细胞在形态、结构和功能上的特化功能
\end{defi}
课件在微助教上传
\section{植物组织}%
\label{sec:植物组织}
\begin{defi}
    组织:细胞经过分裂、分化后形成不同形态和构造的细胞群,相类似的细胞群
\end{defi}
两种分类组织的方式:分生组织、成熟组织,分生组织是具有干细胞特性的组织,可以分裂分化为成熟组织,成熟组织的分类:
\begin{itemize}
    \item 薄壁:吸收、同化(光合)、贮藏、通气、传递,又称营养组织
    \item 保护:避免水分散失,抵御侵袭,分为初生(表皮)和次生(周皮)
    \item 机械:支撑、保护,细胞壁特别厚,分为厚角组织(活细胞,正在生长的组织)和厚壁组织(死细胞,木质化)
    \item 输导:水分、营养,分为管胞(蕨类植物、裸子植物)和导管(被子植物、少数裸子植物)、筛管、伴胞(被子植物)和筛胞(蕨类、裸子)
    \item 分泌:分泌挥发油、树脂、蜜汁等,在内外都有分布
    
\end{itemize}
\begin{defi}
分生组织:具有持续或周期性分离能力的细胞群
\end{defi}
分生组织的位置主要分布于顶端和底端
\subsubsection*{维管束}%
\label{subsub:维管束}
进化到较高级的\textbf{蕨类植物、裸子植物和被子植物}时出现的组织,这三类称为维管植物。

\section{植物器官}%
\label{sec:植物器官}
\begin{defi}
    器官:由不同组织构成的具有一定外部形态和内部结构并执行一定生理功能的部分
\end{defi}
植物的器官分为营养器官和生殖器官。
\subsection{根}%
\label{sub:根}
生长于地下的营养器官,可以吸收土壤中的水分和无机盐到每一个部分;不同的植物分为直根系(人参、甘草、桔梗等双子叶植物)和须根系(百合、大蒜、小麦等单子叶植物)两种,其中直根系有明显的主根,侧根从分支长出

不同植物的根形态不同,如爬山虎的攀援根,吊兰的气生根(茎上产生不定根暴露在空气中)
\subsection{茎}%
\label{sub:茎}
茎上生叶的部位为\textbf{节},还有节间、芽、枝条等结构。茎按质地分类可以分为肉质茎(芦荟、仙人掌)、草质茎(草本植物)和木质茎(木本植物)

茎可以输导根吸收的成分和地上部分的光合作用产物
\subsection{叶}%
\label{sub:叶}
维管植物的营养器官之一,可以进行光合作用,有非常多的形状,但叶尖、叶基的种类并不多

叶序分为互生、对生、轮生、簇生等。叶的变态有苞片(看着很像花瓣,只有叶片)
\subsection{花}%
\label{sub:花}
有花冠、花柱、花蕊等结构。花的类型有:\textbf{完全花、不完全花},南瓜的公花为不完全花
\begin{notation}
花序:花密集或稀疏地按照某种方式有规律地生长在花枝上
\end{notation}
当雄蕊中的划分和雌蕊子房中的胚囊成熟时开花,多年生植物一生只开一次花
\subsection{果实}%
\label{sub:果实}
有些植物经过传粉后不经过受精也能发育为果实,称为无籽果实

根据果实的来源可以分为:单果、聚花果、聚合果
