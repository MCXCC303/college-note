\lecture{1}{02.20}
\section{课程介绍}%
\label{sec:课程介绍}
天然药物研究思路:
\begin{itemize}
    \item 中药学:动植物、矿物、微生物
    \item 天然药物:提取,研究生物活性导向
    \item 波谱分析:结构鉴定
    \item 活性测试:药理学
    
\end{itemize}
\begin{eg}
    银杏提取物、绿茶多酚(牙膏)、紫杉醇、青蒿素(蒿甲醚)
\end{eg}
\subsection{研究任务}%
\label{sub:研究任务}
\begin{itemize}
    \item 资源调查、文献考证、合理利用药学资源
    \item \ldots 
    
\end{itemize}
\section{植物细胞}%
\label{sec:植物细胞}
\subsection{原生质体}%
\label{sub:原生质体}
\begin{defi}
    细胞器:细胞中具有一定形态结构、组成和特定功能的微器官
\end{defi}
\begin{itemize}
    \item 植物特有
        \begin{itemize}
            \item 质体
            \item 液泡
        \end{itemize}
    \item 线粒体
    \item 内质网:合成蛋白质、类脂和多糖
    \item 高尔基体:运输多糖、蛋白质修饰
    \item 核糖体:合成蛋白质
    \item 微管、圆球体、溶酶体、微体等
    
\end{itemize}
\begin{notation}
    近年来发现有植物的细胞器可以固氮
\end{notation}
质体分为白色体、叶绿体和有色体,其中白色体含有合成淀粉的造粉体、合成蛋白质的蛋白质体,叶绿体含有合成氧气的叶绿素等

\subsection{后含物}%
\label{sub:后含物}
1. 淀粉:葡萄糖(醛类糖)缩水缩合,水解得到麦芽糖,完全水解得到葡萄糖;人体含有$\alpha-$淀粉糖苷酶,可以分解$\alpha-1,4$糖苷键;淀粉和水加热沸腾形成糊浆状,不溶于冷水

2. 菊糖:果糖(酮类糖,无还原性)聚合,溶于水,不溶于乙醇

3. 蛋白质:特指贮藏蛋白质,无生命,无积极代谢意义
