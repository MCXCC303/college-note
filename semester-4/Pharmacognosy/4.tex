\lecture{4}{03.13}
\subsection{常见绿藻}%
\label{sub:常见绿藻}
小球藻\textit{C. vulgaris}、衣藻\textit{Chlamydomonas}(有鞭毛)、极地雪藻\textit{Chlamydononas nivalis}(紫外线照射后产生虾青素)、石莼、浒苔\textit{Enteromorpha}(绿潮的原因,可药用,铁含量最高)
\subsection{常见药用藻类植物}%
\label{sub:常见药用藻类植物}
\begin{itemize}
    \item 海人草:其中的海人草酸Kainic Acid有神经生理、毒理作用
    \item 鹧鸪菜:主治蛔虫病
    \item 石花菜:有去火作用,可以抗寄生虫
    \item 麒麟菜、琼枝:清肺化痰等作用
    \item 紫菜\textit{Porphyra}:有软件散结、清热化痰、利尿的作用
    \item 地木耳:普通念珠藻
    \item 葛仙米:有明目、清热作用,约2000元/kg
    \item 海萝、蜈蚣藻:功效类似,蜈蚣藻可以驱虫
    \item 海带:防治缺碘性甲状腺,可以降压降血脂、提高免疫、预防心脑血管疾病,海带多糖有抗辐射作用
    \item 还有铁钉菜、蛋白核小球藻、总状蕨藻、杜氏盐藻等
\end{itemize}
\begin{notation}
    甘露特纳(GV-971),从褐藻中提取,为不均一寡糖,有抑制老年痴呆的作用
\end{notation}
\section{菌类植物}%
\label{sec:菌类植物}
\begin{notation}
    已知的菌类有数十万种
\end{notation}
\subsection{菌类分门}%
\label{sub:菌类分门}
\begin{itemize}
    \item 细菌门:约2000种,用于分解动植物尸体和排泄,部分细菌可以吸取大气中的氮
    
\end{itemize}
\begin{notation}
    细菌作用:

    枯草杆菌能生产蛋白酶和淀粉酶,可以用于鞣制皮革、皮革脱毛、丝绸脱胶、棉布脱浆

    乳酸杆菌和醋酸杆菌可以生产乳酸和醋酸

    谷氨酸短杆菌:产生谷氨酸(味精:谷氨酸钠)和肌苷酸
\end{notation}
