\lecture{5}{03.20}
\begin{notation}
    青霉素:青霉菌青霉属Penicillium,用于治疗梅毒和淋病
\end{notation}
\begin{notation}
    链霉素:链霉菌链霉菌属Streptomyces,用于治疗结核病
\end{notation}
\begin{notation}
环丙沙星的合成、四环素、磺胺嘧啶、万古霉素、氧氟沙星、诺氟沙星
\end{notation}
\subsubsection*{细菌的害处}%
\label{subsub*:细菌的害处}
\begin{notation}
    引起流行病,如痢疾、霍乱、白喉、破伤风等
\end{notation}
\begin{notation}
    农生植物疾病,如水稻叶枯病
\end{notation}
可以利用格兰仕染色法来鉴别革兰氏阳性和阴性两种细菌,革兰氏阳性呈龙胆紫、革兰氏阴性呈红色或粉色;大多数革兰氏阳性菌对青霉素敏感。
