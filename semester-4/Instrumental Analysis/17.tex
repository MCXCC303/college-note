\lecture{17}{05.06}
\section{质谱法}%
\label{sec:质谱法}
\begin{notation}
    结构分析四大谱系:NMR,IR,MS,UV
\end{notation}
\subsection{质谱定义}%
\label{sub:质谱定义}
使用高速电子流击碎样品,通过带电粒子在磁场中的运动来控制不同碎片的运动距离
\begin{notation}
    质谱是物理分析法,通过与GC和HPLC混用可以测定复杂的混合物组成
\end{notation}
\begin{notation}
    质谱是唯一可以给出分子量并确定分子式的分析方法
\end{notation}
质谱的应用范围广(有机/无机),灵敏度高(常用量约1mg),分析速度快(最快千分之一秒),可以实现色谱-质谱共线联用
\subsection{质谱测定过程}%
\label{sub:质谱测定过程}
进样$\to $ 离子化$\to $ 质量分析器$\to $ 检测器$\to $ 输出
\subsubsection{真空系统}%
\label{ssub*:真空系统}
包含样品导入系统、离子源、质量分析器、检测器,真空度需要尽量高,否则会导致氧气烧坏离子源的灯丝、高压放电等
\subsubsection{样品导入系统}%
\label{ssub*:样品导入系统}
在不降低真空度的条件下将样品导入到离子源中
\subsubsection{电离源}%
\label{ssub*:电离源}
有硬电离和软电离两种,作用是使样品转为离子

3.1. 电子轰击源:产生电子流将试样轰击,失去电子并断开化学键变为离子碎片:
\begin{align*}
    \ce{M + e-} &\to \ce{M^{+.} + 2e-}\\
    \ce{M^{+.} + n e-} &\to \ce{(A + B + C + \ldots )^{m+} + k e-}
.\end{align*}
可以通过调节电子的能量来控制电离的强度,一般使用70eV
\begin{notation}
    EI(电子轰击)源会生成大量碎片离子,不适合难挥发、热不稳定的化合物
\end{notation}
3.2. 在样品被轰击之前使用\textbf{高压反应气}稀释,称为CI源
\begin{eg}
    反应气分子受到轰击:\[
        \ce{CH_4 + e- \to CH_4^{+.} + 2e- }\quad \ce{CH_4 + 2e- \to CH_3^{+} + H+}
    .\]
    轰击后的离子可以两两反应,反应后的离子与样品反应生成准分子离子,其中会生成$M+1$ 的碎片离子
\end{eg}
3.3. 快原子轰击源FAB:使用快速氩离子和热的氩原子交换电荷生成快速氩原子,通过氩气快原子轰击样品使其失去电子
\begin{notation}
    FAB是一种软电离方法,适用于热不稳定、大分子量、强极性分子、生物分子、配合物等的分析
\end{notation}
3.4. 基质辅助激光解析电离源MALDI:利用脉冲式激光照射样品使其电离
\subsubsection{质量分析器}%
\label{ssub*:质量分析器}
将离子按照质核比$m / z$ 分开
\begin{notation}
    质量分析器的类型:
    \begin{itemize}
        \item 磁质量分析器
        \item 四级杆质量分析器
        \item 飞行时间质量分析器
        \item \ldots 
    \end{itemize}
\end{notation}
4.1. 磁质量分析器:最简单的分析器,如图所示
\begin{figure}[ht!]
    \centering
    \incfig[0.5]{磁质量分析器}
    \caption{磁质量分析器}
    \label{fig:磁质量分析器}
\end{figure}
\begin{notation}
    易得:\[
        m / z = \frac{H^2 R^2 }{2V} \Rightarrow R = \frac{1.41}{H}\times \sqrt{\frac{mV}{z}}
    .\]
    当$H,V$ 一定时,离子的质荷比越大,偏移(离子色散)越大
\end{notation}
4.2. 双聚焦质量分析器:通过先加一个电场来加速离子,达到了能量和方向上的聚焦,分辨率极高但价格昂贵

4.3. 四极杆质量分析器QMF:有四根平行的金属管,加速离子通过四极杆中间的小孔,不符合要求$m / z$ 的离子撞击到四极杆而被滤除
