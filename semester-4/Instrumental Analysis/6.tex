\lecture{6}{03.14}
第一种:无干扰,a的最大吸光度b不吸收

第二种:a干扰b,b不干扰a;先解出a

第三种:相互干扰;通过计算分光光度法,利用吸光度的加和性来测定

解方程:
\[
    \begin{cases}
        A_1^{a+b} = A_1^{a}+A_1^{b}\\
        A_2^{a+b} = A_2^{a}+A_2^{b}
    \end{cases}
.\]
第二个方法:等吸收双波长法:在吸收峰两侧选择一对吸光度相等的点,其中一个点落在另一个吸收峰的垂线上:
\begin{figure}[ht!]
    \centering
    \incfig[0.5]{等吸收双波长}
    \caption{等吸收双波长}
    \label{fig:等吸收双波长}
\end{figure}

计算:\[
    A_1^{b} = A_2^{b}\quad \Delta A = A_2-A_1 = \left( A_2^{a}+A_2^{b} \right)-\left( A_1^{a}-A_1^{b} \right) = A_2^{a}-A_1^{a}
.\]
\section{荧光分析法}%
\label{sec:荧光分析法}
\subsection{概述}%
\label{sub:概述}
\begin{defi}
    发射光谱:处于激发态的原子离子或分子返回基态发出的能量
\end{defi}
分子发光:能量以辐射的方式释放
\begin{notation}
    按激发模式分为:\textbf{光致发光、生物发光、化学发光}
\end{notation}
光致发光分为\textbf{荧光和磷光},分子荧光分析法通过荧光的位置强度测定物质含量和鉴定
\subsection{特点}%
\label{sub:特点}
\begin{itemize}
    \item 灵敏度高:检出限极低,比UV-vis还低3个数量级
    \item 选择性好
    \item 线性范围宽
    \item 有强荧光的物质不多,应用范围窄,大多用于生物物质
\end{itemize}
\subsection{分子荧光的产生}%
\label{sub:分子荧光的产生}
基态原子吸收光能处于激发态$\left( S_1,S_2 \right)$,返回基态时通过辐射跃迁和无辐射跃迁释放能量,如图:
\begin{figure}[ht!]
    \centering
    \incfig[0.5]{基态分子激发态}
    \caption{基态分子激发态}
    \label{fig:基态分子激发态}
\end{figure}
\subsubsection*{电子能级的多重性}%
\label{subsub*:电子能级的多重性}
受到激发后,电子的自旋方向不变,称为激发单重态,记为$S_{n}$;激发后自旋方向改变,成为三重态,称为激发三重态,记为$T_{n}$ ,一般比$S_{n}$ 的能量低

激发单重态的分子平均寿命短,三重态的较长。单重态激发到三重态为禁阻跃迁,进入的几率较小

\begin{figure}[ht!]
    \centering
    \incfig[]{振动弛豫、内转换和荧光}
    \caption{振动弛豫、内转换和荧光}
    \label{fig:振动弛豫、内转换和荧光}
\end{figure}
