\lecture{15}{04.22}
两种弛豫是NMR信号维持的要素:
\begin{itemize}
    \item 自旋-晶格弛豫
    \item 自旋-自旋弛豫
\end{itemize}
\subsubsection*{自旋-晶格弛豫}%
\label{subsub*:自旋-晶格弛豫}
处于高能态的核自旋体系将能量传递给\textbf{周围环境},又称\textbf{纵向弛豫}
\begin{notation}
    半衰期用$T_1$ 表示,液体的$T_1$ 一般较短,固体的$T_1$ 通常较长
\end{notation}
\subsubsection*{自旋-自旋弛豫}%
\label{subsub*:自旋-自旋弛豫}
处于高能态的核自旋体系将能量传递给\textbf{邻近低能态同类磁性核},又称\textbf{横向弛豫}
\begin{notation}
    半衰期用$T_2$ 表示,一般气体和液体试样的$T_2\approx 1\text{s}$ ,固体试样$T_2\approx 10^{-4}\sim 10^{-5}$
\end{notation}
\subsection{核磁共振仪}%
\label{sub:核磁共振仪}
分类:
\begin{itemize}
    \item 连续波核磁共振仪
    \item 脉冲傅里叶变换共振仪
\end{itemize}
\begin{notation}
    连续波核磁共振仪:有磁铁、射频发生器、射频接收器、扫场线圈,样品管

    使用Helmholtz线圈来制作扫场线圈,可以通过直流电的强度来调整磁场的强度
\end{notation}
固定照射频率、改变磁场强度的称为\textbf{扫场};固定磁场强度,改变照射频率的称为\textbf{扫频}
\begin{notation}
    脉冲傅里叶变换共振仪:用一个强的射频,以脉冲方式使系统偏离平衡,系统在弛豫时会释放信号,幅度随时间衰减,称为\textbf{自由感应衰减信号}或FID,再通过傅里叶变换转为频谱图
\end{notation}
\subsection{溶剂和试样测定}%
\label{sub:溶剂和试样测定}
主要考虑:
\begin{itemize}
    \item 试样溶解度
    \item 是否产生干扰信号
    
\end{itemize}
\begin{eg}
氢谱常用\textbf{氘代试剂}
\end{eg}
\subsection{化学位移}%
\label{sub:化学位移}
根据$\nu_0=\nu$ :$\ce{^1H}$ 在1.4092T时只吸收60MHz的电磁波,但是实验发现不同化学环境的氢核所吸收的频率有差异,数量级在10ppm下
\subsubsection*{屏蔽效应}%
\label{subsub*:屏蔽效应}
原子核感受到的磁场强度比实际施加的磁场更低,原因是核外电子绕核旋转产生与磁场方向相反的感应磁场,抵消一部分磁场

由于屏蔽作用,实际受到的场强:\[
    H = H_0-\sigma H_0 = \left( 1-\sigma \right)H_0
.\] 
称$\sigma$ 为屏蔽常数,带回计算进动频率:
\[
    \nu = \frac{\gamma}{2\pi }\left( 1-\sigma \right)H_0
.\]
使用扫频时,NMR信号出现在低频区;使用扫场时,NMR信号出现在高频区,即:\textbf{右端为高场低频、左端为低场高频}

\begin{notation}
    由于屏蔽效应的存在,不同化学环境的氢核的共振频率不同,称这种现象为\textbf{化学位移}$\delta$,一般定义四甲基硅烷的氢原子的化学位移为参考
\end{notation}
扫场:
\[
    \delta = \frac{H_\text{标准}-H_\text{样品}}{H_\text{标准}}\times 10^{6}\left( \text{ppm} \right)
.\]
扫频:
\[
    \delta = \frac{\nu_\text{样品}-\nu_\text{标准}}{\nu_\text{标准}}\times 10^{6}\left( \text{ppm} \right)
.\]
\subsection{化学位移的影响因素}%
\label{sub:化学位移的影响因素}
\begin{itemize}
    \item 局部屏蔽效应
    \item 溶剂效应
    \item 去屏蔽效应
    \item 电负性:越大,化学位移越大
\end{itemize}
