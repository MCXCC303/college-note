\lecture{11}{04.01}
\subsection{影响红外光谱的因素}%
\label{sub:影响红外光谱的因素}
\begin{itemize}
    \item 诱导效应:朝高波数移动
    \item 共轭效应:电负性越大的共轭效应越强,朝低波数移动
    \item 氢键效应:电子云密度平均化,形成氢键基团的伸缩振动频率降低,峰变宽,强度增加
    \item 振动偶合:一个键的振动 通过公共原子是另一个键的长度发生变化
    \item Fermi振动
    \item 空间位阻效应:取代基使得共轭体系受阻,频率往高频移动
    \item 互变异构效应:如烯醇式互变异构产生共轭
    \item 环张力效应:环状化合物的吸收频率高于同碳的链状化合物,环元素减少环张力增加,环外双键增强,频率增加;环内双键相反
\end{itemize}
\subsection{红外吸收光谱的应用}%
\label{sub:红外吸收光谱的应用}
主要观察:\textbf{峰的位置、数目、吸收强度},可以用于鉴定结构组成,确定化学基团
\begin{notation}
    吸收谱带的吸收强度和分子组成或化学基团的含量有关
\end{notation}
纵坐标为百分透射率$T\%$ ,横坐标使用波数$\sigma\left( \text{cm}^{-1} \right)$
\subsubsection*{一般原则}%
\label{subsub*:一般原则}
\begin{itemize}
    \item 峰位
    \item 峰强
    \item 峰型
\end{itemize}

