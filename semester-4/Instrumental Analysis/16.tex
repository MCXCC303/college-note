\lecture{16}{04.29}
\subsection{耦合常数}%
\label{sub:耦合常数}
\begin{defi}
    偶合裂分:由于相邻磁性核(如$\ce{^1H}$)之间的自旋-自旋耦合($J$耦合),导致单个核的信号峰分裂成多重峰
\end{defi}
\begin{eg}
    裂分规律由相邻等价质子的数量决定
    \begin{itemize}
        \item 相邻有1个质子($n=1$)→ 二重峰(doublet, d)
        \item 相邻有2个等价质子($n=2$)→ 三重峰(triplet, t)
        \item 无相邻质子($n=0$)→ 单峰(singlet, s)。
        
    \end{itemize}
\end{eg}
\begin{defi}
    化学等价:有相同化学环境的核具有相同的化学位移,称为一组化学等价核
\end{defi}
\begin{defi}
    磁等价核:分子中一组化学等价核与分子中其他任何一个核都有相同强弱的偶合,称为一组磁等价核
\end{defi}
如何判断磁等价:
\begin{itemize}
    \item 在对称分子中,处于完全对称位置的氢是磁等价
    \item 若分子内部基团的相对旋转速度大于仪器的响应速度,则磁不等价体现为磁等价
\end{itemize}
    
