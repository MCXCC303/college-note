\lecture{10}{03.28}
\subsection{峰位}%
\label{sub:峰位}
对应振动的能量:\[
    \nu = \frac{1}{2\pi }\sqrt{\frac{k}{\mu}}\quad \mu = \frac{m_1m_2}{m_1+m_2}
.\]
一般情况下:\textbf{不对称伸缩}$>$ 对称伸缩$>$ 弯曲伸缩
\begin{eg}
\begin{figure}[ht!]
    \centering
    \incfig[0.5]{水分子}
    \caption{水分子}
    \label{fig:水分子}
\end{figure}
\end{eg}
\subsection{峰强}%
\label{sub:峰强}
\textbf{瞬间偶极矩变化大,吸收峰强}
\begin{eg}
    有杂原子的时候偶极矩变化大,吸收强
\end{eg}
\textbf{跃迁几率越大,吸收峰越强}
\begin{eg}
    对比:$\nu_{\ce{C=O}}: 1745\text{cm}^{-1}$ ,$\nu_{\ce{C=C}}: 1650\text{cm}^{-1}$ :$\Delta\mu$ 越大,峰强越大

    对比$\nu_{\ce{C=O}}\left( 0\to 1 \right): 1745\text{cm}^{-1}$ ,$\nu_{\ce{C=O}}\left( 0\to 2 \right): 3490\text{cm}^{-1}$:跃迁几率越大,峰强越大
\end{eg}
\begin{notation}
    振动形式:
    \begin{itemize}
        \item 反对称伸缩$>$ 对称伸缩
        \item 伸缩$>$ 面内弯曲
    \end{itemize}
\end{notation}
\begin{eg}
    二氧化碳:对称伸缩振动由于偶极矩为0,没有红外活性;弯曲振动发生简并:
\begin{figure}[ht!]
    \centering
    \incfig[0.5]{二氧化碳红外吸收}
    \caption{二氧化碳红外吸收}
    \label{fig:二氧化碳红外吸收}
\end{figure}
\end{eg}
\subsection{红外光谱的特征性}%
\label{sub:红外光谱的特征性}
\begin{eg}
    \begin{itemize}
        \item $2800\sim 3000\text{cm}^{-1}$ :甲基特征峰
        \item $1600\sim 1850\text{cm}^{-1}$ :羰基特征峰
    \end{itemize}
\end{eg}
根据振动形式分为4个区:
\begin{itemize}
    \item $4000\sim 2500$ :与氢相关伸缩振动区(X-H)
    \item $2500\sim 2000$ :三键、累计双键伸缩振动区
    \item $2000\sim 1500$ :双键伸缩振动区
    \item $1500\sim 400$ :X-Y伸缩振动和X-H变形振动区
\end{itemize}
一般把$4000\sim 1300$ 称为\textbf{特征区},$1300\sim 400$ 称为\textbf{指纹区}:特征区用于确定化合物有哪些基团并确定化合物的类别,吸收峰稀疏,易于辨认;指纹区吸收峰密集、多变且复杂,用于判断化合物的细微结构
\begin{notation}
    诱导效应(I效应):取代基具有不同的电负性,通过静电诱导作用引起分子中电子分布的变化,改变键力常数$k$,使基团的特征频率发生位移
\end{notation}
