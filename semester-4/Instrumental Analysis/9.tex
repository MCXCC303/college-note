\lecture{9}{03.25}
\section{红外吸收光谱法}%
\label{sec:红外吸收光谱法}
\begin{notation}
    振动能级+转动能级跃迁,又称分子振动转动光谱
\end{notation}
红外分为3个区域,能量由高到低:
\begin{itemize}
    \item 近红外区:$0.76\sim 2.5\mu \text{m}$
    \item 中红外区:$2.5\sim 25\mu \text{m}$,最适合进行红外光谱的定性、定量和结构分析
    \item 远红外区:$25\mu \text{m}$ 以上
\end{itemize}
红外吸收光谱一般使用$T-\lambda$ 或$T-\sigma$ 曲线绘制,常用的$\sigma\in [400,4000]\text{cm}^{-1}$
\begin{notation}
    除了单原子和同核分子(如Ne, He, $\ce{O_2}$, $\ce{H_2}$)等之外,几乎所有的化合物在红外光谱区域均有吸收;除了光学异构体、某些高分子量的高聚物和在分子量只有微小差别的化合物外,\textbf{凡是具有结构不同的两个化合物,红外光谱一定不同}
\end{notation}
物质吸收辐射的两个条件:
\begin{itemize}
    \item 红外辐射的能量与分子的振动能级相同
    \item 分子振动时,必须伴随瞬时偶极矩的变化,即$\Delta\mu\neq 0$
\end{itemize}
\subsection{红外吸收光谱产生过程}%
\label{sub:红外吸收光谱产生过程}
当一定$\lambda$ 的光照射分子时,若分子中某个基团振动频率与其一致,二者产生共振或\textbf{振动偶合},此时光的能量通过分子偶极矩变化传递给分子,使分子的基团吸收能量,分子跃迁
\begin{notation}
    常用术语:
    \begin{description}
        \item[基频峰] $v=0$ 跃迁到$v=1$ 时产生的吸收峰
        \item [倍频峰]$v=0$ 到$v=2$ 或$v=3$ 等,分别称为第二倍频、第三倍频等
        \item [和频峰、差频峰]形如$v_1+v_2$ 的称为和频峰,反之为差频峰
        \item [泛频峰]倍频、和频和差频统称为泛频
    \end{description}
\end{notation}
\begin{question}
    分子基团的振动是怎么样的?振动频率取决于什么?
\end{question}
\subsection{化学键力常数}%
\label{sub:化学键力常数}
 \[
    \sigma_{\ce{C#C}}>\sigma_{\ce{C=C}}>\sigma_{\ce{C-C}}
.\]
\textbf{化学键力常数越大,折合相对原子质量越小,化学键的振动频率越大},伸缩振动基频峰的波数越大
\begin{equation}
    \label{eq:vbar}
    \overline{v} = \frac{1}{2\pi c}\sqrt{\frac{k}{\mu}}
\end{equation}
\subsection{两类基本振动形式}%
\label{sub:两类基本振动形式}
\subsubsection*{伸缩振动}%
\label{subsub*:伸缩振动}
键长变化、键角不变,原子沿建州方向伸缩,分为对称和反对称两种
\begin{figure}[ht!]
    \centering
    \incfig[0.5]{反对称伸缩振动}
    \caption{反对称伸缩振动}
    \label{fig:反对称伸缩振动}
\end{figure}
\subsubsection*{变形振动}%
\label{subsub*:变形振动}
分为面内和面外,面内的形式有:
\begin{itemize}
    \item 剪式振动$\delta$
    \item 平面摆动振动$\rho$
\end{itemize}
面外振动的形势有:
\begin{itemize}
    \item 非平面摇摆振动$\omega$
    \item 扭曲运动$\tau$
\end{itemize}
\begin{figure}[ht!]
    \centering
    \incfig[0.5]{变形振动}
    \caption{变形振动}
    \label{fig:变形振动}
\end{figure}
\subsection{峰数}%
\label{sub:峰数}
\begin{notation}
    振动自由度$f$ :\[
        f = 3N \text{(运动自由度)}-\text{平动自由度}-\text{转动自由度}
    .\]
    其中线性分子(平动自由度2, 转动自由度3)为:\[
        f = 3N-5
    .\]
    非线性分子(平动自由度3, 转动自由度3):\[
        f = 3N-6
    .\]
\end{notation}
\begin{eg}
    水分子是非线性分子,$f=3N-6=3$ ,说明水分子有3种基本的振动形式;而二氧化碳为线性分子,$f = 3N-4=4$ ,但是二氧化碳的吸收峰一般只能看到2个峰
\end{eg}
某些振动无法看到的原因:
\begin{itemize}
    \item 峰的简并,仪器分辨率不足无法观测到,如$\beta_{\ce{C=O}}\approx \gamma_{C=O}\approx 666\text{cm}^{-1}$
    \item 非红外活性振动,如$\nu^{s}_{\ce{C=O}}=1340\text{cm}^{-1}$
    
\end{itemize}
\begin{notation}
    Fermi共振:某一个振动的基频和另一个振动的倍频或和频接近时,由于相互作用而\textbf{在该基频峰附近出现两个吸收带}
\end{notation}
\begin{notation}
    振动偶合:两个化学键的振动频率相等或接近时,基频吸收峰\textbf{分裂为两个频率相差较大的吸收峰}
\end{notation}
\begin{eg}
    酸酐在羰基的吸收峰出现两个吸收峰,并且$\Delta\sigma\approx 60\text{cm}^{-1}$
\end{eg}
