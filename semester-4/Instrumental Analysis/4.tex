\lecture{4}{03.07}
回顾:电子跃迁的基础
\begin{notation}
    MOT/分子轨道理论:如果成键必然有成键轨道和反键轨道,如$\sigma$ 和$\sigma^\star $ ,$\pi $ 和$\pi ^\star $,其中反键轨道的能量更高,$E_{\sigma}<E_{\pi }$ ,相应的$E_{\sigma^\star }>E_{\pi ^\star }$
\end{notation}
对于非键轨道$n$ :\[
    E_{\sigma^\star }>E_{\pi ^\star }>E_{n}>E_{\pi }>E_{\sigma}
.\]
其中发生跃迁的可能有:$\sigma\to \sigma^\star $ 、$\pi \to \pi ^\star $ 、$n\to \sigma^\star $、$n\to \pi ^\star $。注意$\sigma$ 无法跃迁到$\pi$ 相关的轨道(空间结构)

\begin{notation}
    生色团:含有不饱和键的基团;助色团:含有$n$ 电子的饱和基团
\end{notation}
\begin{itemize}
    \item R
    \item K
    \item B
    \item E( $\ce{E_1,E_2}$)
\end{itemize}
共轭效应:红移
\begin{figure}[ht!]
    \centering
    \incfig[0.5]{共轭效应电子跃迁}
    \caption{共轭效应电子跃迁}
    \label{fig:共轭效应电子跃迁}
\end{figure}

溶剂效应:
在极性溶剂中$n\to \pi ^\star $ 的跃迁能隙增大,发生蓝移;而$\pi \to \pi ^\star $中,$\pi $ 轨道的极性小于$\pi ^\star $,因此跃迁能隙减小,红移
\begin{figure}[ht!]
    \centering
    \incfig[0.5]{溶剂效应}
    \caption{溶剂效应}
    \label{fig:溶剂效应}
\end{figure}

pH影响:\textbf{质子化和去质子化},质子化时,$\text{p} $ 轨道上的电子不易发生$\text{p} -\pi $共轭。

\subsection{朗伯比尔定律}%
\label{sub:朗伯比尔定律}
一束光通过样品后:\textbf{吸光度、透光率}有哪些关系
\begin{notation}
    光的吸收程度$A$和吸收层厚度$l$成正比,即: \[
        A\propto l
    .\]
    后来比尔提出浓度越高吸收越大,即:\[
        A \propto c
    .\]
\end{notation}
朗伯比尔定律:\begin{equation}
    \label{eq:朗伯比尔定律}
    A = -\lg T = Elc
\end{equation}
其中$l$ 的单位为cm,$E$ 是吸光系数,分为摩尔系数、浓度吸光系数和百分吸光系数,根据溶液浓度的条件更改
\begin{notation}
    朗伯比尔定律描述物质对\textbf{单色光}的吸收
\end{notation}
吸光系数的不同类型:
\begin{itemize}
    \item 摩尔吸光系数:$\varepsilon$ ,表示在\textbf{一定波长}时,溶液浓度为1 mol/L,厚度为1 cm的吸光度
    \item 百分吸光系数$E_{1\text{cm}}^{1\%}$:\textbf{一定波长},溶液浓度为1\% (w/v),厚度为1 cm的吸光度
\end{itemize}
两个之间的换算:\[
    \varepsilon = \frac{M}{10}\cdot E_{1\text{cm}}^{1\%}
.\]
(通过$E = A / lc$ 单位制计算)
\begin{notation}
    摩尔吸光系数需要通过实验测得吸光度$A$来间接计算
\end{notation}
同一物质,在最大吸收波长$\lambda_\text{max}$ 的摩尔吸光系数记为$\epsilon_\text{max}$
\begin{notation}
    朗伯比尔定律的成立条件:
    \begin{itemize}
        \item 单色光
        \item 均匀非散射介质
        \item 吸光物质之间不相互作用
        
    \end{itemize}
\end{notation}
\subsubsection*{朗伯比尔定律的偏离}%
\label{subsub*:朗伯比尔定律的偏离}
\begin{figure}[ht!]
    \centering
    \incfig[0.5]{朗伯比尔定律偏离}
    \caption{朗伯比尔定律偏离}
    \label{fig:朗伯比尔定律偏离}
\end{figure}
影响:
\begin{itemize}
    \item 非单色光
    \item 反射
    \item 散色光
    \item 浓度过高(多聚体)
    \item 溶质离解
    \item 缔合
    \item 互变异构
    \item \ldots 
\end{itemize}
\subsection{紫外可见分光光度计}%
\label{sub:紫外可见分光光度计}
分为\textbf{光源、分光吸收池、探测和数据输出},在分光时通过光栅或棱镜得到一束近似的单色光。
\begin{notation}
    光源需要是连续、有足够强度、稳定的复合光,通过分光器分为不同的波长的光
\end{notation}
