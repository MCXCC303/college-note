\lecture{3}{03.04}
\begin{notation}
    不涉及原子内部的能级跃迁的分析方法称为非光谱分析,如X射线单晶衍射、反射分析
\end{notation}
按照吸收或发射光的分类:
\begin{itemize}
    \item 吸收光谱:光子被原子吸收,产生了能级跃迁
    \item 发射光谱:原子发生能级跃迁,产生一定波长的光
    \item 散射光谱:光子与介质之间产生碰撞,光直接反射的部分称为\textbf{瑞利散射},改变了频率后再被散射出来的部分称为\textbf{拉曼散射}
\end{itemize}
常见的几种跃迁方式:
\begin{notation}
    $\sigma\to \sigma^\star $ 跃迁:碳碳单键中的电子由$\sigma$ 成键轨道跃迁到$\sigma^\star $ 反键轨道,吸收峰在紫外区,能量极高
\end{notation}
\begin{notation}
    $\pi \to \pi ^\star $ 跃迁:碳碳双键上的电子跃迁到双键的反键轨道,摩尔吸光系数较大,即$\varepsilon>10^{4}$ ,为强吸收,如果是共轭双键吸收更强,能量需求越小
\end{notation}
\begin{notation}
    $n\to \pi ^\star $跃迁:含有羟基、氨基、卤素原子、硫原子等杂原子,吸收峰为紫外光,吸收较强
\end{notation}
\begin{notation}
    $n\to \sigma^\star $跃迁:含杂原子的双键,吸收较弱,吸收近紫外光(200-400$\mu$m)
\end{notation}
常见术语:
\begin{defi}
    透光率:\[
        T = \frac{I_\text{t}}{I_0}\times 100\%
    .\]
    其中:\[
        I_0 = I_\text{t} + I_\text{a} + I_\text{r}
    .\]
    $I_\text{t}$ 为透过光,$I_\text{a}$ 为吸收光,$I_\text{r}$为反射光
\end{defi}
当$T=0$,代表光全部吸收
\begin{defi}
    吸光度:\[
        A = \lg \frac{1}{T} = -\lg T = \lg \frac{I_0}{I_\text{t}}
    .\]
    即:\[
        T = 10^{-A}
    .\]
    当$T = 0$ 时,$A = +\infty $
\end{defi}
\begin{figure}[ht!]
    \centering
    \incfig[0.5]{吸收光谱示意图}
    \caption{吸收光谱示意图}
    \label{fig:吸收光谱示意图}
\end{figure}
\begin{notation}
    助色团:一些含有$n$ 电子的饱和集团,如羟基、氨基、亚胺等,本身没有生色功能,当与生色团相连时发生$n-\pi $ 共轭来增强生色团的颜色
\end{notation}
\begin{defi}
    红移:或长移,由于结构改变使得吸收峰向长波长方向改变
\end{defi}
同理有蓝移、增色(上移)、减色(下移)

强吸收:$\varepsilon_\text{max}>10^{4}$
\subsection{吸收带与分子结构的关系}%
\label{sub:吸收带与分子结构的关系}
\begin{defi}
    R带:基团带,由$n\to \pi^\star $ 跃迁产生的吸收带,杂原子的不饱和集团,溶剂极性增强发生蓝移,弱吸收

    K带:共轭带,由$\pi \to \pi ^\star $ 的跃迁产生,波长较短,强吸收,共轭增强发生红移

    B带:芳香带,又称苯的多重吸收带,在极性溶剂中精细结构转为宽峰,在256nm附近,弱吸收

    E带:芳香不饱和带,由$\pi \to \pi ^\star $ 产生,共两个峰:吸收峰为180nm和200nm,均为强吸收
\end{defi}
还有一种是无机物的电子转移:
\begin{notation}
    电荷转移吸收带:同时具有配位体和受体的分子在吸收外来辐射后,电子从给体跃迁到受体,强吸收,可以用于定量分析,\textbf{发生在分子内部,需要光照}
\end{notation}
\begin{defi}
    简并轨道:过渡元素的d和f轨道
\end{defi}
当与配位体配合时,轨道的简并解除,发生能级分裂,轨道如果有没有填充电子的位置,低轨道的电子会吸收能量跃迁到更高的d和f轨道,产生吸收光谱
\subsection{影响吸收带的主要因素}%
\label{sub:影响吸收带的主要因素}
影响形状或位置:
\begin{itemize}
    \item 结构
    \item 状态
    \item 温度
    \item 溶剂极性
    
\end{itemize}
影响吸收带强度:
\begin{itemize}
    \item 能级差
    \item 空间位置
    
\end{itemize}
\begin{notation}
    共轭效应:当$\pi $ 电子共轭增大,如1-3二丁烯,两个$\pi $ 轨道会形成能量一高一低的两个轨道$\pi_1,\pi_2$,同样$\pi ^\star $ 会形成两个轨道$\pi ^\star_3, \pi ^\star_4$,导致$\lambda_\text{max}$ 红移,吸光系数$\varepsilon_\text{max}$ 增大
\end{notation}
\begin{eg}
    萘$\to $ 蒽$\to $ 四并苯$\to $
\end{eg}
\begin{notation}
    超共轭效应:$\sigma$ 电子不受屏蔽效应的离域作用,类似共轭效应

    位阻效应:空间阻碍使得共轭体系破坏,$\lambda_\text{max}$ 蓝移,$\varepsilon_\text{max}$ 减小
\end{notation}
\begin{notation}
    跨环效应:一些$\beta,\gamma$中由于空间结构可以发生$\pi \to \pi ^\star $ ,使得不在同一个平面的两个不饱和键可以发生共轭,强吸收
\end{notation}

