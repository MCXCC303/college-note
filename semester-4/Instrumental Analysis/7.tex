\lecture{7}{03.18}
\begin{notation}
    外转换是磷光和荧光的竞争过程,称为荧光的淬灭
\end{notation}
荧光得分点:
\begin{itemize}
    \item 第一激发态
    \item 荧光波长较激发波长更长
    \item 荧光的波长单一性(带状光谱)
\end{itemize}
磷光得分点:
\begin{itemize}
    \item 先由基态跃迁至$S_1$ ,经体系间跨越到达$T_1$
    \item 第一激发三重态的最低振动能级$\to $ 基态:发生磷光
    \item 磷光寿命更长
\end{itemize}
\begin{notation}
    长余晖材料:磷光材料
\end{notation}
\subsection{荧光的激发光谱和发射光谱}%
\label{sub:荧光的激发光谱和发射光谱}
荧光的激发光谱需要\textbf{固定激发波长},能够激发最多分子的波长称为\textbf{最佳激发波长},使用$\lambda_\text{ex}-F$ 来绘制图像,$F$ 为荧光强度。(图\ref{fig:线粒体红色荧光探针图像示例})
\begin{figure}[ht!]
    \centering
    \incfig[0.5]{线粒体红色荧光探针图像示例}
    \caption{线粒体红色荧光探针图像示例}
    \label{fig:线粒体红色荧光探针图像示例}
\end{figure}
注意到$A$ 和$F$ 之间有一个位移,称为Strokes位移,是激发光谱和发射光谱之间的波长差值,产生原因是\textbf{振动弛豫和内转换消耗了发射光谱的能量}

在高分辨率的荧光光谱(图\ref{fig:高分辨率荧光光谱示例})可以看到每一个能级都有对应的吸收峰和发射峰
\begin{figure}[ht!]
    \centering
    \incfig[0.5]{高分辨率荧光光谱示例}
    \caption{高分辨率荧光光谱示例}
    \label{fig:高分辨率荧光光谱示例}
\end{figure}
\subsection{荧光效率}%
\label{sub:荧光效率}

