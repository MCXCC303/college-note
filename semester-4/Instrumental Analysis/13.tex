\lecture{13}{04.15}
\subsection{原子吸收分光光度计光源}%
\label{sub:原子吸收分光光度计光源}
\begin{notation}
    作用:提供待测元素的特征光谱
\end{notation}
满足以下需求:
\begin{itemize}
    \item 发射锐线
    \item 发射待测元素的共振线
    \item 辐射光强度大、稳定性好
    \item 寿命长
\end{itemize}
\begin{eg}
    空心阴极灯:由阳极钨棒和空心管圆筒形阴极构成,内部充入惰性气体。
\end{eg}
空心阴极灯原理:
\begin{enumerate}
    \item 当施加电压后,空心阴极内壁的电子将流向阳极,与充入的惰性气体碰撞电离,产生正电荷
    \item 在电场作用下,正电荷向阴极内壁轰击,使阴极内壁的金属原子溅射
    \item 溅射出来的金属原子再与惰性气体、电子、离子碰撞而激发
    \item 激发原子返回基态产生特定波长的光
\end{enumerate}
\subsection{原子化器}%
\label{sub:原子化器}
将试样中的待测原子转为原子蒸汽,分为:
\begin{itemize}
    \item 火焰原子化器
    \item 石墨炉原子化器
    \item 其他原子化器
\end{itemize}
\begin{notation}
    火焰原子化器:由三部分组成
    \begin{enumerate}
        \item 雾化器:吸入试样溶液并雾化
        \item 雾化室:除去大雾滴,使燃气与助燃气体充分混合
        \item 燃烧器:使上述物质燃烧产生火焰,使试样转为基态原子
    \end{enumerate}
\end{notation}
火焰的选择:
\begin{itemize}
    \item 尽量低温
    \item 常用空气-乙炔混合火焰(最高温度2600K),可以测定35种元素
\end{itemize}
见书$\text{P}_{226}$ 表14-2(完整火焰温度表)
\begin{notation}
    石墨炉原子化器:通过电磁高温使原子转为基态原子
\end{notation}
优点:温度高、可控、原子化效率高、绝对灵敏度高

缺点:精密度差、记忆效应严重(上一次测定没洗干净),测定速度慢
\begin{notation}
    石墨炉原子化器适用于:难挥发物质
\end{notation}
\subsection{灵敏度、特征浓度、检出限}%
\label{sub:灵敏度、特征浓度、检出限}
\begin{notation}
    灵敏度:
    \[
        S = \frac{\mathrm{d}A}{\mathrm{d}c} \quad S = \frac{\mathrm{d}A}{\mathrm{d}m}
    .\]
    特征浓度:
    \[
        C_0 = \frac{C_{X}\times 0.0044}{A} \left( \mu \text{g}\cdot \ce{mL}^{-1}/1\% \right)
    .\]
\end{notation}
