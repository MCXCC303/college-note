\lecture{14}{04.18}
\begin{notation}
在石墨炉原子化中,使用特征质量表示:
\[
    m_0 = m_{x} \times \frac{0.0044}{A} = C_{x} \times V_{x} \times \frac{0.0044}{A}
.\]
\end{notation}
\textbf{特征浓度或特征质量越小,方法越灵敏}
\begin{notation}
    检出限(LOD):在给定的条件下和某一置信度下,可被检测出的最小浓度或最小质量
    \begin{align*}
        D_{c} &= \frac{c_{x}}{A}\cdot 3\sigma\\
        D_{m} &= \frac{m_{x}}{A}\cdot 3\sigma = \frac{c_{x}v_{x}}{A}\cdot 3\sigma
    .\end{align*}
\end{notation}
\section{核磁共振波谱法}%
\label{sec:核磁共振波谱法}
\begin{defi}
    核磁共振:在外磁场的作用下,具有磁矩的原子核存在不同能级,当用一定频率的射频照射分子时,会引起原子核自旋能级的跃迁,产生核磁共振
\end{defi}
核磁共振波谱提供化学位移、耦合常数、信号强度比等数据
\begin{notation}
    $\ce{^{13}C}$ 的天然丰度较低,样本中的C中$\ce{^{13}C}$ 含量不到1\%
\end{notation}
\begin{notation}
    $\ce{^{13}C}$-NMR用于判断分子中的碳原子数、基团和不同种类的碳原子,常常与$\ce{^{1}H}$-NMR配合使用
\end{notation}
\subsection{核磁共振波谱法基本原理}%
\label{sub:核磁共振波谱法基本原理}
1. 自旋:
\begin{notation}
    质量数和电荷数都是偶数的和为偶-偶核,\textbf{自旋量子数为0},因此在磁场中核磁矩为0, 不产生NMR信号
\end{notation}
\begin{eg}
    $\ce{^{12}_{6}C}$,$\ce{^{16}_{8}O}$
\end{eg}
\begin{notation}
    偶-奇核:质量数为偶数,电荷数为奇数;$I=1,2,3\ldots $ ,可以将原子核的核电分布看作一个椭圆体,有自旋现象
\end{notation}
\begin{notation}
    奇-奇核:质量数为奇数,电荷数可以为奇数,也可以为偶数
    \[
        I = \frac{1}{2}, \frac{3}{2}, \frac{5}{2}\ldots \text{(半整数)}
    .\]
    一般研究$I=\frac{1}{2}$ ,如$\ce{^{^1}H}$ 、$\ce{^{13}C}$ 等
\end{notation}
\begin{notation}
    核磁矩$\mu$ :原子核有自旋现象,因此有自旋角动量$P$ :
    \[
        P = \frac{h}{2\pi }\sqrt{I\left( I+1 \right)}
    .\]
    其中$h$ 为普朗克常数,因此自旋量子数$I\neq 0$ 的都有自旋角动量

    核磁矩的方向服从\textbf{右手法则}:
\begin{figure}[ht!]
    \centering
    \incfig[0.5]{核磁矩右手法则}
    \caption{核磁矩右手法则}
    \label{fig:核磁矩右手法则}
\end{figure}
\end{notation}
\begin{notation}
如果将原子核置于磁场中,核磁矩可以有不同的排列,共$2I+1$ 个取向;以磁量子数$m$ 来表示每一种取向,则:\[
    m = I, I-1, I-2, \ldots, -I+1, -I
.\]
\end{notation}
\begin{eg}
    当$I = \frac{1}{2}$ 时:共有2个取向, 即:$m = I, I-1 = -\frac{1}{2},\frac{1}{2}$,如图:
\begin{figure}[ht!]
    \centering
    \incfig[0.5]{氢原子核磁矩取向}
    \caption{氢原子核磁矩取向}
    \label{fig:氢原子核磁矩取向}
\end{figure}
\end{eg}
\begin{notation}
    核磁矩在磁场$Z$ 轴方向的分量取决于角动量在$Z$ 轴上的分量:\[
        P_Z = \frac{h}{2\pi m}\quad \mu = \gamma P
    .\]
    带入得到:\[
        \mu_Z = \frac{h\gamma}{2\pi m}
    .\]
    核磁矩的能量:\[
        E = -\mu_Z H_0 = -m\gamma \frac{h}{2\pi }H_0
    .\]
    其中$m$ 为取向,不同取向的核能级不同,取向越低,能量越高,带入常用的$m = \frac{1}{2}, -\frac{1}{2}$:
    \begin{align*}
        E_{-\frac{1}{2}} &= -\left( -\frac{1}{2} \right)\gamma \frac{h}{2\pi }H_0\\
    E_{\frac{1}{2}} &= -\left( \frac{1}{2} \right)\gamma \frac{h}{2\pi }
    .\end{align*}
    计算二者的能级差:\[
        \Delta E = E_2-E_1 = \frac{h\gamma}{2\pi}H_0
    .\]
    即二者的能级差随$H_0$ 增大而增大,这个现象称为\textbf{能级分裂}
\begin{figure}[ht!]
    \centering
    \incfig[0.5]{能级分裂示意图}
    \caption{能级分裂示意图}
    \label{fig:能级分裂示意图}
\end{figure}
\end{notation}
\begin{notation}
    进动频率$\nu$ :和磁场$H_0$ 、磁旋比$\gamma$ 相关,使用Larmor方程:
    \[
        \nu = \frac{\gamma}{2\pi } H_0
    .\]
    共振吸收:$\nu_0 = \frac{\gamma}{2\pi }H_0=\Delta E$ ,需要满足$\nu_0=\nu$ 才能实现共振吸收
    \begin{eg}
        对$\ce{^{1}}H$ 在$H_0=1.4092T$ 的磁场中,进动频率为60MHz,因此吸收$\nu_0=60\text{MHz}$
    \end{eg}
\end{notation}
\begin{notation}
    量子力学\textbf{选律}决定只有$\Delta m = \pm 1$ 的能级跃迁才能发生。
\end{notation}
\begin{notation}
    自旋弛豫:基态核数$n_{-}$ 与激发态核数$n_{+}$ 的比例服从Boltzmann分布:\[
    \frac{n_{-}}{n_{+}} = \mathrm{e}^{-\frac{\Delta E}{kT}} = \mathrm{e}^{-\frac{rhH_0}{2\pi kT}}
    .\]
    带入$m = \frac{1}{2}, -\frac{1}{2}$ 计算得到:\[
        \frac{n_{-\frac{1}{2}}}{n_{\frac{1}{2}}} \approx 0.9999
    .\]
\end{notation}
\begin{notation}
    在理论上,在不断施加磁场时,不断产生高能级,没有低能级原子后不再产生信号,称为\textbf{饱和};在实际上由于弛豫,会不断产生新的低能级原子,共振信号不会消失
\end{notation}
