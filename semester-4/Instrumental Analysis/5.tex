\lecture{5}{03.11}
\subsubsection*{光源}%
\label{subsub*:光源}
常用氢灯、氚灯、钨灯,可以发出连续且较强且稳定的光
\subsubsection*{单色器}%
\label{subsub*:单色器}
使用棱镜或光栅分光,将符合光分解为单色光
\subsubsection*{吸收池}%
\label{subsub*:吸收池}
可以使用玻璃、石英、高分子材料制作比色皿,用于盛放溶液
\subsubsection*{检测器}%
\label{subsub*:检测器}
用于检测光的强度并转为电信号
\subsubsection*{信号处理和显示系统}%
\label{subsub*:信号处理和显示系统}
具有放大功能
\begin{figure}[ht!]
    \centering
    \incfig[0.5]{紫外可见分光光度计示意图}
    \caption{紫外可见分光光度计示意图}
    \label{fig:紫外可见分光光度计示意图}
\end{figure}
\subsection{仪器类型}%
\label{sub:仪器类型}
\subsubsection*{单波长单光束分光光度计}%
\label{subsub*:单波长单光束分光光度计}
只使用单光源、一种单色器;优点是简单、廉价;缺点是不能进行全波段扫描,且对单光源和检测器的稳定性有很高要求
\subsubsection*{单波长双光束分光光度计}%
\label{subsub*:单波长双光束分光光度计}
科研最常用,将单色光在第二部分为两束,一份透过参比池,一份经过吸收池,可以通过对比两个池的光强来相互补充;可以进行全波段扫描,自动记录,可以消除一些不稳定性;缺点是造价较高
\subsubsection*{双波长分光光度计}%
\label{subsub*:双波长分光光度计}
使用两个单色器得到两种不同波长,交替通过样品池
\subsection{分光光度计校正}%
\label{sub:分光光度计校正}
1. 波长校正:测定苯蒸汽的特征吸收峰

2. 交替测定:吸收池内交互加入参比溶液和测定溶液,吸光度差1\%
\subsection{有机化合物的紫外吸收光谱}%
\label{sub:有机化合物的紫外吸收光谱}
\subsubsection*{含孤立助色团和生色团的有机化合物}%
\label{subsub*:含孤立助色团和生色团的有机化合物}
饱和碳氢键的氢被杂原子取代,具有$n$ 电子,发生$n\to \sigma^\star $ 的跃迁,吸收波长较长。孤立双键有$\pi \to \pi ^\star $ 的跃迁,吸收峰一般在150-180nm之间,\textbf{其中醛酮有三个吸收峰}($\sigma$ 键、双键、杂原子),一共可以发生:$n\to \pi ^\star $ 、$n\to \sigma^\star $ 、$\pi \to \pi ^\star $ 、$\sigma\to \sigma^\star $ 四种跃迁,但是由于$\sigma\to \sigma^\star $ 能量太大,在紫外可见区,因此一共三个吸收峰
\subsubsection*{共轭烯烃}%
\label{subsub*:共轭烯烃}
同一个分子中有两个双键,由两个亚甲基隔开,因此吸收峰位置一致,能量翻倍;如果生成大$\pi $ 键,则$\pi $ 和$\pi ^\star $ 能级缩小,发生红移;随着共轭体系继续变大,继续红移,化合物逐渐变为有色
\subsubsection*{芳香族}%
\label{subsub*:芳香族}
\subsection{有机物化合物结构分析}%
\label{sub:有机物化合物结构分析}
\begin{enumerate}
    \item 220-800nm 无吸收:脂肪族饱和、胺、无杂原子
    \item 210-250nm 有吸收:可能含有两个共轭单元
    \item 260-300nm 强吸收:含有较多共轭单元
    \item 250-300nm 弱吸收:有羰基
    \item 250-300nm 中等吸收,有振动结构:苯环
\end{enumerate}
\subsubsection*{异构体推定}%
\label{subsub*:异构体推定}
分为结构异构和顺反异构
\subsubsection*{化合物骨架推定}%
\label{subsub*:化合物骨架推定}
\begin{eg}
    维生素B1
\end{eg}
\subsection{定性分析方法}%
\label{sub:定性分析方法}
对比两个物质对光的选择性吸收
\begin{eg}
    在一定条件下:$A\propto c$
\end{eg}
\subsubsection*{定性分析}%
\label{subsub*:定性分析}
常用的定性依据:
\begin{itemize}
    \item $E_{\max }$ :化合物的特性参数
    \item 有机化合物的紫外可见吸收光谱:反映生色团和助色团的特性
    \item 标准谱图库:收集了约46000种化合物的标准谱图
    \item $\lambda_{\max }$ :同上
    
\end{itemize}
\subsubsection*{用于杂质检查}%
\label{subsub*:用于杂质检查}
1. 杂质检查:对比纯物质和杂质物质的吸收图

2. 杂质限量检查:
\begin{eg}
    肾上腺素和肾上腺酮的对比:肾上腺酮多一个杂原子,吸光度较高,规定310nm 处的吸光度$A\le 0.05$
\end{eg}
\subsubsection*{目视比色法}%
\label{subsub*:目视比色法}
配置标准溶液系列然后直接对比颜色,方法简便但准确度低
\subsubsection*{吸光系数法}%
\label{subsub*:吸光系数法}
\[
    A = \varepsilon bc \implies c = \frac{A}{\varepsilon b}
.\]
对照法:\[
    \begin{cases}
        A_1 = \varepsilon bc_1\\
        A_2 = \varepsilon bc_2
    \end{cases} \Rightarrow \frac{A_1}{A_2} = \frac{c_1}{c_2}
.\]
\subsubsection*{同时测定多组分}%
\label{subsub*:同时测定多组分}
\begin{figure}[ht!]
    \centering
    \incfig[0.5]{混合组分的三种情况}
    \caption{混合组分的三种情况}
    \label{fig:混合组分的三种情况}
\end{figure}
