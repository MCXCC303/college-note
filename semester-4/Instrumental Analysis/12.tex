\lecture{12}{04.08}
\section{原子吸收分光光度法}%
\label{sec:原子吸收分光光度法}
基于气态的基态原子外层电子对特定谱线的吸收来测定被测元素含量的分析方法
\begin{notation}
定量方法:朗伯比尔定律
\end{notation}
\subsection{原子光谱谱线变宽的因素}%
\label{sub:原子光谱谱线变宽的因素}
\begin{itemize}
    \item 自然宽度$\Delta \nu_\text{N}$
    \item 多普勒变宽$\Delta \nu_\text{D}$
    \item 压力变宽:
        \begin{itemize}
            \item 劳仑兹变宽$\Delta \nu_\text{L}$:待测原子和非同种粒子碰撞
            \item 赫鲁兹马克变宽$\Delta \nu_\text{R}$:待测原子和同种粒子碰撞
        \end{itemize}
    \item 自吸变宽:光源空心阴极灯发射的共振线被灯内同种基态原子吸收称为自吸现象
    \item 场致变宽:电场和磁场的印象
\end{itemize}
劳仑兹变宽:
\[
    \Delta \nu_\text{L} = 2N_\text{A}\sigma^2 p\sqrt{\frac{2}{\pi RT}\left( \frac{1}{A}+\frac{1}{M} \right)}
.\]
\subsection{原子吸收与原子浓度的关系}%
\label{sub:原子吸收与原子浓度的关系}
\begin{notation}
    积分吸收:钨丝灯光源和氘灯经分光后光谱通带为0.2 nm,但是原子吸收线半宽度仅为$10^{-3}\text{nm}$ ,仅为总入射光的$0.5\%$ 
    \[
        \int K_{p}\mathrm{d}v = KV_{\theta}
    .\]
\end{notation}

