\lecture{19}{05.13}
\begin{notation}
    逆狄尔斯-阿尔德重排:不饱和环的开裂
    \begin{eg}
        环己烯开裂后生成两种稳定离子:丁二烯正离子(更稳定)和乙烯正离子
    \end{eg}
\end{notation}
\subsection{质谱分析法}%
\label{sub:质谱分析法}
\subsection{分子式测定}%
\label{sub:分子式测定}
\begin{notation}
    识别分子离子峰:一般位于质谱最右侧,即:\[
        \ce{M - e- \to M^{+.}}
    .\]
    但不一定是(所有的分子离子都被击碎)

    如何鉴别是不是:首先找质谱中质量数最大的峰,且应该有合理的质量丢失(1-3个H原子,1个碳原子),即\textbf{比分子离子峰小4-14和20-25质量单位处不应该有分子离子峰}

    综上需要考虑以下几点:
    \begin{itemize}
        \item 比分子离子峰小4-14和20-25质量单位处不应该有分子离子峰
        \item 必须符合氮律:如果含偶数个氮或不含氮,其$m / z$ 必为偶数,否则必为奇数
        \item 有机化合物分子离子峰的稳定性:芳香环>共轭多烯>烯>脂环化合物>直链烷烃>醚>酯>胺>酸>醇>高度支链烷烃
    \end{itemize}
\end{notation}
\begin{eg}
    苯胺的分子量:93
    \[
    \chemfig{*6(-=-=-(-NH2)=)}
    .\]
\end{eg}
在测得$M+1$,$M+2$后,可以通过查拜诺表来确定分子式;由于Beynon表中不包含S,Cl,Br等元素,如果存在卤素和硫等杂原子,$M+1$和$M+2$需要去除掉其相对强度后再查询
\begin{itemize}
    \item $M+2$的峰含量在4.4\%以下:不含S,Cl,Br
    \item $\text{$M+2 / M $}\approx 32\%$:含有Cl
    \item $\text{$M+2 / M $}\approx 97\%$ :含有Br
    \item $\text{$M+2 / M $}\approx 4.4\%$ :含有S
\end{itemize}
$M+1$相对强度:
\begin{itemize}
    \item S: 0.8
    \item Cl, Br: 几乎没有
\end{itemize}

