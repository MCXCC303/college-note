\lecture{8}{03.21}
荧光强度:\[
    F=\varphi_\text{f}\times I_\text{a} = \varphi_{\text{f}}\left( I_0-I_\text{t} \right)
.\]
其中$\varphi_\text{f}$ 为荧光效率,$I_\text{a}$ 为吸收光强,$I_0$ 为激发光强,$I_\text{t}$ 为透过光强
\begin{notation}
    当浓度低时,$A\le 0.05$ ,近似处理为:$F=2.3\varphi_\text{f}I_0\varepsilon lc = Kc$,即:\[
        F\propto c\quad \varepsilon lc\le 0.05
    .\]
\end{notation}
\subsection{散射光的影响}%
\label{sub:散射光的影响}
\begin{notation}
    瑞利光:光子和物质发生弹性碰撞,不发生能量交换,$\lambda_\text{入射}=\lambda_\text{散射}$
\end{notation}
\begin{notation}
    拉曼光:非弹性碰撞,$\lambda_\text{入射}\neq \lambda_\text{散射}$
\end{notation}
\begin{figure}[ht!]
    \centering
    \incfig[0.5]{拉曼散射}
    \caption{拉曼散射}
    \label{fig:拉曼散射}
\end{figure}
拉曼光对结果干扰较大,干扰主要来自溶剂,更换溶剂或选择特定的激发波长可以消除拉曼光干扰
\begin{eg}
    硫酸奎宁在320nm和350nm下激发,荧光峰均为448nm;在350nm下水的拉曼散射为400nm,在320nm下为360nm,如果拉曼散射峰不包括在荧光峰内或相差非常大,可以忽略;如果影响比较大会造成荧光峰偏移,需要选择拉曼散射离荧光峰比较远的激发波长
\end{eg}
溶剂的拉曼散射波长可以查表得到
\subsection{定量方法测定荧光}%
\label{sub:定量方法测定荧光}
使用标准曲线法:
\begin{notation}
    配置一系列标准浓度试样测定荧光强度,绘制$F-c$ 标准曲线,通过回归求出在$c_{x}$ 时的荧光强度$F_{x}$
\end{notation}
使用比较法:
\begin{notation}
    在线性范围内,测定标准品和试样的荧光强度比较,\textbf{需要使用同一台仪器}
\end{notation}
\subsubsection*{非荧光物质间接测定}%
\label{subsub*:非荧光物质间接测定}
1. 利用化学反应将非荧光物质转为能用于测定的荧光物质

2. \textbf{荧光猝灭法}
\begin{notation}
    非荧光物质具有可猝灭某个荧光化合物的荧光的作用时,可以通过:\textbf{测定荧光化合物荧光强度的降低来测定非荧光物质的浓度} 
\end{notation}
\begin{eg}
    $\ce{F^-}, \ce{S^{2-}}, \ce{Fe^{3+}}$
\end{eg}
\subsection{荧光分光光度计的结构}%
\label{sub:荧光分光光度计的结构}
光源(通常为汞灯或氙灯)$\to $ 激发单色器(可以选择激发光波长)$\to $ 样品池$\to $ 发射单色器(可以过滤发射光波长)$\to $ 检测器、记录器
\begin{figure}[ht!]
    \centering
    \incfig[0.5]{荧光分光光度计图示}
    \caption{荧光分光光度计图示}
    \label{fig:荧光分光光度计图示}
\end{figure}
\begin{notation}
    提高激发光强可以增大荧光强度,从而\textbf{提高测定灵敏度}
\end{notation}
\subsection{时间分辨荧光}%
\label{sub:时间分辨荧光}
镧系元素等的荧光时间较短,可以结合一个有机复合物将能量传递给这些元素
\subsection{本章重点}%
\label{sub:本章重点}
1. 图\ref{fig:振动弛豫、内转换和荧光}

2. 荧光量子效率

3. 与分子结构的效应(什么分子的荧光强度大:共轭体系、给电子)‘

4. 环境影响:溶剂极性、pH、温度

5. 拉曼散射
