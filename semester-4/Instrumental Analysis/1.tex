\lecture{1}{02.25}
\section{课程内容}%
\label{sec:课程内容}
课程目标:
\begin{itemize}
    \item 方法的基本原理
    \item 仪器的结构、工作原理、功能
    \item 分析步骤
    \item 分析实践
    
\end{itemize}
考试为闭卷考试,成绩组成:
\begin{itemize}
    \item 平时成绩:30\%
        \begin{itemize}
            \item 作业(根据交作业的次数)
        \end{itemize}
    \item 考试成绩:70\%
\end{itemize}
推荐教材:《分析化学》第九版
\subsection{课程概况}%
\label{sub:课程概况}
仪器分析是用特定的科学仪器来测量物质的物理或化学性质,从而对样品进行定性、定量分析的学科,是一门应用广泛的学科。
\begin{notation}
化学分析:利用特定的化学反应及计量关系对物质分析
\end{notation}
仪器分析:测量物质的某些物理或物理化学性质的参数及其变化获取物质的化学组成、含量和结构等。
\[
    \text{分析化学}\begin{cases}
        \text{化学分析}\begin{cases}
            \text{重量分析}\\
            \text{滴定分析}\begin{cases}
                \text{酸碱}\\
                \text{配位}\\
                \text{氧还}\\
                \text{沉淀}
            \end{cases}
        \end{cases}\\
        \text{仪器分析}\begin{cases}
            \text{电化学}\\
            \text{光谱/波谱分析}\\
            \text{色谱}\\
            \text{质谱}
        \end{cases}
    \end{cases}
.\]
光谱分析法的内容:
\begin{itemize}
    \item 原子光谱
        \begin{itemize}
            \item 发射
            \item 吸收
            \item 荧光
            
        \end{itemize}
    \item 分子光谱
        \begin{itemize}
            \item 紫外-可见吸收
            \item 红外吸收
            \item 分子荧光与磷光
            \item 化学发光
            \item 拉曼
            
        \end{itemize}
    \item 其他
        \begin{itemize}
            \item 核磁共振、顺磁共振
            \item X-射线电子能谱、俄歇电子能谱
            \item X-射线荧光、X-射线衍射
            
        \end{itemize}
    
\end{itemize}
仪器分析的特点:灵敏度高、检出限低、重现性好、样品用量少
