\lecture{2}{02.24}
\begin{notation}
    This lecture's goal: inpromptu speaking or suddenly speaks
\end{notation}
\begin{defi}
    Impromptu speaking: when you deliver a speech without any prior preparation. This word means without preparation or organization
\end{defi}
Sometimes you will be called to make a speech suddenly. The bad situation like:
\begin{itemize}
    \item Unprepared, out of control
    \item Unknown of the topic
    \item nervousness
    \item Challenging fr introverts
\end{itemize}
Good situation are:
\begin{itemize}
    \item Skill to communicate
    \item Confidence
    \item People's admire
    \item Share
\end{itemize}
\begin{eg}
    Tell me something about your name.
\end{eg}
the skill of this language skills:
\begin{itemize}
    \item Vivid language: Paints a picture
    \item Emotive language
    \item Inclusive language: Include the audience, friendly and welcoming.
\end{itemize}
Every good speech has a structure. one of the examples like:
\begin{enumerate}
    \item point $\to $ story $\to $ point
    \item intro $\to $ point 1 $\to $ point 2 $\to \ldots $ conclusion
    \item feeling $\to $ anecdote $\to $ tieback
    
\end{enumerate}
Pauses are meaningful. Give your audience to relax, think and breathe.

When you slow down:
\begin{itemize}
    \item language much clearer
    \item stronger
    \item audience pay more attention
\end{itemize}
\begin{eg}
    My favourite season
\end{eg}
\begin{notation}
    In nature nothing exists alone. --Rachel Carson
\end{notation}
