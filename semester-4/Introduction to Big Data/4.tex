\lecture{4}{03.21}
\begin{notation}
    数据脱敏的方法:\textbf{数据替换、无效化、随机化、偏移和取整、对称加密、平均值}
\end{notation}
\subsection{常用数据库}%
\label{sub:常用数据库}
NewSQL、NoSQL:速度快,扩展性好,数据模型多,云计算集成

HDFS、GFS:分布式,规模大
\section{数据挖掘和机器学习}%
\label{sec:数据挖掘和机器学习}
包括:\textbf{聚类、分类、神经网络等方法}
\subsection{分类算法}%
\label{sub:分类算法}
使用已有的样本预测新样本的所属类别
\begin{eg}
    邮件分类为垃圾邮件;患者诊断为患病
\end{eg}
常用算法:朴素贝叶斯、逻辑回归、\textbf{决策树}、随机森林、支持向量机;使用有监督学习(标记样本)
\subsubsection*{决策树}%
\label{subsub*:决策树}
\begin{figure}[ht!]
    \centering
    \incfig[0.5]{决策树示例}
    \caption{决策树示例}
    \label{fig:决策树示例}
\end{figure}
\subsection{聚类算法}%
\label{sub:聚类算法}
无监督学习(不标记样本),有\textbf{划分聚类、层次聚类、基于密度聚类、基于网格聚类};要求根据k个数据划分,每个簇至少有一个数据,每个数据有且仅属于一个簇
\subsubsection*{\textit{K}-均值聚类}%
\label{subsub*:K-均值聚类}
首先随机从数据点中取k个数据作为初始中心,确定分类为k组数据

\subsubsection*{回归分析}%
\label{subsub*:回归分析}
\[
    \bm{w} = \left( \bm{X}^\top \bm{X} \right)^{-1}\bm{X}^\top \bm{y}
.\]
\subsubsection*{关联规则}%
\label{subsub*:关联规则}
\subsubsection*{人工神经网络}%
\label{subsub*:人工神经网络}
简称神经网络(Neural Network),有多种分支,如卷积神经网络CNN

一个神经元至少包含三个部分:输入、计算、输出。输入到神经元的数据首先会通过权重调整,然后传入神经元;传出时需要经过非线性函数对输出归一化
