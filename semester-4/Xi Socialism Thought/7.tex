\lecture{7}{04.25}
{\centering{\section*{考试重点}%
\label{sec*:考试重点}
}}
\section*{导论}%
\label{sec*:导论}
1. 习近平新时代中国特色社会主义思想(习思想)创建的时代背景

2. 习思想是“两个结合”的重大成果

3. 习思想的历史地位

4. 深刻理会“两个全面”的决定性意义
\subsection*{新时代坚持发展中国特色社会主义}%
\label{sub*:新时代坚持发展中国特色社会主义}
1. 中国特色社会主义是社会主义而不是什么其他主义

2. 坚定“四大自信”

3. 中国特色社会主义新时代是我国发展新的历史方位

4. 社会主要矛盾变化是关系全体的历史性变化

5. 统筹推进“五位一体”和协调推进“四个全面”战略布局
\section*{以中国式现代化全面推进中华民族伟大复兴}%
\label{sec*:以中国式现代化全面推进中华民族伟大复兴}
1. 中华民族伟大复兴的中国梦

2. 中国式现代化是中国共产党领导人民长期探索的重大成果

3. 中国式现代化的中国特色

4. 中国式现代化的本质要求

5. 中国式现代化创造人类文明新形态

6. 推进中国式现代化需要牢牢把握的原则

7. 推进中国式现代化需要正确处理的重大关系
\section*{坚持党的全面领导}%
\label{sec*:坚持党的全面领导}
1. 中国最大的国情

2. 中国共产党领导是中国制度的最大优势

3. 党的领导制度是我们的根本领导制度

4. 建立党的全面领导

5. 中国共产党领导是最高政治领导力量

6. 党的领导制度是全面的、系统的、整体的
\section*{坚持以人民为中心}%
\label{sec*:坚持以人民为中心}
1. 人民立场是中国共产党的根本政治立场

2. 人民对美好生活的向往是党的奋斗目标

3. 推动全体人民共同富裕
\section*{全面深化改革}%
\label{sec*:全面深化改革}
1. 改革开放是我们前进的重要法宝

2. 新时代全面深化改革开放是一场深刻革命

3. 坚持全面深化改革总目标

4. 坚持正确方法论
\section*{推动高质量发展}%
\label{sec*:推动高质量发展}
1. 发展新发展理念是关系我国发展全局的一场深刻革命

2. 坚持两个毫不动摇

3. 大力推动构建新发展格局

4. 全面乡村振兴
\section*{社会主义现代化建设}%
\label{sec*:社会主义现代化建设}
1. 深入实施三大战略

2. 科技强则国家强
\section*{发展全过程人民民主}%
\label{sec*:发展全过程人民民主}
1. 人民民主是社会主义的生命

2. 坚定不移走中国特色社会主义政治发展道路

3. 全过程人民民主是社会主义民主政治的伟大创造

4. 全过程人民民主是全覆盖的民主
\section*{建设社会主义文化强国}%
\label{sec*:建设社会主义文化强国}
1. 文化繁荣兴盛是中华民族伟大复兴的要求

2. 文化发展道路

3. 积极塑造主流舆论新格局

4. 广泛执行社会主义核心价值观
\section*{保障民生}%
\label{sec*:保障民生}
1. 坚持在发展中增进民生福祉

2. 完善分配制度

3. 完善社会治理体系
\section*{建设社会主义生态文明}%
\label{sec*:建设社会主义生态文明}
1. 绿水青山就是金山银山

2. 绿色生产和生活方式

3. 积极推动全球可持续发展
\section*{全面从严治党}%
\label{sec*:全面从严治党}
1. 坚定不移全面从严治党

2. 把党的政治建设摆在首位

3. 党的自我革命是跳出历史周期律的第二个答案
\section*{其他}%
\label{sec*:其他}
可能会考察\textbf{二十届三中全会}
\section*{题目格式}%
\label{sec*:题目格式}
问答、选择、判断
