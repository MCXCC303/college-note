\lecture{5}{03.10}
\begin{question}
    如何推动中国式现代化行稳致远?
\end{question}
坚持五大原则,处理好六大关系
\begin{notation}
    十个明确主体内容:
    \begin{itemize}
        \item 坚持和发展中国特色社会主义的总任务
        \item 新时代我国社会主要矛盾
        \item 中国特色社会主义事业总体布局、战略布局
        \item 全面深化改革总目标
        \item 全面推进依法治国总目标
        \item 党在新时代的强军目标
        \item 中国特色大国外交的目标
        \item 中国特色社会主义最本质的特征
        \item 坚持和发展中国特色社会主义的总要求
        \item 中国特色社会主义制度的最大优势
    \end{itemize}
    十四个坚持:
    \begin{itemize}
        \item 党对一切工作的领导
        \item 以人民为中心
        \item 全面深化改革
        \item 新发展理念
        \item 人民当家作主
        \item 全面依法治国
        \item 社会主义核心价值体系
        \item 在发展中保障和改善民生
        \item 人与自然和谐共生
        \item 总体国家安全观
        \item 党对人民军队的绝对领导
        \item “一国两制”和推进祖国统一
        \item 推动构建人类命运共同体
        \item 全面从严治党
    \end{itemize}
\end{notation}
\section{党的领导}%
\label{sec:党的领导}
党的领导包含:\textbf{政治领导、思想领导、组织领导}
