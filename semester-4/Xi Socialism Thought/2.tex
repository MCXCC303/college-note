\lecture{2}{02.24}
科学体系共四个方面:
\begin{itemize}
    \item 十个明确
    \item 十四个坚持
    \item 十三方面伟大成就
    \item 六个必须
    
\end{itemize}

十个明确:
\begin{itemize}
    \item 坚持和发展中国特色社会主义的总任务
    \item 新时代我国社会主要矛盾
    \item 中国特色社会主义事业总体布局、战略布局
    \item 全面深化改革总目标
    \item 全面推进依法治国总目标
    \item 党在新时代的强军目标
    \item 中国特色大国外交要推动构建新型国际关系,推动构建人类命运共同体
    \item 中国特色社会主义最本质的特征和中国特色社会主义制度的最大优势是中国共产党领导
    \item 新时代党的建设总要求
    \item 全面从严治党的战略方针
    
\end{itemize}

十四个坚持:
\begin{itemize}
    \item 党对一切工作的领导
    \item 以人民为中心
    \item 全面深化改革
    \item 新发展理念
    \item 人民当家作主
    \item 全面依法治国
    \item 社会主义核心价值体系
    \item 在发展中保障和改善民生
    \item 人与自然和谐共生
    \item 总体国家安全观
    \item 党对人民军队的绝对领导
    \item “一国两制”和推进祖国统一
    \item 推动构建人类命运共同体
    \item 全面从严治党
    
\end{itemize}

国家建设的十三个方面:\textbf{经济建设、政治建设、文化建设、社会建设、生态文明建设、全面深化改革、全面依法治国、全面从严治党、国防和军队建设、维护国家安全、外交工作、坚持“一国两制”和推进祖国统一、党的建设新的伟大工程}

六个必须坚持:
\begin{itemize}
    \item 人民至上
    \item 自信自立
    \item 守正创新
    \item 问题导向
    \item 系统观念
    \item 胸怀天下
\end{itemize}
