\lecture{4}{03.07}
中国到2035年的总目标:
\begin{itemize}
    \item 经济和科技大幅提升
    \item 国家治理现代化
    \item 文化和社会建设
    \item 生态文明建设
    \item 民生与共同富裕
    \item 国防安全
    \item 对外开放
\end{itemize}
\begin{notation}
    毛泽东同志指出:“扬弃”思想:学习合理的、先进的事物
\end{notation}
小康的定义随着时代的变化而不断变化:一开始是500元,后来是1000元,再后来提出了全民小康水平,包括城乡收入,人均住房,教育水平等方面。
\begin{notation}
    孙中山先生提出《建国方略》,构想了一幅宏伟的建设蓝图;如今虽然这个设想早已实现,但是对于中国的现代化建设仍然没有停止。历史证明:没有代表中国人民根本利益的先进政治力量领导,实现中国式现代化的蓝图都无法实现
\end{notation}
\subsubsection*{中国式现代化发展的几个时期}%
\label{subsub*:中国式现代化发展的几个时期}
新民主主义革命:革命的胜利为实现现代化创造了根本社会条件
\subsubsection*{中国式现代化本质要求}%
\label{subsub*:中国式现代化本质要求}
\begin{itemize}
    \item 坚持中国共产党领导
    \item 坚持中国特色社会主义
    \item 实现高质量发展
    \item 发展全过程人民民主
    \item 丰富人民精神世界
    \item 实现全体人民共同富裕
    \item 促进人与自然和谐共生
    \item 推动构建人类命运共同体
    \item 创造人类文明新形态
\end{itemize}
