\lecture{3}{03.05}
单糖中,果糖是酮糖,其他为醛糖;糖的结构可以简化,将氢原子省略,羟基保留,末端羟基为圆形,末端为醛基表示为三角形
\begin{figure}[ht!]
    \centering
    \incfig[0.5]{葡萄糖简易表示}
    \caption{葡萄糖简易表示}
    \label{fig:葡萄糖简易表示}
\end{figure}
单糖中,除了二羟基丙糖外全部都具有手性
\begin{defi}
    构型:不对称原子上四个不同原子或基团在空间上的排列
\end{defi}
\begin{notation}
    Fischer投影中横键朝前
\end{notation}
D-葡萄糖为R构型(OH  $\to $ CHO $\to \ce{CH_2OH}$)
常见的单糖:
\begin{itemize}
    \item D-甘油醛
    \item D-葡萄糖
    \item D-半乳糖
    \item D-果糖
    \item \ldots 
\end{itemize}
\begin{notation}
    葡萄糖可能在1-5 C之间可以形成氧桥(羟醛缩合)
\end{notation}
\begin{notation}
    fischer和Haworth的转化:

    1. 向右放倒,折成一个环

    2. 末端羟基旋转至平面上方

    3. 羟基氧在环平面内

    4. 羰基氧转为半缩醛羟基

    5. 确定半缩醛羟基的位置
\end{notation}
单糖可在不同构型之间转化,其中$\alpha$ 和$\beta$ 构型占比较高,并且$\beta$ 构型更稳定;同时还可以形成五元环
\subsection{单糖的化学性质}%
\label{sub:单糖的化学性质}
醛糖具有还原性,使用斐林试剂检测是否含有葡萄糖
\begin{notation}
    斐林试剂:0.1g/mL NaOH, 0.05g/mL $\ce{CuSO_4}$, 0.2g/mL 酒石酸钾
\end{notation}
\begin{notation}
    单糖可以酰化、成苷
\end{notation}
成苷作用:半缩醛或半缩酮的羟基与另一分子的羟基或氨基脱水
\subsection{单糖衍生物}%
\label{sub:单糖衍生物}
1. 糖醇

2. 氨基糖:糖中的羟基被氨基或氨基酸取代,在细胞壁中常见
