\lecture{4}{03.11}
{\centering{
\section*{预习1}%
\label{sec*:预习1}
}}
\section*{氨基酸}%
\label{sec*:氨基酸}
氨基酸是蛋白质的基本组成单位,组成人体蛋白质的氨基酸共20中,\textbf{除了甘氨酸之外均为}L-$\alpha$-氨基酸。$\alpha$ 代表顺序,在氨基酸中氨基和羧基直接连接在同一个C原子上;L代表左旋,由氨基酸的手性决定。右旋由D表示

\begin{figure}[ht!]
    \centering
    \incfig[0.5]{氨基酸的基本结构通式}
    \caption{氨基酸的基本结构通式}
    \label{fig:氨基酸的基本结构通式}
\end{figure}
不同的R基决定了不同的氨基酸,最简单的氨基酸为甘氨酸,R=H;由于甘氨酸没有手性因此不具有空间构型的差异,无旋光性
\begin{notation}
    一碳单位:含有一个碳原子的\textbf{基团}
\end{notation}
\textbf{各种蛋白质的含氮量均大约16\%}
\begin{eg}
    脯氨酸(脯萄亚):氨基同时连接两个碳原子,形成环状结构,碳原子少一个,因此称为亚氨基酸;

    同理:羟脯氨酸在脯氨酸的亚甲基上加上一个羟基
\end{eg}
\begin{eg}
    半胱氨酸:\textbf{极性最强的氨基酸},即还原性最强,很容易发生反应被氧化,因为其具有\textbf{巯基}。与自己反应可以生成二硫键,两个半胱氨酸结合形成胱氨酸(常见存在形式),但是半胱氨酸也是存在的
\end{eg}
\begin{notation}
    同型半胱氨酸\textbf{并不是半胱氨酸}
\end{notation}
\begin{notation}
    甲硫氨酸又称为\textbf{蛋氨酸},含有硫原子
\end{notation}
\subsection{氨基酸的分类}%
\label{sub:氨基酸的分类}
必须熟练背诵:
\begin{description}
    \item[酸性] 2种:天冬氨酸、谷氨酸(酸性天冬谷,多一个羧基)
    \item [碱性] 3种:赖氨酸、精氨酸、组氨酸(碱性赖精组)
    \item [芳香族] 3种:酪氨酸、色氨酸、苯丙氨酸(老色本,芳香族)
    \item [非极性疏水性] 7种:异亮氨酸、亮氨酸、甲硫氨酸、丙氨酸、甘氨酸、脯氨酸、缬氨酸(一两假饼干腹泻)
    \item [极性中性] 5种:谷氨酰胺、半胱氨酸、天冬酰胺、丝氨酸、苏氨酸(孤半天始苏)
\end{description}
其他分类:
\begin{description}
    \item[必需氨基酸] 9种:缬、异、亮、苯、蛋、色、苏、赖、组(组一两本单色书来写)
    \item [含有硫] 3种:半胱、胱、蛋
    \item [有支链] 3种:缬、异、亮
    \item [可以被磷酸化] 酪、丝、苏
\end{description}
按氨基酸的转化关系:
\begin{description}
    \item[生糖兼生酮] (可以转化为葡萄糖、酮体)异、苯、酪、色、苏(一本老色素)
    \item [一碳单位] 丝、色、组、甘(施舍猪肝)
    \item [生酮] 亮、赖
    \item [不参与转氨基] 脯、羟脯、苏、赖(步枪书来)
    
\end{description}
氨基酸三字母缩写对照:
\begin{description}
    \item [丙氨酸] Ala
    \item [精氨酸] Arg
    \item [天冬酰胺] Asn
    \item [天冬氨酸] Asp
    \item [半胱氨酸] Cys
    \item [谷氨酰胺] Gln
    \item [谷氨酸] Glu
    \item [甘氨酸] Gly
    \item [组氨酸] His
    \item [异亮氨酸] Ile
    \item [亮氨酸] Leu
    \item [赖氨酸] Lys
    \item [甲硫氨酸] Met
    \item [苯丙氨酸] Phe
    \item [脯氨酸] Pro
    \item [丝氨酸] Ser
    \item [苏氨酸] Thr
    \item [色氨酸] Trp
    \item [酪氨酸] Tyr
    \item [缬氨酸] Val
\end{description}
\subsection{其他考点}%
\label{sub:其他考点}
\textbf{容易使肽链形成折角的氨基酸}:脯氨酸($\beta$ 转角结构,180度回转,第二个位置);\textbf{修饰氨基酸}(不能一步到位翻译出来,必须先翻译一种氨基酸,通过其他加工修饰得到):羟脯氨酸、羟赖氨酸、胱氨酸;\textbf{含有共轭双键的氨基酸}:一般是芳香族氨基酸
\subsection{氨基酸理化性质}%
\label{sub:氨基酸理化性质}
\subsubsection*{两性解离}%
\label{subsub*:两性解离}
氨基酸在pH中解离阳离子和阴离子,如果呈电中性,称此时的pH为等电点pI

如果不相等:带负电时为羧酸根解离,带正电为氨基结合氢离子,即:
\begin{enumerate}
    \item 带负电时pH>pI
    \item 带正电时pH<pI
\end{enumerate}
\subsubsection*{紫外吸收峰}%
\label{subsub*:紫外吸收峰}
由于某些氨基酸有共轭双键因此有吸收峰,如酪氨酸和色氨酸最大吸收峰为280nm 左右,而苯丙氨酸的最大吸收峰在260nm
\begin{notation}
    核酸的紫外吸收峰也为260nm左右
\end{notation}
\subsubsection*{茚三酮反应}%
\label{subsub*:茚三酮反应}
氨基酸和茚三酮加热生成的蓝紫色化合物,最大吸收峰为570nm
\section*{糖代谢}%
\label{sec*:糖代谢}

