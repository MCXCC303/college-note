\lecture{2}{02.26}
\begin{notation}
    萌芽时期:Emil Fischer
\end{notation}
\begin{notation}
    分子生物学时期:Watson and Crick, Wilkins, Franklin
\end{notation}
\section{糖}%
\label{sec:糖}
\begin{defi}
    糖:由C,H,O组成,形如$\ce{( CH_2O )_{n}}$,基本结构为多羟基醛/酮
\end{defi}
基本结构单元为单糖,可以使用fischer投影、haworth投影和椅式构象表示
\begin{eg}
D-葡萄糖:$\ce{C_6H_{12}O_6}$
\end{eg}
糖广泛分布于植物中,其次为微生物(约10\%至30\%,细胞壁),动物中静态含糖较少(约2\%),但动态含量较高
\begin{notation}
    最简单的单糖:丙糖(3元环),戊糖是核糖的成分,己糖(6元环)是含量最多的(葡萄糖、果糖、半乳糖\ldots )

    寡糖:分为二糖、三糖(棉籽糖)和其他糖
\end{notation}
\subsection{糖的生物学功能}%
\label{sub:糖的生物学功能}
\begin{itemize}
    \item 最主要的能源物质:通过氧化提供能量
    \item 贮藏能量:糖原、淀粉
    \item 结构组成物质:细胞壁(糖蛋白、糖脂)、细胞间质(粘多糖)、植物茎杆(纤维素)
    \item 药物活性成分
\end{itemize}
\subsection{单糖}%
\label{sub:单糖}

