\lecture{8}{04.09}
\begin{notation}
    维生素C:抗坏血酸,每日摄入量约400mg,其中的羟基酶辅助分子可以参与体内多种羟基化反应

    功能:
    \begin{itemize}
        \item 基本作用
            \begin{itemize}
                \item 促进胶原蛋白合成
                \item 参与胆固醇转化,促进维生素D生成(防治坏血病)
                \item 参与芳香族氨基酸的代谢
            \end{itemize}
        \item 还原作用
            \begin{itemize}
                \item 维持巯基酶的巯基处于还原状态
                \item 还原高铁血红蛋白
                \item 保护维生素A, E, B不被氧化,促进叶酸转为活性四氢叶酸
            \end{itemize}
        \item 免疫作用
            \begin{itemize}
                \item 促进淋巴细胞增值趋化作用
                \item 提高吞噬细胞吞噬能力
                \item 促进免疫球蛋白的合成
            \end{itemize}
        
    \end{itemize}
\end{notation}
\section{蛋白质}%
\label{sec:蛋白质}
\begin{defi}
    有许多氨基酸通过肽键连接形成的高分子
\end{defi}
\begin{notation}
    氨基酸的平均分子质量约为110
\end{notation}
\subsection{蛋白质基本单元}%
\label{sub:蛋白质基本单元}
人体内的自然氨基酸单元均为$\alpha$-L-氨基酸
\begin{notation}
    八种必须氨基酸:异亮、甲硫、缬、亮\ldots :一家写两三本书来
\end{notation}
\subsection{氨基酸物理性质}%
\label{sub:氨基酸物理性质}
晶体、熔点较高、一般不溶解于纯水,易溶于碱或酸溶液;具有旋光性
\subsection{氨基酸化学性质}%
\label{sub:氨基酸化学性质}
\begin{notation}
    两性解离、等电点:在某一个pH下,溶解后的蛋白质的氨基正电与羧基负电平衡,该pH为等电点,记为p$I$
\end{notation}
\begin{figure}[ht!]
    \centering
    \incfig[0.5]{甘氨酸滴定示例}
    \caption{甘氨酸滴定示例}
    \label{fig:甘氨酸滴定示例}
\end{figure}
对中性氨基酸:\[
    \text{p}I = \text{p}K_\text{R}
.\]
对酸性氨基酸:\[
    \text{p}I = \frac{\left( \text{p}K_1+\text{p}K_\text{R} \right)}{2} 
.\]
对碱性氨基酸:\[
    \text{p}I = \frac{\left( \text{p}K_2+\text{p}K_\text{R} \right)}{2}
.\]
\begin{notation}
    茚三酮反应:鉴定氨基酸
\end{notation}
\subsection{氨基酸紫外吸收性质}%
\label{sub:氨基酸紫外吸收性质}
色氨酸、酪氨酸、苯丙氨酸在280nm附近有紫外吸收(苯环共轭)
