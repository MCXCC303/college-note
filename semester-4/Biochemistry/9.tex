\lecture{9}{04.16}
\subsection{蛋白质常见化学反应}%
\label{sub:蛋白质常见化学反应}
\begin{notation}
    $\alpha$-氨基反应:
    \begin{itemize}
        \item 与亚硝酸反应
        \item 与酰化试剂反应
        \item 羟基化反应(将氨基中的一个氢原子转为羟基)
        \item 脱氨基反应:变为酮酸
    \end{itemize}
\end{notation}
\begin{notation}
    $\alpha$-羧基的反应:
    \begin{itemize}
        \item 成盐、成酯
        \item 脱羧(氨基酸代谢,生成一级胺)
    \end{itemize}
\end{notation}
\begin{notation}
    $\alpha$-氨基、羧基共同参与:茚三酮反应,氨基酸与茚三酮在酸性溶液中加热,形成\textbf{紫色物质}

    对于脯氨酸和羟基脯氨酸,与茚三酮反应不释放氨,生成\textbf{黄色化合物}

    可以定性定量测定各种氨基酸和蛋白质,通过测定释放的二氧化碳来计算氨基酸的量
\end{notation}
\begin{notation}
    侧链R参与:主要由比较活泼的基团参与,如:羟基、酚羟基、巯基、吲哚基、咪唑基、甲硫基、羧基和氨基
    \begin{eg}
        蛋氨酸侧链的甲硫基可以与羟化试剂形成锍盐
    \end{eg}
    \begin{eg}
        半胱氨酸的巯基可以形成二硫键、具有强还原性,可以稳定蛋白质的空间结构
    \end{eg}
\end{notation}
\subsubsection*{生物活性肽}%
\label{subsub*:生物活性肽}
\begin{notation}
    谷胱甘肽可以与生物毒性物质结合,消除毒性,反应为通过活性氧由GSH转为GSSH 
\end{notation}
\begin{notation}
    多肽类激素和神经肽:
    \begin{eg}
        促甲状腺激素释放激素(TRH):末端为酰胺
    \end{eg}
\end{notation}
\subsection{蛋白质一级结构}%
\label{sub:蛋白质一级结构}
氨基酸的分类:酸性、中性、碱性
\begin{notation}
    作为蛋白质切割的判断
\end{notation}
\subsection{蛋白质构象}%
\label{sub:蛋白质构象}
\begin{defi}
    蛋白质分子中原子和基团在三维空间上的排列、分布和肽链的走向
\end{defi}
维持蛋白质构象的化学键:
\begin{itemize}
    \item 氢键
    \item 疏水键
    \item 离子键
    \item 配位键
    \item 范德华力
\end{itemize}
\subsection{蛋白质三级结构}%
\label{sub:蛋白质三级结构}
\begin{defi}
    结构域:由几个基序结构单元组合形成的具有独特空间构象的独立折叠单元
\end{defi}
结构域之间以共价键连接
\subsection{蛋白质四级结构}%
\label{sub:蛋白质四级结构}
\begin{notation}
    不是每一个蛋白质分子都有四级结构
\end{notation}
\begin{defi}
    亚基:由两条或多条多肽链组成的活性蛋白质中具有完整三级结构的\textbf{独立多肽链}
\end{defi}
\begin{eg}
    胰岛素的三条多肽链不构成亚基:由共价键连接,不是独立的多肽链
\end{eg}
\begin{defi}
    四级结构:以亚基为基础,亚基间通过\textbf{非共价键}相互作用形成更复杂的空间结构
\end{defi}
亚基间通常由疏水作用力连接
\begin{notation}
    寡聚体蛋白:由2-10个亚基组成的蛋白质

    多聚体蛋白:由$>10$ 个亚基组成的蛋白质
\end{notation}
\begin{eg}
    免疫球蛋白(IgG)
\end{eg}
