\lecture{7}{04.02}
\section{脂类}%
\label{sec:脂类}
\section{维生素}%
\label{sec:维生素}
\begin{itemize}
    \item 脂溶性
    \item 水溶性
\end{itemize}
\subsection{维生素定义}%
\label{sub:维生素定义}
\begin{itemize}
    \item \textbf{维持人和动物生理正常功能必须的一类物质}
    \item 体内不能合成或合成量不足,需要食物供给
    \item 低分子量有机化合物
    \item 天然存在于食品中,多以前体形式存在
    \item 不参与构成机体组织或细胞组成,主要参与机体代谢调节
    \item 机体需要量很小,但不可或缺
\end{itemize}
\subsection{维生素命名}%
\label{sub:维生素命名}
有按发现顺序、按化学结构特点等命名方法:
\begin{notation}
    按发现顺序:维生素A, B, C, \ldots 

    按结构特点:视黄醇(维生素A),硫胺素等

    按生理功能特点:抗干眼病维生素(维生素A),抗癞皮病维生素(维生素B),抗坏血酸维生素(维生素C)
\end{notation}
\subsection{维生素的需要量}%
\label{sub:维生素的需要量}
\begin{defi}
    能保持人体健康,达到机体应有发育水平和充分发挥效率的完成各项体力和脑力活动的,人体所需要的维生素的必须量
\end{defi}
\subsection{脂溶性维生素}%
\label{sub:脂溶性维生素}
一般是自身发生作用
\subsubsection{维生素K}%
\label{subsub:维生素K}
维生素K1:绿色蔬菜,称为叶绿甲基萘醌

维生素K2:肠道细菌代谢产物

维生素K3:维持体内凝血因子的正常水平,缺乏易导致皮下出血
\subsection{水溶性维生素}%
\label{sub:水溶性维生素}
一般参与酶的功能作用,含维生素B族和维生素C
\begin{notation}
    脂溶性维生素和水溶性维生素结构差别较大。

    水溶性维生素在体内无储存,随尿液排出
\end{notation}
\subsubsection{维生素B族}%
\label{ssub*:维生素B族}
\begin{notation}
    维生素$\text{B}_1$:硫胺素,在谷类和豆类中含量较高

    参与$\alpha$-酮酸氧化脱羧酶的辅酶代谢糖,参与磷酸戊糖的代谢,参与神经传导并可以抑制胆碱酯酶
\end{notation}
\begin{notation}
    维生素$\text{B}_2$ :核黄素,参与体内氧还反应
\end{notation}
\begin{notation}
    维生素$\text{B}_3$ :烟酸、烟酰胺,又称维生素PP,活化后变为$\text{NAD}^{+}$和$\text{NADP}^+$
\end{notation}
\begin{notation}
    维生素$\text{B}_5$ :泛酸,参与合成辅酶A,缺乏会导致肠胃功能障碍、肢体神经综合征等
\end{notation}
\begin{notation}
    维生素$\text{B}_7$:生物素,分为$\alpha$ 和$\beta$ 两种;鸡蛋清中的抗生物素蛋白可以和生物素结合
\end{notation}
\begin{notation}
    维生素$\text{B}_9$ :叶酸,经代谢分别变为二氢叶酸和四氢叶酸,参与DNA合成、氨基酸代谢,缺乏导致巨幼红细胞贫血
\end{notation}
\begin{notation}
    维生素 $\text{B}_{12}$ :钴胺素
\end{notation}
