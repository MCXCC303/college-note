\lecture{1}{02.19}
\section{概论}%
\label{sec:概论}
生物化学共两部分:生化1(必修)+生化2(选修)
\subsection{药学院研究方向}%
\label{sub:药学院研究方向}
著名杂志:Medicinal Chemistry
\subsection{课程信息}%
\label{sub:课程信息}
使用教材:生物化学第八版,人民卫生出版社

推荐参考教材:生物化学第四版,高等教育出版社
\subsection{课程介绍}%
\label{sub:课程介绍}
\begin{itemize}
    \item 绪论:什么是生化
    \item 生命的分子基础
    \item 生命的物质的代谢和能量转换:生命的维持
    \item 遗传信息:生命的延续
    \item 药学生化
\end{itemize}
\subsection{课程特点与学习方法}%
\label{sub:课程特点与学习方法}
\subsubsection*{课程特点}%
\label{subsub:课程特点}
\begin{itemize}
    \item 知识点多
    \item 章节之间相互独立
    \item 能进行逻辑推理的部分少
\end{itemize}
\subsubsection*{学习方法}%
\label{subsub:学习方法}
\begin{enumerate}
    \item 抓重点(相对重点,所有知识点都很重要)
    \item 理解记忆
\end{enumerate}
\subsection{考核方式}%
\label{sub:考核方式}
课程使用\textbf{闭卷考试}
\begin{table}[htpb]
    \centering
    \caption{成绩组成}
    \label{tab:成绩组成}
    \begin{tabular}{ccc}
    \toprule
    \multirow{2}{*}{考试} & \multicolumn{2}{c}{平时成绩}\\
                          & 出勤 & 作业\\
    \midrule
    60\% & 20\% & 20\%\\
    \bottomrule
    \end{tabular}
\end{table}
\section{生化概论}%
\label{sec:生化概论}
\begin{question}
    生命是如何定义的?
\end{question}
\begin{sol}
    生命由化学定义,生命是一台巧妙的化学机器。生命之所以存在是因为体内的原子不断进行化学反应
\end{sol}
\begin{notation}
    每秒生命体内发生超过5000万亿次化学反应
\end{notation}
\begin{question}
细胞间如何沟通?
\end{question}
\textit{绪论结构:}
\[
    \text{绪论}
    \begin{cases}
        \text{含义}\\
        \text{研究内容}\\
        \text{发展简史}\\
        \ldots 
    \end{cases}
.\]
\subsection{什么是生物化学}%
\label{sub:什么是生物化学}
\begin{defi}
    化学是研究物质组成、化学变化过程及其变化过程中的能量变化的学科,生物化学在化学基础上限制范围在\textbf{生命体的物质组成}和\textbf{生命过程中的化学变化}
\end{defi}
\begin{notation}
    生命体的化学复杂度远高于无机体的化学变化
\end{notation}
从研究尺度分类:物理研究原子,化学研究分子,生物研究细胞

\subsection{生物化学的研究内容}%
\label{sub:生物化学的研究内容}
应用:
\begin{itemize}
    \item 营养学(糖化学、蛋白质)
    \item 药学(生物分子药物)
    \item 生命健康
\end{itemize}
研究内容:
\begin{itemize}
    \item 生物体化学组成
    \item 物质代谢及其调节
    \item 遗传信息传递及其调控
\end{itemize}
\subsection{物质代谢和能量转换}%
\label{sub:物质代谢和能量转换}
食物摄取$\to $ (少量)能量释放$\to $ 消化系统$\to $ 酶催化化学反应$\to $ 能量储存(ATP、脂肪、糖等物质形式)
\begin{notation}
    三大营养物质:\textbf{蛋白质、碳水(糖)、脂肪};蛋白质一部分产生氨气、尿素排除体内,大部分转为氨基酸;脂肪产生脂肪酸,进入三羧酸循环
\end{notation}
\begin{eg}
    由不平衡营养摄入导致的疾病:肥胖症(Obesity);肥胖症可能导致:
    \begin{itemize}
        \item 高血压
        \item 高血脂
        \item 肝脏疾病
        \item 心血管疾病
        \item 心理问题
        \item 癌症
        \item II型糖尿病
        \item \ldots 
    \end{itemize}
\end{eg}
\subsection{遗传信息传递及调控}%
\label{sub:遗传信息传递及调控}
\begin{notation}
    遗传与进化是生命最基本的特征之一
\end{notation}
遗传信息以生物分子为载体,一般生物的载体为DNA,即\textbf{核酸是遗传信息的载体}
\begin{notation}
    \textbf{中心法则}:$\text{DNA}\xrightarrow[]{\text{Transcription}}\text{RNA}\xrightarrow[]{\text{Translation}}\text{Prot}$
\end{notation}
\subsection{发展简史}%
\label{sub:发展简史}
古代:杜康酿酒

近代:屠呦呦发现青蒿素
