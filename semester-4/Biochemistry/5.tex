\lecture{5}{03.12}
\section{多糖化学}%
\label{sec:多糖化学}
主要研究\textbf{活性多糖药物}、\textbf{构效关系}
\begin{notation}
基本单位:单糖
\end{notation}
\subsection{寡糖}%
\label{sub:寡糖}
\begin{defi}
    少数单糖通过糖苷键结合的聚合物,2-20个单糖;2个单糖结合的又称二糖
\end{defi}
\begin{eg}
    由$\alpha$-葡萄糖和$\beta$-葡萄糖通过糖苷键($\alpha$-(1-4)糖苷键)连接后形成麦芽糖
\end{eg}
\begin{eg}
由$\alpha$-葡萄糖和$\beta$-果糖形成$\alpha$-(1-2)-$\beta$糖苷键
\end{eg}
同理:乳糖和海藻糖分别形成$\beta$-(1-4)糖苷键和$\alpha$-(1-1)糖苷键
\subsection{多糖分类}%
\label{sub:多糖分类}
按用途分类:
\begin{notation}
贮存多糖:\textbf{淀粉、糖原}

结构多糖:\textbf{纤维素、几丁质}
\end{notation}
按组成成分分类:
\begin{notation}
    同聚多糖、杂聚多糖、粘多糖、糖复合物
\end{notation}
\subsubsection*{同聚多糖}%
\label{subsub*:同聚多糖}
\begin{eg}
    淀粉:直链淀粉只有$\alpha$-(1-4)糖苷键,支链淀粉在支链处为$\alpha$-(1-6)糖苷键
\end{eg}
\begin{eg}
糖原:通过$\alpha$-(1,4)糖苷键和$\alpha$-(1-6)糖苷键连接而成的短支链
\end{eg}
\begin{eg}
    纤维素
\end{eg}
\begin{eg}
几丁质的单体:N-乙酰氨基葡萄糖,由$\alpha$-(1-4)糖苷键结合
\end{eg}
\subsubsection*{杂聚多糖}%
\label{subsub*:杂聚多糖}
\begin{eg}
    琼脂:由D-半乳糖和L-半乳糖聚合而成,D型:L型=2:1
\end{eg}
\subsubsection*{粘多糖}%
\label{subsub*:粘多糖}
由含氮单元缩合而成的不均一多糖
\begin{eg}
    透明质酸:由$\beta$-D-葡糖醛酸和N-乙酰氨基葡萄糖聚合
\end{eg}
\begin{eg}
    硫酸软骨素:由$\beta$-D-葡糖醛酸和N-乙酰氨基半乳糖或$\beta$-L-艾杜糖醛酸和N-乙酰氨基半乳糖组成
\end{eg}
\begin{eg}
    肝素Heparin:由硫酸氨基葡萄糖硫酸酯、葡糖醛糖、硫酸氨基葡萄糖和艾杜糖醛酸硫酸酯缩合而成
\end{eg}
\subsubsection*{糖复合物}%
\label{subsub*:糖复合物}
可以有非糖物质和糖结合到一起
\begin{notation}
    蛋白聚糖:结缔组织
\end{notation}
\begin{notation}
    肽聚糖:多糖连接在氨基酸链形成的大分子
\end{notation}
\begin{notation}
糖脂:糖和脂类分子的组合,脂分子为主导,如脑苷脂、甘油糖脂、鞘糖脂
\end{notation}
\begin{notation}
    脂多糖:糖为主体的糖脂,在免疫系统有重要作用
\end{notation}
\subsection{多糖含量和分子量的测定}%
\label{sub:多糖含量和分子量的测定}

