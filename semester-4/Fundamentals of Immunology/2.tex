\lecture{2}{02.24}
回忆:固有免疫的成员:
\begin{itemize}
    \item 巨噬细胞
    \item 中性粒
    \item 补体蛋白
    \item 树突状细胞
    \item 自然杀伤细胞(NK细胞)
\end{itemize}
\begin{notation}
    固有免疫使用模式识别,适应性免疫使用特异性识别
\end{notation}
特异性免疫的特点:
\begin{itemize}
    \item 自我调节
    \item 有记忆性
    \item 多样性
    \item 适应性
    \item \ldots
\end{itemize}
\section{免疫系统组成}%
\label{sec:免疫系统组成}
\[
    \text{免疫系统}\begin{cases}
        \text{免疫器官}\begin{cases}
            \text{中枢}\\
            \text{外周}
        \end{cases}\\
        \text{免疫细胞}\begin{cases}
            \text{固有免疫细胞}\begin{cases}
                \text{单核巨噬}\\
                \text{树突状}\\
                \text{自然杀伤细胞}\\
                \text{中性粒}\\
                \text{嗜酸碱性粒}\\
                \text{肥大}
            \end{cases}\\
            \text{适应性}
        \end{cases}\\
        \text{免疫分子}\begin{cases}
            \text{体液中}\begin{cases}
                \text{免疫球蛋白}\\
                \text{补体}\\
                \text{细胞因子}
            \end{cases}\\
            \text{细胞膜}\begin{cases}
                \text{MHC}\\
                \text{CD}\\
                \text{粘附}\\
                \text{膜受体}
            \end{cases}
        \end{cases}
    \end{cases}
.\]
\subsection{免疫器官}%
\label{sub:免疫器官}
通过血液循环和淋巴循环相互联系,\textbf{胸腺和骨髓}为中枢免疫器官,\textbf{淋巴结、脾脏和黏膜等淋巴组织}为外周免疫器官。

中枢免疫器官又称初级淋巴器官
\subsubsection*{中枢免疫器官}%
\label{subsub:中枢免疫器官}
\begin{notation}
    骨髓:各类血细胞的发源地,B细胞的发育成熟的场所,\textbf{再次免疫应答}后抗体产生的关键部位
\end{notation}
骨髓分为黄骨髓和红骨髓,其中红骨髓可以造血,由造血组织和血窦组成
\begin{defi}
    血窦:一种特殊的血管结构
\end{defi}
造血组织中含有基质细胞,如网状细胞、成纤维细胞、巨噬细胞等,共同组成造血内环境(HIM)

骨髓有以下功能:
\begin{itemize}
    \item 制造血细胞和免疫细胞:通过造血干细胞分化
    \item 哺乳动物产生B细胞
    \item 免疫应答
    
\end{itemize}
细胞因子可以看作是小蛋白,各细胞因子的作用:
\begin{itemize}
    \item 造血干细胞(HSC):发送SCF使髓样干细胞分化
    \item \ldots 
\end{itemize}
\begin{notation}
    胸腺(Thymus):造血干细胞在骨髓中发育为T细胞,迁移至胸腺继续发育
\end{notation}
胸腺细胞由一下细胞组成:
\begin{itemize}
    \item 胸腺细胞:不同分化的T细胞
    \item 胸腺基质细胞
\end{itemize}
胸腺具有皮质和髓质两部分:

\textit{皮质:}
\begin{itemize}
    \item 浅皮质、深皮质
    \item 90\%为胸腺细胞
\end{itemize}

\textit{髓质:}
\begin{itemize}
    \item 胸腺小体
    \item 巨噬细胞、DC细胞(树突状细胞),T细胞
    
\end{itemize}
\begin{notation}
    胸腺的功能:T细胞分化成熟的场所,通过细胞因子和胸腺肽类进行免疫调节,引发阴性选择(只有10\%的T细胞才能存活)
\end{notation}
\subsubsection*{外周免疫器官}%
\label{subsub:外周免疫器官}
\begin{notation}
淋巴结:分布最广泛,沿血管排列,分布于颈部、腹股沟等
\end{notation}
淋巴结是结构最完备的外周免疫器官:
\begin{itemize}
    \item 皮质区:浅皮质区有大量未受抗原刺激的初始B细胞,并形成初级淋巴滤泡,当抗原刺激时形成生发中心,称为次级淋巴滤泡;深皮质区在浅皮质区和髓质之间,是T细胞的居住场所,含有从自组织迁移来的DC细胞,含有毛细管后微静脉(PCV)或高内皮微静脉(HEV)
    \item 髓质:含有B细胞、浆细胞和巨噬细胞
    
\end{itemize}
淋巴结的功能:
\begin{itemize}
    \item T,B细胞的定居处
    \item 过滤作用
    \item 免疫应答的主要场所
    \item 参与淋巴细胞再循环
\end{itemize}
\begin{notation}
    脾:胚胎时期的造血器官,在左腹部处,人体最大的外周免疫器官
\end{notation}
脾脏的结构:

1. 白髓:T细胞集中区域(胸腺依赖区),其中有一些淋巴小结中有B细胞 

2. 红髓:含巨噬细胞

脾脏可以对\textbf{血源性抗原}产生免疫应答
\begin{notation}
    黏膜相关淋巴组织:包含肠、鼻、支气管相关的淋巴组织,具有黏膜局部防御功能,可以产生分泌性IgA(可以穿过肠道表面),口服抗原介导的免疫耐受
\end{notation}
\subsection{免疫细胞}%
\label{sub:免疫细胞}
不做细讲。
\subsection{免疫分子}%
\label{sub:免疫分子}
\subsubsection*{膜分子}%
\label{subsub:膜分子}
细胞表面的标记物,主要包括MHC,协同刺激分子等
\begin{notation}
    白细胞分化抗原:最开始从白细胞开始研究,但是在其他细胞中也存在。细胞分化成长中,膜分子在不断变化
\end{notation}
通过膜表面的分子可以判断细胞的种类,根据胞外区结构分为6类:
\begin{itemize}
    \item IgSF(免疫球蛋白超家族)
    \item CKRF
    \item C型凝集素超家族
    \item 整合素家族
    \item 选择素家族
    \item TNFSF
    
\end{itemize}
\begin{notation}
    对于胞内环境:

    ITAM:用于活化
    
    ITIM:用于抑制
\end{notation}
常见的免疫细胞膜分子参与的细胞间相互作用:
\begin{itemize}
    \item 胸腺细胞和胸腺基质细胞
    \item 淋巴细胞再循环
    \item APC抗原活化T细胞
    \item T,B细胞
    \item CTL杀伤靶细胞
    \item \ldots 
    
\end{itemize}
\begin{notation}
    分化群(Cluster of Differentiation, CD):以单克隆抗体鉴定为主,由国际白细胞分化抗原会议将不同的单克隆抗体识别的同一种抗原归类为同一个分化群,由CD1到CD363
\end{notation}
常见的分化群:$\text{CD4}^{-}, \text{CD8}^{-}, \text{CD4}^{+}, \text{CD8}^{+}$(T细胞发育)

人白细胞分化抗原的功能:

受体:
\begin{itemize}
    \item 特异性识别抗原的受体和共受体
    \item 模式识别
    \item 细胞因子
    \item 补体
    \item NK细胞
    \item IgFc段
\end{itemize}

粘附分子:
\begin{itemize}
    \item 共刺激/抑制因子
    \item 归巢受体
    \item 血管地址素(用于定位分子如何走向指定位置)
\end{itemize}
\subsubsection*{粘附分子}%
\label{subsub:粘附分子}
\begin{defi}
    粘附分子(Adhesino Molecules, AM),介导细胞和细胞间或细胞与基质之间接触和结合的一类分子,大部分为糖蛋白。
\end{defi}
分类:
\begin{itemize}
    \item IgSF
    \item 整合素家族
    \item 选择素家族
    \item 钙黏蛋白家族
\end{itemize}
具有与免疫球蛋白相似的V区样或C区样结构域的分子归于IgSF
\begin{notation}
    V区样/C区样:对于一个膜分子,下半部分一般不变,可变区域在上半部分
\end{notation}
\begin{notation}
    选择素家族:有L,P,E三个成员,选择素的配体为\textbf{路易斯-X寡糖}
\end{notation}
粘附分子的功能:多样、重叠
\begin{eg}
    中性粒细胞穿过血管壁:
    \begin{enumerate}
        \item ICAM:细胞表面,将细胞与血管壁细胞粘附
        \item SLIG
        \item SEL:选择素
        \item INT:整合素
        
    \end{enumerate}
    通过SEL和INT把细胞吸附到血管壁表面,然后穿过细胞膜
\end{eg}
\subsection{淋巴细胞归巢和淋巴细胞再循环}%
\label{sub:淋巴细胞归巢和淋巴细胞再循环}
\begin{defi}
    淋巴细胞归巢:成熟淋巴细胞从中枢免疫器官经血液循环迁移定居于外周免疫器官或组织的特定区域
\end{defi}
见书$\text{P}_{13}$
