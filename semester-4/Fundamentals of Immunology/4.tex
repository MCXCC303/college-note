\lecture{4}{03.10}
三种激活途径:
\begin{itemize}
    \item 经典途径
    \item 旁路途径(C3蛋白自发裂解)
    \item 凝集素途径(不依赖于适应性免疫)
\end{itemize}
一旦激活,一直到C5转换酶形成催化转化C5才到达结束
\section{抗原}%
\label{sec:抗原}
\begin{notation}
    能引起机体适应性免疫,且可以和适应性免疫活性物质相作用的物质称为\textbf{抗原}
\end{notation}
如果一个物质只能被固有免疫识别,这个物质不一定是抗原(铁屑)
\subsection{性质和分子基础}%
\label{sub:性质和分子基础}
\begin{defi}
    能刺激机体的免疫系统使之产生特异性免疫应答
\end{defi}
\begin{notation}
    半抗原:只有免疫原性没有免疫反应性或反之的抗原

    全抗原:两个都有(T细胞只识别蛋白抗原,用于制作疫苗)
\end{notation}
决定抗原特异性的分子结构基础:\textbf{抗原表位}/抗原决定基
\begin{notation}
    抗原决定基的分类:
    \begin{itemize}
        \item 顺序表位/线性表位:连续线性排列的氨基酸
        \item 构象表位:不连续,通过空间结构组成,拉伸后消失
    \end{itemize}
顺序表位可以被T细胞和B细胞识别,约8-10个($\ce{CD 8^+}$)或13-17个($\ce{CD 4^+}$)并且\textbf{必须要MHC分子参与};构象表位只能被B细胞识别,约5-15个氨基酸,且不需要MHC分子
\end{notation}
\subsection{共同抗原表位与交叉反应}%
\label{sub:共同抗原表位与交叉反应}
\begin{defi}
    共同抗原表位:不同抗原之间含有的相同或相似抗原单位

    交叉抗原:含有共同抗原表位的不同抗原
\end{defi}
\begin{eg}
    链球菌感染导致的心肌受到攻击
\end{eg}
\subsection{抗原分子的理化与结构性质}%
\label{sub:抗原分子的理化与结构性质}
\begin{description}
    \item[异物性] 结构差异越大越强
    \item [化学属性] \ldots 
    \item [分子量]越大含抗原表位越多,免疫原性越强
    \item [分子结构]复杂分子结构免疫原性强,环状的比链状的强
    \item [构象] \ldots 
    \item [易接近性] \ldots 
    \item [物理性状]聚合、颗粒更强
    \item [进入人体的方式]皮下注射最强
\end{description}
\begin{eg}
    间氨基磺酸/间氨基砷酸苯
\end{eg}
\subsection{抗原与机体的亲缘关系}%
\label{sub:抗原与机体的亲缘关系}
\begin{itemize}
    \item 异种抗原:两重性,有抗毒素作用
    \item 异嗜性抗原:感染溶血型链球菌可能导致肾小球肾炎和心肌炎
    \item \ldots 
    
\end{itemize}
\subsection{不同的抗原}%
\label{sub:不同的抗原}
\subsubsection*{超抗原}%
\label{subsub*:超抗原}
\begin{defi}
    可以一次性激活大量T细胞的抗原(SAg)
\end{defi}
\begin{eg}
    金黄色葡萄球菌蛋白A(SPA)、肠毒素A/B(SEA/SEB)
\end{eg}
\subsubsection*{佐剂}%
\label{subsub*:佐剂}
\begin{defi}
    将佐剂和抗原同时注入体内,来增强抗原特异性免疫应答或应答类型
\end{defi}
\begin{eg}
    卡介苗(BCG),氢氧化铝等
\end{eg}
通过改变抗原的物理形状来缓解抗原降解,其他机理等
\subsubsection*{丝裂原}%
\label{subsub*:丝裂原}
\begin{defi}
与丝裂原受体结合,促进淋巴母细胞转化并有丝分裂
\end{defi}
