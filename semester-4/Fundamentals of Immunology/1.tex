\lecture{1}{02.17}
\section{第一节课:什么是免疫}%
\label{sec:第一节课:什么是免疫}
\begin{notation}
    免疫:immune,免疫力:immunity,免疫学:immunology
\end{notation}
\begin{defi}
    传统免疫定义:机体对病原微生物及其有害产物的侵入所引起的发病具有抵抗力
\end{defi}
\subsection{免疫学发展简史}%
\label{sub:免疫学发展简史}
\textit{经验免疫学}
\begin{notation}
    认为:患某种传染病康复后,不会再患同种疾病
\end{notation}
\begin{eg}
    使用人痘、牛痘(Edward Jenner, 1796)预防天花病毒,1978年天花绝迹
\end{eg}
\textit{初盛时期}
\begin{eg}
    巴斯德:主动免疫,注射减毒或灭活的病毒

    贝林、北里:被动免疫,白喉杆菌、破伤风杆菌
\end{eg}
\begin{notation}
    体液免疫学说(保尔$\cdot $欧立希):体液中产生的抗体能清除病原微生物,中和细菌毒素
\end{notation}
\begin{notation}
    补体的发现(Pfeiffer,1894;Bordet,1895):一系列蛋白,对机体的异常细胞进行清理、对病原体抵抗
\end{notation}
\textit{近代免疫学时期}
\begin{notation}
    \begin{itemize}
        \item 抗体研究:揭示结构(1959)
        \item 杂交产生单克隆抗体(1975)
        \item 免疫试验技术
    \end{itemize}
\end{notation}
\begin{notation}
抗体的产生机制:克隆选择学说是近代免疫学发展的基石
\begin{itemize}
    \item 抗原的作用只是选择并激活相应的免疫活性细胞克隆
    \item 细胞受体和该细胞后代所分泌的产物(抗体)具有相同的特异性,与自身抗原反应的抗体被清除
\end{itemize}
\end{notation}
\subsection{免疫系统基本功能}%
\label{sub:免疫系统基本功能}
\begin{table}[htpb]
    \centering
    \caption{基本功能}
    \label{tab:基本功能}
    \begin{tabular}{ccc}
    \toprule
    功能 & 生理性 & 危害\\
    \midrule
    免疫防御 & 防御危害 & 超敏反应\\
    免疫稳态 & 清除衰老损伤细胞 & 自身免疫病\\
    免疫监视 & 清除异常细胞 & 癌变、感染\\
    \bottomrule
    \end{tabular}
\end{table}
\subsection{免疫系统三大防线}%
\label{sub:免疫系统三大防线}
\begin{enumerate}
    \item 物理屏障(皮肤、呼吸道、消化道、生殖道)
    \item 固有免疫系统(巨噬细胞、中性粒细胞、补体蛋白、NK细胞)
    \item 适应性免疫系统(B细胞、T细胞)
\end{enumerate}
\subsection{免疫应答种类和特点}%
\label{sub:免疫应答种类和特点}
\textit{固有性免疫}

与生俱来的防御,对所有病原都具有攻击性

\textit{适应性免疫}

一般经过三个阶段:识别$\to $ 活化$\to $ 效应,表现为细胞免疫和体液免疫

\begin{notation}
    适应性免疫具有\textbf{特异性}和\textbf{记忆效应}
\end{notation}
