\learn{4}{10.20}
\subsection{伯努利模型}%
\label{sub:伯努利模型}
\begin{notation}
    独立试验序列:$E_1,E_2,\ldots,E_n$ 彼此相互独立

    $n$ 重独立试验: $E_1,E_1,\ldots,E_1$ 做$n$ 遍,且相互独立,记作$E_1^{n}$

    伯努利试验:结果只有两种:$\Omega=\left\{ A,\bar{A} \right\} $
\end{notation}
\begin{eg}
    抛硬币是伯努利试验
\end{eg}
\begin{notation}
    $n$ 重伯努利试验:做$n$ 次结果只有两种的独立试验
\end{notation}
\begin{cor}
    $A$ 发生的概率为$ p$,$p\in (0,1)$ ,则$\bar{A}$ 的概率为$1-p$ ,在$n$ 重伯努利试验中, $A$ 发生$ k$次的概率为:
    \[
        P_n\left( k \right) =\mathrm{C}_{n}^{k}p^{k}\left( 1-p \right) ^{1-k}
    .\] 

    或:
    \[
        P_n\left( k \right) =\mathrm{C}_{n}^{k}p^{k}q^{1-k} \hspace{0.5cm} q=1-p
    .\] 

    该公式称为\textit{二项概率公式}
\end{cor}
\begin{notation}
    为何称为二项概率公式:
    \begin{align*}
        \sum_{k=1}^{n} P_n\left( k \right) &=\sum_{k=1}^{n} \mathrm{C}_{n}^{k}p^{k}q^{1-k}\\
        &= \left( p+q \right) ^{n} \\
        &= \left( p+1-p \right) ^{n} \\
        &= 1
    .\end{align*}

    本质为二项式展开
\end{notation}
\begin{eg}
    一批产品的废品率为$0.1$ ,良品率为$0.9$,每次取一个产品又放回,共取三次,求:

    1. 恰有一次取到废品的概率

    2. 恰有两次取到废品的概率

    3. 三次都取到废品的概率

    4. 三次都取到良品的概率
\end{eg}
1. $\mathrm{C}_{3}^{1}0.1^{1}\cdot 0.9^2$ 

2. $\mathrm{C}_{3}^{2}0.1^2\cdot 0.9^{1}$ 

3. $\mathrm{C}_{3}^{3}0.1^3$ 

4. $\mathrm{C}_{3}^{0}0.9^{3}$
\begin{eg}
    彩票每周开奖一次,中奖率十万分之一 ,十年一共买了520次彩票,求十年从未中奖的概率
\end{eg}
解:$p=0.99999$, $P_{520}\left( 520 \right) =\mathrm{C}_{520}^{520}0.99999^{520}\approx 0.9948$ 
\begin{notation}
    为了更快计算二项公式的结果可以使用两种估算方法
\end{notation}

\section{随机变量及分布}%
\label{sec:随机变量及分布}
\begin{defi}
    随机变量:把试验的结果转化为符号语言

    即:用一个映射函数将试验的所有结果映射为数$X=X\left( \omega \right) $
\end{defi}
\begin{eg}
    $\left\{ \omega|X\left( \omega \right) =a \right\} $ 表示一个事件,也可以写成:$\left\{ X=a \right\} $ 

    事件的概率:$P\left\{ X=a \right\} $ 或$P\left( X=a \right) $
\end{eg}

