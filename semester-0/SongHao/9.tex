\learn{9}{11.25}
\section{正态总体下的抽样分布}%
\label{sec:正态总体下的抽样分布}
\begin{cor}
    $X\sim N\left( \mu,\sigma^2  \right)$ 且$X_1,X_2,\ldots ,X_{n}$ 为样本,定义:\begin{enumerate}
        \item $\overline{X}=\frac{1}{n}\sum_{i=1}^{n} X_{i}$
        \item $S^2 =\frac{1}{n-1}\sum_{i=1}^{n} \left( X_{i}-\overline{X} \right)$
    \end{enumerate}
    结论:
    \begin{enumerate}
        \item $\overline{X}\sim N\left( \mu,\frac{\sigma^2}{n} \right)$ ,标准化后$\frac{\overline{X}-\mu}{\sigma}\sqrt{n}\sim N\left( 0,1 \right)$
        \item $E\overline{X}=E\left( \frac{1}{n}\sum_{i=1}^{n} X_{i} \right)=\frac{1}{n}\sum_{i=1}^{n} EX_{i}=\frac{1}{n}\cdot n\mu=\mu$
        \item $D\overline{X}=D\left( \frac{1}{n}\sum_{i=1}^{n} X_{i} \right)=\frac{1}{n^2 }\sum_{i=1}^{n} DX_{i}=\frac{1}{n^2 }\cdot n\sigma^2 =\frac{\sigma}{n}$
        \item $\frac{\left( n-1 \right)S^2 }{\sigma^2 }=\frac{1}{\sigma^2 }\sum_{i=1}^{n} \left( X_{i}-\overline{X} \right)^2 \sim \chi^2 \left( n-1 \right)$
        \item $\overline{X}$ 和$S^2 $ 相互独立
    \end{enumerate}
\end{cor}
\begin{notation}
    均值的方差比原来的分布的方差小:少量样本的平均数波动会较大(取到两个很大的+取到两个很小的),大量样本的平均数波动不大
\end{notation}
\begin{cor}
    \begin{enumerate}
        \item $\frac{1}{\sigma^2 }\sum_{i=1}^{n} \left( X_{i}-\mu \right)^2 \sim \chi^2 \left( n \right)$
        \item $\frac{\overline{X}-\mu}{S}\sqrt{n}\sim t\left( n-1 \right)$
    \end{enumerate}
    \begin{proof}
        把$X_{i}$ 标准化后求和:$\sum_{i=1}^{n} \left( \frac{X_{i}-\mu}{\sigma} \right)^2 \sim \chi^2 \left( n \right)$ ,提取后即得$\frac{1}{\sigma^2 }\left( X_{i}-\mu \right)^2 \sim \chi^2 \left( n \right)$
    \end{proof}
\end{cor}
\begin{notation}
区分:$\overline{X}$ 为样本均值,自由度为$n-1$ ,$\mu$ 为总体期望,自由度为$n$ ;样本均值为$\frac{1}{n}\sum_{i=1}^{n} X_{i}$ ,相当于多了一个约束(多了一个方程),使得自由未知量少一个
\end{notation}
\begin{eg}
\[
    \bm{Ax}=\begin{pmatrix}
        1\\
        2
    \end{pmatrix}\quad \bm{A}=\begin{pmatrix}
    1 & 0 & 1 & 1\\
    0 & 1 & 1 & 1
    \end{pmatrix}
.\]
自由未知量为$x_3,x_4$
\end{eg}
\begin{proof}
    $\frac{\overline{X}-\mu}{S}\sqrt{n}\sim t\left( n-1 \right)$ :

    需要用到: \begin{enumerate}
        \item $\frac{\left( n-1 \right)S^2 }{\sigma^2 }\sim \chi^2 \left( n-1 \right)$ :$S$ 和$\sigma$ 的关系
        \item $\frac{X}{\sqrt{Y /\left( n-1 \right)}}\sim t\left( n-1 \right)$,其中$X\sim N\left( 0,1 \right),Y\sim \chi^2 \left( n-1 \right)$
    \end{enumerate}
    由于$\frac{\overline{X}-\mu}{\sigma}\sqrt{n}\sim N\left( 0,1 \right),  \frac{\left( n-1 \right)S^2 }{\sigma^2 }\sim \chi^2 \left( n-1 \right)$,构造$t$ 分布:\[
        \frac{\frac{X-\mu}{\sigma}\sqrt{n}}{\sqrt{\frac{\left( n-1 \right)S^2 }{\sigma^2 } /\left( n-1 \right)}}=\frac{\frac{X-\mu}{\sigma}\sqrt{n}\cdot \sigma}{\sqrt{\left( n-1 \right)S^2 /\left( n-1 \right)}}=\frac{X-\mu}{S}\sqrt{n}\sim t\left( n-1 \right)
    .\]
    即结论得证。
\end{proof}
以上为一个正态分布总体;以下为两个正态分布总体:
\begin{cor}
    $X\sim N\left( \mu_1,\sigma_1^2  \right),Y\sim N\left( \mu_1,\sigma_2^2  \right)$ 

    有两个正态分布总体样本:$\left\{ X_1,X_2,\ldots ,X_{n_1} \right\},\left\{ Y_1,Y_2,\ldots ,Y_{n_2} \right\}$ ,有以下结论:
    \begin{enumerate}
        \item $\frac{\left( \overline{X}-\overline{Y}\right) -\left( \mu_1-\mu_2 \right)}{\sqrt{\frac{\sigma_1^2 }{n_1}+\frac{\sigma_2^2 }{n_2}}}$
        \item $\frac{S_1^2 /\sigma_1^2 }{S_2^2 /\sigma_2^2}\sim F\left( n_1-1,n_2-1 \right)$
        \item 当$\sigma_1^2 =\sigma_2^2 =\sigma^2  $ 时:$T\left( \overline{X},\overline{Y},\mu_1,\mu_2,S_1,S_2 \right)$ 的某个形式为$t$ 分布,形式见证明(太复杂了)
    \end{enumerate}
\end{cor}
\begin{proof}
    第一个结论:$\frac{\left( \overline{X}-\overline{Y} \right)-\left( \mu_1-\mu_2 \right)}{\sqrt{\frac{\sigma_1^2}{n_1}+\frac{\sigma_2^2 }{n_2} }}$
    
    由一个总体的正态分布结论有:$\overline{X}\sim N\left( \mu_1,\frac{\sigma_1^2}{n} \right),\overline{Y}\sim N\left( \mu_2,\frac{\sigma_2^2}{n} \right)$,则由线性可加性:\[
        \overline{X}-\overline{Y}\sim N\left( \mu_1-\mu_2,\frac{\sigma_1^2 }{n_1}+\frac{\sigma_2^2 }{n_2} \right)
    .\]
    接着将这个分布标准化即可:\[
        \frac{\left( \overline{X}-\overline{Y} \right)-\left( \mu_1-\mu_2 \right)}{\sqrt{\frac{\sigma_1^2 }{n_1}+\frac{\sigma_2^2 }{n_2}}} \sim N\left( 0,1 \right)
    .\]
\end{proof}
\begin{proof}
    第二个结论:$\frac{S_1^2 /\sigma_1^2 }{S_2^2 /\sigma_2^2 }\sim F\left( n_1-1,n_2-1 \right)$ :

    需要用到结论:$\frac{\left( n-1 \right)S^2 }{\sigma^2 }\sim \chi^2 \left( n-1 \right)$ 

    两个卡方分布的比值即为F分布:\[
        \frac{\frac{\left( n_1-1 \right)S_1^2 }{\sigma_1^2 } /\left( n_1-1 \right)}{\frac{\left( n_2-1 \right)S_2^2 }{\sigma_2^2 } /\left( n_2-1 \right)}\sim F\left( n_1-1,n_2-1 \right)
    .\]
\end{proof}
\begin{notation}
    $t$ 分布注意:开根号、除自由度:\[
        \frac{X}{\sqrt{Y /\bm{n}}}\sim t\left( \bm{n} \right)
    .\]
    F分布注意:除参数:\[
        \frac{X /\bm{n}_1}{Y /\bm{n}_2}\sim F\left( \bm{n}_1,\bm{n}_2 \right)
    .\]
\end{notation}
\begin{notation}
    某个复杂形式符合$t$ 分布的证明:\[
        T=\frac{\left( \overline{X}-\overline{Y} \right)-\left( \mu_1-\mu_2 \right)}{\sqrt{\frac{\left( n_1-1 \right)S_1^2 +\left( n_2-1 \right)S_2^2 }{n_1+n_2-2}}\cdot \sqrt{\frac{1}{n_1}+\frac{1}{n_2}}}\sim t\left( n_1+n_2-2 \right)
    .\]
    或者更直白的形式:\[
        T=\frac{\frac{\left( \overline{X}-\overline{Y} \right)-\left( \mu_1-\mu_2 \right)}{\sqrt{\frac{\sigma_1^2}{n_1}+\frac{\sigma_2^2}{n_2}}}}{\sqrt{\frac{\left( \frac{\left( n_1-1 \right)S_1^2}{\sigma^2 } +\frac{\left( n_2-1 \right)S_2^2 }{\sigma^2 } \right)}{n_1-1+n_2-1}}} \sim t\left( n_1-1+n_2-1 \right)
    .\]
    不难发现:上半部分为结论展示的正态分布,下半部分中,卡方分布部分利用卡方分布可加性:$\chi^2 \left( n_1 \right)+\chi^2 \left( n_2 \right)=\chi^2 \left( n_1+n_2 \right)$,用$\frac{\left( n-1 \right)S^2 }{\sigma^2 }$ 组合:\[
        \begin{cases}
            \frac{\left( n_1-1 \right)S_1^2 }{\sigma^2 }\sim \chi^2 \left( n_1-1 \right)\\
            \frac{\left( n_2-1 \right)S_2^2 }{\sigma_2^2 }\sim \chi^2 \left( n_2-1 \right)
        \end{cases}\Rightarrow \frac{\left( n_1-1 \right)S_1^2 }{\sigma_2^2 }+\frac{\left( n_2-1 \right)S_2^2 }{\sigma_2^2 }\sim \chi^2 \left( n_1+n_2-2 \right)
    .\]
    最后通过构造$t$ 分布将两部分组合即可。
\end{notation}
\begin{eg}
    $X\sim N\left( \mu,\sigma^2  \right),\overline{X}$和$S^2 $ 为样本均值和方差,当$n=16$ 时,求$k$ 使$P\left( \overline{X}>\mu+kS \right)=0.95$
\end{eg}
解:把原式转换:\[
    P\left( \overline{X}>\mu+kS \right)=P\left( \frac{\overline{X}-\mu}{S}\sqrt{16}>k\cdot \sqrt{16} \right)
.\]
由结论可得$\frac{\overline{X}-\mu}{S}\cdot \sqrt{16}\sim t\left( 15 \right)$ ,由$t$ 分布的对称性:$P\left( T>4k \right)=1-P\left( T>-4k \right)$ 即$P\left( T>-4k \right)=0.05$ ,因此查表$t_{0.05}\left( 15 \right)$ 即可
\begin{figure}[ht]
    \centering
    \incfig[0.7]{t分布对称性}
    \caption{$t$分布对称性}
    \label{fig:t分布对称性}
\end{figure}
\section{参数估计}%
\label{sec:参数估计}
\begin{defi}
    参数:对于分布的参数举例如下:

    通过抽样$X_1,X_2,\ldots X_{n}$ 构造函数
\end{defi}
\begin{table}[htpb]
    \centering
    \caption{参数-分布关系}
    \label{tab:参数-分布关系}
    \begin{tabular}{cc}
    \toprule
    总体分布 & 总体参数\\
    \midrule
    $N\left( \mu,\sigma^2  \right)$ & $\mu,\sigma^2 $\\
    $P\left( \lambda \right)$ & $\lambda$ \\
    $U[a,b]$ & $a,b$ \\
    \bottomrule
    \end{tabular}
\end{table}
\begin{eg}
抽取50个同学量身高,构造身高函数,通过观测值得到的估计参数来接近理论上的参数
\end{eg}
\begin{defi}
参数空间:参数的取值范围
\end{defi}
\subsection{点估计}%
\label{sub:点估计}
\begin{defi}
    点估计:估计值为一个数

    对比区间估计:估计值是一个区间
\end{defi}
点估计构造的函数记为$\hat{\theta}=\hat{\theta}\left( X_1,X_2,\ldots ,X_{n} \right)$
\subsection{矩估计}%
\label{sub:矩估计}
\begin{defi}
    使用随机变量的矩估计参数

    使用样本矩代替总体的矩,一般使用一阶矩样本矩代替一阶总体矩,二阶样本矩代替二阶总体矩
\end{defi}
    对于总体:

    一阶(中心/原点)矩$EX$ $\leftarrow$ 一阶样本矩$\overline{X}=\frac{1}{n}\sum_{i=1}^{n} X_{i}$ 

    二阶矩$EX^2 \leftarrow$ 二阶样本矩$A_2=\frac{1}{n}\sum_{i=1}^{n} X_{i}^2 $
\begin{eg}
    $X\sim N\left( \mu,\sigma^2  \right)$ ,取样$X_1,X_2,\ldots X_{n}$ ,使用矩估计估计参数$\mu,\sigma$
\end{eg}
解:$EX=\mu, \overline{X}=\frac{1}{n}\sum_{i=1}^{n} X_{i}$ (一阶样本$\to $ 一阶总体)

即:$\hat{\mu}=\overline{X}$(估计值)

同理:$DX=E\left( X-EX \right)=EX^2 -\left( EX \right)^2 $,转换后:$EX^2 =DX+\left( EX \right)^2 =\sigma^2 +\mu^2 $,而$A_2 =\frac{1}{n}\sum_{i=1}^{n} X_{i}^2 $

即:$\hat{\sigma}^2 +\hat{\mu}^2 =A_2\Rightarrow \hat{\sigma}^2 =A_2-\hat{\mu}^2 $ (估计值)

算出$\hat{\sigma}$ 即得:\[
    \begin{cases}
        \hat{\mu}=\frac{1}{n}\sum_{i=1}^{n} X_{i}\\
        \hat{\sigma}=\sqrt{\frac{1}{n}\sum_{i=1}^{n} X_{i}^2 -\left( \frac{1}{n}\sum_{i=1}^{n} X_{i} \right)^2 }
    \end{cases}
.\]
