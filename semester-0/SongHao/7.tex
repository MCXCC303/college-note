\learn{7}{11.23}
\section{抽样分布}%
\label{sec:抽样分布}
\begin{defi}
    卡方分布$\chi^2 \left( n \right)$ :标准正态分布平方后的和是卡方分布
\end{defi}
即:$X_1,X_2,\ldots X_{i}$ 相互独立且都是标准正态分布,则\[
    \sum_{i=1}^{n} X_{i}^2 \sim \chi^2 \left( n \right)=\chi^2 
.\]
\begin{eg}
    $X_1,X_2,\ldots ,X_8$ 独立同分布于$N\left( 0,1 \right)$ ,则$\sum_{i=1}^{8} X_{i}\sim \chi^2 \left( 8 \right)$
\end{eg}
\begin{cor}
    卡方分布的期望$EX=n$ ,方差$DX=2n$
\end{cor}
\begin{figure}[ht]
    \centering
    \incfig[]{卡方分布图像}
    \caption{卡方分布图像}
    \label{fig:卡方分布图像}
\end{figure}
\begin{cor}
    中心极限定理可得:如果$X\sim \chi^2 \left( n \right)$ 且$n$ 充分大的时候:$\frac{X-n}{\sqrt{2n}}$ 近似符合标准正态分布$N\left( 0,1 \right)$
\end{cor}
\begin{notation}
    独立同分布的中心极限定理不管分布种类
\end{notation}
\begin{cor}
    $X\sim \chi^2 \left( n \right),Y\sim \chi^2 \left( m \right)$ ,且$X,Y$ 独立,则$X+Y\sim \chi^2 \left( m+n \right)$
\end{cor}
\begin{notation}
推论:$X_{i}\sim \chi^2 \left( m_{i} \right)$ ,则\[
    \sum_{i=1}^{n} X_{i}\sim \chi^2 \left( \sum_{i=1}^{n} m_{i} \right)
.\]
即卡方分布符合可加性
\end{notation}
\begin{defi}
上$\alpha$ 分位数:$P\left( \chi^2 >\chi^2_{\alpha} \left( n \right) \right)=\alpha$ :在分布图某点右边的概率为$\alpha$ ,这个点的值
\end{defi}
\begin{eg}
    $\chi^2 _{0.05}\left( 10 \right)=18.3,\chi^2 _{0.1}\left( 25 \right)=34.4$
\end{eg}
上$\alpha$ 分位数查表可得:$n$ 已知,确定所需的概率$\alpha$ 
\begin{eg}
    已知$X\sim \chi^2 \left( 10 \right)$ 求:当$P\left( X>a \right)=0.025,P\left( X<b \right)=0.05$ 时的$a,b$
\end{eg}
解:由题:$n=10,\alpha_1=0.025$ ,查表可得:$\chi^2 _{0.025}\left( 10 \right)=20.5=a$ 

易得$P\left( X<b \right)=0.05=1-P\left( X\ge b \right)$ 即$P\left( X\ge b \right)=0.95=\alpha_2$ ,查表:$\chi^2 _{\alpha_2}\left( 10 \right)=3.94=b$ 
\begin{notation}
卡方分布为单峰曲线,$n-2$ 时取最大值
\end{notation}
\subsection{\textit{t}分布}%
\label{sub:t分布}
\begin{defi}
    $t$ 分布由标准正态分布和卡方分布组成:$X\sim N\left( 0,1 \right),Y\sim \chi^2 \left( n \right)$,且$X,Y$ 相互独立,则$\frac{X}{\sqrt{Y /n}}\sim t\left( n \right)$
\end{defi}
\begin{notation}
    上$\alpha$ 分位数:$P\left( T>t_{\alpha}\left( n \right) \right)=\alpha$ :$\alpha$ 为一个数,由于$t$ 分布的对称性可得:$t_{\alpha}\left( n \right)=1-t_{1-\alpha}\left( n \right)$
\end{notation}
\begin{eg}
    $X\sim N\left( 2,1 \right),\left( Y_1,Y_2,\ldots ,Y_4 \right)\sim N\left( 0,4 \right)$ ,且相互独立,令$T=\frac{4\left( X-2 \right)}{\sqrt{\sum_{i=1}^{4} Y_{i}^2 }}$ ,求$T$ 的分布并求$P\left( \left| T \right|>t_0 \right)=0.01$ 时的$t_0$
\end{eg}
