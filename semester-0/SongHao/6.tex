\learn{6}{11.16}
\section{大数定律}%
\label{sec:大数定律}
\begin{cor}
    切比雪夫不等式:随机变量中任意$X$离期望的距离比$\epsilon$ 大的概率比$\frac{DX}{\epsilon^2 }$ 小
\end{cor}
\begin{proof}
    假设$X$ 为连续型,由概率定义:\[
        P\left( \left| X-EX \right|\ge \epsilon \right)=\int_{\left| x-EX \right|\ge \epsilon}^{} f\left( x \right) \mathrm{d}x
    .\]
    这里因为$\left| X-EX \right|\ge \epsilon$ ,同时平方:$\left( X-EX \right)^2 /\epsilon^2 \ge 1$ ,即原式可以转为:\[
        P\left( \left| X-EX \right|\ge \epsilon \right)=\int_{\left| x-EX \right|\ge \epsilon}^{} 1\cdot f\left( x \right) \mathrm{d}x \le \int_{\left| x-EX \right|\ge \epsilon}^{} \frac{\left( x-EX \right)^2 }{\epsilon^2 }f\left( x \right) \mathrm{d}x
    .\]
    这个积分有积分域,大小必然比从$+\infty $ 到$-\infty $ 更小(或等于):因为密度函数$f\left( x \right)\ge 0$,因此不需要考虑被积函数正负问题,即:\[
        P\left( \left| X-EX \right|\ge \epsilon \right)\le \int_{-\infty }^{+\infty} \frac{\left( x-EX \right)^2 }{\epsilon^2 }f\left( x \right) \mathrm{d}x=\frac{\int_{-\infty}^{+\infty} \left( x-EX \right)^2f\left( x \right)  \mathrm{d}x}{\epsilon^2 }=\frac{DX}{\epsilon^2 }
    .\]
\end{proof}
该结论等价于:
\begin{enumerate}
    \item $P\left( \left| X-EX \right|<\epsilon \right)\ge 1-\frac{DX}{\epsilon^2 }$
    \item $P\left( \left| X-EX \right|\le \epsilon \right)\ge 1-\frac{DX}{\epsilon^2 }$
\end{enumerate}
当$DX$ 越小时$P$ 越大,即方差越小,$X$ 落在外面的概率越小
\begin{eg}
    一个电网有10000盏灯,每一盏灯打开的概率为0.7,求一个晚上开着的灯数量在6800到7200盏中的概率
\end{eg}
解:期望为7000:$EX=7000$ ,即$X\sim B\left( 10000,0.7 \right)$ ,由于二项分布$EX=7000,DX=np\left( 1-p \right)=10000\times 0.7\times \left( 1-0.7\right)=2100$ 

要求:$P\left\{ X\in [6800,7200] \right\}$ :转换后可得:$P\left\{ X\in [6800,7200] \right\}=P\left\{ \left| X-7000 \right|\le 200 \right\}$

由切比雪夫不等式:\[
    P\left( \left| X-7000 \right|\le 200 \right)\ge 1-\frac{2100}{200^2 }\approx 0.9475
.\]
\begin{eg}
    白细胞平均数为7300,标准差为700,求白细胞数在5200到9400的概率
\end{eg}
解:$EX=7300,DX=700^2 =490000$ ,由切比雪夫不等式:\[
    P\left( 5200\le X\le 9400 \right)=P\left( \left| X-7300 \right|<2100 \right)\ge 1-\frac{DX}{2100^2 }
.\]
\begin{notation}
    用切比雪夫不等式做题:两边取到的距离应该相等
\end{notation}
\begin{eg}
    $3\sigma$ 原则:求$P\left( \left| X-\mu \right|\ge 3\sigma \right)$
\end{eg}
解:由切比雪夫不等式:\[
    P\left( \left| X-\mu \right|\le 3\sigma \right)\ge 1-\frac{\sigma^2 }{(3\sigma)^2 } \quad P\left( \left| X-\mu \right|\ge 3\sigma \right)\le \frac{\sigma^2 }{\left( 3\sigma \right)^2 }=\frac{1}{9}
.\]
\begin{notation}
    当$X\sim N\left( \mu,\sigma^2  \right)$时,这个概率约等于0.0027
\end{notation}
\subsection{切比雪夫大数定律}%
\label{sub:切比雪夫大数定律}
\begin{defi}
    依概率收敛:有一个随机变量序列,越到$n$ 大时$X_{n}$能取得某数$a$的概率越来越接近1,即$X_{n}$的取值密集在$a$ 的附近,记为:\[
        X_{n}\xrightarrow[n\to +\infty ]{P}a
    .\]
\end{defi}
