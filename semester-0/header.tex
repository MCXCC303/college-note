%─────────────────────%
% Undefined Variables %
%─────────────────────%
\ifx\noter\undefined
    \def\noter{THF}
\else
\fi
\ifx\lecturer\undefined
    \def\lecturer{None}
\else
\fi

%──────────────%
% New Commands %
%──────────────%
\newcommand{\learn}[2]{% lecture sign
    \def\@learn{Learn #1}%
    \subsection*{\@learn}
    \marginpar{\small\textsf{\text{#2}}}
}

%───────────────────%
% Document Settings %
%───────────────────%
\documentclass[10pt,a4paper]{article}

%─────────────────%
% Package Imports %
%─────────────────%
\usepackage[]{amsmath}
\usepackage[]{amssymb}
\usepackage[]{amsthm}
\usepackage[]{array}
\usepackage[]{bm}
\usepackage[]{booktabs}
\usepackage[]{fancyhdr}
\usepackage[]{float}
\usepackage[]{geometry}
\usepackage[]{graphicx}
\usepackage[]{hyperref}
\usepackage[]{import}
\usepackage[]{inputenc}
\usepackage[]{mathrsfs}
\usepackage[]{multirow}
\usepackage[]{pdfpages}
\usepackage[]{pgfplots}
\usepackage[]{stmaryrd}
\usepackage[]{tabu}
\usepackage[]{tcolorbox}
\usepackage[]{textcomp}
\usepackage[]{thmtools}
\usepackage[]{tikz}
\usepackage[]{tkz-euclide}
\usepackage[]{url}
\usepackage[]{wrapfig}
\usepackage[]{xifthen}
\usepackage[]{yhmath}
\usepackage[UTF8]{ctex}
\usepackage[dvipsnames]{xcolor}
\usepackage[notransparent]{svg}
\usepackage[version=4]{mhchem}

%───────────────────%
% Inkscape Settings %
%───────────────────%
\newcommand{\incfig}[2][1]{%
    \def\svgwidth{#1\columnwidth}
    \import{./figures/}{#2.pdf_tex}
}

%────────────────%
% Fancy Settings %
%────────────────%
\fancypagestyle{CustomStyle}{%
    \fancyhf{}
    \setlength{\headheight}{14.49998pt}
    \fancyhead[R]{\thepage}
    \fancyhead[L]{\lecturer: \className}
    \fancyfoot[C]{\@learn}
}

%──────────────────%
% pgfplot Settings %
%──────────────────%
\usepgfplotslibrary{external}
\pgfarrowsdeclarecombine{twolatex'}{twolatex'}{latex'}{latex'}{latex'}{latex'}
\pgfplotsset{compat=1.12}

%───────────────%
% Tikz Settings %
%───────────────%
\usetikzlibrary{arrows.meta}
\usetikzlibrary{decorations.markings}
\usetikzlibrary{decorations.pathmorphing}
\usetikzlibrary{positioning}
\usetikzlibrary{fadings}
\usetikzlibrary{intersections}
\usetikzlibrary{cd}
\tikzset{->/.style = {decoration={markings,mark=at position 1 with {\arrow[scale=2]{latex'}}},postaction={decorate}}}
\tikzset{<-/.style = {decoration={markings,mark=at position 0 with {\arrowreversed[scale=2]{latex'}}},postaction={decorate}}}
\tikzset{<->/.style = {decoration={markings,mark=at position 0 with {\arrowreversed[scale=2]{latex'}},mark=at position 1 with {\arrow[scale=2]{latex'}}},postaction={decorate}}}
\tikzset{->-/.style = {decoration={markings,mark=at position #1 with {\arrow[scale=2]{latex'}}},postaction={decorate}}}
\tikzset{-<-/.style = {decoration={markings,mark=at position #1 with {\arrowreversed[scale=2]{latex'}}},postaction={decorate}}}
\tikzset{->>/.style = {decoration={markings,mark=at position 1 with {\arrow[scale=2]{latex'}}}postaction={decorate}}}
\tikzset{<<-/.style = {decoration={markings,mark=at position 0 with {\arrowreversed[scale=2]{twolatex'}}},postaction={decorate}}}
\tikzset{<<->>/.style = {decoration={markings,mark=at position 0 with {\arrowreversed[scale=2]{twolatex'}},mark=at position 1 with {\arrow[scale=2]{twolatex'}}},postaction={decorate}}}
\tikzset{->>-/.style = {decoration={markings,mark=at position #1 with {\arrow[scale=2]{twolatex'}}},postaction={decorate}}}
\tikzset{-<<-/.style = {decoration={markings,mark=at position #1 with {\arrowreversed[scale=2]{twolatex'}}},postaction={decorate}}}
\tikzset{circ/.style = {fill, circle, inner sep = 0, minimum size = 3}}
\tikzset{scirc/.style = {fill, circle, inner sep = 0, minimum size = 1.5}}
\tikzset{mstate/.style={circle, draw, blue, text=black, minimum width=0.7cm}}
\tikzset{eqpic/.style={baseline={([yshift=-.5ex]current bounding box.center)}}}
\tikzset{commutative diagrams/.cd,cdmap/.style={/tikz/column 1/.append style={anchor=base east},/tikz/column 2/.append style={anchor=base west},row sep=tiny}}

%──────────────────────%
% Theorem Environments %
%──────────────────────%
\theoremstyle{definition}
\declaretheoremstyle[
    headfont=\bfseries\sffamily\color{ForestGreen!70!black}, bodyfont=\normalfont,
    mdframed={
        linewidth=2pt,
        rightline=false, topline=false, bottomline=false,
        linecolor=ForestGreen, backgroundcolor=ForestGreen!5,
    }
]{thmDefi} % Defi = Green box
\declaretheoremstyle[
    headfont=\bfseries\sffamily\color{RawSienna!70!black}, bodyfont=\normalfont,
    numbered=no,
    mdframed={
        linewidth=2pt,
        rightline=false, topline=false, bottomline=false,
        linecolor=RawSienna, backgroundcolor=RawSienna!1,
    },
    qed=\qedsymbol
]{thmProof} % Proof = Red box
\declaretheoremstyle[
    headfont=\bfseries\sffamily\color{NavyBlue!70!black}, bodyfont=\normalfont,
    mdframed={
        linewidth=2pt,
        rightline=false, topline=false, bottomline=false,
        linecolor=NavyBlue
    }
]{thmNotation} % Notation = Blue line
% \declaretheoremstyle[
%     headfont=\bfseries\sffamily\color{NavyBlue!70!black}, bodyfont=\normalfont,
%     mdframed={
%         linewidth=2pt,
%         rightline=false, topline=false, bottomline=false,
%         linecolor=NavyBlue, backgroundcolor=NavyBlue!5,
%     }
% ]{thmbluebox}
% \declaretheoremstyle[
%     headfont=\bfseries\sffamily\color{RawSienna!70!black}, bodyfont=\normalfont,
%     mdframed={
%         linewidth=2pt,
%         rightline=false, topline=false, bottomline=false,
%         linecolor=RawSienna, backgroundcolor=RawSienna!5,
%     }
% ]{thmredbox}
% \declaretheoremstyle[
%     headfont=\bfseries\sffamily\color{NavyBlue!70!black}, bodyfont=\normalfont,
%     numbered=no,
%     mdframed={
%         linewidth=2pt,
%         rightline=false, topline=false, bottomline=false,
%         linecolor=NavyBlue, backgroundcolor=NavyBlue!1,
%     },
% ]{thmexplanationbox}
% \declaretheorem[style=thmNotation,numbered=no,name=Notation]{notation}
\declaretheorem[style=thmDefi,numbered=no,name=Definition]{defi}
% \declaretheorem[style=thmProof,numbered=no,name=Proof]{replacementproof}
\newtheorem*{aim}{Aim}
\newtheorem*{assumption}{Assumption}
\newtheorem*{axiom}{Axiom}
\newtheorem*{claim}{Claim}
\newtheorem*{conjecture}{Conjecture}
\newtheorem*{cor}{Corollary}
\newtheorem*{eg}{Example}
\newtheorem*{exercise}{Exercise}
\newtheorem*{ex}{Exercise}
\newtheorem*{fact}{Fact}
\newtheorem*{law}{Law}
\newtheorem*{lemma}{Lemma}
\newtheorem*{notation}{Notation}
\newtheorem*{prop}{Proposition}
\newtheorem*{question}{Question}
\newtheorem*{remark}{Remark}
\newtheorem*{rrule}{Rule}
\newtheorem*{thm}{Theorem}
\newtheorem*{warning}{Warning}
% \newtheorem{ncor}[nthm]{Corollary}
% \newtheorem{nlemma}[nthm]{Lemma}
% \newtheorem{nprop}[nthm]{Proposition}
% \newtheorem{nthm}{Theorem}[section]
% \renewenvironment{proof}[1][\proofname]{\vspace{-10pt}\begin{replacementproof}}{\end{replacementproof}}

%─────────────%
% Math Symbol %
%─────────────%

%────────────%
% Beginnings %
%────────────%
\title{\textbf{Part III-B: \className}}
\author{Lecture by \lecturer\\Note by \noter}
\pagestyle{CustomStyle}
%────────────────────────────────%
% Page Settings (set when print) %
%────────────────────────────────%
% \addtolength{\parskip}{-1mm}
% \addtolength{\parindent}{-2mm}
% \geometry{left=0.5cm,right=0.5cm,top=0.5cm,bottom=0.5cm}
