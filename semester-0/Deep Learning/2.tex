\learn{2}{01.30}
\begin{notation}
    $L^2 $ 范数可以写为$\left\lVert \bm{x} \right\rVert _2$ ,也可以简写为$\left\lVert \bm{x} \right\rVert$
\end{notation}
\begin{notation}
    $L^{1}$ 范数常用于区分0值和距离0很近的值,使用$L^{2}$ 范数,距离太近几乎没有区别
\end{notation}
\begin{defi}
    最大范数:$p\to \infty $ 时影响范数的最大因素为最大值:\[
        \left\lVert \bm{x} \right\rVert_\infty = \max_{i}\left| x_{i} \right|
    .\]
\end{defi}
\begin{eg}
    \textbf{Frobenius范数:}使用范数衡量矩阵的大小,类似于$L^2 $ 范数的计算:
    \[
        \left\lVert \bm{A} \right\rVert_F = \sqrt{\sum_{i,j}^{} A_{i,j}^2 }
    .\]
\end{eg}
\begin{eg}
    使用范数表示向量的内积:\[
        \bm{x}^\top \bm{y}=\left\lVert \bm{x} \right\rVert\cdot \left\lVert y \right\rVert\cos\theta
    .\]
    即:\[
        \bm{x}\cdot \bm{y}=xy\cos\theta
    .\]
\end{eg}
\subsection{特殊的矩阵和向量}%
\label{sub:特殊的矩阵和向量}
\begin{eg}
    对角矩阵,用$\bm{D}$ 表示,除主对角线上的元素其他的元素均为零:
    \[
        \bm{D}=\begin{bmatrix}
            a_{1,1} & 0 & \cdots & 0\\
            0 & a_{2,2} & \cdots  & 0\\
            \vdots & \vdots & \ddots & \vdots\\
            0 & 0 & \cdots & a_{n,n}
        \end{bmatrix}
    .\]
    \begin{notation}
        对角矩阵不一定是方阵,非方阵超出方阵的部分全部为0
    \end{notation}
\end{eg}
\begin{defi}
    对角转换diag:$\mathrm{diag}(\bm{v})$ 将$\bm{v}$ 中的元素放入对角矩阵中
\end{defi}
对角转换有以下性质:
\begin{itemize}
    \item $\mathrm{diag}(\bm{v})\bm{x}=\bm{v}\odot\bm{x}$
    \item $\mathrm{diag}(\bm{v})^{-1}=\mathrm{diag}(\begin{bmatrix}
            \frac{1}{\bm{v_1}} & \ldots  & \frac{1}{\bm{v_{n}}}\\
    \end{bmatrix}^\top )$
\end{itemize}
\begin{eg}
    对称矩阵:$\bm{A} = \bm{A}^\top $ ,常用于某些不依赖参数顺序的双参数函数,交换$i,j$ 后结果不变
\end{eg}
\begin{eg}
    单位向量:范数为1的向量,即:$\left\lVert \bm{x} \right\rVert_{n}=1$
\end{eg}
\begin{eg}
    向量的正交:$\bm{x}^\top \bm{y}=0$ ,代表着$\left\lVert x \right\rVert_{n}$ 和$\left\lVert y \right\rVert_{n}$ 都不为零时$\theta=\frac{\pi}{2}$

    标准正交:$\theta=\frac{\pi}{2}$ 且$\left\lVert \bm{x} \right\rVert_{n}=\left\lVert \bm{y} \right\rVert_{n}=1$
\end{eg}
\begin{eg}
    正交矩阵:行向量和列向量都分别标准正交,有以下性质:\[
        \bm{A}^\top \bm{A}=\bm{A}\bm{A}^\top =\bm{I}\Rightarrow \bm{A}^{-1}=\bm{A}^\top 
    .\]
    可得:正交矩阵求逆非常容易,可以用来求解线性方程组
\end{eg}
\subsection{特征分解}%
\label{sub:特征分解}
\begin{defi}
    特征向量和特征值:\[
        \bm{A}\bm{x}=\lambda\bm{x}
    .\]
    则$\lambda$ 为$\bm{A}$ 的特征值,$\bm{x}$ 为$\bm{A}$ 的一个(右)特征向量
    \[
        \bm{x}^\top \bm{A}=\lambda\bm{x}^\top 
    .\]
    称$\bm{x}$ 为左特征向量
\end{defi}
\begin{defi}
    特征分解:如果$\bm{A}$ 有$n$ 个线性无关的特征向量$\left\{ \bm{v}^{\left( 1 \right)},\ldots ,\bm{v}^{\left( n \right)} \right\}$ ,对应$n$ 个特征值$\left\{ \lambda_1,\ldots ,\lambda_{n} \right\}$ ,将特征向量(列向量)排列为一个矩阵$\bm{V}=\begin{bmatrix}
        \bm{v}^{(1)} & \ldots & \bm{v}^{(n)}\\
    \end{bmatrix}$,并将特征值排列为一个向量$\bm{\lambda}=\begin{bmatrix}
        \lambda_1 & \ldots & \lambda_{n}
    \end{bmatrix}^\top $,则:\[
        \bm{A}=\bm{V}\mathrm{diag}(\bm{\lambda})\bm{V}^{-1}
    .\]
\end{defi}
特征向量、值、分解的性质:
\begin{itemize}
    \item $\bm{A}$ 对$\bm{x}$ 变换相当于对$\bm{x}$ 缩放$\lambda$ 倍
    \item $\bm{v}$ 是$\bm{A}$ 的特征向量,则$c\bm{v}$ 也是,且特征值一样
    \item 实对称矩阵一定可以分解为实特征向量和实特征值,即$\bm{A}=\bm{Q\Lambda Q}^\top $
\end{itemize}
\begin{notation}
    实对称矩阵的分解:\[
        \bm{A}=\bm{Q\Lambda Q}^\top  
    .\]
    其中$\bm{Q}$ 为$\bm{A}$ 的特征向量$\bm{v}^{(i)}$组成的\textbf{正交矩阵},$\bm{\Lambda}$ 为对角矩阵,且$\Lambda_{i,i}=\lambda_{i}$ 所对应的特征向量是$\bm{Q}$ 的第$i$ 个列向量$\bm{Q}_{:,i}$
\end{notation}
使用$\bm{A}$ 进行矩阵乘法可以看作是:将空间各自沿$\bm{v}^{\left( i \right)}$ 延展$\lambda_{i}$ 倍
\begin{figure}[ht]
    \centering
    \incfig[0.7]{实对称矩阵的乘法}
    \caption{实对称矩阵的乘法}
    \label{fig:实对称矩阵的乘法}
\end{figure}
