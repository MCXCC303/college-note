\learn{1}{01.28}
\section{线性代数复习}%
\label{sec:线性代数复习}
\begin{defi}
    张量:多维数组或多维立方体(二维为矩阵,一维为向量,零维为常量),用字体不同的粗体表示:\textbf{\textsf{A}}

    对比:矩阵$\bm{A}$,张量\textbf{\textsf{A}}
\end{defi}
\begin{defi}
    向量:一般指列向量,转置后为横向量
\end{defi}
\begin{eg}
    $\bm{x}=\left( 1,2,3 \right)^\top=\begin{bmatrix}
        1\\
        2\\
        3
    \end{bmatrix}$,转置后:$\bm{x}^\top=\begin{bmatrix}
    1 & 2 & 3\\
    \end{bmatrix}$
\end{eg}
\begin{defi}
    矩阵乘法:左矩阵的行遍历右矩阵的列
\end{defi}
\begin{eg}
    两个向量的点乘(内积):$\bm{x}\cdot \bm{y}=\bm{x}^\top\bm{y}$,如下图:
\begin{figure}[ht]
    \centering
    \incfig[0.6]{点积}
    \caption{点积}
    \label{fig:点积}
\end{figure}

同理,两个向量的外积:$\bm{x}\times \bm{y}=\bm{x}\bm{y}^\top $ ,如下图:
\begin{figure}[ht]
    \centering
    \incfig[0.6]{外积}
    \caption{外积}
    \label{fig:外积}
\end{figure}
\end{eg}
\begin{defi}
    广播:把向量转置后加到矩阵的每一行
\end{defi}
\begin{eg}
    \begin{align*}
        &\bm{A}&+&\bm{b}&=&\bm{C}\\
        \Rightarrow& \begin{bmatrix}
            a_{11} & a_{12} & a_{13}\\
            a_{21} & a_{22} & a_{23}
        \end{bmatrix} &+& \begin{bmatrix}
            b_1 & b_2 & b_3
        \end{bmatrix}^\top &=& \begin{bmatrix}
            a_{11}+b_1 & a_{12}+b_2 & a_{13}+b_3\\
            a_{21}+b_1 & a_{22}+b_2 & a_{23}+b_3
        \end{bmatrix}
    .\end{align*}
\end{eg}
\begin{defi}
    范数:曼哈顿距离$\Rightarrow $ 欧氏距离$\Rightarrow \ldots $,$L^p$ 范数形如:\[
        \left\lVert \bm{x} \right\rVert_{p} = \left( \sum_{i} \left| x_{i} \right|^p \right)^\frac{1}{p}
    .\]
\end{defi}
\begin{eg}
    曼哈顿距离:$L^1:\left\lVert \bm{x} \right\rVert_{1} = \sum_{i}^{} \left| x_{i} \right|$ ,欧几里得距离:$L^2:\left\lVert \bm{x} \right\rVert_{2} = \sqrt{\sum_{i}^{} x_{i}^2 }$
\end{eg}
