\learn{3}{01.31}
\begin{notation}
    $\bm{x}^\top \bm{Ax}$ 表示二次型或者二次方程($\bm{x}$ 为二维向量),如:
    \begin{align*}
        \bm{x} &=\begin{bmatrix}
            x_1\\
            x_2
        \end{bmatrix}\quad \bm{A}=\begin{bmatrix}
        a & b\\
        b & c
        \end{bmatrix}\\
        \bm{x}^\top \bm{Ax}&= \begin{bmatrix}
            x_1 & x_2\\
        \end{bmatrix}\begin{bmatrix}
            a & b\\
            b & c
        \end{bmatrix}\begin{bmatrix}
            x_1\\
            x_2
        \end{bmatrix} \\
        &= \begin{bmatrix}
            x_1a+x_2b & x_1b+x_2c\\
        \end{bmatrix}\begin{bmatrix}
            x_1\\
            x_2
        \end{bmatrix} \\
        &= ax_1^2 +2bx_1x_2+cx_2^2
    .\end{align*}
\end{notation}
\begin{defi}
    所有特征值都是非负数的矩阵为\textbf{半正定}矩阵,对于半正定矩阵有:\[
        \forall \bm{x},\bm{x}^\top \bm{A}\bm{x}\ge 0
    .\]
    对于正定矩阵($A_{i,j}>0$)有:\[
        \bm{x}^\top \bm{Ax}=0\Rightarrow \bm{x}=\bm{0}
    .\]
\end{defi}
\begin{notation}
    当$\bm{x}$ 是$\bm{A}$ 的某个特征向量$\bm{v}$ 时,由于:\[
        \bm{Av}=\lambda\bm{v}
    .\]
    且$\bm{x}=\bm{v}=\begin{bmatrix}
        x_1 & x_2\\
    \end{bmatrix}^\top $ ,二次型可以写为:$f=\bm{v}^\top \bm{Av}$,带入得:
    \begin{align*}
        f&= \bm{v}^\top \bm{Av} \\
        &= \bm{v}^\top \lambda\bm{v} \\
        &= \lambda\bm{v}^\top \bm{v} \\
        &= \lambda\left( x_1^2 +x_2^2  \right)
    .\end{align*}
    如果限定$\left\lVert \bm{x} \right\rVert_{2}=1$ ,即$x_1^2 +x_2^2 =1$ ,则$f=\lambda$ ,即$\max\left( f \right)=\max\left( \lambda \right),\min\left( f \right)=\min\left( \lambda \right)$
\end{notation}
\subsection{奇异值分解}%
\label{sub:奇异值分解}
除特征分解外,还有一种分解叫做奇异值分解,将一个实对称矩阵分解为\textbf{奇异向量}和\textbf{奇异值}
\begin{notation}
    特征分解:$\bm{A}=\bm{V}\mathrm{diag}(\bm{\lambda})\bm{V}^{-1}$
\end{notation}
\begin{defi}
    奇异值分解(SVD):\[
        \bm{A}=\bm{UDV}^\top \quad \text{or} \quad \bm{A}=\bm{U\Sigma V}^\top 
    .\]
\end{defi}
\begin{notation}
    每个实数矩阵都有一个奇异值分解
\end{notation}
