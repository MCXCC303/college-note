\learn{5}{02.18}
\subsubsection*{奇异值分解的各部分来源}%
\label{subsub:奇异值分解的各部分来源}
假设任意$m\times n$矩阵$\bm{A}$

\textit{1. 求出两个对称矩阵}

由于:$\left( \bm{A}^\top \bm{A} \right)^\top =\bm{A}\bm{A}^\top $,即这两个矩阵(\textbf{方阵})都是对称的,令$\bm{S}_L:m\times m=\bm{A}\bm{A}^\top ,\bm{S}_R:n\times n=\bm{A}^\top \bm{A}$ ,提取出两个对称矩阵的特征向量组:\[
    \bm{S}_L\Rightarrow \begin{bmatrix}
        \bm{u_1} & \bm{u_2} & \ldots \bm{u_{m}}
    \end{bmatrix}=\bm{U}\quad \bm{S}_R \Rightarrow \begin{bmatrix}
        \bm{v_1} & \bm{v_2} & \ldots \bm{v_{n}}\\
    \end{bmatrix}=\bm{V}
.\]
如果求出$\bm{S}_L$ 和$\bm{S}_R$ 的特征值,并顺序排列,若$m>n$ ,则会有:$\bm{S}_L$ 的前$n$ 个特征值与$\bm{S}_R$ 的$n$ 个特征值一一对应,$\bm{S}_L$ 的剩余特征值都为0
\begin{notation}
    $\bm{A}$ 的\textbf{非零奇异值的平方}是$\bm{A}^\top \bm{A}$ 或$\bm{A}\bm{A}^\top $ 的特征值
\end{notation}
对非0的特征值开根号即为奇异值,即:$\lambda_{i}=\sigma_{i}^2 $
\begin{notation}
    $\bm{\Sigma}$中的奇异值为非负值,且按照降序排列,即: \[
        \sigma_1\ge \sigma_2\ge \ldots \ge \sigma_{r}>0
    .\]
    其中:\[
        \bm{\Sigma} = \begin{bmatrix}
            \sigma_1 & 0 & \cdots & 0\\
            0 & \sigma_2 & \cdots & 0\\
            \vdots & \vdots & \ddots & \vdots \\
            0 & 0 & \cdots & \sigma_{r}
        \end{bmatrix}
    .\]
\end{notation}
\subsection{Moore-Penrose伪逆}%
\label{sub:Moore-Penrose伪逆}

