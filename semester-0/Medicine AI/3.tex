\learn{3}{10.18}
编码规则:
\begin{align*}
    \begin{cases}
        \text{二级结构}
        \begin{cases}
            \text{H:}\alpha\text{螺旋}\to \left( 0,1,0 \right) \\
            \text{E:}\beta\text{折叠}\to \left( 1,0,0 \right)  \\
            \text{C:其他结构}\to \left( 0,0,1 \right) 
        \end{cases}\\
        \text{溶剂可及性}\begin{cases}
            \text{b: buried(包埋)}\to \left( 1,0 \right) \\
            \text{e: exposed(暴露)}\to \left( 0,1 \right) 
        \end{cases}
    \end{cases}
.\end{align*}
\begin{table}[htpb]
    \centering
    \begin{tabular}{|c|c|c|c|c|c|c|c|c|c|c|c|}
    \hline
    \multicolumn{2}{|c|}{Prot.} & M & V & L & S & P & A & D & K & T & N \\
    \hline
    \multicolumn{2}{|c|}{Sec.} & C & C & C & C & E & H & E & E & H & H \\
    \hline
    \hline
    \multirow{5}{*}{PSSSA}        &\multirow{3}{*}{PSS} & 0 & 0 & 0 & 0 & 1 & 0 & 1 & 1 & 0 & 0 \\
                                  & & 0 & 0 & 0 & 0 & 0 & 1 & 0 & 0 & 1 & 1 \\
                                  & & 1 & 1 & 1 & 1 & 0 & 0 & 0 & 0 & 0 & 0 \\
                                  &\multirow{2}{*}{PSA} & 0 & 0 & 1 & 0 & 0 & 0 & 1 & 1 & 0 & 0 \\
                                  & & 1 & 1 & 0 & 1 & 1 & 1 & 0 & 0 & 1 & 1 \\
                         \hline
                         \hline
    \multicolumn{2}{|c|}{S.A.} & e & e & b & e & e & b & b & e & e & e \\
    \hline
    \end{tabular}
\end{table}
\begin{eg}
    有一条10氨基酸长度的蛋白质序列:

    PSSSA使用$5\times 1000$ 的矩阵编码蛋白质,每一个氨基酸由一个5维向量表示

    用PSSSA编码时,一般取序列羧基的一侧开始的1000个氨基酸编码,如不满1000个使用0向量补齐
\end{eg}
\subsection{核酸物质表征}%
\label{sub:核酸物质表征}
\begin{notation}
    基本知识:碱基与核酸
\end{notation}
\subsubsection{碱基}%
\label{subsub:-碱基}
\begin{table}[htpb]
    \centering
    \caption{常见碱基}
    \label{tab:常见碱基}
    \begin{tabular}{ccc}
    \toprule
    种类 & DNA & RNA \\
    \midrule
    \multirow{2}{*}{嘌呤族(R)} & \multicolumn{2}{c}{腺嘌呤(A)} \\
                                 & \multicolumn{2}{c}{鸟嘌呤(G)} \\
                                 \midrule
    \multirow{2}{*}{嘧啶族(Y)} & \multicolumn{2}{c}{胞嘧啶(C)} \\
                                 & 胸腺嘧啶(T)& 尿嘧啶(U)\\
    \bottomrule
    \end{tabular}
\end{table}
\begin{notation}
    碱基配对方式:
    \begin{align*}
        \begin{cases}
            \text{DNA}
            \begin{cases}
                A= T \\
                C\equiv G
            \end{cases}\\
            \text{RNA}
            \begin{cases}
                A= U \\
                C\equiv G
            \end{cases}
        \end{cases}
    .\end{align*}
\end{notation}
\begin{notation}
    K-mer

    K: DNA或RNA中一个长度为K的序列

    以该序列为子序列,遍历核酸序列,计算该长度的所有子序列组合出现的频率
\end{notation}
\begin{eg}
    长度为$K$ 的K-mer种类共有$4^{k}$ 种可能

    如长度为3的子序列,子序列每个位置有A,G,C,U四种选择,共$4^3$ 种组合

    一段15个核酸的RNA序列如下:
    \begin{table}[htpb]
        \centering
        \caption{3-mer RNA}
        \label{tab:3-mer-RNA}
        \begin{tabular}{|c|c|c|c|c|c|c|c|c|c|c|c|c|c|}
        \hline
        C & A & T & C & G & G & T & A & A & C & C & C & A & T \\
        \hline
        \end{tabular}
    \end{table}

    所有可能的长度为3的子序列及其频率:
    \begin{table}[htpb]
        \centering
        \caption{3-mers}
        \label{tab:3-mers}
        \begin{tabular}{ccc}
        \toprule
        & RNA seq. & freq. \\
        \midrule
            1 & CAT & 0.111\\
            2 & ATC & 0.056\\
            3 & TCG & 0.056\\
            4 & CGG & 0.056\\
            $\ldots$ & $\ldots$ & $\ldots$\\
            12 & CCA & 0.056\\
            13 & ATG & 0\\
            $\ldots$ & $\ldots$ & $\ldots$ \\
            64 & $\ldots$ & 0 \\
        \bottomrule
        \end{tabular}
    \end{table}
\end{eg}


