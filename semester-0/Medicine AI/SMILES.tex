{\centering{\subsection*{SMILES RULE}%
\label{sub:SMILES RULE}}}
\subsubsection*{1. 简单规则}%
\label{subsub:1-简单规则}
原子:原子缩写符号
\begin{eg}
    Au, Pt, C, N
\end{eg}
离子:原子加上电荷数,外接中括号
\begin{eg}
    $\text{Fe}^{3+}$: [Fe+++]

    $\text{C}^{-}$ : [C-]

    $\text{Pt}^{6+}$ : [Pt++++++]
\end{eg}
H原子:省略

相邻原子:直接连接
\begin{eg}
    Dodecane: CCCCCCCCCCCC (12 Carbons)
\end{eg}
分支:以小括号表示
\begin{eg}
    Write in git style:
    \begin{center}
        \begin{tikzpicture}
            \draw [] (-2,0)--(0,0);
            \draw [] (0,0)--(2,0);
            \draw [] (2,0)--(4,0);
            \draw [] (0,0)--(1,2);
            \draw [] (1,2)--(3,2);
            \draw [] (3,2)--(5,2);
            \filldraw [color=white] (-2,0) circle [radius=0.5] node at(-2,0) {$ $};
            \filldraw [color=white] (0,0) circle [radius=0.5] node at(0,0) {$ $};
            \filldraw [color=white] (2,0) circle [radius=0.5] node at(2,0) {$ $};
            \filldraw [color=white] (4,0) circle [radius=0.5] node at(4,0) {$ $};
            \filldraw [color=white] (1,2) circle [radius=0.5] node at(1,2) {$ $};
            \filldraw [color=white] (3,2) circle [radius=0.5] node at(3,2) {$ $};
            \filldraw [color=white] (5,2) circle [radius=0.5] node at(5,2) {$ $};
            \node at(-2,0) {$A$};
            \node at(0,0) {$B$};
            \node at(2,0) {$C$};
            \node at(4,0) {$D$};
            \node at(1,2) {$E$};
            \node at(3,2) {$F$};
            \node at(5,2) {$G$};
        \end{tikzpicture}
    \end{center}
    SMILES: AB(EFG)CD
\end{eg}
单键:直接省略

双键:“=”

三键:“\#”

芳香键=单键(直接省略)
\begin{notation}
    部分软件芳香键使用单双键交替表示

    芳香原子使用小写字母
\end{notation}
\begin{eg}
    hex-2-en-4-yne/戊-2-烯-4-炔(不分顺反):CC=CC\#CC

    toluene: Cc1ccccc1
\end{eg}
\subsubsection*{2. 立体结构}%
\label{subsub:2-立体结构}
环状结构:将环断开形成线性结构,以数字标记断开的原子
\begin{eg}
    Cyclohexane: C1CCCCC1
\end{eg}
同位素:[核电荷数+元素符号]
\begin{eg}
    $^{13}\text{C}$: [13C]
\end{eg}
Z/E构象:使用“/”和“$\backslash$”代表单键方向
\begin{eg}
    (2E)-hex-2-en-4-yne: C/C=C/C\#CC

    (2Z)-hex-2-en-4-yne: C/C=C$\backslash$C\#CC
\end{eg}
手性异构:@表示S,@@代表R
\begin{eg}
\begin{figure}[htpb]
    \centering
    \includegraphics[width=0.2\textwidth]{figures/S&R}
    \caption{S\&R}
    \label{fig:fig-S-R}
\end{figure}
$-\text{CH}_{3}$最小,放在最后,对基团大小比较:
\[
    \text{F}>\text{NH}_{2}>\text{COOH}
.\] 
为R构型,即:N[C@@](F)(C)C(=O)O
\end{eg}
\subsubsection*{3. 算法与生成}%
\label{subsub:3-算法与生成}
\begin{notation}
    大部分SMILES生成算法为商业算法,如Morgan算法、Canonical SMILES算法等

    生成SMILES主要使用深度优先搜素(DFS)算法遍历分子图
\end{notation}

