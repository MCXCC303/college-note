\learn{5}{10.23}
\subsection{模型评估和性能度量}%
\label{sub:模型评估和性能度量}
\begin{notation}
    留出法(\textit{hold-out}):

    将原始数据集$D$ 分为两个互斥的子集$S,T$,$S$作为训练数据集,$T$ 作为测试数据集:$D=S\cup T,S\cap T=\varnothing $
\end{notation}
在划分任务时要尽量保证$S$ 和$T$ 中的样本类别比例相似
\begin{eg}
    \[
        D\left( a,b \right)  \to S\left( \lambda a,\lambda b \right) \cup T\left( \left( 1-\lambda \right) a,\left( 1-\lambda \right) b \right)
    .\] 

    该过程称为分层采样法,其中$\lambda\in [\frac{2}{3} , \frac{4}{5} ]$

    使用$S$ 训练模型,$T$ 进行模型测试,多次随机划分$a,b$ 在$S$ 和$T$ 内的内容,多次实验取测试结果平均值
\end{eg}
\begin{notation}
    交叉验证法/\textit{k}折交叉验证(\textit{cross validation}/\textit{k-fold cross validation}):\[
        D=D_1\cup D_2\cup \ldots\cup D_k\text{ 且 }D_I\cap D_j=\varnothing \left( i\neq j \right)
    .\]   

    此处$\forall D_i$ 由$D$ 分层采样得到

    每次实验使用$k-1$ 个子集的并集训练,剩下的一个子集作为测试集:\[
        S=\sum_{i=1}^{m-1} D_i + \sum_{i=m+1}^{k}D_i \hspace{0.5cm} T=D_m
    .\] 

    取不同的$m$ 值共可以得到$k$ 组“训练集-测试集”,得到$k$ 个结果,取$k$ 个结果的平均值
\end{notation}
\begin{eg}
    5折交叉验证的数据划分:
    \begin{table}[htpb]
        \centering
        \begin{tabular}{|c|c|c|c|c|}
        \hline
        $D_1$ & $D_2$ & $D_3$ & $D_4$ & $\bm{D}_5$ \\
        \hline
        $D_1$ & $D_2$ & $D_3$ & $\bm{D}_4$ & $D_5$ \\
        \hline
        $D_1$ & $D_2$ & $\bm{D}_3$ & $D_4$ & $D_5$ \\
        \hline
        $D_1$ & $\bm{D}_2$ & $D_3$ & $D_4$ & $D_5$ \\
        \hline
        $\bm{D}_1$ & $D_2$ & $D_3$ & $D_4$ & $D_5$ \\
        \hline
        \end{tabular}
        $\Rightarrow \begin{cases}
            Res_1\\
            Res_2\\
            Res_3\\
            Res_4\\
            Res_5
        \end{cases}$
        \ce{->[Avg.]}Result
    \end{table}
\end{eg}
\begin{notation}
    若样本量$m$ 等于子集数$k$ ,交叉验证法等同于留一法(\textit{leave one out}, LOO)

    留一法的优点:训练结果更准确

    缺点:样本量太大的时候消耗过多资源
\end{notation}
\subsection{模型性能度量}%
\label{sub:模型性能度量}
\begin{notation}
    错误率:\[
        E=\frac{1}{m} N\left( f\left( x_{i} \right) \neq y_{i} \right) 
    .\] 

    准确率:\[
        \text{Acc}=\frac{1}{m} N\left( f\left( x_{i} \right) =y_{i} \right) 
    .\] 

    $m$ 为样本总数,$N\left( f\left( x_{i} \right) =y \right) $ 表示符合特征$f:x\to y$的样本数量
\end{notation}
\begin{notation}
    二分类问题:
\end{notation}







