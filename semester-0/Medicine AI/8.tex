\learn{8}{11.04}
\begin{notation}
    当$L_{\bm{P}-\bm{R}}^{(1)}$ 完全包裹$L_{\bm{P}-\bm{R}}^{(2)}$ 时,代表模型1在各个阈值下查全率和查准率都较模型2更好,但当$L_{\bm{P}-\bm{R}}^{(m,n)}$相交时,无法通过曲线直接判断
\end{notation}
缺点:未知曲线的面积不好求,无法判断相交曲线之间的性能关系,因此采用其他方法评估$\bm{P}-\bm{R}$值的关联
\begin{notation}
    平衡点BEP:

    作平衡线(一般为$y=ax,a\in [0,+\infty ]$),交曲线$L_{\bm{P}-\bm{R}}^{(m,n)}$于两个点,判断点的高低

    缺点:太过简单
\end{notation}
\begin{notation}
    $\bm{F}_1$ 度量:

\end{notation}

