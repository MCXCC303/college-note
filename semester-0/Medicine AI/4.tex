\learn{4}{10.20}
\subsection{数据预处理}%
\label{sub:数据预处理}
\subsubsection{标准化}%
\label{subsub:标准化}
\begin{notation}
    变量离差标准化:标准化后所有变量范围都在[0,1]内
    \[
        y_{i}=\frac{x_{i}-x_{\min}}{x_{\max}-x_{\min}}
    .\] 
\end{notation}
\begin{eg}
    一组变量如下:
    \[
        X=\left( 1.5,1.7,2.2,1.2,1.6,1.4,1.1 \right) 
    .\] 

    易得$x_{\min}=1.1,x_{\max}=2.2$
    \begin{align*}
        y_{i}&= \frac{x_{i}-x_{\min}}{x_{\max}-x_{\min}}\\
        &= \frac{x_{i}-1.1}{2.2-1.1} \\
        &= \frac{x_{i}-1.1}{1.1} \\
        &= \frac{x_{i}}{1.1}-1
    .\end{align*}

    得$Y=\left( 0.364,0.545,1,0.091,0.455,0.273,0 \right) $
\end{eg}
\begin{notation}
    \textit{Z-score}(变量标准差)标准化

    经过标准化后平均值为0,标准差为1
    \[
        z_i=\frac{x_{i}-\bar{x}}{s} \quad s=\sqrt{\frac{1}{n-1}\sum_{i=1}^{n} \left( x_{i}-\bar{x} \right) ^2} 
    .\] 

    可以看出$s$ 为原数据的标准差,$z_i$ 值其实等同于标准正态分布中的$u$ 值:
    \[
        u=\frac{x-\mu}{\sigma} \quad y=\frac{1}{\sigma\sqrt{2\pi} }\mathrm{e}^{-u^2}/2
    .\] 
\end{notation}
\subsubsection{插补缺失值}%
\label{subsub:插补缺失值}
\begin{notation}
    均值插补

    1. 数值性变量:采用平均值插补

    2. 离散型:采用众数插补
\end{notation}
\begin{notation}
    同类均值插补:使用层次聚类方法归类缺失值的样本,用该类别的特征均值插补
\end{notation}
\begin{notation}
    KNN(\textit{K-nearest neighbor})缺失值插补:找到与含缺失值样本相似的$K$个样本,使用这$K$ 个样本在该缺失变量上的均值填充
\end{notation}
../../semester-0/Medicine AI/KNN.tex

