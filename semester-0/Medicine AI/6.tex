\learn{6}{10.31}
\begin{notation}
    二分类问题:
    
    将一个样本分至两个类别的问题,如:鉴定邮件是否为垃圾邮件,预测某人是否会患上某种疾病等问题
\end{notation}
对于二分类问题,真实结果有两种,使用模型预测也会产生两种结果,组合得到混淆矩阵:
\begin{align*}
    \begin{bmatrix}
        \text{真阳性(TP)} & \text{假阴性(FN)}\\
        \text{假阳性(FP)} & \text{真阴性(TN)}\\
    \end{bmatrix}
.\end{align*}

其中:阳性/阴性为模型预测结果,真/假为真实结果

准确率(Acc)根据混淆矩阵的计算:
\[
    \text{Acc}=\frac{\text{TP}+\text{TN}}{\text{TP}+\text{TN}+\text{FP}+\text{FN}} 
.\] 
\begin{notation}
    马修斯相关系数(Matthews Correlation Coefficient, MCC):

    MCC比Acc更加全面(正负数据不平衡)
\end{notation}
\[
    \text{MCC}=\frac{\text{TP}\times \text{TN}-\text{FP}\times \text{FN}}{\sqrt{\left( \text{TP}+\text{FP} \right) \left( \text{TP}+\text{FN} \right) \left( \text{TN}+\text{FP} \right) \left( \text{TN}+\text{FN} \right) } } \in [-1,1]
.\] 
\begin{notation}
    MCC结果解读:

    $\circ$ $\text{FP}=\text{FN}=0$ :无误判结果,代入得:$\text{MCC}=1$ ,表示模型完美

    $\circ$ $\text{TP}=\text{TN}=0$ :全部误判,代入得:$\text{MCC}=-1$ ,表示最差

    $\circ$ $\text{TP}\times \text{TN}=\text{FP}\times \text{FN}$ ,即$\text{MCC}=0$ ,表示模型完全随机判断

    当样本中阴性样本远少于阳性样本时,Acc计算不能涉及到假阴性与假阳性而MCC可以

    若第一个模型对阳性和阴性样本判断接近,而第二个模型对阳性样本表现极佳但对阴性样本表现极差,则$\text{MCC}_1>\text{MCC}_2$,而Acc可能接近
\end{notation}
\begin{notation}
    查准率$\bm{P}$ ,查全率$\bm{R}$,$\bm{F}_1$ 度量:
    
    $\circ$ 查准率(precision, $\bm{P}$) :又叫精确率
\end{notation}
\begin{align*}
    \bm{P} &=\frac{N_\text{TP}}{N_{\text{P}_p}} \\
    &= \frac{\text{TP}}{\text{TP}+\text{FP}}
.\end{align*}
\begin{notation}
    $\circ$ 查全率(recall, $\bm{R}$):又叫召回率
\end{notation}
\begin{align*}
    \bm{R}&= \frac{N_\text{TP}}{N_{\text{P}_a}}  \\
    &= \frac{\text{TP}}{\text{TP}+\text{FN}}
.\end{align*}

一般情况下:查全率和查准率相矛盾
