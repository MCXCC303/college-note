%───────────────────%
% Document Settings %
%───────────────────%
\documentclass[12pt,a4paper]{article}

%─────────────────%
% Package Imports %
%─────────────────%
\usepackage[]{amssymb}
\usepackage[]{amsthm}
\usepackage[]{amsmath}
\usepackage[]{array}
\usepackage[]{bm}
\usepackage[]{booktabs}
\usepackage[UTF8]{ctex}
\usepackage[]{geometry}
\usepackage[]{hyperref}
\usepackage[]{import}
\usepackage[]{mathrsfs}
\usepackage[]{multirow}
\usepackage[]{pdfpages}
\usepackage[]{pgfplots}
\usepackage[]{tabu}
\usepackage[]{tikz}
\usepackage[]{tkz-euclide}
\usepackage[]{wrapfig}
\usepackage[]{xifthen}
\usepackage[]{yhmath}

%──────────────────%
% pgfplot Settings %
%──────────────────%
\usepgfplotslibrary{external}
\pgfarrowsdeclarecombine{twolatex'}{twolatex'}{latex'}{latex'}{latex'}{latex'}
\pgfplotsset{compat=1.12}

%───────────────%
% Tikz Settings %
%───────────────%
\usetikzlibrary{arrows.meta}
\usetikzlibrary{decorations.markings}
\usetikzlibrary{decorations.pathmorphing}
\usetikzlibrary{positioning}
\usetikzlibrary{fadings}
\usetikzlibrary{intersections}
\usetikzlibrary{cd}
\tikzset{->/.style = {decoration={markings,mark=at position 1 with {\arrow[scale=2]{latex'}}},postaction={decorate}}}
\tikzset{<-/.style = {decoration={markings,mark=at position 0 with {\arrowreversed[scale=2]{latex'}}},postaction={decorate}}}
\tikzset{<->/.style = {decoration={markings,mark=at position 0 with {\arrowreversed[scale=2]{latex'}},mark=at position 1 with {\arrow[scale=2]{latex'}}},postaction={decorate}}}
\tikzset{->-/.style = {decoration={markings,mark=at position #1 with {\arrow[scale=2]{latex'}}},postaction={decorate}}}
\tikzset{-<-/.style = {decoration={markings,mark=at position #1 with {\arrowreversed[scale=2]{latex'}}},postaction={decorate}}}
\tikzset{->>/.style = {decoration={markings,mark=at position 1 with {\arrow[scale=2]{latex'}}}postaction={decorate}}}
\tikzset{<<-/.style = {decoration={markings,mark=at position 0 with {\arrowreversed[scale=2]{twolatex'}}},postaction={decorate}}}
\tikzset{<<->>/.style = {decoration={markings,mark=at position 0 with {\arrowreversed[scale=2]{twolatex'}},mark=at position 1 with {\arrow[scale=2]{twolatex'}}},postaction={decorate}}}
\tikzset{->>-/.style = {decoration={markings,mark=at position #1 with {\arrow[scale=2]{twolatex'}}},postaction={decorate}}}
\tikzset{-<<-/.style = {decoration={markings,mark=at position #1 with {\arrowreversed[scale=2]{twolatex'}}},postaction={decorate}}}
\tikzset{circ/.style = {fill, circle, inner sep = 0, minimum size = 3}}
\tikzset{scirc/.style = {fill, circle, inner sep = 0, minimum size = 1.5}}
\tikzset{mstate/.style={circle, draw, blue, text=black, minimum width=0.7cm}}
\tikzset{eqpic/.style={baseline={([yshift=-.5ex]current bounding box.center)}}}
\tikzset{commutative diagrams/.cd,cdmap/.style={/tikz/column 1/.append style={anchor=base east},/tikz/column 2/.append style={anchor=base west},row sep=tiny}}

%───────────────────%
% Inkscape Settings %
%───────────────────%
% \newcommand{\incfig}[1]{%
%     \def\svgwidth{\columnwidth}
%     \import{./fig/}{#1.pdf_tex}
% }

%──────────────────────%
% Theorem Environments %
%──────────────────────%
\theoremstyle{definition}
\newtheorem*{aim}{Aim}
\newtheorem*{axiom}{Axiom}
\newtheorem*{claim}{Claim}
\newtheorem*{cor}{Corollary}
\newtheorem*{conjecture}{Conjecture}
\newtheorem*{defi}{Definition}
\newtheorem*{eg}{Example}
\newtheorem*{ex}{Exercise}
\newtheorem*{fact}{Fact}
\newtheorem*{law}{Law}
\newtheorem*{lemma}{Lemma}
\newtheorem*{notation}{Notation}
\newtheorem*{prop}{Proposition}
\newtheorem*{question}{Question}
\newtheorem*{rrule}{Rule}
\newtheorem*{thm}{Theorem}
\newtheorem*{assumption}{Assumption}
\newtheorem*{remark}{Remark}
\newtheorem*{warning}{Warning}
\newtheorem*{exercise}{Exercise}
\newtheorem{nthm}{Theorem}[section]
\newtheorem{nlemma}[nthm]{Lemma}
\newtheorem{nprop}[nthm]{Proposition}
\newtheorem{ncor}[nthm]{Corollary}

%────────────%
% Beginnings %
%────────────%
\title{\textbf{Part III-B: \className}}
\author{Lecture by \lecturer\\Note by \noter}

%───────────────%
% Page Settings %
%───────────────%
% \addtolength{\parskip}{-1mm}
% \addtolength{\parindent}{-2mm}
% \geometry{left=0.5cm,right=0.5cm,top=0.5cm,bottom=0.5cm}
